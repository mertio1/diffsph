%% Generated by Sphinx.
\def\sphinxdocclass{report}
\documentclass[letterpaper,10pt,english]{sphinxmanual}
\ifdefined\pdfpxdimen
   \let\sphinxpxdimen\pdfpxdimen\else\newdimen\sphinxpxdimen
\fi \sphinxpxdimen=.75bp\relax
\ifdefined\pdfimageresolution
    \pdfimageresolution= \numexpr \dimexpr1in\relax/\sphinxpxdimen\relax
\fi
%% let collapsable pdf bookmarks panel have high depth per default
\PassOptionsToPackage{bookmarksdepth=5}{hyperref}

\PassOptionsToPackage{warn}{textcomp}
\usepackage[utf8]{inputenc}
\ifdefined\DeclareUnicodeCharacter
% support both utf8 and utf8x syntaxes
  \ifdefined\DeclareUnicodeCharacterAsOptional
    \def\sphinxDUC#1{\DeclareUnicodeCharacter{"#1}}
  \else
    \let\sphinxDUC\DeclareUnicodeCharacter
  \fi
  \sphinxDUC{00A0}{\nobreakspace}
  \sphinxDUC{2500}{\sphinxunichar{2500}}
  \sphinxDUC{2502}{\sphinxunichar{2502}}
  \sphinxDUC{2514}{\sphinxunichar{2514}}
  \sphinxDUC{251C}{\sphinxunichar{251C}}
  \sphinxDUC{2572}{\textbackslash}
\fi
\usepackage{cmap}
\usepackage[T1]{fontenc}
\usepackage{amsmath,amssymb,amstext}
\usepackage{babel}



\usepackage{tgtermes}
\usepackage{tgheros}
\renewcommand{\ttdefault}{txtt}



\usepackage[Bjarne]{fncychap}
\usepackage{sphinx}

\fvset{fontsize=auto}
\usepackage{geometry}


% Include hyperref last.
\usepackage{hyperref}
% Fix anchor placement for figures with captions.
\usepackage{hypcap}% it must be loaded after hyperref.
% Set up styles of URL: it should be placed after hyperref.
\urlstyle{same}

\addto\captionsenglish{\renewcommand{\contentsname}{Contents:}}

\usepackage{sphinxmessages}
\setcounter{tocdepth}{2}



\title{diffsph}
\date{Nov 23, 2023}
\release{1.0.0.}
\author{Martin Vollmann, Finn Welzmueller, Lovorka Gajovic}
\newcommand{\sphinxlogo}{\vbox{}}
\renewcommand{\releasename}{Release}
\makeindex
\begin{document}

\pagestyle{empty}
\sphinxmaketitle
\pagestyle{plain}
\sphinxtableofcontents
\pagestyle{normal}
\phantomsection\label{\detokenize{index::doc}}


\sphinxAtStartPar
\sphinxcode{\sphinxupquote{diffsph}} is a Python package that computes diffuse signals (e.g. brightness) emitted by Milky\sphinxhyphen{}Way satellite dwarf spheroidal galaxies. The underlining physics is mostly based on %
\begin{footnote}[1]\sphinxAtStartFootnote
M. Vollmann, “Universal profiles for radio searches of dark matter in dwarf galaxies”,
doi:\sphinxhref{https://iopscience.iop.org/article/10.1088/1475-7516/2021/04/068}{10.1088/1475\sphinxhyphen{}7516/2021/04/068}
{[}arXiv:\sphinxhref{https://arxiv.org/abs/2011.11947}{2011.11947 {[}astro\sphinxhyphen{}ph.HE{]}}{]}.
%
\end{footnote}.


\chapter{diffsph}
\label{\detokenize{modules:diffsph}}\label{\detokenize{modules::doc}}

\section{diffsph package}
\label{\detokenize{diffsph:diffsph-package}}\label{\detokenize{diffsph::doc}}

\subsection{Subpackages}
\label{\detokenize{diffsph:subpackages}}

\subsubsection{diffsph.profiles package}
\label{\detokenize{diffsph.profiles:diffsph-profiles-package}}\label{\detokenize{diffsph.profiles::doc}}

\paragraph{Submodules}
\label{\detokenize{diffsph.profiles:submodules}}

\paragraph{diffsph.profiles.analytics module}
\label{\detokenize{diffsph.profiles:module-diffsph.profiles.analytics}}\label{\detokenize{diffsph.profiles:diffsph-profiles-analytics-module}}\index{module@\spxentry{module}!diffsph.profiles.analytics@\spxentry{diffsph.profiles.analytics}}\index{diffsph.profiles.analytics@\spxentry{diffsph.profiles.analytics}!module@\spxentry{module}}\index{cobrA() (in module diffsph.profiles.analytics)@\spxentry{cobrA()}\spxextra{in module diffsph.profiles.analytics}}

\begin{fulllineitems}
\phantomsection\label{\detokenize{diffsph.profiles:diffsph.profiles.analytics.cobrA}}\pysiglinewithargsret{\sphinxcode{\sphinxupquote{diffsph.profiles.analytics.}}\sphinxbfcode{\sphinxupquote{cobrA}}}{\emph{\DUrole{n}{t}}, \emph{\DUrole{n}{rs}}, \emph{\DUrole{n}{rh}}}{}
\sphinxAtStartPar
Brightness H\sphinxhyphen{}function for the ‘constant’ top\sphinxhyphen{}hat source in the regime\sphinxhyphen{}A approximation
\begin{quote}\begin{description}
\item[{Parameters}] \leavevmode\begin{itemize}
\item {} 
\sphinxAtStartPar
\sphinxstyleliteralstrong{\sphinxupquote{t}} \textendash{} \(D\sin(\theta)/r_h\), where \(\theta\) (\sphinxcode{\sphinxupquote{theta}}), \(r_h\) (\sphinxcode{\sphinxupquote{rh}}) and \(D\) (\sphinxcode{\sphinxupquote{dist}}) are defined below

\item {} 
\sphinxAtStartPar
\sphinxstyleliteralstrong{\sphinxupquote{theta}} \textendash{} angular radius in rad

\item {} 
\sphinxAtStartPar
\sphinxstyleliteralstrong{\sphinxupquote{rs}} \textendash{} scale radius

\item {} 
\sphinxAtStartPar
\sphinxstyleliteralstrong{\sphinxupquote{rh}} \textendash{} diffusion radius parameter

\item {} 
\sphinxAtStartPar
\sphinxstyleliteralstrong{\sphinxupquote{dist}} \textendash{} distance to the source

\end{itemize}

\end{description}\end{quote}

\end{fulllineitems}

\index{cobrC() (in module diffsph.profiles.analytics)@\spxentry{cobrC()}\spxextra{in module diffsph.profiles.analytics}}

\begin{fulllineitems}
\phantomsection\label{\detokenize{diffsph.profiles:diffsph.profiles.analytics.cobrC}}\pysiglinewithargsret{\sphinxcode{\sphinxupquote{diffsph.profiles.analytics.}}\sphinxbfcode{\sphinxupquote{cobrC}}}{\emph{\DUrole{n}{t}}, \emph{\DUrole{n}{rs}}, \emph{\DUrole{n}{rh}}}{}
\sphinxAtStartPar
Brightness H\sphinxhyphen{}function for the ‘constant’ source in the regime\sphinxhyphen{}C approximation
\begin{quote}\begin{description}
\item[{Parameters}] \leavevmode\begin{itemize}
\item {} 
\sphinxAtStartPar
\sphinxstyleliteralstrong{\sphinxupquote{t}} \textendash{} \(D\sin(\theta)/r_h\), where \(\theta\) (\sphinxcode{\sphinxupquote{theta}}), \(r_h\) (\sphinxcode{\sphinxupquote{rh}}) and \(D\) (\sphinxcode{\sphinxupquote{dist}}) are defined below

\item {} 
\sphinxAtStartPar
\sphinxstyleliteralstrong{\sphinxupquote{theta}} \textendash{} angular radius in rad

\item {} 
\sphinxAtStartPar
\sphinxstyleliteralstrong{\sphinxupquote{rs}} \textendash{} scale radius

\item {} 
\sphinxAtStartPar
\sphinxstyleliteralstrong{\sphinxupquote{rh}} \textendash{} diffusion radius parameter

\end{itemize}

\end{description}\end{quote}

\end{fulllineitems}

\index{cofdA() (in module diffsph.profiles.analytics)@\spxentry{cofdA()}\spxextra{in module diffsph.profiles.analytics}}

\begin{fulllineitems}
\phantomsection\label{\detokenize{diffsph.profiles:diffsph.profiles.analytics.cofdA}}\pysiglinewithargsret{\sphinxcode{\sphinxupquote{diffsph.profiles.analytics.}}\sphinxbfcode{\sphinxupquote{cofdA}}}{\emph{\DUrole{n}{theta}}, \emph{\DUrole{n}{rs}}, \emph{\DUrole{n}{rh}}, \emph{\DUrole{n}{dist}}}{}
\sphinxAtStartPar
Flux\sphinxhyphen{}density H\sphinxhyphen{}function for the ‘constant’ top\sphinxhyphen{}hat source in the regime\sphinxhyphen{}A approximation
\begin{quote}\begin{description}
\item[{Parameters}] \leavevmode\begin{itemize}
\item {} 
\sphinxAtStartPar
\sphinxstyleliteralstrong{\sphinxupquote{theta}} \textendash{} angular radius in rad

\item {} 
\sphinxAtStartPar
\sphinxstyleliteralstrong{\sphinxupquote{rs}} \textendash{} scale radius

\item {} 
\sphinxAtStartPar
\sphinxstyleliteralstrong{\sphinxupquote{rh}} \textendash{} diffusion radius parameter

\item {} 
\sphinxAtStartPar
\sphinxstyleliteralstrong{\sphinxupquote{dist}} \textendash{} distance to the source

\end{itemize}

\end{description}\end{quote}

\end{fulllineitems}

\index{cofdAmax() (in module diffsph.profiles.analytics)@\spxentry{cofdAmax()}\spxextra{in module diffsph.profiles.analytics}}

\begin{fulllineitems}
\phantomsection\label{\detokenize{diffsph.profiles:diffsph.profiles.analytics.cofdAmax}}\pysiglinewithargsret{\sphinxcode{\sphinxupquote{diffsph.profiles.analytics.}}\sphinxbfcode{\sphinxupquote{cofdAmax}}}{\emph{\DUrole{n}{rs}}, \emph{\DUrole{n}{rh}}, \emph{\DUrole{n}{dist}}}{}
\sphinxAtStartPar
Maximum value for the flux\sphinxhyphen{}density H\sphinxhyphen{}function for the ‘constant’ top\sphinxhyphen{}hat source in the regime\sphinxhyphen{}A approximation
\begin{quote}\begin{description}
\item[{Parameters}] \leavevmode\begin{itemize}
\item {} 
\sphinxAtStartPar
\sphinxstyleliteralstrong{\sphinxupquote{rs}} \textendash{} scale radius

\item {} 
\sphinxAtStartPar
\sphinxstyleliteralstrong{\sphinxupquote{rh}} \textendash{} diffusion radius parameter

\item {} 
\sphinxAtStartPar
\sphinxstyleliteralstrong{\sphinxupquote{dist}} \textendash{} distance to the source

\end{itemize}

\end{description}\end{quote}

\end{fulllineitems}

\index{cofdC() (in module diffsph.profiles.analytics)@\spxentry{cofdC()}\spxextra{in module diffsph.profiles.analytics}}

\begin{fulllineitems}
\phantomsection\label{\detokenize{diffsph.profiles:diffsph.profiles.analytics.cofdC}}\pysiglinewithargsret{\sphinxcode{\sphinxupquote{diffsph.profiles.analytics.}}\sphinxbfcode{\sphinxupquote{cofdC}}}{\emph{\DUrole{n}{theta}}, \emph{\DUrole{n}{rs}}, \emph{\DUrole{n}{rh}}, \emph{\DUrole{n}{dist}}}{}
\sphinxAtStartPar
Flux\sphinxhyphen{}density H\sphinxhyphen{}function for the ‘constant’ top\sphinxhyphen{}hat source in the regime\sphinxhyphen{}C approximation
\begin{quote}\begin{description}
\item[{Parameters}] \leavevmode\begin{itemize}
\item {} 
\sphinxAtStartPar
\sphinxstyleliteralstrong{\sphinxupquote{theta}} \textendash{} angular radius in rad

\item {} 
\sphinxAtStartPar
\sphinxstyleliteralstrong{\sphinxupquote{rs}} \textendash{} scale radius

\item {} 
\sphinxAtStartPar
\sphinxstyleliteralstrong{\sphinxupquote{rh}} \textendash{} diffusion radius parameter

\item {} 
\sphinxAtStartPar
\sphinxstyleliteralstrong{\sphinxupquote{dist}} \textendash{} distance to the source

\end{itemize}

\end{description}\end{quote}

\end{fulllineitems}

\index{cofdCmax() (in module diffsph.profiles.analytics)@\spxentry{cofdCmax()}\spxextra{in module diffsph.profiles.analytics}}

\begin{fulllineitems}
\phantomsection\label{\detokenize{diffsph.profiles:diffsph.profiles.analytics.cofdCmax}}\pysiglinewithargsret{\sphinxcode{\sphinxupquote{diffsph.profiles.analytics.}}\sphinxbfcode{\sphinxupquote{cofdCmax}}}{\emph{\DUrole{n}{rs}}, \emph{\DUrole{n}{rh}}, \emph{\DUrole{n}{dist}}}{}
\sphinxAtStartPar
Maximum value for the flux\sphinxhyphen{}density H\sphinxhyphen{}function for the ‘constant’ top\sphinxhyphen{}hat source in the regime\sphinxhyphen{}C approximation
\begin{quote}\begin{description}
\item[{Parameters}] \leavevmode\begin{itemize}
\item {} 
\sphinxAtStartPar
\sphinxstyleliteralstrong{\sphinxupquote{rs}} \textendash{} scale radius

\item {} 
\sphinxAtStartPar
\sphinxstyleliteralstrong{\sphinxupquote{rh}} \textendash{} diffusion radius parameter

\item {} 
\sphinxAtStartPar
\sphinxstyleliteralstrong{\sphinxupquote{dist}} \textendash{} distance to the source

\end{itemize}

\end{description}\end{quote}

\end{fulllineitems}

\index{psbrA() (in module diffsph.profiles.analytics)@\spxentry{psbrA()}\spxextra{in module diffsph.profiles.analytics}}

\begin{fulllineitems}
\phantomsection\label{\detokenize{diffsph.profiles:diffsph.profiles.analytics.psbrA}}\pysiglinewithargsret{\sphinxcode{\sphinxupquote{diffsph.profiles.analytics.}}\sphinxbfcode{\sphinxupquote{psbrA}}}{\emph{\DUrole{n}{theta}}, \emph{\DUrole{n}{rs}}, \emph{\DUrole{n}{rh}}, \emph{\DUrole{n}{dist}}}{}
\sphinxAtStartPar
Brigthness H\sphinxhyphen{}function for point sources in the regime\sphinxhyphen{}A approximation
\begin{quote}\begin{description}
\item[{Parameters}] \leavevmode\begin{itemize}
\item {} 
\sphinxAtStartPar
\sphinxstyleliteralstrong{\sphinxupquote{theta}} \textendash{} angular radius in rad

\item {} 
\sphinxAtStartPar
\sphinxstyleliteralstrong{\sphinxupquote{rs}} \textendash{} scale radius

\item {} 
\sphinxAtStartPar
\sphinxstyleliteralstrong{\sphinxupquote{rh}} \textendash{} diffusion radius parameter

\item {} 
\sphinxAtStartPar
\sphinxstyleliteralstrong{\sphinxupquote{dist}} \textendash{} distance to the source

\end{itemize}

\end{description}\end{quote}

\end{fulllineitems}

\index{psbrC() (in module diffsph.profiles.analytics)@\spxentry{psbrC()}\spxextra{in module diffsph.profiles.analytics}}

\begin{fulllineitems}
\phantomsection\label{\detokenize{diffsph.profiles:diffsph.profiles.analytics.psbrC}}\pysiglinewithargsret{\sphinxcode{\sphinxupquote{diffsph.profiles.analytics.}}\sphinxbfcode{\sphinxupquote{psbrC}}}{\emph{\DUrole{n}{t}}, \emph{\DUrole{n}{rh}}}{}
\sphinxAtStartPar
Brigthness H\sphinxhyphen{}function for point sources in the regime\sphinxhyphen{}C approximation
Variable t is defined as
\begin{quote}\begin{description}
\item[{Parameters}] \leavevmode\begin{itemize}
\item {} 
\sphinxAtStartPar
\sphinxstyleliteralstrong{\sphinxupquote{t}} \textendash{} \(D\sin(\theta)/r_h\), where \(\theta\) (\sphinxcode{\sphinxupquote{theta}}), \(r_h\) (\sphinxcode{\sphinxupquote{rh}}) and \(D\) (\sphinxcode{\sphinxupquote{dist}}) are defined below

\item {} 
\sphinxAtStartPar
\sphinxstyleliteralstrong{\sphinxupquote{theta}} \textendash{} angular radius in rad

\item {} 
\sphinxAtStartPar
\sphinxstyleliteralstrong{\sphinxupquote{rh}} \textendash{} diffusion radius parameter

\item {} 
\sphinxAtStartPar
\sphinxstyleliteralstrong{\sphinxupquote{dist}} \textendash{} distance to the source

\end{itemize}

\end{description}\end{quote}

\end{fulllineitems}

\index{psfdC() (in module diffsph.profiles.analytics)@\spxentry{psfdC()}\spxextra{in module diffsph.profiles.analytics}}

\begin{fulllineitems}
\phantomsection\label{\detokenize{diffsph.profiles:diffsph.profiles.analytics.psfdC}}\pysiglinewithargsret{\sphinxcode{\sphinxupquote{diffsph.profiles.analytics.}}\sphinxbfcode{\sphinxupquote{psfdC}}}{\emph{\DUrole{n}{theta}}, \emph{\DUrole{n}{rh}}, \emph{\DUrole{n}{dist}}}{}
\sphinxAtStartPar
Flux\sphinxhyphen{}density H\sphinxhyphen{}function for point sources in the regime\sphinxhyphen{}C approximation
\begin{quote}\begin{description}
\item[{Parameters}] \leavevmode\begin{itemize}
\item {} 
\sphinxAtStartPar
\sphinxstyleliteralstrong{\sphinxupquote{theta}} \textendash{} angular radius in rad

\item {} 
\sphinxAtStartPar
\sphinxstyleliteralstrong{\sphinxupquote{rh}} \textendash{} diffusion radius parameter

\item {} 
\sphinxAtStartPar
\sphinxstyleliteralstrong{\sphinxupquote{dist}} \textendash{} distance to the source

\end{itemize}

\end{description}\end{quote}

\end{fulllineitems}

\index{psfdCmax() (in module diffsph.profiles.analytics)@\spxentry{psfdCmax()}\spxextra{in module diffsph.profiles.analytics}}

\begin{fulllineitems}
\phantomsection\label{\detokenize{diffsph.profiles:diffsph.profiles.analytics.psfdCmax}}\pysiglinewithargsret{\sphinxcode{\sphinxupquote{diffsph.profiles.analytics.}}\sphinxbfcode{\sphinxupquote{psfdCmax}}}{\emph{\DUrole{n}{rh}}, \emph{\DUrole{n}{dist}}}{}
\sphinxAtStartPar
Maximum value for the flux\sphinxhyphen{}density H\sphinxhyphen{}function for point sources in the regime\sphinxhyphen{}C approximation
\begin{quote}\begin{description}
\item[{Parameters}] \leavevmode\begin{itemize}
\item {} 
\sphinxAtStartPar
\sphinxstyleliteralstrong{\sphinxupquote{rh}} \textendash{} diffusion radius parameter

\item {} 
\sphinxAtStartPar
\sphinxstyleliteralstrong{\sphinxupquote{dist}} \textendash{} distance to the source

\end{itemize}

\end{description}\end{quote}

\end{fulllineitems}

\index{sisbrA() (in module diffsph.profiles.analytics)@\spxentry{sisbrA()}\spxextra{in module diffsph.profiles.analytics}}

\begin{fulllineitems}
\phantomsection\label{\detokenize{diffsph.profiles:diffsph.profiles.analytics.sisbrA}}\pysiglinewithargsret{\sphinxcode{\sphinxupquote{diffsph.profiles.analytics.}}\sphinxbfcode{\sphinxupquote{sisbrA}}}{\emph{\DUrole{n}{t}}, \emph{\DUrole{n}{sigmav}}, \emph{\DUrole{n}{rh}}}{}
\sphinxAtStartPar
Brightness H\sphinxhyphen{}function for the singular isothermal source in the regime\sphinxhyphen{}A approximation
\begin{quote}\begin{description}
\item[{Parameters}] \leavevmode\begin{itemize}
\item {} 
\sphinxAtStartPar
\sphinxstyleliteralstrong{\sphinxupquote{t}} \textendash{} \(D\sin(\theta)/r_h\), where \(\theta\) (\sphinxcode{\sphinxupquote{theta}}), \(r_h\) (\sphinxcode{\sphinxupquote{rh}}) and \(D\) (\sphinxcode{\sphinxupquote{dist}}) are defined below

\item {} 
\sphinxAtStartPar
\sphinxstyleliteralstrong{\sphinxupquote{theta}} \textendash{} angular radius in rad

\item {} 
\sphinxAtStartPar
\sphinxstyleliteralstrong{\sphinxupquote{sigmav}} \textendash{} velocity dispersion parameter

\item {} 
\sphinxAtStartPar
\sphinxstyleliteralstrong{\sphinxupquote{rh}} \textendash{} diffusion radius parameter

\item {} 
\sphinxAtStartPar
\sphinxstyleliteralstrong{\sphinxupquote{dist}} \textendash{} distance to the source

\end{itemize}

\end{description}\end{quote}

\end{fulllineitems}

\index{sisbrC() (in module diffsph.profiles.analytics)@\spxentry{sisbrC()}\spxextra{in module diffsph.profiles.analytics}}

\begin{fulllineitems}
\phantomsection\label{\detokenize{diffsph.profiles:diffsph.profiles.analytics.sisbrC}}\pysiglinewithargsret{\sphinxcode{\sphinxupquote{diffsph.profiles.analytics.}}\sphinxbfcode{\sphinxupquote{sisbrC}}}{\emph{\DUrole{n}{t}}, \emph{\DUrole{n}{sigmav}}, \emph{\DUrole{n}{rh}}}{}
\sphinxAtStartPar
Brightness H\sphinxhyphen{}function for the singular isothermal  source in the regime\sphinxhyphen{}C approximation
\begin{quote}\begin{description}
\item[{Parameters}] \leavevmode\begin{itemize}
\item {} 
\sphinxAtStartPar
\sphinxstyleliteralstrong{\sphinxupquote{t}} \textendash{} \(D\sin(\theta)/r_h\), where \(\theta\) (\sphinxcode{\sphinxupquote{theta}}), \(r_h\) (\sphinxcode{\sphinxupquote{rh}}) and \(D\) (\sphinxcode{\sphinxupquote{dist}}) are defined below

\item {} 
\sphinxAtStartPar
\sphinxstyleliteralstrong{\sphinxupquote{theta}} \textendash{} angular radius in rad

\item {} 
\sphinxAtStartPar
\sphinxstyleliteralstrong{\sphinxupquote{sigmav}} \textendash{} velocity dispersion parameter

\item {} 
\sphinxAtStartPar
\sphinxstyleliteralstrong{\sphinxupquote{rh}} \textendash{} diffusion radius parameter

\item {} 
\sphinxAtStartPar
\sphinxstyleliteralstrong{\sphinxupquote{dist}} \textendash{} distance to the source

\end{itemize}

\end{description}\end{quote}

\end{fulllineitems}



\paragraph{diffsph.profiles.hfactors module}
\label{\detokenize{diffsph.profiles:module-diffsph.profiles.hfactors}}\label{\detokenize{diffsph.profiles:diffsph-profiles-hfactors-module}}\index{module@\spxentry{module}!diffsph.profiles.hfactors@\spxentry{diffsph.profiles.hfactors}}\index{diffsph.profiles.hfactors@\spxentry{diffsph.profiles.hfactors}!module@\spxentry{module}}\index{D\_factor() (in module diffsph.profiles.hfactors)@\spxentry{D\_factor()}\spxextra{in module diffsph.profiles.hfactors}}

\begin{fulllineitems}
\phantomsection\label{\detokenize{diffsph.profiles:diffsph.profiles.hfactors.D_factor}}\pysiglinewithargsret{\sphinxcode{\sphinxupquote{diffsph.profiles.hfactors.}}\sphinxbfcode{\sphinxupquote{D\_factor}}}{\emph{\DUrole{n}{theta}}, \emph{\DUrole{n}{dist}}, \emph{\DUrole{n}{rad\_temp}}, \emph{\DUrole{o}{**}\DUrole{n}{kwargs}}}{}
\sphinxAtStartPar
Generic “D” factor
\begin{quote}\begin{description}
\item[{Parameters}] \leavevmode\begin{itemize}
\item {} 
\sphinxAtStartPar
\sphinxstyleliteralstrong{\sphinxupquote{theta}} \textendash{} angular distance in rad units

\item {} 
\sphinxAtStartPar
\sphinxstyleliteralstrong{\sphinxupquote{dist}} \textendash{} distance to earth

\item {} 
\sphinxAtStartPar
\sphinxstyleliteralstrong{\sphinxupquote{rad\_temp}} \textendash{} radial template

\end{itemize}

\end{description}\end{quote}

\sphinxAtStartPar
Keyword arguments
\begin{quote}\begin{description}
\item[{Parameters}] \leavevmode\begin{itemize}
\item {} 
\sphinxAtStartPar
\sphinxstyleliteralstrong{\sphinxupquote{rs}} \textendash{} scale radius

\item {} 
\sphinxAtStartPar
\sphinxstyleliteralstrong{\sphinxupquote{rhos}} \textendash{} characteristic density

\item {} 
\sphinxAtStartPar
\sphinxstyleliteralstrong{\sphinxupquote{alpha}} \textendash{} exponent \(\alpha\) in the {\hyperref[\detokenize{diffsph.profiles:diffsph.profiles.templates.hdz}]{\sphinxcrossref{\sphinxcode{\sphinxupquote{diffsph.profiles.templates.hdz()}}}}} profile

\item {} 
\sphinxAtStartPar
\sphinxstyleliteralstrong{\sphinxupquote{beta}} \textendash{} exponent \(\beta\) in the {\hyperref[\detokenize{diffsph.profiles:diffsph.profiles.templates.hdz}]{\sphinxcrossref{\sphinxcode{\sphinxupquote{diffsph.profiles.templates.hdz()}}}}} profile

\item {} 
\sphinxAtStartPar
\sphinxstyleliteralstrong{\sphinxupquote{gamma}} \textendash{} exponent \(\gamma\) in the {\hyperref[\detokenize{diffsph.profiles:diffsph.profiles.templates.hdz}]{\sphinxcrossref{\sphinxcode{\sphinxupquote{diffsph.profiles.templates.hdz()}}}}} profile

\item {} 
\sphinxAtStartPar
\sphinxstyleliteralstrong{\sphinxupquote{alphaE}} \textendash{} parameter \(\alpha_E\) in the {\hyperref[\detokenize{diffsph.profiles:diffsph.profiles.templates.enst}]{\sphinxcrossref{\sphinxcode{\sphinxupquote{diffsph.profiles.templates.enst()}}}}} profile

\item {} 
\sphinxAtStartPar
\sphinxstyleliteralstrong{\sphinxupquote{rc}} \textendash{} core radius parameter \(r_c\) in the {\hyperref[\detokenize{diffsph.profiles:diffsph.profiles.templates.cnfw}]{\sphinxcrossref{\sphinxcode{\sphinxupquote{diffsph.profiles.templates.cnfw()}}}}} profile

\item {} 
\sphinxAtStartPar
\sphinxstyleliteralstrong{\sphinxupquote{sigmav}} \textendash{} velocity dispersion parameter \(\sigma_v\) in the {\hyperref[\detokenize{diffsph.profiles:diffsph.profiles.templates.sis}]{\sphinxcrossref{\sphinxcode{\sphinxupquote{diffsph.profiles.templates.sis()}}}}} profile

\end{itemize}

\item[{Returns}] \leavevmode
\sphinxAtStartPar
D factor

\end{description}\end{quote}

\end{fulllineitems}

\index{H\_brightness() (in module diffsph.profiles.hfactors)@\spxentry{H\_brightness()}\spxextra{in module diffsph.profiles.hfactors}}

\begin{fulllineitems}
\phantomsection\label{\detokenize{diffsph.profiles:diffsph.profiles.hfactors.H_brightness}}\pysiglinewithargsret{\sphinxcode{\sphinxupquote{diffsph.profiles.hfactors.}}\sphinxbfcode{\sphinxupquote{H\_brightness}}}{\emph{\DUrole{n}{theta}}, \emph{\DUrole{n}{dist}}, \emph{\DUrole{n}{rh}}, \emph{\DUrole{n}{hyp}}, \emph{\DUrole{n}{rad\_temp}}, \emph{\DUrole{n}{regime}}, \emph{\DUrole{o}{**}\DUrole{n}{kwargs}}}{}
\sphinxAtStartPar
Generic emissivity halo/bulge function in the Regime “A”, “B” or “C” approximations
\begin{quote}\begin{description}
\item[{Parameters}] \leavevmode\begin{itemize}
\item {} 
\sphinxAtStartPar
\sphinxstyleliteralstrong{\sphinxupquote{theta}} \textendash{} angular distance in rad units

\item {} 
\sphinxAtStartPar
\sphinxstyleliteralstrong{\sphinxupquote{dist}} \textendash{} distance to earth

\item {} 
\sphinxAtStartPar
\sphinxstyleliteralstrong{\sphinxupquote{rh}} \textendash{} diffusion halo/bulge radius

\item {} 
\sphinxAtStartPar
\sphinxstyleliteralstrong{\sphinxupquote{hyp}} (\sphinxstyleliteralemphasis{\sphinxupquote{str}}) \textendash{} hypothesis: \sphinxcode{\sphinxupquote{\textquotesingle{}wimp\textquotesingle{}}} (\sphinxstylestrong{default}), \sphinxcode{\sphinxupquote{\textquotesingle{}decay\textquotesingle{}}} or \sphinxcode{\sphinxupquote{\textquotesingle{}generic\textquotesingle{}}})

\item {} 
\sphinxAtStartPar
\sphinxstyleliteralstrong{\sphinxupquote{halo\_model}} \textendash{} DM halo model

\item {} 
\sphinxAtStartPar
\sphinxstyleliteralstrong{\sphinxupquote{rad\_temp}} \textendash{} radial template

\item {} 
\sphinxAtStartPar
\sphinxstyleliteralstrong{\sphinxupquote{regime}} \textendash{} regime of the approximation. Must be either upper or lower case a, b, c or I/II/III.

\end{itemize}

\end{description}\end{quote}

\sphinxAtStartPar
Keyword arguments
\begin{quote}\begin{description}
\item[{Parameters}] \leavevmode\begin{itemize}
\item {} 
\sphinxAtStartPar
\sphinxstyleliteralstrong{\sphinxupquote{rs}} \textendash{} scale radius

\item {} 
\sphinxAtStartPar
\sphinxstyleliteralstrong{\sphinxupquote{rhos}} \textendash{} characteristic density

\item {} 
\sphinxAtStartPar
\sphinxstyleliteralstrong{\sphinxupquote{alpha}} \textendash{} exponent \(\alpha\) in the {\hyperref[\detokenize{diffsph.profiles:diffsph.profiles.templates.hdz}]{\sphinxcrossref{\sphinxcode{\sphinxupquote{diffsph.profiles.templates.hdz()}}}}} profile

\item {} 
\sphinxAtStartPar
\sphinxstyleliteralstrong{\sphinxupquote{beta}} \textendash{} exponent \(\beta\) in the {\hyperref[\detokenize{diffsph.profiles:diffsph.profiles.templates.hdz}]{\sphinxcrossref{\sphinxcode{\sphinxupquote{diffsph.profiles.templates.hdz()}}}}} profile

\item {} 
\sphinxAtStartPar
\sphinxstyleliteralstrong{\sphinxupquote{gamma}} \textendash{} exponent \(\gamma\) in the {\hyperref[\detokenize{diffsph.profiles:diffsph.profiles.templates.hdz}]{\sphinxcrossref{\sphinxcode{\sphinxupquote{diffsph.profiles.templates.hdz()}}}}} profile

\item {} 
\sphinxAtStartPar
\sphinxstyleliteralstrong{\sphinxupquote{alphaE}} \textendash{} parameter \(\alpha_E\) in the {\hyperref[\detokenize{diffsph.profiles:diffsph.profiles.templates.enst}]{\sphinxcrossref{\sphinxcode{\sphinxupquote{diffsph.profiles.templates.enst()}}}}} profile

\item {} 
\sphinxAtStartPar
\sphinxstyleliteralstrong{\sphinxupquote{rc}} \textendash{} core radius parameter \(r_c\) in the {\hyperref[\detokenize{diffsph.profiles:diffsph.profiles.templates.cnfw}]{\sphinxcrossref{\sphinxcode{\sphinxupquote{diffsph.profiles.templates.cnfw()}}}}} profile

\item {} 
\sphinxAtStartPar
\sphinxstyleliteralstrong{\sphinxupquote{sigmav}} \textendash{} velocity dispersion parameter \(\sigma_v\) in the {\hyperref[\detokenize{diffsph.profiles:diffsph.profiles.templates.sis}]{\sphinxcrossref{\sphinxcode{\sphinxupquote{diffsph.profiles.templates.sis()}}}}} profile

\end{itemize}

\item[{Returns}] \leavevmode
\sphinxAtStartPar
brightness halo/bulge function

\end{description}\end{quote}

\end{fulllineitems}

\index{H\_emissivity() (in module diffsph.profiles.hfactors)@\spxentry{H\_emissivity()}\spxextra{in module diffsph.profiles.hfactors}}

\begin{fulllineitems}
\phantomsection\label{\detokenize{diffsph.profiles:diffsph.profiles.hfactors.H_emissivity}}\pysiglinewithargsret{\sphinxcode{\sphinxupquote{diffsph.profiles.hfactors.}}\sphinxbfcode{\sphinxupquote{H\_emissivity}}}{\emph{\DUrole{n}{r}}, \emph{\DUrole{n}{rh}}, \emph{\DUrole{n}{hyp}}, \emph{\DUrole{n}{rad\_temp}}, \emph{\DUrole{n}{regime}}, \emph{\DUrole{o}{**}\DUrole{n}{kwargs}}}{}
\sphinxAtStartPar
Generic emissivity halo/bulge function in the Regime “A”, “B” or “C” approximations
\begin{quote}\begin{description}
\item[{Parameters}] \leavevmode\begin{itemize}
\item {} 
\sphinxAtStartPar
\sphinxstyleliteralstrong{\sphinxupquote{r}} \textendash{} galactocentric distance

\item {} 
\sphinxAtStartPar
\sphinxstyleliteralstrong{\sphinxupquote{rh}} \textendash{} diffusion halo/bulge radius

\item {} 
\sphinxAtStartPar
\sphinxstyleliteralstrong{\sphinxupquote{hyp}} (\sphinxstyleliteralemphasis{\sphinxupquote{str}}) \textendash{} hypothesis: \sphinxcode{\sphinxupquote{\textquotesingle{}wimp\textquotesingle{}}} (\sphinxstylestrong{default}), \sphinxcode{\sphinxupquote{\textquotesingle{}decay\textquotesingle{}}} or \sphinxcode{\sphinxupquote{\textquotesingle{}generic\textquotesingle{}}})

\item {} 
\sphinxAtStartPar
\sphinxstyleliteralstrong{\sphinxupquote{rad\_temp}} \textendash{} radial template

\item {} 
\sphinxAtStartPar
\sphinxstyleliteralstrong{\sphinxupquote{regime}} \textendash{} regime of the approximation (upper/lower case a, b, c or I/II/III).

\end{itemize}

\end{description}\end{quote}

\sphinxAtStartPar
Keyword arguments
\begin{quote}\begin{description}
\item[{Parameters}] \leavevmode\begin{itemize}
\item {} 
\sphinxAtStartPar
\sphinxstyleliteralstrong{\sphinxupquote{rs}} \textendash{} scale radius

\item {} 
\sphinxAtStartPar
\sphinxstyleliteralstrong{\sphinxupquote{rhos}} \textendash{} characteristic density

\item {} 
\sphinxAtStartPar
\sphinxstyleliteralstrong{\sphinxupquote{alpha}} \textendash{} exponent \(\alpha\) in the {\hyperref[\detokenize{diffsph.profiles:diffsph.profiles.templates.hdz}]{\sphinxcrossref{\sphinxcode{\sphinxupquote{diffsph.profiles.templates.hdz()}}}}} profile

\item {} 
\sphinxAtStartPar
\sphinxstyleliteralstrong{\sphinxupquote{beta}} \textendash{} exponent \(\beta\) in the {\hyperref[\detokenize{diffsph.profiles:diffsph.profiles.templates.hdz}]{\sphinxcrossref{\sphinxcode{\sphinxupquote{diffsph.profiles.templates.hdz()}}}}} profile

\item {} 
\sphinxAtStartPar
\sphinxstyleliteralstrong{\sphinxupquote{gamma}} \textendash{} exponent \(\gamma\) in the {\hyperref[\detokenize{diffsph.profiles:diffsph.profiles.templates.hdz}]{\sphinxcrossref{\sphinxcode{\sphinxupquote{diffsph.profiles.templates.hdz()}}}}} profile

\item {} 
\sphinxAtStartPar
\sphinxstyleliteralstrong{\sphinxupquote{alphaE}} \textendash{} parameter \(\alpha_E\) in the {\hyperref[\detokenize{diffsph.profiles:diffsph.profiles.templates.enst}]{\sphinxcrossref{\sphinxcode{\sphinxupquote{diffsph.profiles.templates.enst()}}}}} profile

\item {} 
\sphinxAtStartPar
\sphinxstyleliteralstrong{\sphinxupquote{rc}} \textendash{} core radius parameter \(r_c\) in the {\hyperref[\detokenize{diffsph.profiles:diffsph.profiles.templates.cnfw}]{\sphinxcrossref{\sphinxcode{\sphinxupquote{diffsph.profiles.templates.cnfw()}}}}} profile

\item {} 
\sphinxAtStartPar
\sphinxstyleliteralstrong{\sphinxupquote{sigmav}} \textendash{} velocity dispersion parameter \(\sigma_v\) in the {\hyperref[\detokenize{diffsph.profiles:diffsph.profiles.templates.sis}]{\sphinxcrossref{\sphinxcode{\sphinxupquote{diffsph.profiles.templates.sis()}}}}} profile

\end{itemize}

\item[{Returns}] \leavevmode
\sphinxAtStartPar
emissivity halo/bulge function

\end{description}\end{quote}

\end{fulllineitems}

\index{H\_fluxdens() (in module diffsph.profiles.hfactors)@\spxentry{H\_fluxdens()}\spxextra{in module diffsph.profiles.hfactors}}

\begin{fulllineitems}
\phantomsection\label{\detokenize{diffsph.profiles:diffsph.profiles.hfactors.H_fluxdens}}\pysiglinewithargsret{\sphinxcode{\sphinxupquote{diffsph.profiles.hfactors.}}\sphinxbfcode{\sphinxupquote{H\_fluxdens}}}{\emph{\DUrole{n}{theta}}, \emph{\DUrole{n}{dist}}, \emph{\DUrole{n}{rh}}, \emph{\DUrole{n}{hyp}}, \emph{\DUrole{n}{rad\_temp}}, \emph{\DUrole{n}{regime}}, \emph{\DUrole{o}{**}\DUrole{n}{kwargs}}}{}
\sphinxAtStartPar
Generic flux\sphinxhyphen{}density halo/bulge function in the Regime “A”, “B” or “C” approximations
\begin{quote}\begin{description}
\item[{Parameters}] \leavevmode\begin{itemize}
\item {} 
\sphinxAtStartPar
\sphinxstyleliteralstrong{\sphinxupquote{theta}} \textendash{} angular distance in rad units

\item {} 
\sphinxAtStartPar
\sphinxstyleliteralstrong{\sphinxupquote{dist}} \textendash{} distance to earth

\item {} 
\sphinxAtStartPar
\sphinxstyleliteralstrong{\sphinxupquote{rh}} \textendash{} diffusion halo/bulge radius

\item {} 
\sphinxAtStartPar
\sphinxstyleliteralstrong{\sphinxupquote{hyp}} (\sphinxstyleliteralemphasis{\sphinxupquote{str}}) \textendash{} hypothesis: \sphinxcode{\sphinxupquote{\textquotesingle{}wimp\textquotesingle{}}} (\sphinxstylestrong{default}), \sphinxcode{\sphinxupquote{\textquotesingle{}decay\textquotesingle{}}} or \sphinxcode{\sphinxupquote{\textquotesingle{}generic\textquotesingle{}}})

\item {} 
\sphinxAtStartPar
\sphinxstyleliteralstrong{\sphinxupquote{halo\_model}} \textendash{} DM halo model

\item {} 
\sphinxAtStartPar
\sphinxstyleliteralstrong{\sphinxupquote{rad\_temp}} \textendash{} radial template

\item {} 
\sphinxAtStartPar
\sphinxstyleliteralstrong{\sphinxupquote{regime}} \textendash{} regime of the approximation. Must be either upper or lower case a, b, c or I/II/III.

\end{itemize}

\end{description}\end{quote}

\sphinxAtStartPar
Keyword arguments
\begin{quote}\begin{description}
\item[{Parameters}] \leavevmode\begin{itemize}
\item {} 
\sphinxAtStartPar
\sphinxstyleliteralstrong{\sphinxupquote{rs}} \textendash{} scale radius

\item {} 
\sphinxAtStartPar
\sphinxstyleliteralstrong{\sphinxupquote{rhos}} \textendash{} characteristic density

\item {} 
\sphinxAtStartPar
\sphinxstyleliteralstrong{\sphinxupquote{alpha}} \textendash{} exponent \(\alpha\) in the {\hyperref[\detokenize{diffsph.profiles:diffsph.profiles.templates.hdz}]{\sphinxcrossref{\sphinxcode{\sphinxupquote{diffsph.profiles.templates.hdz()}}}}} profile

\item {} 
\sphinxAtStartPar
\sphinxstyleliteralstrong{\sphinxupquote{beta}} \textendash{} exponent \(\beta\) in the {\hyperref[\detokenize{diffsph.profiles:diffsph.profiles.templates.hdz}]{\sphinxcrossref{\sphinxcode{\sphinxupquote{diffsph.profiles.templates.hdz()}}}}} profile

\item {} 
\sphinxAtStartPar
\sphinxstyleliteralstrong{\sphinxupquote{gamma}} \textendash{} exponent \(\gamma\) in the {\hyperref[\detokenize{diffsph.profiles:diffsph.profiles.templates.hdz}]{\sphinxcrossref{\sphinxcode{\sphinxupquote{diffsph.profiles.templates.hdz()}}}}} profile

\item {} 
\sphinxAtStartPar
\sphinxstyleliteralstrong{\sphinxupquote{alphaE}} \textendash{} parameter \(\alpha_E\) in the {\hyperref[\detokenize{diffsph.profiles:diffsph.profiles.templates.enst}]{\sphinxcrossref{\sphinxcode{\sphinxupquote{diffsph.profiles.templates.enst()}}}}} profile

\item {} 
\sphinxAtStartPar
\sphinxstyleliteralstrong{\sphinxupquote{rc}} \textendash{} core radius parameter \(r_c\) in the {\hyperref[\detokenize{diffsph.profiles:diffsph.profiles.templates.cnfw}]{\sphinxcrossref{\sphinxcode{\sphinxupquote{diffsph.profiles.templates.cnfw()}}}}} profile

\item {} 
\sphinxAtStartPar
\sphinxstyleliteralstrong{\sphinxupquote{sigmav}} \textendash{} velocity dispersion parameter \(\sigma_v\) in the {\hyperref[\detokenize{diffsph.profiles:diffsph.profiles.templates.sis}]{\sphinxcrossref{\sphinxcode{\sphinxupquote{diffsph.profiles.templates.sis()}}}}} profile

\end{itemize}

\item[{Returns}] \leavevmode
\sphinxAtStartPar
flux density halo/bulge function

\end{description}\end{quote}

\end{fulllineitems}

\index{H\_fluxdens\_approx() (in module diffsph.profiles.hfactors)@\spxentry{H\_fluxdens\_approx()}\spxextra{in module diffsph.profiles.hfactors}}

\begin{fulllineitems}
\phantomsection\label{\detokenize{diffsph.profiles:diffsph.profiles.hfactors.H_fluxdens_approx}}\pysiglinewithargsret{\sphinxcode{\sphinxupquote{diffsph.profiles.hfactors.}}\sphinxbfcode{\sphinxupquote{H\_fluxdens\_approx}}}{\emph{\DUrole{n}{theta}}, \emph{\DUrole{n}{dist}}, \emph{\DUrole{n}{rh}}, \emph{\DUrole{n}{hyp}}, \emph{\DUrole{n}{rad\_temp}}, \emph{\DUrole{n}{regime}}, \emph{\DUrole{o}{**}\DUrole{n}{kwargs}}}{}
\sphinxAtStartPar
Generic flux\sphinxhyphen{}density halo/bulge function in the Regime “A”, “B” or “C” approximations (alternative formula)
\begin{quote}\begin{description}
\item[{Parameters}] \leavevmode\begin{itemize}
\item {} 
\sphinxAtStartPar
\sphinxstyleliteralstrong{\sphinxupquote{theta}} \textendash{} angular distance in rad units

\item {} 
\sphinxAtStartPar
\sphinxstyleliteralstrong{\sphinxupquote{dist}} \textendash{} distance to earth

\item {} 
\sphinxAtStartPar
\sphinxstyleliteralstrong{\sphinxupquote{rh}} \textendash{} diffusion halo/bulge radius

\item {} 
\sphinxAtStartPar
\sphinxstyleliteralstrong{\sphinxupquote{hyp}} (\sphinxstyleliteralemphasis{\sphinxupquote{str}}) \textendash{} hypothesis: \sphinxcode{\sphinxupquote{\textquotesingle{}wimp\textquotesingle{}}} (\sphinxstylestrong{default}), \sphinxcode{\sphinxupquote{\textquotesingle{}decay\textquotesingle{}}} or \sphinxcode{\sphinxupquote{\textquotesingle{}generic\textquotesingle{}}})

\item {} 
\sphinxAtStartPar
\sphinxstyleliteralstrong{\sphinxupquote{halo\_model}} \textendash{} DM halo model

\item {} 
\sphinxAtStartPar
\sphinxstyleliteralstrong{\sphinxupquote{rad\_temp}} \textendash{} radial template

\item {} 
\sphinxAtStartPar
\sphinxstyleliteralstrong{\sphinxupquote{regime}} \textendash{} regime of the approximation. Must be either upper or lower case a, b, c or I/II/III.

\end{itemize}

\end{description}\end{quote}

\sphinxAtStartPar
Keyword arguments
\begin{quote}\begin{description}
\item[{Parameters}] \leavevmode\begin{itemize}
\item {} 
\sphinxAtStartPar
\sphinxstyleliteralstrong{\sphinxupquote{rs}} \textendash{} scale radius

\item {} 
\sphinxAtStartPar
\sphinxstyleliteralstrong{\sphinxupquote{rhos}} \textendash{} characteristic density

\item {} 
\sphinxAtStartPar
\sphinxstyleliteralstrong{\sphinxupquote{alpha}} \textendash{} exponent \(\alpha\) in the {\hyperref[\detokenize{diffsph.profiles:diffsph.profiles.templates.hdz}]{\sphinxcrossref{\sphinxcode{\sphinxupquote{diffsph.profiles.templates.hdz()}}}}} profile

\item {} 
\sphinxAtStartPar
\sphinxstyleliteralstrong{\sphinxupquote{beta}} \textendash{} exponent \(\beta\) in the {\hyperref[\detokenize{diffsph.profiles:diffsph.profiles.templates.hdz}]{\sphinxcrossref{\sphinxcode{\sphinxupquote{diffsph.profiles.templates.hdz()}}}}} profile

\item {} 
\sphinxAtStartPar
\sphinxstyleliteralstrong{\sphinxupquote{gamma}} \textendash{} exponent \(\gamma\) in the {\hyperref[\detokenize{diffsph.profiles:diffsph.profiles.templates.hdz}]{\sphinxcrossref{\sphinxcode{\sphinxupquote{diffsph.profiles.templates.hdz()}}}}} profile

\item {} 
\sphinxAtStartPar
\sphinxstyleliteralstrong{\sphinxupquote{alphaE}} \textendash{} parameter \(\alpha_E\) in the {\hyperref[\detokenize{diffsph.profiles:diffsph.profiles.templates.enst}]{\sphinxcrossref{\sphinxcode{\sphinxupquote{diffsph.profiles.templates.enst()}}}}} profile

\item {} 
\sphinxAtStartPar
\sphinxstyleliteralstrong{\sphinxupquote{rc}} \textendash{} core radius parameter \(r_c\) in the {\hyperref[\detokenize{diffsph.profiles:diffsph.profiles.templates.cnfw}]{\sphinxcrossref{\sphinxcode{\sphinxupquote{diffsph.profiles.templates.cnfw()}}}}} profile

\item {} 
\sphinxAtStartPar
\sphinxstyleliteralstrong{\sphinxupquote{sigmav}} \textendash{} velocity dispersion parameter \(\sigma_v\) in the {\hyperref[\detokenize{diffsph.profiles:diffsph.profiles.templates.sis}]{\sphinxcrossref{\sphinxcode{\sphinxupquote{diffsph.profiles.templates.sis()}}}}} profile

\end{itemize}

\item[{Returns}] \leavevmode
\sphinxAtStartPar
flux density halo/bulge function

\end{description}\end{quote}

\end{fulllineitems}

\index{Hem\_A() (in module diffsph.profiles.hfactors)@\spxentry{Hem\_A()}\spxextra{in module diffsph.profiles.hfactors}}

\begin{fulllineitems}
\phantomsection\label{\detokenize{diffsph.profiles:diffsph.profiles.hfactors.Hem_A}}\pysiglinewithargsret{\sphinxcode{\sphinxupquote{diffsph.profiles.hfactors.}}\sphinxbfcode{\sphinxupquote{Hem\_A}}}{\emph{\DUrole{n}{r}}, \emph{\DUrole{n}{rh}}, \emph{\DUrole{n}{hyp}}, \emph{\DUrole{n}{rad\_temp}}, \emph{\DUrole{o}{**}\DUrole{n}{kwargs}}}{}
\sphinxAtStartPar
Generic emissivity halo/bulge function for Regime A
\begin{quote}\begin{description}
\item[{Parameters}] \leavevmode\begin{itemize}
\item {} 
\sphinxAtStartPar
\sphinxstyleliteralstrong{\sphinxupquote{r}} \textendash{} galactocentric distance

\item {} 
\sphinxAtStartPar
\sphinxstyleliteralstrong{\sphinxupquote{rh}} \textendash{} diffusion halo/bulge radius

\item {} 
\sphinxAtStartPar
\sphinxstyleliteralstrong{\sphinxupquote{hyp}} (\sphinxstyleliteralemphasis{\sphinxupquote{str}}) \textendash{} hypothesis: \sphinxcode{\sphinxupquote{\textquotesingle{}wimp\textquotesingle{}}} (\sphinxstylestrong{default}), \sphinxcode{\sphinxupquote{\textquotesingle{}decay\textquotesingle{}}} or \sphinxcode{\sphinxupquote{\textquotesingle{}generic\textquotesingle{}}})

\item {} 
\sphinxAtStartPar
\sphinxstyleliteralstrong{\sphinxupquote{rad\_temp}} \textendash{} radial template

\end{itemize}

\end{description}\end{quote}

\sphinxAtStartPar
Keyword arguments
\begin{quote}\begin{description}
\item[{Parameters}] \leavevmode\begin{itemize}
\item {} 
\sphinxAtStartPar
\sphinxstyleliteralstrong{\sphinxupquote{rs}} \textendash{} scale radius

\item {} 
\sphinxAtStartPar
\sphinxstyleliteralstrong{\sphinxupquote{rhos}} \textendash{} characteristic density

\item {} 
\sphinxAtStartPar
\sphinxstyleliteralstrong{\sphinxupquote{alpha}} \textendash{} exponent \(\alpha\) in the {\hyperref[\detokenize{diffsph.profiles:diffsph.profiles.templates.hdz}]{\sphinxcrossref{\sphinxcode{\sphinxupquote{diffsph.profiles.templates.hdz()}}}}} profile

\item {} 
\sphinxAtStartPar
\sphinxstyleliteralstrong{\sphinxupquote{beta}} \textendash{} exponent \(\beta\) in the {\hyperref[\detokenize{diffsph.profiles:diffsph.profiles.templates.hdz}]{\sphinxcrossref{\sphinxcode{\sphinxupquote{diffsph.profiles.templates.hdz()}}}}} profile

\item {} 
\sphinxAtStartPar
\sphinxstyleliteralstrong{\sphinxupquote{gamma}} \textendash{} exponent \(\gamma\) in the {\hyperref[\detokenize{diffsph.profiles:diffsph.profiles.templates.hdz}]{\sphinxcrossref{\sphinxcode{\sphinxupquote{diffsph.profiles.templates.hdz()}}}}} profile

\item {} 
\sphinxAtStartPar
\sphinxstyleliteralstrong{\sphinxupquote{alphaE}} \textendash{} parameter \(\alpha_E\) in the {\hyperref[\detokenize{diffsph.profiles:diffsph.profiles.templates.enst}]{\sphinxcrossref{\sphinxcode{\sphinxupquote{diffsph.profiles.templates.enst()}}}}} profile

\item {} 
\sphinxAtStartPar
\sphinxstyleliteralstrong{\sphinxupquote{rc}} \textendash{} core radius parameter \(r_c\) in the {\hyperref[\detokenize{diffsph.profiles:diffsph.profiles.templates.cnfw}]{\sphinxcrossref{\sphinxcode{\sphinxupquote{diffsph.profiles.templates.cnfw()}}}}} profile

\item {} 
\sphinxAtStartPar
\sphinxstyleliteralstrong{\sphinxupquote{sigmav}} \textendash{} velocity dispersion parameter \(\sigma_v\) in the {\hyperref[\detokenize{diffsph.profiles:diffsph.profiles.templates.sis}]{\sphinxcrossref{\sphinxcode{\sphinxupquote{diffsph.profiles.templates.sis()}}}}} profile

\end{itemize}

\item[{Returns}] \leavevmode
\sphinxAtStartPar
emissivity halo/bulge function using the Regime\sphinxhyphen{}A approximation

\end{description}\end{quote}

\end{fulllineitems}

\index{Hem\_B() (in module diffsph.profiles.hfactors)@\spxentry{Hem\_B()}\spxextra{in module diffsph.profiles.hfactors}}

\begin{fulllineitems}
\phantomsection\label{\detokenize{diffsph.profiles:diffsph.profiles.hfactors.Hem_B}}\pysiglinewithargsret{\sphinxcode{\sphinxupquote{diffsph.profiles.hfactors.}}\sphinxbfcode{\sphinxupquote{Hem\_B}}}{\emph{\DUrole{n}{r}}, \emph{\DUrole{n}{rh}}, \emph{\DUrole{n}{hyp}}, \emph{\DUrole{n}{rad\_temp}}, \emph{\DUrole{o}{**}\DUrole{n}{kwargs}}}{}
\sphinxAtStartPar
Generic emissivity halo/bulge function for Regime B
\begin{quote}\begin{description}
\item[{Parameters}] \leavevmode\begin{itemize}
\item {} 
\sphinxAtStartPar
\sphinxstyleliteralstrong{\sphinxupquote{r}} \textendash{} galactocentric distance

\item {} 
\sphinxAtStartPar
\sphinxstyleliteralstrong{\sphinxupquote{rh}} \textendash{} diffusion halo/bulge radius

\item {} 
\sphinxAtStartPar
\sphinxstyleliteralstrong{\sphinxupquote{hyp}} (\sphinxstyleliteralemphasis{\sphinxupquote{str}}) \textendash{} hypothesis: \sphinxcode{\sphinxupquote{\textquotesingle{}wimp\textquotesingle{}}} (\sphinxstylestrong{default}), \sphinxcode{\sphinxupquote{\textquotesingle{}decay\textquotesingle{}}} or \sphinxcode{\sphinxupquote{\textquotesingle{}generic\textquotesingle{}}})

\item {} 
\sphinxAtStartPar
\sphinxstyleliteralstrong{\sphinxupquote{rad\_temp}} \textendash{} radial template

\end{itemize}

\end{description}\end{quote}

\sphinxAtStartPar
Keyword arguments
\begin{quote}\begin{description}
\item[{Parameters}] \leavevmode\begin{itemize}
\item {} 
\sphinxAtStartPar
\sphinxstyleliteralstrong{\sphinxupquote{rs}} \textendash{} scale radius

\item {} 
\sphinxAtStartPar
\sphinxstyleliteralstrong{\sphinxupquote{rhos}} \textendash{} characteristic density

\item {} 
\sphinxAtStartPar
\sphinxstyleliteralstrong{\sphinxupquote{alpha}} \textendash{} exponent \(\alpha\) in the {\hyperref[\detokenize{diffsph.profiles:diffsph.profiles.templates.hdz}]{\sphinxcrossref{\sphinxcode{\sphinxupquote{diffsph.profiles.templates.hdz()}}}}} profile

\item {} 
\sphinxAtStartPar
\sphinxstyleliteralstrong{\sphinxupquote{beta}} \textendash{} exponent \(\beta\) in the {\hyperref[\detokenize{diffsph.profiles:diffsph.profiles.templates.hdz}]{\sphinxcrossref{\sphinxcode{\sphinxupquote{diffsph.profiles.templates.hdz()}}}}} profile

\item {} 
\sphinxAtStartPar
\sphinxstyleliteralstrong{\sphinxupquote{gamma}} \textendash{} exponent \(\gamma\) in the {\hyperref[\detokenize{diffsph.profiles:diffsph.profiles.templates.hdz}]{\sphinxcrossref{\sphinxcode{\sphinxupquote{diffsph.profiles.templates.hdz()}}}}} profile

\item {} 
\sphinxAtStartPar
\sphinxstyleliteralstrong{\sphinxupquote{alphaE}} \textendash{} parameter \(\alpha_E\) in the {\hyperref[\detokenize{diffsph.profiles:diffsph.profiles.templates.enst}]{\sphinxcrossref{\sphinxcode{\sphinxupquote{diffsph.profiles.templates.enst()}}}}} profile

\item {} 
\sphinxAtStartPar
\sphinxstyleliteralstrong{\sphinxupquote{rc}} \textendash{} core radius parameter \(r_c\) in the {\hyperref[\detokenize{diffsph.profiles:diffsph.profiles.templates.cnfw}]{\sphinxcrossref{\sphinxcode{\sphinxupquote{diffsph.profiles.templates.cnfw()}}}}} profile

\item {} 
\sphinxAtStartPar
\sphinxstyleliteralstrong{\sphinxupquote{sigmav}} \textendash{} velocity dispersion parameter \(\sigma_v\) in the {\hyperref[\detokenize{diffsph.profiles:diffsph.profiles.templates.sis}]{\sphinxcrossref{\sphinxcode{\sphinxupquote{diffsph.profiles.templates.sis()}}}}} profile

\end{itemize}

\item[{Returns}] \leavevmode
\sphinxAtStartPar
emissivity halo/bulge function using the Regime\sphinxhyphen{}B approximation

\end{description}\end{quote}

\end{fulllineitems}

\index{Hem\_C() (in module diffsph.profiles.hfactors)@\spxentry{Hem\_C()}\spxextra{in module diffsph.profiles.hfactors}}

\begin{fulllineitems}
\phantomsection\label{\detokenize{diffsph.profiles:diffsph.profiles.hfactors.Hem_C}}\pysiglinewithargsret{\sphinxcode{\sphinxupquote{diffsph.profiles.hfactors.}}\sphinxbfcode{\sphinxupquote{Hem\_C}}}{\emph{\DUrole{n}{r}}, \emph{\DUrole{n}{rh}}, \emph{\DUrole{n}{hyp}}, \emph{\DUrole{n}{rad\_temp}}, \emph{\DUrole{o}{**}\DUrole{n}{kwargs}}}{}
\sphinxAtStartPar
Generic emissivity halo/bulge function for Regime C
\begin{quote}\begin{description}
\item[{Parameters}] \leavevmode\begin{itemize}
\item {} 
\sphinxAtStartPar
\sphinxstyleliteralstrong{\sphinxupquote{r}} \textendash{} galactocentric distance

\item {} 
\sphinxAtStartPar
\sphinxstyleliteralstrong{\sphinxupquote{rh}} \textendash{} diffusion halo/bulge radius

\item {} 
\sphinxAtStartPar
\sphinxstyleliteralstrong{\sphinxupquote{hyp}} (\sphinxstyleliteralemphasis{\sphinxupquote{str}}) \textendash{} hypothesis: \sphinxcode{\sphinxupquote{\textquotesingle{}wimp\textquotesingle{}}} (\sphinxstylestrong{default}), \sphinxcode{\sphinxupquote{\textquotesingle{}decay\textquotesingle{}}} or \sphinxcode{\sphinxupquote{\textquotesingle{}generic\textquotesingle{}}})

\item {} 
\sphinxAtStartPar
\sphinxstyleliteralstrong{\sphinxupquote{rad\_temp}} \textendash{} radial template

\end{itemize}

\end{description}\end{quote}

\sphinxAtStartPar
Keyword arguments
\begin{quote}\begin{description}
\item[{Parameters}] \leavevmode\begin{itemize}
\item {} 
\sphinxAtStartPar
\sphinxstyleliteralstrong{\sphinxupquote{rs}} \textendash{} scale radius

\item {} 
\sphinxAtStartPar
\sphinxstyleliteralstrong{\sphinxupquote{rhos}} \textendash{} characteristic density

\item {} 
\sphinxAtStartPar
\sphinxstyleliteralstrong{\sphinxupquote{alpha}} \textendash{} exponent \(\alpha\) in the {\hyperref[\detokenize{diffsph.profiles:diffsph.profiles.templates.hdz}]{\sphinxcrossref{\sphinxcode{\sphinxupquote{diffsph.profiles.templates.hdz()}}}}} profile

\item {} 
\sphinxAtStartPar
\sphinxstyleliteralstrong{\sphinxupquote{beta}} \textendash{} exponent \(\beta\) in the {\hyperref[\detokenize{diffsph.profiles:diffsph.profiles.templates.hdz}]{\sphinxcrossref{\sphinxcode{\sphinxupquote{diffsph.profiles.templates.hdz()}}}}} profile

\item {} 
\sphinxAtStartPar
\sphinxstyleliteralstrong{\sphinxupquote{gamma}} \textendash{} exponent \(\gamma\) in the {\hyperref[\detokenize{diffsph.profiles:diffsph.profiles.templates.hdz}]{\sphinxcrossref{\sphinxcode{\sphinxupquote{diffsph.profiles.templates.hdz()}}}}} profile

\item {} 
\sphinxAtStartPar
\sphinxstyleliteralstrong{\sphinxupquote{alphaE}} \textendash{} parameter \(\alpha_E\) in the {\hyperref[\detokenize{diffsph.profiles:diffsph.profiles.templates.enst}]{\sphinxcrossref{\sphinxcode{\sphinxupquote{diffsph.profiles.templates.enst()}}}}} profile

\item {} 
\sphinxAtStartPar
\sphinxstyleliteralstrong{\sphinxupquote{rc}} \textendash{} core radius parameter \(r_c\) in the {\hyperref[\detokenize{diffsph.profiles:diffsph.profiles.templates.cnfw}]{\sphinxcrossref{\sphinxcode{\sphinxupquote{diffsph.profiles.templates.cnfw()}}}}} profile

\item {} 
\sphinxAtStartPar
\sphinxstyleliteralstrong{\sphinxupquote{sigmav}} \textendash{} velocity dispersion parameter \(\sigma_v\) in the {\hyperref[\detokenize{diffsph.profiles:diffsph.profiles.templates.sis}]{\sphinxcrossref{\sphinxcode{\sphinxupquote{diffsph.profiles.templates.sis()}}}}} profile

\end{itemize}

\item[{Returns}] \leavevmode
\sphinxAtStartPar
emissivity halo/bulge function using the Regime\sphinxhyphen{}C approximation

\end{description}\end{quote}

\end{fulllineitems}

\index{J\_factor() (in module diffsph.profiles.hfactors)@\spxentry{J\_factor()}\spxextra{in module diffsph.profiles.hfactors}}

\begin{fulllineitems}
\phantomsection\label{\detokenize{diffsph.profiles:diffsph.profiles.hfactors.J_factor}}\pysiglinewithargsret{\sphinxcode{\sphinxupquote{diffsph.profiles.hfactors.}}\sphinxbfcode{\sphinxupquote{J\_factor}}}{\emph{\DUrole{n}{theta}}, \emph{\DUrole{n}{dist}}, \emph{\DUrole{n}{rad\_temp}}, \emph{\DUrole{o}{**}\DUrole{n}{kwargs}}}{}
\sphinxAtStartPar
Generic “J” factor
\begin{quote}\begin{description}
\item[{Parameters}] \leavevmode\begin{itemize}
\item {} 
\sphinxAtStartPar
\sphinxstyleliteralstrong{\sphinxupquote{theta}} \textendash{} angular distance in rad units

\item {} 
\sphinxAtStartPar
\sphinxstyleliteralstrong{\sphinxupquote{dist}} \textendash{} distance to earth

\item {} 
\sphinxAtStartPar
\sphinxstyleliteralstrong{\sphinxupquote{rad\_temp}} \textendash{} radial template

\end{itemize}

\end{description}\end{quote}

\sphinxAtStartPar
Keyword arguments
\begin{quote}\begin{description}
\item[{Parameters}] \leavevmode\begin{itemize}
\item {} 
\sphinxAtStartPar
\sphinxstyleliteralstrong{\sphinxupquote{rs}} \textendash{} scale radius

\item {} 
\sphinxAtStartPar
\sphinxstyleliteralstrong{\sphinxupquote{rhos}} \textendash{} characteristic density

\item {} 
\sphinxAtStartPar
\sphinxstyleliteralstrong{\sphinxupquote{alpha}} \textendash{} exponent \(\alpha\) in the {\hyperref[\detokenize{diffsph.profiles:diffsph.profiles.templates.hdz}]{\sphinxcrossref{\sphinxcode{\sphinxupquote{diffsph.profiles.templates.hdz()}}}}} profile

\item {} 
\sphinxAtStartPar
\sphinxstyleliteralstrong{\sphinxupquote{beta}} \textendash{} exponent \(\beta\) in the {\hyperref[\detokenize{diffsph.profiles:diffsph.profiles.templates.hdz}]{\sphinxcrossref{\sphinxcode{\sphinxupquote{diffsph.profiles.templates.hdz()}}}}} profile

\item {} 
\sphinxAtStartPar
\sphinxstyleliteralstrong{\sphinxupquote{gamma}} \textendash{} exponent \(\gamma\) in the {\hyperref[\detokenize{diffsph.profiles:diffsph.profiles.templates.hdz}]{\sphinxcrossref{\sphinxcode{\sphinxupquote{diffsph.profiles.templates.hdz()}}}}} profile

\item {} 
\sphinxAtStartPar
\sphinxstyleliteralstrong{\sphinxupquote{alphaE}} \textendash{} parameter \(\alpha_E\) in the {\hyperref[\detokenize{diffsph.profiles:diffsph.profiles.templates.enst}]{\sphinxcrossref{\sphinxcode{\sphinxupquote{diffsph.profiles.templates.enst()}}}}} profile

\item {} 
\sphinxAtStartPar
\sphinxstyleliteralstrong{\sphinxupquote{rc}} \textendash{} core radius parameter \(r_c\) in the {\hyperref[\detokenize{diffsph.profiles:diffsph.profiles.templates.cnfw}]{\sphinxcrossref{\sphinxcode{\sphinxupquote{diffsph.profiles.templates.cnfw()}}}}} profile

\item {} 
\sphinxAtStartPar
\sphinxstyleliteralstrong{\sphinxupquote{sigmav}} \textendash{} velocity dispersion parameter \(\sigma_v\) in the {\hyperref[\detokenize{diffsph.profiles:diffsph.profiles.templates.sis}]{\sphinxcrossref{\sphinxcode{\sphinxupquote{diffsph.profiles.templates.sis()}}}}} profile

\end{itemize}

\item[{Returns}] \leavevmode
\sphinxAtStartPar
J factor

\end{description}\end{quote}

\end{fulllineitems}

\index{halo\_factor() (in module diffsph.profiles.hfactors)@\spxentry{halo\_factor()}\spxextra{in module diffsph.profiles.hfactors}}

\begin{fulllineitems}
\phantomsection\label{\detokenize{diffsph.profiles:diffsph.profiles.hfactors.halo_factor}}\pysiglinewithargsret{\sphinxcode{\sphinxupquote{diffsph.profiles.hfactors.}}\sphinxbfcode{\sphinxupquote{halo\_factor}}}{\emph{\DUrole{n}{n}}, \emph{\DUrole{n}{rh}}, \emph{\DUrole{n}{hyp}}, \emph{\DUrole{n}{rad\_temp}}, \emph{\DUrole{o}{**}\DUrole{n}{kwargs}}}{}
\sphinxAtStartPar
n\sphinxhyphen{}th order halo/bulge factor h\_n for a given model (e.g. NFW, Einasto, Plummer, …) 
Arguments \sphinxcode{\sphinxupquote{\textquotesingle{}n\textquotesingle{}}}, \sphinxcode{\sphinxupquote{\textquotesingle{}rh\textquotesingle{}}}, \sphinxcode{\sphinxupquote{\textquotesingle{}hyp\textquotesingle{}}} and \sphinxcode{\sphinxupquote{\textquotesingle{}rad\_temp\textquotesingle{}}} are necessary. Remaining arguments depend on the 
adopted halo model.
\begin{quote}\begin{description}
\item[{Parameters}] \leavevmode\begin{itemize}
\item {} 
\sphinxAtStartPar
\sphinxstyleliteralstrong{\sphinxupquote{n}} \textendash{} order of the halo/bulge factor

\item {} 
\sphinxAtStartPar
\sphinxstyleliteralstrong{\sphinxupquote{rh}} \textendash{} diffusion halo/bulge radius

\item {} 
\sphinxAtStartPar
\sphinxstyleliteralstrong{\sphinxupquote{hyp}} (\sphinxstyleliteralemphasis{\sphinxupquote{str}}) \textendash{} hypothesis: \sphinxcode{\sphinxupquote{\textquotesingle{}wimp\textquotesingle{}}} (\sphinxstylestrong{default}), \sphinxcode{\sphinxupquote{\textquotesingle{}decay\textquotesingle{}}} or \sphinxcode{\sphinxupquote{\textquotesingle{}generic\textquotesingle{}}})

\item {} 
\sphinxAtStartPar
\sphinxstyleliteralstrong{\sphinxupquote{rad\_temp}} \textendash{} radial template

\end{itemize}

\end{description}\end{quote}

\sphinxAtStartPar
Keyword arguments
\begin{quote}\begin{description}
\item[{Parameters}] \leavevmode\begin{itemize}
\item {} 
\sphinxAtStartPar
\sphinxstyleliteralstrong{\sphinxupquote{rs}} \textendash{} scale radius

\item {} 
\sphinxAtStartPar
\sphinxstyleliteralstrong{\sphinxupquote{rhos}} \textendash{} characteristic density

\item {} 
\sphinxAtStartPar
\sphinxstyleliteralstrong{\sphinxupquote{alpha}} \textendash{} exponent \(\alpha\) in the {\hyperref[\detokenize{diffsph.profiles:diffsph.profiles.templates.hdz}]{\sphinxcrossref{\sphinxcode{\sphinxupquote{diffsph.profiles.templates.hdz()}}}}} profile

\item {} 
\sphinxAtStartPar
\sphinxstyleliteralstrong{\sphinxupquote{beta}} \textendash{} exponent \(\beta\) in the {\hyperref[\detokenize{diffsph.profiles:diffsph.profiles.templates.hdz}]{\sphinxcrossref{\sphinxcode{\sphinxupquote{diffsph.profiles.templates.hdz()}}}}} profile

\item {} 
\sphinxAtStartPar
\sphinxstyleliteralstrong{\sphinxupquote{gamma}} \textendash{} exponent \(\gamma\) in the {\hyperref[\detokenize{diffsph.profiles:diffsph.profiles.templates.hdz}]{\sphinxcrossref{\sphinxcode{\sphinxupquote{diffsph.profiles.templates.hdz()}}}}} profile

\item {} 
\sphinxAtStartPar
\sphinxstyleliteralstrong{\sphinxupquote{alphaE}} \textendash{} parameter \(\alpha_E\) in the {\hyperref[\detokenize{diffsph.profiles:diffsph.profiles.templates.enst}]{\sphinxcrossref{\sphinxcode{\sphinxupquote{diffsph.profiles.templates.enst()}}}}} profile

\item {} 
\sphinxAtStartPar
\sphinxstyleliteralstrong{\sphinxupquote{rc}} \textendash{} core radius parameter \(r_c\) in the {\hyperref[\detokenize{diffsph.profiles:diffsph.profiles.templates.cnfw}]{\sphinxcrossref{\sphinxcode{\sphinxupquote{diffsph.profiles.templates.cnfw()}}}}} profile

\item {} 
\sphinxAtStartPar
\sphinxstyleliteralstrong{\sphinxupquote{sigmav}} \textendash{} velocity dispersion parameter \(\sigma_v\) in the {\hyperref[\detokenize{diffsph.profiles:diffsph.profiles.templates.sis}]{\sphinxcrossref{\sphinxcode{\sphinxupquote{diffsph.profiles.templates.sis()}}}}} profile

\end{itemize}

\item[{Returns}] \leavevmode
\sphinxAtStartPar
halo/bulge factor

\end{description}\end{quote}

\end{fulllineitems}



\paragraph{diffsph.profiles.massmodels module}
\label{\detokenize{diffsph.profiles:module-diffsph.profiles.massmodels}}\label{\detokenize{diffsph.profiles:diffsph-profiles-massmodels-module}}\index{module@\spxentry{module}!diffsph.profiles.massmodels@\spxentry{diffsph.profiles.massmodels}}\index{diffsph.profiles.massmodels@\spxentry{diffsph.profiles.massmodels}!module@\spxentry{module}}\index{D() (in module diffsph.profiles.massmodels)@\spxentry{D()}\spxextra{in module diffsph.profiles.massmodels}}

\begin{fulllineitems}
\phantomsection\label{\detokenize{diffsph.profiles:diffsph.profiles.massmodels.D}}\pysiglinewithargsret{\sphinxcode{\sphinxupquote{diffsph.profiles.massmodels.}}\sphinxbfcode{\sphinxupquote{D}}}{\emph{\DUrole{n}{tharcmin}}, \emph{\DUrole{n}{galaxy}}, \emph{\DUrole{n}{rad\_temp}}, \emph{\DUrole{n}{manual}\DUrole{o}{=}\DUrole{default_value}{False}}, \emph{\DUrole{o}{**}\DUrole{n}{kwargs}}}{}
\sphinxAtStartPar
Model\sphinxhyphen{}specific D factor in GeV/cm \({}^2\)
\begin{quote}\begin{description}
\item[{Parameters}] \leavevmode\begin{itemize}
\item {} 
\sphinxAtStartPar
\sphinxstyleliteralstrong{\sphinxupquote{tharcmin}} \textendash{} angular radius in arcmin

\item {} 
\sphinxAtStartPar
\sphinxstyleliteralstrong{\sphinxupquote{galaxy}} (\sphinxstyleliteralemphasis{\sphinxupquote{str}}) \textendash{} name of the galaxy

\item {} 
\sphinxAtStartPar
\sphinxstyleliteralstrong{\sphinxupquote{rad\_temp}} (\sphinxstyleliteralemphasis{\sphinxupquote{str}}) \textendash{} radial template (\sphinxcode{\sphinxupquote{\textquotesingle{}NFW\textquotesingle{}}}, \sphinxcode{\sphinxupquote{\textquotesingle{}Einasto\textquotesingle{}}}, etc.)

\item {} 
\sphinxAtStartPar
\sphinxstyleliteralstrong{\sphinxupquote{manual}} (\sphinxstyleliteralemphasis{\sphinxupquote{bool}}) \textendash{} manual input of parameter values in rad\_temp (default value = \sphinxcode{\sphinxupquote{\textquotesingle{}False\textquotesingle{}}})

\end{itemize}

\end{description}\end{quote}

\sphinxAtStartPar
Keyword arguments
\begin{itemize}
\item {} 
\sphinxAtStartPar
\sphinxcode{\sphinxupquote{manual = \textquotesingle{}False\textquotesingle{}}}

\end{itemize}
\begin{quote}\begin{description}
\item[{Parameters}] \leavevmode
\sphinxAtStartPar
\sphinxstyleliteralstrong{\sphinxupquote{ref}} \textendash{} reference used (\sphinxcode{\sphinxupquote{\textquotesingle{}Martinez\textquotesingle{}}} or \sphinxcode{\sphinxupquote{\textquotesingle{}1309.2641\textquotesingle{}}}, \sphinxcode{\sphinxupquote{\textquotesingle{}Geringer\sphinxhyphen{}Sameth\textquotesingle{}}} or \sphinxcode{\sphinxupquote{\textquotesingle{}1408.0002\textquotesingle{}}}, etc.)

\end{description}\end{quote}
\begin{itemize}
\item {} 
\sphinxAtStartPar
\sphinxcode{\sphinxupquote{manual = \textquotesingle{}True\textquotesingle{}}}

\end{itemize}
\begin{quote}\begin{description}
\item[{Parameters}] \leavevmode\begin{itemize}
\item {} 
\sphinxAtStartPar
\sphinxstyleliteralstrong{\sphinxupquote{rs}} \textendash{} scale radius

\item {} 
\sphinxAtStartPar
\sphinxstyleliteralstrong{\sphinxupquote{rhos}} \textendash{} characteristic density

\item {} 
\sphinxAtStartPar
\sphinxstyleliteralstrong{\sphinxupquote{alpha}} \textendash{} exponent \(\alpha\) in the {\hyperref[\detokenize{diffsph.profiles:diffsph.profiles.templates.hdz}]{\sphinxcrossref{\sphinxcode{\sphinxupquote{diffsph.profiles.templates.hdz()}}}}} profile

\item {} 
\sphinxAtStartPar
\sphinxstyleliteralstrong{\sphinxupquote{beta}} \textendash{} exponent \(\beta\) in the {\hyperref[\detokenize{diffsph.profiles:diffsph.profiles.templates.hdz}]{\sphinxcrossref{\sphinxcode{\sphinxupquote{diffsph.profiles.templates.hdz()}}}}} profile

\item {} 
\sphinxAtStartPar
\sphinxstyleliteralstrong{\sphinxupquote{gamma}} \textendash{} exponent \(\gamma\) in the {\hyperref[\detokenize{diffsph.profiles:diffsph.profiles.templates.hdz}]{\sphinxcrossref{\sphinxcode{\sphinxupquote{diffsph.profiles.templates.hdz()}}}}} profile

\item {} 
\sphinxAtStartPar
\sphinxstyleliteralstrong{\sphinxupquote{alphaE}} \textendash{} parameter \(\alpha_E\) in the {\hyperref[\detokenize{diffsph.profiles:diffsph.profiles.templates.enst}]{\sphinxcrossref{\sphinxcode{\sphinxupquote{diffsph.profiles.templates.enst()}}}}} profile

\item {} 
\sphinxAtStartPar
\sphinxstyleliteralstrong{\sphinxupquote{rc}} \textendash{} core radius parameter \(r_c\) in the {\hyperref[\detokenize{diffsph.profiles:diffsph.profiles.templates.cnfw}]{\sphinxcrossref{\sphinxcode{\sphinxupquote{diffsph.profiles.templates.cnfw()}}}}} profile

\item {} 
\sphinxAtStartPar
\sphinxstyleliteralstrong{\sphinxupquote{sigmav}} \textendash{} velocity dispersion parameter \(\sigma_v\) in the {\hyperref[\detokenize{diffsph.profiles:diffsph.profiles.templates.sis}]{\sphinxcrossref{\sphinxcode{\sphinxupquote{diffsph.profiles.templates.sis()}}}}} profile

\end{itemize}

\item[{Returns}] \leavevmode
\sphinxAtStartPar
D factor

\end{description}\end{quote}

\end{fulllineitems}

\index{Hbr() (in module diffsph.profiles.massmodels)@\spxentry{Hbr()}\spxextra{in module diffsph.profiles.massmodels}}

\begin{fulllineitems}
\phantomsection\label{\detokenize{diffsph.profiles:diffsph.profiles.massmodels.Hbr}}\pysiglinewithargsret{\sphinxcode{\sphinxupquote{diffsph.profiles.massmodels.}}\sphinxbfcode{\sphinxupquote{Hbr}}}{\emph{\DUrole{n}{tharcmin}}, \emph{\DUrole{n}{galaxy}}, \emph{\DUrole{n}{rad\_temp}}, \emph{\DUrole{n}{hyp}}, \emph{\DUrole{n}{ratio}}, \emph{\DUrole{n}{regime}\DUrole{o}{=}\DUrole{default_value}{\textquotesingle{}B\textquotesingle{}}}, \emph{\DUrole{n}{manual}\DUrole{o}{=}\DUrole{default_value}{False}}, \emph{\DUrole{o}{**}\DUrole{n}{kwargs}}}{}
\sphinxAtStartPar
Model\sphinxhyphen{}specific brightness halo/bulge function in the Regime “A”, “B” or “C” approximations
\begin{quote}\begin{description}
\item[{Parameters}] \leavevmode\begin{itemize}
\item {} 
\sphinxAtStartPar
\sphinxstyleliteralstrong{\sphinxupquote{tharcmin}} \textendash{} angular radius in arcmin

\item {} 
\sphinxAtStartPar
\sphinxstyleliteralstrong{\sphinxupquote{galaxy}} (\sphinxstyleliteralemphasis{\sphinxupquote{str}}) \textendash{} name of the galaxy

\item {} 
\sphinxAtStartPar
\sphinxstyleliteralstrong{\sphinxupquote{rad\_temp}} (\sphinxstyleliteralemphasis{\sphinxupquote{str}}) \textendash{} radial template (\sphinxcode{\sphinxupquote{\textquotesingle{}NFW\textquotesingle{}}}, \sphinxcode{\sphinxupquote{\textquotesingle{}Einasto\textquotesingle{}}}, etc.)

\item {} 
\sphinxAtStartPar
\sphinxstyleliteralstrong{\sphinxupquote{hyp}} (\sphinxstyleliteralemphasis{\sphinxupquote{str}}) \textendash{} hypothesis: \sphinxcode{\sphinxupquote{\textquotesingle{}wimp\textquotesingle{}}} (\sphinxstylestrong{default}), \sphinxcode{\sphinxupquote{\textquotesingle{}decay\textquotesingle{}}} or \sphinxcode{\sphinxupquote{\textquotesingle{}generic\textquotesingle{}}}

\item {} 
\sphinxAtStartPar
\sphinxstyleliteralstrong{\sphinxupquote{ratio}} \textendash{} ratio between the diffusion halo/bulge and the half\sphinxhyphen{}light radius (default value = 1)

\item {} 
\sphinxAtStartPar
\sphinxstyleliteralstrong{\sphinxupquote{regime}} \textendash{} regime of the approximation. Must be either upper or lower case a, b, c or I/II/III.

\item {} 
\sphinxAtStartPar
\sphinxstyleliteralstrong{\sphinxupquote{manual}} (\sphinxstyleliteralemphasis{\sphinxupquote{bool}}) \textendash{} manual input of parameter values in rad\_temp (default value = \sphinxcode{\sphinxupquote{\textquotesingle{}False\textquotesingle{}}})

\end{itemize}

\end{description}\end{quote}

\sphinxAtStartPar
Keyword arguments
\begin{itemize}
\item {} 
\sphinxAtStartPar
\sphinxcode{\sphinxupquote{manual = \textquotesingle{}False\textquotesingle{}}}

\end{itemize}
\begin{quote}\begin{description}
\item[{Parameters}] \leavevmode
\sphinxAtStartPar
\sphinxstyleliteralstrong{\sphinxupquote{ref}} \textendash{} reference used (\sphinxcode{\sphinxupquote{\textquotesingle{}Martinez\textquotesingle{}}} or \sphinxcode{\sphinxupquote{\textquotesingle{}1309.2641\textquotesingle{}}}, \sphinxcode{\sphinxupquote{\textquotesingle{}Geringer\sphinxhyphen{}Sameth\textquotesingle{}}} or \sphinxcode{\sphinxupquote{\textquotesingle{}1408.0002\textquotesingle{}}}, etc.)

\end{description}\end{quote}
\begin{itemize}
\item {} 
\sphinxAtStartPar
\sphinxcode{\sphinxupquote{manual = \textquotesingle{}True\textquotesingle{}}}

\end{itemize}
\begin{quote}\begin{description}
\item[{Parameters}] \leavevmode\begin{itemize}
\item {} 
\sphinxAtStartPar
\sphinxstyleliteralstrong{\sphinxupquote{rs}} \textendash{} scale radius

\item {} 
\sphinxAtStartPar
\sphinxstyleliteralstrong{\sphinxupquote{rhos}} \textendash{} characteristic density

\item {} 
\sphinxAtStartPar
\sphinxstyleliteralstrong{\sphinxupquote{alpha}} \textendash{} exponent \(\alpha\) in the {\hyperref[\detokenize{diffsph.profiles:diffsph.profiles.templates.hdz}]{\sphinxcrossref{\sphinxcode{\sphinxupquote{diffsph.profiles.templates.hdz()}}}}} profile

\item {} 
\sphinxAtStartPar
\sphinxstyleliteralstrong{\sphinxupquote{beta}} \textendash{} exponent \(\beta\) in the {\hyperref[\detokenize{diffsph.profiles:diffsph.profiles.templates.hdz}]{\sphinxcrossref{\sphinxcode{\sphinxupquote{diffsph.profiles.templates.hdz()}}}}} profile

\item {} 
\sphinxAtStartPar
\sphinxstyleliteralstrong{\sphinxupquote{gamma}} \textendash{} exponent \(\gamma\) in the {\hyperref[\detokenize{diffsph.profiles:diffsph.profiles.templates.hdz}]{\sphinxcrossref{\sphinxcode{\sphinxupquote{diffsph.profiles.templates.hdz()}}}}} profile

\item {} 
\sphinxAtStartPar
\sphinxstyleliteralstrong{\sphinxupquote{alphaE}} \textendash{} parameter \(\alpha_E\) in the {\hyperref[\detokenize{diffsph.profiles:diffsph.profiles.templates.enst}]{\sphinxcrossref{\sphinxcode{\sphinxupquote{diffsph.profiles.templates.enst()}}}}} profile

\item {} 
\sphinxAtStartPar
\sphinxstyleliteralstrong{\sphinxupquote{rc}} \textendash{} core radius parameter \(r_c\) in the {\hyperref[\detokenize{diffsph.profiles:diffsph.profiles.templates.cnfw}]{\sphinxcrossref{\sphinxcode{\sphinxupquote{diffsph.profiles.templates.cnfw()}}}}} profile

\item {} 
\sphinxAtStartPar
\sphinxstyleliteralstrong{\sphinxupquote{sigmav}} \textendash{} velocity dispersion parameter \(\sigma_v\) in the {\hyperref[\detokenize{diffsph.profiles:diffsph.profiles.templates.sis}]{\sphinxcrossref{\sphinxcode{\sphinxupquote{diffsph.profiles.templates.sis()}}}}} profile

\end{itemize}

\item[{Returns}] \leavevmode
\sphinxAtStartPar
brightness halo/bulge function

\end{description}\end{quote}

\end{fulllineitems}

\index{Hem() (in module diffsph.profiles.massmodels)@\spxentry{Hem()}\spxextra{in module diffsph.profiles.massmodels}}

\begin{fulllineitems}
\phantomsection\label{\detokenize{diffsph.profiles:diffsph.profiles.massmodels.Hem}}\pysiglinewithargsret{\sphinxcode{\sphinxupquote{diffsph.profiles.massmodels.}}\sphinxbfcode{\sphinxupquote{Hem}}}{\emph{\DUrole{n}{r}}, \emph{\DUrole{n}{galaxy}}, \emph{\DUrole{n}{rad\_temp}}, \emph{\DUrole{n}{hyp}}, \emph{\DUrole{n}{ratio}}, \emph{\DUrole{n}{regime}\DUrole{o}{=}\DUrole{default_value}{\textquotesingle{}B\textquotesingle{}}}, \emph{\DUrole{n}{manual}\DUrole{o}{=}\DUrole{default_value}{False}}, \emph{\DUrole{o}{**}\DUrole{n}{kwargs}}}{}
\sphinxAtStartPar
Model\sphinxhyphen{}specific emissivity halo/bulge function in the Regime “A”, “B” or “C” approximations
\begin{quote}\begin{description}
\item[{Parameters}] \leavevmode\begin{itemize}
\item {} 
\sphinxAtStartPar
\sphinxstyleliteralstrong{\sphinxupquote{r}} \textendash{} galactocentric distance in kpc

\item {} 
\sphinxAtStartPar
\sphinxstyleliteralstrong{\sphinxupquote{galaxy}} (\sphinxstyleliteralemphasis{\sphinxupquote{str}}) \textendash{} name of the galaxy

\item {} 
\sphinxAtStartPar
\sphinxstyleliteralstrong{\sphinxupquote{rad\_temp}} (\sphinxstyleliteralemphasis{\sphinxupquote{str}}) \textendash{} radial template (\sphinxcode{\sphinxupquote{\textquotesingle{}NFW\textquotesingle{}}}, \sphinxcode{\sphinxupquote{\textquotesingle{}Einasto\textquotesingle{}}}, etc.)

\item {} 
\sphinxAtStartPar
\sphinxstyleliteralstrong{\sphinxupquote{hyp}} (\sphinxstyleliteralemphasis{\sphinxupquote{str}}) \textendash{} hypothesis: \sphinxcode{\sphinxupquote{\textquotesingle{}wimp\textquotesingle{}}} (\sphinxstylestrong{default}), \sphinxcode{\sphinxupquote{\textquotesingle{}decay\textquotesingle{}}} or \sphinxcode{\sphinxupquote{\textquotesingle{}generic\textquotesingle{}}}

\item {} 
\sphinxAtStartPar
\sphinxstyleliteralstrong{\sphinxupquote{ratio}} \textendash{} ratio between the diffusion halo/bulge and the half\sphinxhyphen{}light radius (default value = 1)

\item {} 
\sphinxAtStartPar
\sphinxstyleliteralstrong{\sphinxupquote{regime}} \textendash{} regime of the approximation. Must be either upper or lower case a, b, c or I/II/III.

\item {} 
\sphinxAtStartPar
\sphinxstyleliteralstrong{\sphinxupquote{manual}} (\sphinxstyleliteralemphasis{\sphinxupquote{bool}}) \textendash{} manual input of parameter values in rad\_temp (default value = \sphinxcode{\sphinxupquote{\textquotesingle{}False\textquotesingle{}}})

\end{itemize}

\end{description}\end{quote}

\sphinxAtStartPar
Keyword arguments
\begin{itemize}
\item {} 
\sphinxAtStartPar
\sphinxcode{\sphinxupquote{manual = \textquotesingle{}False\textquotesingle{}}}

\end{itemize}
\begin{quote}\begin{description}
\item[{Parameters}] \leavevmode
\sphinxAtStartPar
\sphinxstyleliteralstrong{\sphinxupquote{ref}} \textendash{} reference used (\sphinxcode{\sphinxupquote{\textquotesingle{}Martinez\textquotesingle{}}} or \sphinxcode{\sphinxupquote{\textquotesingle{}1309.2641\textquotesingle{}}}, \sphinxcode{\sphinxupquote{\textquotesingle{}Geringer\sphinxhyphen{}Sameth\textquotesingle{}}} or \sphinxcode{\sphinxupquote{\textquotesingle{}1408.0002\textquotesingle{}}}, etc.)

\end{description}\end{quote}
\begin{itemize}
\item {} 
\sphinxAtStartPar
\sphinxcode{\sphinxupquote{manual = \textquotesingle{}True\textquotesingle{}}}

\end{itemize}
\begin{quote}\begin{description}
\item[{Parameters}] \leavevmode\begin{itemize}
\item {} 
\sphinxAtStartPar
\sphinxstyleliteralstrong{\sphinxupquote{rs}} \textendash{} scale radius

\item {} 
\sphinxAtStartPar
\sphinxstyleliteralstrong{\sphinxupquote{rhos}} \textendash{} characteristic density

\item {} 
\sphinxAtStartPar
\sphinxstyleliteralstrong{\sphinxupquote{alpha}} \textendash{} exponent \(\alpha\) in the {\hyperref[\detokenize{diffsph.profiles:diffsph.profiles.templates.hdz}]{\sphinxcrossref{\sphinxcode{\sphinxupquote{diffsph.profiles.templates.hdz()}}}}} profile

\item {} 
\sphinxAtStartPar
\sphinxstyleliteralstrong{\sphinxupquote{beta}} \textendash{} exponent \(\beta\) in the {\hyperref[\detokenize{diffsph.profiles:diffsph.profiles.templates.hdz}]{\sphinxcrossref{\sphinxcode{\sphinxupquote{diffsph.profiles.templates.hdz()}}}}} profile

\item {} 
\sphinxAtStartPar
\sphinxstyleliteralstrong{\sphinxupquote{gamma}} \textendash{} exponent \(\gamma\) in the {\hyperref[\detokenize{diffsph.profiles:diffsph.profiles.templates.hdz}]{\sphinxcrossref{\sphinxcode{\sphinxupquote{diffsph.profiles.templates.hdz()}}}}} profile

\item {} 
\sphinxAtStartPar
\sphinxstyleliteralstrong{\sphinxupquote{alphaE}} \textendash{} parameter \(\alpha_E\) in the {\hyperref[\detokenize{diffsph.profiles:diffsph.profiles.templates.enst}]{\sphinxcrossref{\sphinxcode{\sphinxupquote{diffsph.profiles.templates.enst()}}}}} profile

\item {} 
\sphinxAtStartPar
\sphinxstyleliteralstrong{\sphinxupquote{rc}} \textendash{} core radius parameter \(r_c\) in the {\hyperref[\detokenize{diffsph.profiles:diffsph.profiles.templates.cnfw}]{\sphinxcrossref{\sphinxcode{\sphinxupquote{diffsph.profiles.templates.cnfw()}}}}} profile

\item {} 
\sphinxAtStartPar
\sphinxstyleliteralstrong{\sphinxupquote{sigmav}} \textendash{} velocity dispersion parameter \(\sigma_v\) in the {\hyperref[\detokenize{diffsph.profiles:diffsph.profiles.templates.sis}]{\sphinxcrossref{\sphinxcode{\sphinxupquote{diffsph.profiles.templates.sis()}}}}} profile

\end{itemize}

\item[{Returns}] \leavevmode
\sphinxAtStartPar
emissivity halo/bulge function

\end{description}\end{quote}

\end{fulllineitems}

\index{Hfd() (in module diffsph.profiles.massmodels)@\spxentry{Hfd()}\spxextra{in module diffsph.profiles.massmodels}}

\begin{fulllineitems}
\phantomsection\label{\detokenize{diffsph.profiles:diffsph.profiles.massmodels.Hfd}}\pysiglinewithargsret{\sphinxcode{\sphinxupquote{diffsph.profiles.massmodels.}}\sphinxbfcode{\sphinxupquote{Hfd}}}{\emph{\DUrole{n}{tharcmin}}, \emph{\DUrole{n}{galaxy}}, \emph{\DUrole{n}{rad\_temp}}, \emph{\DUrole{n}{hyp}}, \emph{\DUrole{n}{ratio}}, \emph{\DUrole{n}{regime}\DUrole{o}{=}\DUrole{default_value}{\textquotesingle{}B\textquotesingle{}}}, \emph{\DUrole{n}{manual}\DUrole{o}{=}\DUrole{default_value}{False}}, \emph{\DUrole{o}{**}\DUrole{n}{kwargs}}}{}
\sphinxAtStartPar
Model\sphinxhyphen{}specific flux\sphinxhyphen{}density halo/bulge function in the Regime “A”, “B” or “C” approximations
\begin{quote}\begin{description}
\item[{Parameters}] \leavevmode\begin{itemize}
\item {} 
\sphinxAtStartPar
\sphinxstyleliteralstrong{\sphinxupquote{tharcmin}} \textendash{} angular radius in arcmin

\item {} 
\sphinxAtStartPar
\sphinxstyleliteralstrong{\sphinxupquote{galaxy}} (\sphinxstyleliteralemphasis{\sphinxupquote{str}}) \textendash{} name of the galaxy

\item {} 
\sphinxAtStartPar
\sphinxstyleliteralstrong{\sphinxupquote{rad\_temp}} (\sphinxstyleliteralemphasis{\sphinxupquote{str}}) \textendash{} radial template (\sphinxcode{\sphinxupquote{\textquotesingle{}NFW\textquotesingle{}}}, \sphinxcode{\sphinxupquote{\textquotesingle{}Einasto\textquotesingle{}}}, etc.)

\item {} 
\sphinxAtStartPar
\sphinxstyleliteralstrong{\sphinxupquote{hyp}} (\sphinxstyleliteralemphasis{\sphinxupquote{str}}) \textendash{} hypothesis: \sphinxcode{\sphinxupquote{\textquotesingle{}wimp\textquotesingle{}}} (\sphinxstylestrong{default}), \sphinxcode{\sphinxupquote{\textquotesingle{}decay\textquotesingle{}}} or \sphinxcode{\sphinxupquote{\textquotesingle{}generic\textquotesingle{}}}

\item {} 
\sphinxAtStartPar
\sphinxstyleliteralstrong{\sphinxupquote{ratio}} \textendash{} ratio between the diffusion halo/bulge and the half\sphinxhyphen{}light radius (default value = 1)

\item {} 
\sphinxAtStartPar
\sphinxstyleliteralstrong{\sphinxupquote{regime}} \textendash{} regime of the approximation. Must be either upper or lower case a, b, c or I/II/III.

\item {} 
\sphinxAtStartPar
\sphinxstyleliteralstrong{\sphinxupquote{manual}} (\sphinxstyleliteralemphasis{\sphinxupquote{bool}}) \textendash{} manual input of parameter values in rad\_temp (default value = \sphinxcode{\sphinxupquote{\textquotesingle{}False\textquotesingle{}}})

\end{itemize}

\end{description}\end{quote}

\sphinxAtStartPar
Keyword arguments
\begin{itemize}
\item {} 
\sphinxAtStartPar
\sphinxcode{\sphinxupquote{manual = \textquotesingle{}False\textquotesingle{}}}

\end{itemize}
\begin{quote}\begin{description}
\item[{Parameters}] \leavevmode
\sphinxAtStartPar
\sphinxstyleliteralstrong{\sphinxupquote{ref}} \textendash{} reference used (\sphinxcode{\sphinxupquote{\textquotesingle{}Martinez\textquotesingle{}}} or \sphinxcode{\sphinxupquote{\textquotesingle{}1309.2641\textquotesingle{}}}, \sphinxcode{\sphinxupquote{\textquotesingle{}Geringer\sphinxhyphen{}Sameth\textquotesingle{}}} or \sphinxcode{\sphinxupquote{\textquotesingle{}1408.0002\textquotesingle{}}}, etc.)

\end{description}\end{quote}
\begin{itemize}
\item {} 
\sphinxAtStartPar
\sphinxcode{\sphinxupquote{manual = \textquotesingle{}True\textquotesingle{}}}

\end{itemize}
\begin{quote}\begin{description}
\item[{Parameters}] \leavevmode\begin{itemize}
\item {} 
\sphinxAtStartPar
\sphinxstyleliteralstrong{\sphinxupquote{rs}} \textendash{} scale radius

\item {} 
\sphinxAtStartPar
\sphinxstyleliteralstrong{\sphinxupquote{rhos}} \textendash{} characteristic density

\item {} 
\sphinxAtStartPar
\sphinxstyleliteralstrong{\sphinxupquote{alpha}} \textendash{} exponent \(\alpha\) in the {\hyperref[\detokenize{diffsph.profiles:diffsph.profiles.templates.hdz}]{\sphinxcrossref{\sphinxcode{\sphinxupquote{diffsph.profiles.templates.hdz()}}}}} profile

\item {} 
\sphinxAtStartPar
\sphinxstyleliteralstrong{\sphinxupquote{beta}} \textendash{} exponent \(\beta\) in the {\hyperref[\detokenize{diffsph.profiles:diffsph.profiles.templates.hdz}]{\sphinxcrossref{\sphinxcode{\sphinxupquote{diffsph.profiles.templates.hdz()}}}}} profile

\item {} 
\sphinxAtStartPar
\sphinxstyleliteralstrong{\sphinxupquote{gamma}} \textendash{} exponent \(\gamma\) in the {\hyperref[\detokenize{diffsph.profiles:diffsph.profiles.templates.hdz}]{\sphinxcrossref{\sphinxcode{\sphinxupquote{diffsph.profiles.templates.hdz()}}}}} profile

\item {} 
\sphinxAtStartPar
\sphinxstyleliteralstrong{\sphinxupquote{alphaE}} \textendash{} parameter \(\alpha_E\) in the {\hyperref[\detokenize{diffsph.profiles:diffsph.profiles.templates.enst}]{\sphinxcrossref{\sphinxcode{\sphinxupquote{diffsph.profiles.templates.enst()}}}}} profile

\item {} 
\sphinxAtStartPar
\sphinxstyleliteralstrong{\sphinxupquote{rc}} \textendash{} core radius parameter \(r_c\) in the {\hyperref[\detokenize{diffsph.profiles:diffsph.profiles.templates.cnfw}]{\sphinxcrossref{\sphinxcode{\sphinxupquote{diffsph.profiles.templates.cnfw()}}}}} profile

\item {} 
\sphinxAtStartPar
\sphinxstyleliteralstrong{\sphinxupquote{sigmav}} \textendash{} velocity dispersion parameter \(\sigma_v\) in the {\hyperref[\detokenize{diffsph.profiles:diffsph.profiles.templates.sis}]{\sphinxcrossref{\sphinxcode{\sphinxupquote{diffsph.profiles.templates.sis()}}}}} profile

\end{itemize}

\item[{Returns}] \leavevmode
\sphinxAtStartPar
brightness halo/bulge function

\end{description}\end{quote}

\end{fulllineitems}

\index{J() (in module diffsph.profiles.massmodels)@\spxentry{J()}\spxextra{in module diffsph.profiles.massmodels}}

\begin{fulllineitems}
\phantomsection\label{\detokenize{diffsph.profiles:diffsph.profiles.massmodels.J}}\pysiglinewithargsret{\sphinxcode{\sphinxupquote{diffsph.profiles.massmodels.}}\sphinxbfcode{\sphinxupquote{J}}}{\emph{\DUrole{n}{tharcmin}}, \emph{\DUrole{n}{galaxy}}, \emph{\DUrole{n}{rad\_temp}}, \emph{\DUrole{n}{manual}\DUrole{o}{=}\DUrole{default_value}{False}}, \emph{\DUrole{o}{**}\DUrole{n}{kwargs}}}{}
\sphinxAtStartPar
Model\sphinxhyphen{}specific J factor in Gev \({}^2\)/cm \({}^5\)
\begin{quote}\begin{description}
\item[{Parameters}] \leavevmode\begin{itemize}
\item {} 
\sphinxAtStartPar
\sphinxstyleliteralstrong{\sphinxupquote{tharcmin}} \textendash{} angular radius in arcmin

\item {} 
\sphinxAtStartPar
\sphinxstyleliteralstrong{\sphinxupquote{galaxy}} (\sphinxstyleliteralemphasis{\sphinxupquote{str}}) \textendash{} name of the galaxy

\item {} 
\sphinxAtStartPar
\sphinxstyleliteralstrong{\sphinxupquote{rad\_temp}} (\sphinxstyleliteralemphasis{\sphinxupquote{str}}) \textendash{} radial template (\sphinxcode{\sphinxupquote{\textquotesingle{}NFW\textquotesingle{}}}, \sphinxcode{\sphinxupquote{\textquotesingle{}Einasto\textquotesingle{}}}, etc.)

\item {} 
\sphinxAtStartPar
\sphinxstyleliteralstrong{\sphinxupquote{manual}} (\sphinxstyleliteralemphasis{\sphinxupquote{bool}}) \textendash{} manual input of parameter values in rad\_temp (default value = \sphinxcode{\sphinxupquote{\textquotesingle{}False\textquotesingle{}}})

\end{itemize}

\end{description}\end{quote}

\sphinxAtStartPar
Keyword arguments
\begin{itemize}
\item {} 
\sphinxAtStartPar
\sphinxcode{\sphinxupquote{manual = \textquotesingle{}False\textquotesingle{}}}

\end{itemize}
\begin{quote}\begin{description}
\item[{Parameters}] \leavevmode
\sphinxAtStartPar
\sphinxstyleliteralstrong{\sphinxupquote{ref}} \textendash{} reference used (\sphinxcode{\sphinxupquote{\textquotesingle{}Martinez\textquotesingle{}}} or \sphinxcode{\sphinxupquote{\textquotesingle{}1309.2641\textquotesingle{}}}, \sphinxcode{\sphinxupquote{\textquotesingle{}Geringer\sphinxhyphen{}Sameth\textquotesingle{}}} or \sphinxcode{\sphinxupquote{\textquotesingle{}1408.0002\textquotesingle{}}}, etc.)

\end{description}\end{quote}
\begin{itemize}
\item {} 
\sphinxAtStartPar
\sphinxcode{\sphinxupquote{manual = \textquotesingle{}True\textquotesingle{}}}

\end{itemize}
\begin{quote}\begin{description}
\item[{Parameters}] \leavevmode\begin{itemize}
\item {} 
\sphinxAtStartPar
\sphinxstyleliteralstrong{\sphinxupquote{rs}} \textendash{} scale radius

\item {} 
\sphinxAtStartPar
\sphinxstyleliteralstrong{\sphinxupquote{rhos}} \textendash{} characteristic density

\item {} 
\sphinxAtStartPar
\sphinxstyleliteralstrong{\sphinxupquote{alpha}} \textendash{} exponent \(\alpha\) in the {\hyperref[\detokenize{diffsph.profiles:diffsph.profiles.templates.hdz}]{\sphinxcrossref{\sphinxcode{\sphinxupquote{diffsph.profiles.templates.hdz()}}}}} profile

\item {} 
\sphinxAtStartPar
\sphinxstyleliteralstrong{\sphinxupquote{beta}} \textendash{} exponent \(\beta\) in the {\hyperref[\detokenize{diffsph.profiles:diffsph.profiles.templates.hdz}]{\sphinxcrossref{\sphinxcode{\sphinxupquote{diffsph.profiles.templates.hdz()}}}}} profile

\item {} 
\sphinxAtStartPar
\sphinxstyleliteralstrong{\sphinxupquote{gamma}} \textendash{} exponent \(\gamma\) in the {\hyperref[\detokenize{diffsph.profiles:diffsph.profiles.templates.hdz}]{\sphinxcrossref{\sphinxcode{\sphinxupquote{diffsph.profiles.templates.hdz()}}}}} profile

\item {} 
\sphinxAtStartPar
\sphinxstyleliteralstrong{\sphinxupquote{alphaE}} \textendash{} parameter \(\alpha_E\) in the {\hyperref[\detokenize{diffsph.profiles:diffsph.profiles.templates.enst}]{\sphinxcrossref{\sphinxcode{\sphinxupquote{diffsph.profiles.templates.enst()}}}}} profile

\item {} 
\sphinxAtStartPar
\sphinxstyleliteralstrong{\sphinxupquote{rc}} \textendash{} core radius parameter \(r_c\) in the {\hyperref[\detokenize{diffsph.profiles:diffsph.profiles.templates.cnfw}]{\sphinxcrossref{\sphinxcode{\sphinxupquote{diffsph.profiles.templates.cnfw()}}}}} profile

\item {} 
\sphinxAtStartPar
\sphinxstyleliteralstrong{\sphinxupquote{sigmav}} \textendash{} velocity dispersion parameter \(\sigma_v\) in the {\hyperref[\detokenize{diffsph.profiles:diffsph.profiles.templates.sis}]{\sphinxcrossref{\sphinxcode{\sphinxupquote{diffsph.profiles.templates.sis()}}}}} profile

\end{itemize}

\item[{Returns}] \leavevmode
\sphinxAtStartPar
J factor

\end{description}\end{quote}

\end{fulllineitems}

\index{h() (in module diffsph.profiles.massmodels)@\spxentry{h()}\spxextra{in module diffsph.profiles.massmodels}}

\begin{fulllineitems}
\phantomsection\label{\detokenize{diffsph.profiles:diffsph.profiles.massmodels.h}}\pysiglinewithargsret{\sphinxcode{\sphinxupquote{diffsph.profiles.massmodels.}}\sphinxbfcode{\sphinxupquote{h}}}{\emph{\DUrole{n}{n}}, \emph{\DUrole{n}{galaxy}}, \emph{\DUrole{n}{rad\_temp}}, \emph{\DUrole{n}{hyp}}, \emph{\DUrole{n}{ratio}}, \emph{\DUrole{n}{manual}\DUrole{o}{=}\DUrole{default_value}{False}}, \emph{\DUrole{o}{**}\DUrole{n}{kwargs}}}{}
\sphinxAtStartPar
Model\sphinxhyphen{}specific n\sphinxhyphen{}th halo factor
\begin{quote}\begin{description}
\item[{Parameters}] \leavevmode\begin{itemize}
\item {} 
\sphinxAtStartPar
\sphinxstyleliteralstrong{\sphinxupquote{n}} \textendash{} order of the halo/bulge factor

\item {} 
\sphinxAtStartPar
\sphinxstyleliteralstrong{\sphinxupquote{rh}} \textendash{} diffusion halo/bulge radius

\item {} 
\sphinxAtStartPar
\sphinxstyleliteralstrong{\sphinxupquote{rad\_temp}} \textendash{} radial template

\item {} 
\sphinxAtStartPar
\sphinxstyleliteralstrong{\sphinxupquote{hyp}} (\sphinxstyleliteralemphasis{\sphinxupquote{str}}) \textendash{} hypothesis: \sphinxcode{\sphinxupquote{\textquotesingle{}wimp\textquotesingle{}}} (\sphinxstylestrong{default}), \sphinxcode{\sphinxupquote{\textquotesingle{}decay\textquotesingle{}}} or \sphinxcode{\sphinxupquote{\textquotesingle{}generic\textquotesingle{}}}

\item {} 
\sphinxAtStartPar
\sphinxstyleliteralstrong{\sphinxupquote{ratio}} \textendash{} ratio between the diffusion halo/bulge and the half\sphinxhyphen{}light radius

\item {} 
\sphinxAtStartPar
\sphinxstyleliteralstrong{\sphinxupquote{manual}} (\sphinxstyleliteralemphasis{\sphinxupquote{bool}}) \textendash{} manual input of parameter values in rad\_temp (default value = \sphinxcode{\sphinxupquote{\textquotesingle{}False\textquotesingle{}}})

\end{itemize}

\end{description}\end{quote}

\sphinxAtStartPar
Keyword arguments
\begin{itemize}
\item {} 
\sphinxAtStartPar
\sphinxcode{\sphinxupquote{manual = \textquotesingle{}False\textquotesingle{}}}

\end{itemize}
\begin{quote}\begin{description}
\item[{Parameters}] \leavevmode
\sphinxAtStartPar
\sphinxstyleliteralstrong{\sphinxupquote{ref}} \textendash{} reference used (\sphinxcode{\sphinxupquote{\textquotesingle{}Martinez\textquotesingle{}}} or \sphinxcode{\sphinxupquote{\textquotesingle{}1309.2641\textquotesingle{}}}, \sphinxcode{\sphinxupquote{\textquotesingle{}Geringer\sphinxhyphen{}Sameth\textquotesingle{}}} or \sphinxcode{\sphinxupquote{\textquotesingle{}1408.0002\textquotesingle{}}}, etc.)

\end{description}\end{quote}
\begin{itemize}
\item {} 
\sphinxAtStartPar
\sphinxcode{\sphinxupquote{manual = \textquotesingle{}True\textquotesingle{}}}

\end{itemize}
\begin{quote}\begin{description}
\item[{Parameters}] \leavevmode\begin{itemize}
\item {} 
\sphinxAtStartPar
\sphinxstyleliteralstrong{\sphinxupquote{rs}} \textendash{} scale radius

\item {} 
\sphinxAtStartPar
\sphinxstyleliteralstrong{\sphinxupquote{rhos}} \textendash{} characteristic density

\item {} 
\sphinxAtStartPar
\sphinxstyleliteralstrong{\sphinxupquote{alpha}} \textendash{} exponent \(\alpha\) in the {\hyperref[\detokenize{diffsph.profiles:diffsph.profiles.templates.hdz}]{\sphinxcrossref{\sphinxcode{\sphinxupquote{diffsph.profiles.templates.hdz()}}}}} profile

\item {} 
\sphinxAtStartPar
\sphinxstyleliteralstrong{\sphinxupquote{beta}} \textendash{} exponent \(\beta\) in the {\hyperref[\detokenize{diffsph.profiles:diffsph.profiles.templates.hdz}]{\sphinxcrossref{\sphinxcode{\sphinxupquote{diffsph.profiles.templates.hdz()}}}}} profile

\item {} 
\sphinxAtStartPar
\sphinxstyleliteralstrong{\sphinxupquote{gamma}} \textendash{} exponent \(\gamma\) in the {\hyperref[\detokenize{diffsph.profiles:diffsph.profiles.templates.hdz}]{\sphinxcrossref{\sphinxcode{\sphinxupquote{diffsph.profiles.templates.hdz()}}}}} profile

\item {} 
\sphinxAtStartPar
\sphinxstyleliteralstrong{\sphinxupquote{alphaE}} \textendash{} parameter \(\alpha_E\) in the {\hyperref[\detokenize{diffsph.profiles:diffsph.profiles.templates.enst}]{\sphinxcrossref{\sphinxcode{\sphinxupquote{diffsph.profiles.templates.enst()}}}}} profile

\item {} 
\sphinxAtStartPar
\sphinxstyleliteralstrong{\sphinxupquote{rc}} \textendash{} core radius parameter \(r_c\) in the {\hyperref[\detokenize{diffsph.profiles:diffsph.profiles.templates.cnfw}]{\sphinxcrossref{\sphinxcode{\sphinxupquote{diffsph.profiles.templates.cnfw()}}}}} profile

\item {} 
\sphinxAtStartPar
\sphinxstyleliteralstrong{\sphinxupquote{sigmav}} \textendash{} velocity dispersion parameter \(\sigma_v\) in the {\hyperref[\detokenize{diffsph.profiles:diffsph.profiles.templates.sis}]{\sphinxcrossref{\sphinxcode{\sphinxupquote{diffsph.profiles.templates.sis()}}}}} profile

\end{itemize}

\item[{Returns}] \leavevmode
\sphinxAtStartPar
halo factor

\end{description}\end{quote}

\end{fulllineitems}

\index{rho() (in module diffsph.profiles.massmodels)@\spxentry{rho()}\spxextra{in module diffsph.profiles.massmodels}}

\begin{fulllineitems}
\phantomsection\label{\detokenize{diffsph.profiles:diffsph.profiles.massmodels.rho}}\pysiglinewithargsret{\sphinxcode{\sphinxupquote{diffsph.profiles.massmodels.}}\sphinxbfcode{\sphinxupquote{rho}}}{\emph{\DUrole{n}{r}}, \emph{\DUrole{n}{rad\_temp}}, \emph{\DUrole{n}{manual}\DUrole{o}{=}\DUrole{default_value}{False}}, \emph{\DUrole{o}{**}\DUrole{n}{kwargs}}}{}
\sphinxAtStartPar
Dark matter density
\begin{quote}\begin{description}
\item[{Parameters}] \leavevmode\begin{itemize}
\item {} 
\sphinxAtStartPar
\sphinxstyleliteralstrong{\sphinxupquote{r}} \textendash{} galactocentric distance

\item {} 
\sphinxAtStartPar
\sphinxstyleliteralstrong{\sphinxupquote{rad\_temp}} \textendash{} template (‘NFW’, ‘Einasto’, etc.)

\item {} 
\sphinxAtStartPar
\sphinxstyleliteralstrong{\sphinxupquote{manual}} (\sphinxstyleliteralemphasis{\sphinxupquote{bool}}) \textendash{} manual input of parameter values in rad\_temp (default value = \sphinxcode{\sphinxupquote{\textquotesingle{}False\textquotesingle{}}})

\end{itemize}

\end{description}\end{quote}

\sphinxAtStartPar
Keyword arguments
\begin{itemize}
\item {} 
\sphinxAtStartPar
\sphinxcode{\sphinxupquote{manual = \textquotesingle{}False\textquotesingle{}}}

\end{itemize}
\begin{quote}\begin{description}
\item[{Parameters}] \leavevmode
\sphinxAtStartPar
\sphinxstyleliteralstrong{\sphinxupquote{ref}} \textendash{} reference used (\sphinxcode{\sphinxupquote{\textquotesingle{}Martinez\textquotesingle{}}} or \sphinxcode{\sphinxupquote{\textquotesingle{}1309.2641\textquotesingle{}}}, \sphinxcode{\sphinxupquote{\textquotesingle{}Geringer\sphinxhyphen{}Sameth\textquotesingle{}}} or \sphinxcode{\sphinxupquote{\textquotesingle{}1408.0002\textquotesingle{}}}, etc.)

\end{description}\end{quote}
\begin{itemize}
\item {} 
\sphinxAtStartPar
\sphinxcode{\sphinxupquote{manual = \textquotesingle{}True\textquotesingle{}}}

\end{itemize}
\begin{quote}\begin{description}
\item[{Parameters}] \leavevmode\begin{itemize}
\item {} 
\sphinxAtStartPar
\sphinxstyleliteralstrong{\sphinxupquote{rs}} \textendash{} scale radius

\item {} 
\sphinxAtStartPar
\sphinxstyleliteralstrong{\sphinxupquote{rhos}} \textendash{} characteristic density

\item {} 
\sphinxAtStartPar
\sphinxstyleliteralstrong{\sphinxupquote{alpha}} \textendash{} exponent \(\alpha\) in the {\hyperref[\detokenize{diffsph.profiles:diffsph.profiles.templates.hdz}]{\sphinxcrossref{\sphinxcode{\sphinxupquote{diffsph.profiles.templates.hdz()}}}}} profile

\item {} 
\sphinxAtStartPar
\sphinxstyleliteralstrong{\sphinxupquote{beta}} \textendash{} exponent \(\beta\) in the {\hyperref[\detokenize{diffsph.profiles:diffsph.profiles.templates.hdz}]{\sphinxcrossref{\sphinxcode{\sphinxupquote{diffsph.profiles.templates.hdz()}}}}} profile

\item {} 
\sphinxAtStartPar
\sphinxstyleliteralstrong{\sphinxupquote{gamma}} \textendash{} exponent \(\gamma\) in the {\hyperref[\detokenize{diffsph.profiles:diffsph.profiles.templates.hdz}]{\sphinxcrossref{\sphinxcode{\sphinxupquote{diffsph.profiles.templates.hdz()}}}}} profile

\item {} 
\sphinxAtStartPar
\sphinxstyleliteralstrong{\sphinxupquote{alphaE}} \textendash{} parameter \(\alpha_E\) in the {\hyperref[\detokenize{diffsph.profiles:diffsph.profiles.templates.enst}]{\sphinxcrossref{\sphinxcode{\sphinxupquote{diffsph.profiles.templates.enst()}}}}} profile

\item {} 
\sphinxAtStartPar
\sphinxstyleliteralstrong{\sphinxupquote{rc}} \textendash{} core radius parameter \(r_c\) in the {\hyperref[\detokenize{diffsph.profiles:diffsph.profiles.templates.cnfw}]{\sphinxcrossref{\sphinxcode{\sphinxupquote{diffsph.profiles.templates.cnfw()}}}}} profile

\item {} 
\sphinxAtStartPar
\sphinxstyleliteralstrong{\sphinxupquote{sigmav}} \textendash{} velocity dispersion parameter \(\sigma_v\) in the {\hyperref[\detokenize{diffsph.profiles:diffsph.profiles.templates.sis}]{\sphinxcrossref{\sphinxcode{\sphinxupquote{diffsph.profiles.templates.sis()}}}}} profile

\end{itemize}

\item[{Returns}] \leavevmode
\sphinxAtStartPar
dark matter density

\end{description}\end{quote}

\end{fulllineitems}



\paragraph{diffsph.profiles.templates module}
\label{\detokenize{diffsph.profiles:module-diffsph.profiles.templates}}\label{\detokenize{diffsph.profiles:diffsph-profiles-templates-module}}\index{module@\spxentry{module}!diffsph.profiles.templates@\spxentry{diffsph.profiles.templates}}\index{diffsph.profiles.templates@\spxentry{diffsph.profiles.templates}!module@\spxentry{module}}\index{bkrt() (in module diffsph.profiles.templates)@\spxentry{bkrt()}\spxextra{in module diffsph.profiles.templates}}

\begin{fulllineitems}
\phantomsection\label{\detokenize{diffsph.profiles:diffsph.profiles.templates.bkrt}}\pysiglinewithargsret{\sphinxcode{\sphinxupquote{diffsph.profiles.templates.}}\sphinxbfcode{\sphinxupquote{bkrt}}}{\emph{\DUrole{n}{r}}, \emph{\DUrole{n}{rs}}, \emph{\DUrole{n}{rhos}}}{}
\sphinxAtStartPar
Burkert dark\sphinxhyphen{}matter halo profile.
\begin{equation*}
\begin{split}\rho(r) = \frac{\rho_s}{(1+r/r_s)(1+r^2/r_s^2)}\end{split}
\end{equation*}\begin{quote}\begin{description}
\item[{Parameters}] \leavevmode\begin{itemize}
\item {} 
\sphinxAtStartPar
\sphinxstyleliteralstrong{\sphinxupquote{r}} \textendash{} main variable (galactocentric distance)

\item {} 
\sphinxAtStartPar
\sphinxstyleliteralstrong{\sphinxupquote{rs}} \textendash{} scale radius

\item {} 
\sphinxAtStartPar
\sphinxstyleliteralstrong{\sphinxupquote{rhos}} \textendash{} charactieristic density

\end{itemize}

\item[{Returns}] \leavevmode
\sphinxAtStartPar
density at galactocentric distance \(r\)

\end{description}\end{quote}

\end{fulllineitems}

\index{cnfw() (in module diffsph.profiles.templates)@\spxentry{cnfw()}\spxextra{in module diffsph.profiles.templates}}

\begin{fulllineitems}
\phantomsection\label{\detokenize{diffsph.profiles:diffsph.profiles.templates.cnfw}}\pysiglinewithargsret{\sphinxcode{\sphinxupquote{diffsph.profiles.templates.}}\sphinxbfcode{\sphinxupquote{cnfw}}}{\emph{\DUrole{n}{r}}, \emph{\DUrole{n}{rs}}, \emph{\DUrole{n}{rhos}}, \emph{\DUrole{n}{rc}}}{}
\sphinxAtStartPar
Cored Navarro/Frenk/White dark\sphinxhyphen{}matter halo template.
\begin{equation*}
\begin{split}\rho(r) = \frac{\rho_s}{(r/r_s+r_c/r_s)(1+r/r_s)^2}\end{split}
\end{equation*}\begin{quote}\begin{description}
\item[{Parameters}] \leavevmode\begin{itemize}
\item {} 
\sphinxAtStartPar
\sphinxstyleliteralstrong{\sphinxupquote{r}} \textendash{} main variable (galactocentric distance)

\item {} 
\sphinxAtStartPar
\sphinxstyleliteralstrong{\sphinxupquote{rs}} \textendash{} scale radius

\item {} 
\sphinxAtStartPar
\sphinxstyleliteralstrong{\sphinxupquote{rhos}} \textendash{} chraracteristic density

\item {} 
\sphinxAtStartPar
\sphinxstyleliteralstrong{\sphinxupquote{rc}} \textendash{} core radius

\end{itemize}

\item[{Returns}] \leavevmode
\sphinxAtStartPar
density at galactocentric distance \(r\)

\end{description}\end{quote}

\end{fulllineitems}

\index{const() (in module diffsph.profiles.templates)@\spxentry{const()}\spxextra{in module diffsph.profiles.templates}}

\begin{fulllineitems}
\phantomsection\label{\detokenize{diffsph.profiles:diffsph.profiles.templates.const}}\pysiglinewithargsret{\sphinxcode{\sphinxupquote{diffsph.profiles.templates.}}\sphinxbfcode{\sphinxupquote{const}}}{\emph{\DUrole{n}{r}}, \emph{\DUrole{n}{rs}}}{}
\sphinxAtStartPar
Constant (top\sphinxhyphen{}hat) template
\begin{equation*}
\begin{split}\rho(r) = \frac3{4\pi r_s^3}\Theta(r_s-r)\end{split}
\end{equation*}\begin{quote}\begin{description}
\item[{Parameters}] \leavevmode
\sphinxAtStartPar
\sphinxstyleliteralstrong{\sphinxupquote{rs}} \textendash{} characteristic radius

\item[{Returns}] \leavevmode
\sphinxAtStartPar
constant density

\end{description}\end{quote}

\end{fulllineitems}

\index{enst() (in module diffsph.profiles.templates)@\spxentry{enst()}\spxextra{in module diffsph.profiles.templates}}

\begin{fulllineitems}
\phantomsection\label{\detokenize{diffsph.profiles:diffsph.profiles.templates.enst}}\pysiglinewithargsret{\sphinxcode{\sphinxupquote{diffsph.profiles.templates.}}\sphinxbfcode{\sphinxupquote{enst}}}{\emph{\DUrole{n}{r}}, \emph{\DUrole{n}{rs}}, \emph{\DUrole{n}{rhos}}, \emph{\DUrole{n}{alphaE}\DUrole{o}{=}\DUrole{default_value}{0.17}}}{}
\sphinxAtStartPar
Einasto dark\sphinxhyphen{}matter halo profile.
\begin{equation*}
\begin{split}\rho(r) = \rho_s\exp\left[-\frac{2}{\alpha_E}\left(\frac{r^{\alpha_E}}{r_s^{\alpha_E}}-1\right)\right]\end{split}
\end{equation*}\begin{quote}\begin{description}
\item[{Parameters}] \leavevmode\begin{itemize}
\item {} 
\sphinxAtStartPar
\sphinxstyleliteralstrong{\sphinxupquote{r}} \textendash{} main variable (galactocentric distance)

\item {} 
\sphinxAtStartPar
\sphinxstyleliteralstrong{\sphinxupquote{rs}} \textendash{} scale radius

\item {} 
\sphinxAtStartPar
\sphinxstyleliteralstrong{\sphinxupquote{rhos}} \textendash{} charactieristic density

\item {} 
\sphinxAtStartPar
\sphinxstyleliteralstrong{\sphinxupquote{alphaE}} \textendash{} power\sphinxhyphen{}law slope of the Einasto profile, (default value = 0.17)

\end{itemize}

\item[{Returns}] \leavevmode
\sphinxAtStartPar
density at galactocentric distance \(r\)

\end{description}\end{quote}

\end{fulllineitems}

\index{hdz() (in module diffsph.profiles.templates)@\spxentry{hdz()}\spxextra{in module diffsph.profiles.templates}}

\begin{fulllineitems}
\phantomsection\label{\detokenize{diffsph.profiles:diffsph.profiles.templates.hdz}}\pysiglinewithargsret{\sphinxcode{\sphinxupquote{diffsph.profiles.templates.}}\sphinxbfcode{\sphinxupquote{hdz}}}{\emph{\DUrole{n}{r}}, \emph{\DUrole{n}{rs}}, \emph{\DUrole{n}{rhos}}, \emph{\DUrole{n}{alpha}}, \emph{\DUrole{n}{beta}}, \emph{\DUrole{n}{gamma}}}{}
\sphinxAtStartPar
Hernquist/Diemand/Zhao dark\sphinxhyphen{}matter halo template.
\begin{equation*}
\begin{split}\rho(r) = \frac{\rho_s}{(r/r_s)^\gamma(1+(r/r_s)^\alpha)^{\frac{\beta-\gamma}\alpha}}\end{split}
\end{equation*}
\sphinxAtStartPar
Using default values alpha = 1, beta = 3 and gamma = 1 results in the default NFW halo profile.
\begin{quote}\begin{description}
\item[{Parameters}] \leavevmode\begin{itemize}
\item {} 
\sphinxAtStartPar
\sphinxstyleliteralstrong{\sphinxupquote{r}} \textendash{} main variable (galactocentric distance)

\item {} 
\sphinxAtStartPar
\sphinxstyleliteralstrong{\sphinxupquote{rs}} \textendash{} scale radius

\item {} 
\sphinxAtStartPar
\sphinxstyleliteralstrong{\sphinxupquote{rhos}} \textendash{} chraracteristic density

\item {} 
\sphinxAtStartPar
\sphinxstyleliteralstrong{\sphinxupquote{alpha}} \textendash{} inner exponent

\item {} 
\sphinxAtStartPar
\sphinxstyleliteralstrong{\sphinxupquote{beta}} \textendash{} large\sphinxhyphen{}r exponent

\item {} 
\sphinxAtStartPar
\sphinxstyleliteralstrong{\sphinxupquote{gamma}} \textendash{} small\sphinxhyphen{}r exponent

\end{itemize}

\item[{Returns}] \leavevmode
\sphinxAtStartPar
density at galactocentric distance r

\end{description}\end{quote}

\end{fulllineitems}

\index{nfw() (in module diffsph.profiles.templates)@\spxentry{nfw()}\spxextra{in module diffsph.profiles.templates}}

\begin{fulllineitems}
\phantomsection\label{\detokenize{diffsph.profiles:diffsph.profiles.templates.nfw}}\pysiglinewithargsret{\sphinxcode{\sphinxupquote{diffsph.profiles.templates.}}\sphinxbfcode{\sphinxupquote{nfw}}}{\emph{\DUrole{n}{r}}, \emph{\DUrole{n}{rs}}, \emph{\DUrole{n}{rhos}}}{}
\sphinxAtStartPar
Navarro/Frenk/White dark\sphinxhyphen{}matter halo template.
\begin{equation*}
\begin{split}\rho(r) = \frac{\rho_s}{(r/r_s)(1+r/r_s)^2}\end{split}
\end{equation*}\begin{quote}\begin{description}
\item[{Parameters}] \leavevmode\begin{itemize}
\item {} 
\sphinxAtStartPar
\sphinxstyleliteralstrong{\sphinxupquote{r}} \textendash{} main variable (galactocentric distance)

\item {} 
\sphinxAtStartPar
\sphinxstyleliteralstrong{\sphinxupquote{rs}} \textendash{} scale radius

\item {} 
\sphinxAtStartPar
\sphinxstyleliteralstrong{\sphinxupquote{rhos}} \textendash{} chraracteristic density

\end{itemize}

\item[{Returns}] \leavevmode
\sphinxAtStartPar
density at galactocentric distance \(r\)

\end{description}\end{quote}

\end{fulllineitems}

\index{plmm() (in module diffsph.profiles.templates)@\spxentry{plmm()}\spxextra{in module diffsph.profiles.templates}}

\begin{fulllineitems}
\phantomsection\label{\detokenize{diffsph.profiles:diffsph.profiles.templates.plmm}}\pysiglinewithargsret{\sphinxcode{\sphinxupquote{diffsph.profiles.templates.}}\sphinxbfcode{\sphinxupquote{plmm}}}{\emph{\DUrole{n}{r}}, \emph{\DUrole{n}{rs}}}{}
\sphinxAtStartPar
Plummer template
\begin{equation*}
\begin{split}\rho(r) = \frac3{4\pi r_s^3}\frac1{(1+r^2/r_s^2)^{5/2}}\end{split}
\end{equation*}\begin{quote}\begin{description}
\item[{Parameters}] \leavevmode\begin{itemize}
\item {} 
\sphinxAtStartPar
\sphinxstyleliteralstrong{\sphinxupquote{r}} \textendash{} main variable (distance to the center)

\item {} 
\sphinxAtStartPar
\sphinxstyleliteralstrong{\sphinxupquote{rs}} \textendash{} Plummer radius

\item {} 
\sphinxAtStartPar
\sphinxstyleliteralstrong{\sphinxupquote{rhoa}} \textendash{} central density

\end{itemize}

\item[{Returns}] \leavevmode
\sphinxAtStartPar
density of the Plummer sphere at distance \(r\)

\end{description}\end{quote}

\end{fulllineitems}

\index{ps() (in module diffsph.profiles.templates)@\spxentry{ps()}\spxextra{in module diffsph.profiles.templates}}

\begin{fulllineitems}
\phantomsection\label{\detokenize{diffsph.profiles:diffsph.profiles.templates.ps}}\pysiglinewithargsret{\sphinxcode{\sphinxupquote{diffsph.profiles.templates.}}\sphinxbfcode{\sphinxupquote{ps}}}{\emph{\DUrole{n}{r}}, \emph{\DUrole{n}{rs}}}{}
\sphinxAtStartPar
Point source template
\begin{equation*}
\begin{split}\rho(r) = \frac1{4\pi r^2}\delta(r)\end{split}
\end{equation*}\begin{quote}\begin{description}
\item[{Parameters}] \leavevmode\begin{itemize}
\item {} 
\sphinxAtStartPar
\sphinxstyleliteralstrong{\sphinxupquote{r}} \textendash{} main variable (galactocentric distance)

\item {} 
\sphinxAtStartPar
\sphinxstyleliteralstrong{\sphinxupquote{rs}} \textendash{} characteristic radius

\end{itemize}

\item[{Returns}] \leavevmode
\sphinxAtStartPar
density at galactocentric distance \(r\)

\end{description}\end{quote}

\end{fulllineitems}

\index{ps\_iso() (in module diffsph.profiles.templates)@\spxentry{ps\_iso()}\spxextra{in module diffsph.profiles.templates}}

\begin{fulllineitems}
\phantomsection\label{\detokenize{diffsph.profiles:diffsph.profiles.templates.ps_iso}}\pysiglinewithargsret{\sphinxcode{\sphinxupquote{diffsph.profiles.templates.}}\sphinxbfcode{\sphinxupquote{ps\_iso}}}{\emph{\DUrole{n}{r}}, \emph{\DUrole{n}{rs}}, \emph{\DUrole{n}{rhos}}}{}
\sphinxAtStartPar
Pseudo\sphinxhyphen{}isothermal sphere dark\sphinxhyphen{}matter halo profile.
\begin{equation*}
\begin{split}\rho(r) = \frac{\rho_s}{1+r^2/r_s^2}\end{split}
\end{equation*}\begin{quote}\begin{description}
\item[{Parameters}] \leavevmode\begin{itemize}
\item {} 
\sphinxAtStartPar
\sphinxstyleliteralstrong{\sphinxupquote{r}} \textendash{} main variable (galactocentric distance)

\item {} 
\sphinxAtStartPar
\sphinxstyleliteralstrong{\sphinxupquote{rs}} \textendash{} scale radius

\item {} 
\sphinxAtStartPar
\sphinxstyleliteralstrong{\sphinxupquote{rhos}} \textendash{} characteristic density

\end{itemize}

\item[{Returns}] \leavevmode
\sphinxAtStartPar
density at galactocentric distance \(r\)

\end{description}\end{quote}

\end{fulllineitems}

\index{sis() (in module diffsph.profiles.templates)@\spxentry{sis()}\spxextra{in module diffsph.profiles.templates}}

\begin{fulllineitems}
\phantomsection\label{\detokenize{diffsph.profiles:diffsph.profiles.templates.sis}}\pysiglinewithargsret{\sphinxcode{\sphinxupquote{diffsph.profiles.templates.}}\sphinxbfcode{\sphinxupquote{sis}}}{\emph{\DUrole{n}{r}}, \emph{\DUrole{n}{sigmav}}}{}
\sphinxAtStartPar
Singular isothermal sphere
\begin{equation*}
\begin{split}\rho(r) = \frac{\sigma_v^2}{2\pi G r^2}    \end{split}
\end{equation*}\begin{quote}\begin{description}
\item[{Parameters}] \leavevmode\begin{itemize}
\item {} 
\sphinxAtStartPar
\sphinxstyleliteralstrong{\sphinxupquote{r}} \textendash{} main variable (galactocentric distance)

\item {} 
\sphinxAtStartPar
\sphinxstyleliteralstrong{\sphinxupquote{sigmav}} \textendash{} velocity dispersion

\end{itemize}

\item[{Returns}] \leavevmode
\sphinxAtStartPar
density at galactocentric distance \(r\)

\end{description}\end{quote}

\end{fulllineitems}



\paragraph{Module contents}
\label{\detokenize{diffsph.profiles:module-diffsph.profiles}}\label{\detokenize{diffsph.profiles:module-contents}}\index{module@\spxentry{module}!diffsph.profiles@\spxentry{diffsph.profiles}}\index{diffsph.profiles@\spxentry{diffsph.profiles}!module@\spxentry{module}}

\subsubsection{diffsph.spectra package}
\label{\detokenize{diffsph.spectra:diffsph-spectra-package}}\label{\detokenize{diffsph.spectra::doc}}

\paragraph{Submodules}
\label{\detokenize{diffsph.spectra:submodules}}

\paragraph{diffsph.spectra.analytics module}
\label{\detokenize{diffsph.spectra:module-diffsph.spectra.analytics}}\label{\detokenize{diffsph.spectra:diffsph-spectra-analytics-module}}\index{module@\spxentry{module}!diffsph.spectra.analytics@\spxentry{diffsph.spectra.analytics}}\index{diffsph.spectra.analytics@\spxentry{diffsph.spectra.analytics}!module@\spxentry{module}}\index{Fav() (in module diffsph.spectra.analytics)@\spxentry{Fav()}\spxextra{in module diffsph.spectra.analytics}}

\begin{fulllineitems}
\phantomsection\label{\detokenize{diffsph.spectra:diffsph.spectra.analytics.Fav}}\pysiglinewithargsret{\sphinxcode{\sphinxupquote{diffsph.spectra.analytics.}}\sphinxbfcode{\sphinxupquote{Fav}}}{\emph{\DUrole{n}{x}}}{}
\sphinxAtStartPar
Synchrotron\sphinxhyphen{}power function for randomly\sphinxhyphen{}oriented magnetic fields %
\begin{footnote}[*]\sphinxAtStartFootnote
Formula extracted from \sphinxhref{https://ui.adsabs.harvard.edu/abs/1988ApJ...334L...5G\%2F/}{Ghisellini et al, 1988}
%
\end{footnote}.
\begin{equation*}
\begin{split}F(x) = x^2 \left(K_{4/3}(x) K_{1/3}(x) - \frac35 x [K_{4/3}^2(x) - K_{1/3}^2(x)]\right)\end{split}
\end{equation*}\begin{quote}\begin{description}
\item[{Returns}] \leavevmode
\sphinxAtStartPar
Pitch\sphinxhyphen{}angle averaged synchrotron function as a function of \(x\)

\end{description}\end{quote}

\end{fulllineitems}

\index{M\_C() (in module diffsph.spectra.analytics)@\spxentry{M\_C()}\spxextra{in module diffsph.spectra.analytics}}

\begin{fulllineitems}
\phantomsection\label{\detokenize{diffsph.spectra:diffsph.spectra.analytics.M_C}}\pysiglinewithargsret{\sphinxcode{\sphinxupquote{diffsph.spectra.analytics.}}\sphinxbfcode{\sphinxupquote{M\_C}}}{\emph{\DUrole{n}{xi}}, \emph{\DUrole{n}{eta}}, \emph{\DUrole{n}{delta}}}{}
\sphinxAtStartPar
Master function in the Regime\sphinxhyphen{}C limit
\begin{equation*}
\begin{split}\mathcal M_C(\xi,\eta,\delta) = \frac{\xi^\delta}{(1-\delta)\eta} F(\xi^2)\end{split}
\end{equation*}
\end{fulllineitems}

\index{M\_i() (in module diffsph.spectra.analytics)@\spxentry{M\_i()}\spxextra{in module diffsph.spectra.analytics}}

\begin{fulllineitems}
\phantomsection\label{\detokenize{diffsph.spectra:diffsph.spectra.analytics.M_i}}\pysiglinewithargsret{\sphinxcode{\sphinxupquote{diffsph.spectra.analytics.}}\sphinxbfcode{\sphinxupquote{M\_i}}}{\emph{\DUrole{n}{xi}}, \emph{\DUrole{n}{eta}}, \emph{\DUrole{n}{delta}}}{}
\sphinxAtStartPar
Master function in the large \(\eta\) limit
\begin{equation*}
\begin{split}\mathcal M_i(\xi,\eta,\delta) = \frac{\Gamma^2(1/3)\eta^{-\frac{5}{3(1-\delta)}}}{5\sqrt[3]{2}(1-\delta)}\Gamma\left(\frac{5}{3(1-\delta)},\eta\, \xi^{1 - \delta}\right)\exp\left(\eta\,  \xi^{1-\delta}\right)\end{split}
\end{equation*}
\end{fulllineitems}

\index{M\_raw() (in module diffsph.spectra.analytics)@\spxentry{M\_raw()}\spxextra{in module diffsph.spectra.analytics}}

\begin{fulllineitems}
\phantomsection\label{\detokenize{diffsph.spectra:diffsph.spectra.analytics.M_raw}}\pysiglinewithargsret{\sphinxcode{\sphinxupquote{diffsph.spectra.analytics.}}\sphinxbfcode{\sphinxupquote{M\_raw}}}{\emph{\DUrole{n}{xi}}, \emph{\DUrole{n}{eta}}, \emph{\DUrole{n}{delta}}}{}
\sphinxAtStartPar
“Raw” master function
\begin{equation*}
\begin{split}\mathcal M(\xi,\eta,\delta) = \int_\xi^\infty dx F(x^2)\exp\left(-\eta\,[x^{1-\delta}-\xi^{1-\delta}]\right)\end{split}
\end{equation*}\begin{quote}\begin{description}
\item[{Returns}] \leavevmode
\sphinxAtStartPar
above integral

\end{description}\end{quote}

\end{fulllineitems}

\index{anltc\_Mst() (in module diffsph.spectra.analytics)@\spxentry{anltc\_Mst()}\spxextra{in module diffsph.spectra.analytics}}

\begin{fulllineitems}
\phantomsection\label{\detokenize{diffsph.spectra:diffsph.spectra.analytics.anltc_Mst}}\pysiglinewithargsret{\sphinxcode{\sphinxupquote{diffsph.spectra.analytics.}}\sphinxbfcode{\sphinxupquote{anltc\_Mst}}}{\emph{\DUrole{n}{xi}}, \emph{\DUrole{n}{eta}}, \emph{\DUrole{n}{delta}}}{}
\sphinxAtStartPar
Master function
\begin{equation*}
\begin{split}\mathcal M(\xi,\eta,\delta) = \int_\xi^\infty dx F(x^2)\exp\left(-\eta\,[x^{1-\delta}-\xi^{1-\delta}]\right)\end{split}
\end{equation*}
\begin{sphinxadmonition}{note}{Note:}
\sphinxAtStartPar
Function evaluates the above integral only for those values where no numerical errors are present. Otherwise, it uses the approximate formulas {\hyperref[\detokenize{diffsph.spectra:diffsph.spectra.analytics.M_C}]{\sphinxcrossref{\sphinxcode{\sphinxupquote{diffsph.spectra.analytics.M\_C()}}}}} or {\hyperref[\detokenize{diffsph.spectra:diffsph.spectra.analytics.M_i}]{\sphinxcrossref{\sphinxcode{\sphinxupquote{diffsph.spectra.analytics.M\_i()}}}}}
\end{sphinxadmonition}

\end{fulllineitems}

\index{btot() (in module diffsph.spectra.analytics)@\spxentry{btot()}\spxextra{in module diffsph.spectra.analytics}}

\begin{fulllineitems}
\phantomsection\label{\detokenize{diffsph.spectra:diffsph.spectra.analytics.btot}}\pysiglinewithargsret{\sphinxcode{\sphinxupquote{diffsph.spectra.analytics.}}\sphinxbfcode{\sphinxupquote{btot}}}{\emph{\DUrole{n}{E}}, \emph{\DUrole{n}{B}}}{}
\sphinxAtStartPar
Total energy loss function in GeV/s
\begin{quote}\begin{description}
\item[{Parameters}] \leavevmode\begin{itemize}
\item {} 
\sphinxAtStartPar
\sphinxstyleliteralstrong{\sphinxupquote{E}} \textendash{} cosmic\sphinxhyphen{}ray energy in GeV

\item {} 
\sphinxAtStartPar
\sphinxstyleliteralstrong{\sphinxupquote{B}} \textendash{} magnitude of the magnetic field’s smooth component in \(\mu\)G

\end{itemize}

\item[{Returns}] \leavevmode
\sphinxAtStartPar
energy\sphinxhyphen{}loss rate in GeV/s

\end{description}\end{quote}

\end{fulllineitems}

\index{lam() (in module diffsph.spectra.analytics)@\spxentry{lam()}\spxextra{in module diffsph.spectra.analytics}}

\begin{fulllineitems}
\phantomsection\label{\detokenize{diffsph.spectra:diffsph.spectra.analytics.lam}}\pysiglinewithargsret{\sphinxcode{\sphinxupquote{diffsph.spectra.analytics.}}\sphinxbfcode{\sphinxupquote{lam}}}{\emph{\DUrole{n}{E}}, \emph{\DUrole{n}{B}}, \emph{\DUrole{n}{D0}}, \emph{\DUrole{n}{delta}\DUrole{o}{=}\DUrole{default_value}{0.3333333333333333}}}{}
\sphinxAtStartPar
Syrovatskii variable in kpc$^{\text{2}}$
\begin{quote}\begin{description}
\item[{Parameters}] \leavevmode\begin{itemize}
\item {} 
\sphinxAtStartPar
\sphinxstyleliteralstrong{\sphinxupquote{E}} \textendash{} cosmic\sphinxhyphen{}ray energy in GeV

\item {} 
\sphinxAtStartPar
\sphinxstyleliteralstrong{\sphinxupquote{B}} \textendash{} magnitude of the magnetic field’s smooth component in \(\mu\)G

\item {} 
\sphinxAtStartPar
\sphinxstyleliteralstrong{\sphinxupquote{D0}} \textendash{} magnitude of the diffusion coefficient for a 1 GeV CRE in cm$^{\text{2}}$/s

\item {} 
\sphinxAtStartPar
\sphinxstyleliteralstrong{\sphinxupquote{delta}} \textendash{} power\sphinxhyphen{}law exponent of the diffusion coefficient as a function of the CRE’s energy (default value = 1/3)

\end{itemize}

\item[{Returns}] \leavevmode
\sphinxAtStartPar
Syrovatskii variable in kpc$^{\text{2}}$

\end{description}\end{quote}

\end{fulllineitems}



\paragraph{diffsph.spectra.synchrotron module}
\label{\detokenize{diffsph.spectra:module-diffsph.spectra.synchrotron}}\label{\detokenize{diffsph.spectra:diffsph-spectra-synchrotron-module}}\index{module@\spxentry{module}!diffsph.spectra.synchrotron@\spxentry{diffsph.spectra.synchrotron}}\index{diffsph.spectra.synchrotron@\spxentry{diffsph.spectra.synchrotron}!module@\spxentry{module}}\index{Enu() (in module diffsph.spectra.synchrotron)@\spxentry{Enu()}\spxextra{in module diffsph.spectra.synchrotron}}

\begin{fulllineitems}
\phantomsection\label{\detokenize{diffsph.spectra:diffsph.spectra.synchrotron.Enu}}\pysiglinewithargsret{\sphinxcode{\sphinxupquote{diffsph.spectra.synchrotron.}}\sphinxbfcode{\sphinxupquote{Enu}}}{\emph{\DUrole{n}{B}}, \emph{\DUrole{n}{nu}}}{}
\sphinxAtStartPar
Typical particle energy in GeV for synchrotron radiation at the frequency nu in GHz and for a magnetic field B in \(\mu\)G
\begin{quote}\begin{description}
\item[{Parameters}] \leavevmode\begin{itemize}
\item {} 
\sphinxAtStartPar
\sphinxstyleliteralstrong{\sphinxupquote{B}} \textendash{} magnitude of the magnetic field’s smooth component in \(\mu\)G

\item {} 
\sphinxAtStartPar
\sphinxstyleliteralstrong{\sphinxupquote{nu}} \textendash{} frequency in GHz

\end{itemize}

\item[{Returns}] \leavevmode
\sphinxAtStartPar
Particle energy in GeV.

\end{description}\end{quote}

\end{fulllineitems}

\index{Mst() (in module diffsph.spectra.synchrotron)@\spxentry{Mst()}\spxextra{in module diffsph.spectra.synchrotron}}

\begin{fulllineitems}
\phantomsection\label{\detokenize{diffsph.spectra:diffsph.spectra.synchrotron.Mst}}\pysiglinewithargsret{\sphinxcode{\sphinxupquote{diffsph.spectra.synchrotron.}}\sphinxbfcode{\sphinxupquote{Mst}}}{\emph{\DUrole{n}{xi}}, \emph{\DUrole{n}{eta}}, \emph{\DUrole{n}{delta}}}{}
\sphinxAtStartPar
Interpolation function for the kernel function \(\hat{\mathcal M}(\xi,\eta,\delta)\)
\begin{quote}\begin{description}
\item[{Parameters}] \leavevmode\begin{itemize}
\item {} 
\sphinxAtStartPar
\sphinxstyleliteralstrong{\sphinxupquote{xi}} \textendash{} \(\xi\)

\item {} 
\sphinxAtStartPar
\sphinxstyleliteralstrong{\sphinxupquote{eta}} \textendash{} \(\eta\)

\item {} 
\sphinxAtStartPar
\sphinxstyleliteralstrong{\sphinxupquote{delta}} \textendash{} \(\delta\)

\end{itemize}

\item[{Returns}] \leavevmode
\sphinxAtStartPar
Spectral\sphinxhyphen{}function kernel (as an interpolation function)

\end{description}\end{quote}

\end{fulllineitems}

\index{Mst\_DM() (in module diffsph.spectra.synchrotron)@\spxentry{Mst\_DM()}\spxextra{in module diffsph.spectra.synchrotron}}

\begin{fulllineitems}
\phantomsection\label{\detokenize{diffsph.spectra:diffsph.spectra.synchrotron.Mst_DM}}\pysiglinewithargsret{\sphinxcode{\sphinxupquote{diffsph.spectra.synchrotron.}}\sphinxbfcode{\sphinxupquote{Mst\_DM}}}{\emph{\DUrole{n}{xi}}, \emph{\DUrole{n}{eta}}, \emph{\DUrole{n}{m}}, \emph{\DUrole{n}{delta}}, \emph{\DUrole{n}{channel}}}{}
\sphinxAtStartPar
Master function for dark\sphinxhyphen{}matter hypotheses
\begin{quote}\begin{description}
\item[{Parameters}] \leavevmode\begin{itemize}
\item {} 
\sphinxAtStartPar
\sphinxstyleliteralstrong{\sphinxupquote{xi}} \textendash{} \(\xi\)

\item {} 
\sphinxAtStartPar
\sphinxstyleliteralstrong{\sphinxupquote{eta}} \textendash{} \(\eta\)

\item {} 
\sphinxAtStartPar
\sphinxstyleliteralstrong{\sphinxupquote{delta}} \textendash{} \(\delta\)

\item {} 
\sphinxAtStartPar
\sphinxstyleliteralstrong{\sphinxupquote{m}} \textendash{} WIMP mass in GeV

\item {} 
\sphinxAtStartPar
\sphinxstyleliteralstrong{\sphinxupquote{channel}} \textendash{} annihilation/decay channel

\end{itemize}

\item[{Returns}] \leavevmode
\sphinxAtStartPar
Master function (as an interpolation function) for DM hypotheses

\end{description}\end{quote}

\end{fulllineitems}

\index{Mst\_pw() (in module diffsph.spectra.synchrotron)@\spxentry{Mst\_pw()}\spxextra{in module diffsph.spectra.synchrotron}}

\begin{fulllineitems}
\phantomsection\label{\detokenize{diffsph.spectra:diffsph.spectra.synchrotron.Mst_pw}}\pysiglinewithargsret{\sphinxcode{\sphinxupquote{diffsph.spectra.synchrotron.}}\sphinxbfcode{\sphinxupquote{Mst\_pw}}}{\emph{\DUrole{n}{eta}}, \emph{\DUrole{n}{Gamma}}, \emph{\DUrole{n}{delta}}}{}
\sphinxAtStartPar
Master function for the generic power\sphinxhyphen{}law hypothesis
\begin{quote}\begin{description}
\item[{Parameters}] \leavevmode\begin{itemize}
\item {} 
\sphinxAtStartPar
\sphinxstyleliteralstrong{\sphinxupquote{eta}} \textendash{} \(\eta\)

\item {} 
\sphinxAtStartPar
\sphinxstyleliteralstrong{\sphinxupquote{Gamma}} \textendash{} \(\Gamma\)

\item {} 
\sphinxAtStartPar
\sphinxstyleliteralstrong{\sphinxupquote{delta}} \textendash{} \(\delta\)

\end{itemize}

\item[{Returns}] \leavevmode
\sphinxAtStartPar
Master function (as an interpolation function) for the generic power\sphinxhyphen{}law hypothesis

\end{description}\end{quote}

\end{fulllineitems}

\index{X() (in module diffsph.spectra.synchrotron)@\spxentry{X()}\spxextra{in module diffsph.spectra.synchrotron}}

\begin{fulllineitems}
\phantomsection\label{\detokenize{diffsph.spectra:diffsph.spectra.synchrotron.X}}\pysiglinewithargsret{\sphinxcode{\sphinxupquote{diffsph.spectra.synchrotron.}}\sphinxbfcode{\sphinxupquote{X}}}{\emph{\DUrole{n}{nu}}, \emph{\DUrole{n}{tau}}, \emph{\DUrole{n}{delta}}, \emph{\DUrole{n}{B}}, \emph{\DUrole{n}{hyp}}, \emph{\DUrole{o}{**}\DUrole{n}{kwargs}}}{}
\sphinxAtStartPar
Spectral function in erg/GHz for all hypotheses built in diffsph
\begin{quote}\begin{description}
\item[{Parameters}] \leavevmode\begin{itemize}
\item {} 
\sphinxAtStartPar
\sphinxstyleliteralstrong{\sphinxupquote{nu}} \textendash{} frequency in GHz

\item {} 
\sphinxAtStartPar
\sphinxstyleliteralstrong{\sphinxupquote{tau}} \textendash{} diffusion time\sphinxhyphen{}scale parameter for a 1 GeV CRE in s

\item {} 
\sphinxAtStartPar
\sphinxstyleliteralstrong{\sphinxupquote{delta}} \textendash{} power\sphinxhyphen{}law exponent of the diffusion coefficient as a function of the CRE’s energy

\item {} 
\sphinxAtStartPar
\sphinxstyleliteralstrong{\sphinxupquote{B}} \textendash{} magnitude of the magnetic field’s smooth component in \(\mu\)G

\item {} 
\sphinxAtStartPar
\sphinxstyleliteralstrong{\sphinxupquote{hyp}} (\sphinxstyleliteralemphasis{\sphinxupquote{str}}) \textendash{} hypothesis: \sphinxcode{\sphinxupquote{\textquotesingle{}wimp\textquotesingle{}}}, \sphinxcode{\sphinxupquote{\textquotesingle{}decay\textquotesingle{}}} or \sphinxcode{\sphinxupquote{\textquotesingle{}generic\textquotesingle{}}}

\end{itemize}

\end{description}\end{quote}

\begin{DUlineblock}{0em}
\item[] 
\end{DUlineblock}

\sphinxAtStartPar
\sphinxstylestrong{Keyword arguments:}

\begin{DUlineblock}{0em}
\item[] 
\end{DUlineblock}
\begin{itemize}
\item {} 
\sphinxAtStartPar
If \sphinxcode{\sphinxupquote{hyp = \textquotesingle{}wimp\textquotesingle{}}} or \sphinxcode{\sphinxupquote{\textquotesingle{}decay\textquotesingle{}}}

\end{itemize}
\begin{quote}\begin{description}
\item[{Parameters}] \leavevmode\begin{itemize}
\item {} 
\sphinxAtStartPar
\sphinxstyleliteralstrong{\sphinxupquote{mchi}} \textendash{} mass of the DM particle in GeV/\(c^2\)

\item {} 
\sphinxAtStartPar
\sphinxstyleliteralstrong{\sphinxupquote{channel}} (\sphinxstyleliteralemphasis{\sphinxupquote{str}}) \textendash{} annihilation/decay channel: \(b\bar b\) (\sphinxcode{\sphinxupquote{\textquotesingle{}bb\textquotesingle{}}}), \(\mu^+ \mu^-\) (\sphinxcode{\sphinxupquote{\textquotesingle{}mumu\textquotesingle{}}}), \(W^+ W^-\) (\sphinxcode{\sphinxupquote{\textquotesingle{}WW\textquotesingle{}}}), etc.

\end{itemize}

\end{description}\end{quote}
\begin{itemize}
\item {} 
\sphinxAtStartPar
If \sphinxcode{\sphinxupquote{hyp = \textquotesingle{}generic\textquotesingle{}}}

\end{itemize}
\begin{quote}\begin{description}
\item[{Parameters}] \leavevmode
\sphinxAtStartPar
\sphinxstyleliteralstrong{\sphinxupquote{Gamma}} \textendash{} power\sphinxhyphen{}law exponent of the generic CRE source (\(1.1 < \Gamma < 3\))

\end{description}\end{quote}

\begin{DUlineblock}{0em}
\item[] 
\end{DUlineblock}
\begin{quote}\begin{description}
\item[{Returns}] \leavevmode
\sphinxAtStartPar
spectral function in erg/GHz

\end{description}\end{quote}

\end{fulllineitems}

\index{X\_DM() (in module diffsph.spectra.synchrotron)@\spxentry{X\_DM()}\spxextra{in module diffsph.spectra.synchrotron}}

\begin{fulllineitems}
\phantomsection\label{\detokenize{diffsph.spectra:diffsph.spectra.synchrotron.X_DM}}\pysiglinewithargsret{\sphinxcode{\sphinxupquote{diffsph.spectra.synchrotron.}}\sphinxbfcode{\sphinxupquote{X\_DM}}}{\emph{\DUrole{n}{k}}, \emph{\DUrole{n}{mchi}}, \emph{\DUrole{n}{channel}}, \emph{\DUrole{n}{nu}}, \emph{\DUrole{n}{tau}}, \emph{\DUrole{n}{delta}}, \emph{\DUrole{n}{B}}}{}
\sphinxAtStartPar
Spectral function in erg/GHz for all DM hypotheses built in diffsph
\begin{quote}\begin{description}
\item[{Parameters}] \leavevmode\begin{itemize}
\item {} 
\sphinxAtStartPar
\sphinxstyleliteralstrong{\sphinxupquote{k}} \textendash{} hypothesis index (k=1 for decay and k=2 for annihilation)

\item {} 
\sphinxAtStartPar
\sphinxstyleliteralstrong{\sphinxupquote{mchi}} \textendash{} mass of the DM particle in GeV/\(c^2\)

\item {} 
\sphinxAtStartPar
\sphinxstyleliteralstrong{\sphinxupquote{channel}} \textendash{} annihilation/decay channel: \(b\bar b\) (\sphinxcode{\sphinxupquote{\textquotesingle{}bb\textquotesingle{}}}), \(\mu^+ \mu^-\) (\sphinxcode{\sphinxupquote{\textquotesingle{}mumu\textquotesingle{}}}), \(W^+ W^-\) (\sphinxcode{\sphinxupquote{\textquotesingle{}WW\textquotesingle{}}}), etc.

\item {} 
\sphinxAtStartPar
\sphinxstyleliteralstrong{\sphinxupquote{nu}} \textendash{} frequency in GHz

\item {} 
\sphinxAtStartPar
\sphinxstyleliteralstrong{\sphinxupquote{tau}} \textendash{} diffusion time\sphinxhyphen{}scale parameter for a 1 GeV CRE in s

\item {} 
\sphinxAtStartPar
\sphinxstyleliteralstrong{\sphinxupquote{delta}} \textendash{} power\sphinxhyphen{}law exponent of the diffusion coefficient as a function of the CRE’s energy

\item {} 
\sphinxAtStartPar
\sphinxstyleliteralstrong{\sphinxupquote{B}} \textendash{} magnitude of the magnetic field’s smooth component in \(\mu\)G

\end{itemize}

\item[{Returns}] \leavevmode
\sphinxAtStartPar
spectral function in erg/GHz

\end{description}\end{quote}

\end{fulllineitems}

\index{X\_gen() (in module diffsph.spectra.synchrotron)@\spxentry{X\_gen()}\spxextra{in module diffsph.spectra.synchrotron}}

\begin{fulllineitems}
\phantomsection\label{\detokenize{diffsph.spectra:diffsph.spectra.synchrotron.X_gen}}\pysiglinewithargsret{\sphinxcode{\sphinxupquote{diffsph.spectra.synchrotron.}}\sphinxbfcode{\sphinxupquote{X\_gen}}}{\emph{\DUrole{n}{Emin}}, \emph{\DUrole{n}{Emax}}, \emph{\DUrole{n}{S\_func}}, \emph{\DUrole{n}{nu}}, \emph{\DUrole{n}{tau}}, \emph{\DUrole{n}{delta}}, \emph{\DUrole{n}{B}}}{}
\sphinxAtStartPar
Spectral function in erg/GHz for generic CRE sources
\begin{equation*}
\begin{split}X_\text{gen}(\nu) = \int_{E_m}^{E_M}dE'\hat X(\nu, E')S(E')\end{split}
\end{equation*}\begin{quote}\begin{description}
\item[{Parameters}] \leavevmode\begin{itemize}
\item {} 
\sphinxAtStartPar
\sphinxstyleliteralstrong{\sphinxupquote{Emin}} \textendash{} low\sphinxhyphen{}E cutoff energy in GeV of the CRE source \sphinxcode{\sphinxupquote{\textquotesingle{}S\_func\textquotesingle{}}}

\item {} 
\sphinxAtStartPar
\sphinxstyleliteralstrong{\sphinxupquote{Emax}} \textendash{} high\sphinxhyphen{}E cutoff energy in GeV of the CRE source \sphinxcode{\sphinxupquote{\textquotesingle{}S\_func\textquotesingle{}}}

\item {} 
\sphinxAtStartPar
\sphinxstyleliteralstrong{\sphinxupquote{S\_func}} \textendash{} CRE source function

\item {} 
\sphinxAtStartPar
\sphinxstyleliteralstrong{\sphinxupquote{nu}} \textendash{} frequency in GHz

\item {} 
\sphinxAtStartPar
\sphinxstyleliteralstrong{\sphinxupquote{tau}} \textendash{} diffusion time\sphinxhyphen{}scale parameter for a 1 GeV CRE in s

\item {} 
\sphinxAtStartPar
\sphinxstyleliteralstrong{\sphinxupquote{delta}} \textendash{} power\sphinxhyphen{}law exponent of the diffusion coefficient as a function of the CRE’s energy

\item {} 
\sphinxAtStartPar
\sphinxstyleliteralstrong{\sphinxupquote{B}} \textendash{} magnitude of the magnetic field’s smooth component in \(\mu\)G

\end{itemize}

\item[{Returns}] \leavevmode
\sphinxAtStartPar
spectral function in erg/GHz

\end{description}\end{quote}

\end{fulllineitems}

\index{X\_pw() (in module diffsph.spectra.synchrotron)@\spxentry{X\_pw()}\spxextra{in module diffsph.spectra.synchrotron}}

\begin{fulllineitems}
\phantomsection\label{\detokenize{diffsph.spectra:diffsph.spectra.synchrotron.X_pw}}\pysiglinewithargsret{\sphinxcode{\sphinxupquote{diffsph.spectra.synchrotron.}}\sphinxbfcode{\sphinxupquote{X\_pw}}}{\emph{\DUrole{n}{Gamma}}, \emph{\DUrole{n}{nu}}, \emph{\DUrole{n}{tau}}, \emph{\DUrole{n}{delta}}, \emph{\DUrole{n}{B}}}{}
\sphinxAtStartPar
Spectral function in erg/GHz for the generic power\sphinxhyphen{}law hypothesis
\begin{quote}\begin{description}
\item[{Parameters}] \leavevmode\begin{itemize}
\item {} 
\sphinxAtStartPar
\sphinxstyleliteralstrong{\sphinxupquote{Gamma}} \textendash{} power\sphinxhyphen{}law exponent of the generic CRE source (\(1.1 < \Gamma < 3\))

\item {} 
\sphinxAtStartPar
\sphinxstyleliteralstrong{\sphinxupquote{nu}} \textendash{} frequency in GHz

\item {} 
\sphinxAtStartPar
\sphinxstyleliteralstrong{\sphinxupquote{tau}} \textendash{} diffusion time\sphinxhyphen{}scale parameter for a 1 GeV CRE in s

\item {} 
\sphinxAtStartPar
\sphinxstyleliteralstrong{\sphinxupquote{delta}} \textendash{} power\sphinxhyphen{}law exponent of the diffusion coefficient as a function of the CRE’s energy

\item {} 
\sphinxAtStartPar
\sphinxstyleliteralstrong{\sphinxupquote{B}} \textendash{} magnitude of the magnetic field’s smooth component in \(\mu\)G

\end{itemize}

\item[{Returns}] \leavevmode
\sphinxAtStartPar
spectral function in erg/GHz

\end{description}\end{quote}

\end{fulllineitems}

\index{eta() (in module diffsph.spectra.synchrotron)@\spxentry{eta()}\spxextra{in module diffsph.spectra.synchrotron}}

\begin{fulllineitems}
\phantomsection\label{\detokenize{diffsph.spectra:diffsph.spectra.synchrotron.eta}}\pysiglinewithargsret{\sphinxcode{\sphinxupquote{diffsph.spectra.synchrotron.}}\sphinxbfcode{\sphinxupquote{eta}}}{\emph{\DUrole{n}{E}}, \emph{\DUrole{n}{B}}, \emph{\DUrole{n}{tau}}, \emph{\DUrole{n}{delta}}}{}
\sphinxAtStartPar
\(\eta\) variable as a function of the CRE’s energy, magnetic field, tau and delta parameters
\begin{quote}\begin{description}
\item[{Parameters}] \leavevmode\begin{itemize}
\item {} 
\sphinxAtStartPar
\sphinxstyleliteralstrong{\sphinxupquote{E}} \textendash{} CRE energy in GeV

\item {} 
\sphinxAtStartPar
\sphinxstyleliteralstrong{\sphinxupquote{B}} \textendash{} magnetic field strength in µG

\item {} 
\sphinxAtStartPar
\sphinxstyleliteralstrong{\sphinxupquote{tau}} \textendash{} diffusion time\sphinxhyphen{}scale parameter for a 1 GeV CRE in s

\item {} 
\sphinxAtStartPar
\sphinxstyleliteralstrong{\sphinxupquote{delta}} \textendash{} power\sphinxhyphen{}law exponent of the diffusion coefficient as a function of the CRE’s energy

\end{itemize}

\item[{Returns}] \leavevmode
\sphinxAtStartPar
\(\eta\) variable

\end{description}\end{quote}

\end{fulllineitems}

\index{htX() (in module diffsph.spectra.synchrotron)@\spxentry{htX()}\spxextra{in module diffsph.spectra.synchrotron}}

\begin{fulllineitems}
\phantomsection\label{\detokenize{diffsph.spectra:diffsph.spectra.synchrotron.htX}}\pysiglinewithargsret{\sphinxcode{\sphinxupquote{diffsph.spectra.synchrotron.}}\sphinxbfcode{\sphinxupquote{htX}}}{\emph{\DUrole{n}{E}}, \emph{\DUrole{n}{nu}}, \emph{\DUrole{n}{tau}}, \emph{\DUrole{n}{delta}}, \emph{\DUrole{n}{B}}, \emph{\DUrole{n}{fast\_comp}\DUrole{o}{=}\DUrole{default_value}{True}}}{}
\sphinxAtStartPar
Spectral function kernel in erg/GHz \(\hat X\)
\begin{quote}\begin{description}
\item[{Parameters}] \leavevmode\begin{itemize}
\item {} 
\sphinxAtStartPar
\sphinxstyleliteralstrong{\sphinxupquote{E}} \textendash{} CRE energy in GeV

\item {} 
\sphinxAtStartPar
\sphinxstyleliteralstrong{\sphinxupquote{nu}} \textendash{} frequency in GHz

\item {} 
\sphinxAtStartPar
\sphinxstyleliteralstrong{\sphinxupquote{tau}} \textendash{} diffusion time\sphinxhyphen{}scale parameter for a 1 GeV CRE in s

\item {} 
\sphinxAtStartPar
\sphinxstyleliteralstrong{\sphinxupquote{delta}} \textendash{} power\sphinxhyphen{}law exponent of the diffusion coefficient as a function of the CRE’s energy

\item {} 
\sphinxAtStartPar
\sphinxstyleliteralstrong{\sphinxupquote{B}} \textendash{} magnitude of the magnetic field’s smooth component in \(\mu\)G

\item {} 
\sphinxAtStartPar
\sphinxstyleliteralstrong{\sphinxupquote{fast\_comp}} (\sphinxstyleliteralemphasis{\sphinxupquote{bool}}) \textendash{} if \sphinxcode{\sphinxupquote{\textquotesingle{}True\textquotesingle{}}}, employs the interpolating method (default value = \sphinxcode{\sphinxupquote{\textquotesingle{}True\textquotesingle{}}})

\end{itemize}

\item[{Returns}] \leavevmode
\sphinxAtStartPar
spectral kernel in erg/GHz

\end{description}\end{quote}

\end{fulllineitems}

\index{lMst() (in module diffsph.spectra.synchrotron)@\spxentry{lMst()}\spxextra{in module diffsph.spectra.synchrotron}}

\begin{fulllineitems}
\phantomsection\label{\detokenize{diffsph.spectra:diffsph.spectra.synchrotron.lMst}}\pysiglinewithargsret{\sphinxcode{\sphinxupquote{diffsph.spectra.synchrotron.}}\sphinxbfcode{\sphinxupquote{lMst}}}{\emph{\DUrole{n}{Lxi}}, \emph{\DUrole{n}{Leta}}, \emph{\DUrole{n}{delta}}}{}
\sphinxAtStartPar
Interpolation function for (kernel) \(\log(\hat{\mathcal M})\)
\begin{quote}\begin{description}
\item[{Parameters}] \leavevmode\begin{itemize}
\item {} 
\sphinxAtStartPar
\sphinxstyleliteralstrong{\sphinxupquote{Lxi}} \textendash{} \(\log(\xi)\)

\item {} 
\sphinxAtStartPar
\sphinxstyleliteralstrong{\sphinxupquote{Leta}} \textendash{} \(\log(\eta)\)

\item {} 
\sphinxAtStartPar
\sphinxstyleliteralstrong{\sphinxupquote{delta}} \textendash{} \(\delta\)

\end{itemize}

\item[{Returns}] \leavevmode
\sphinxAtStartPar
\(\log(\hat{\mathcal M})\) as a function of \(\log(\xi)\), \(\log(\eta)\) and \(\delta\)

\end{description}\end{quote}

\end{fulllineitems}

\index{lMst\_DM() (in module diffsph.spectra.synchrotron)@\spxentry{lMst\_DM()}\spxextra{in module diffsph.spectra.synchrotron}}

\begin{fulllineitems}
\phantomsection\label{\detokenize{diffsph.spectra:diffsph.spectra.synchrotron.lMst_DM}}\pysiglinewithargsret{\sphinxcode{\sphinxupquote{diffsph.spectra.synchrotron.}}\sphinxbfcode{\sphinxupquote{lMst\_DM}}}{\emph{\DUrole{n}{Lxi}}, \emph{\DUrole{n}{Leta}}, \emph{\DUrole{n}{Lm}}, \emph{\DUrole{n}{delta}}, \emph{\DUrole{n}{channel}}}{}
\sphinxAtStartPar
Interpolation function \(\log(\mathcal M)\) for DM hypotheses
\begin{quote}\begin{description}
\item[{Parameters}] \leavevmode\begin{itemize}
\item {} 
\sphinxAtStartPar
\sphinxstyleliteralstrong{\sphinxupquote{Lxi}} \textendash{} \(\log(\xi)\)

\item {} 
\sphinxAtStartPar
\sphinxstyleliteralstrong{\sphinxupquote{Leta}} \textendash{} \(\log(\eta)\)

\item {} 
\sphinxAtStartPar
\sphinxstyleliteralstrong{\sphinxupquote{Lm}} \textendash{} \(\log(m/\text{GeV})\) (\(m\) is the WIMP mass)

\item {} 
\sphinxAtStartPar
\sphinxstyleliteralstrong{\sphinxupquote{delta}} \textendash{} \(\delta\)

\item {} 
\sphinxAtStartPar
\sphinxstyleliteralstrong{\sphinxupquote{channel}} \textendash{} annihilation/decay channel

\end{itemize}

\item[{Returns}] \leavevmode
\sphinxAtStartPar
\(\log(\mathcal M)\) as a function of \(\log(\xi)\), \(\log(\eta)\), \(\log(m)\) and \(\delta\)

\end{description}\end{quote}

\end{fulllineitems}

\index{lMst\_pw() (in module diffsph.spectra.synchrotron)@\spxentry{lMst\_pw()}\spxextra{in module diffsph.spectra.synchrotron}}

\begin{fulllineitems}
\phantomsection\label{\detokenize{diffsph.spectra:diffsph.spectra.synchrotron.lMst_pw}}\pysiglinewithargsret{\sphinxcode{\sphinxupquote{diffsph.spectra.synchrotron.}}\sphinxbfcode{\sphinxupquote{lMst\_pw}}}{\emph{\DUrole{n}{Leta}}, \emph{\DUrole{n}{Gamma}}, \emph{\DUrole{n}{delta}}}{}
\sphinxAtStartPar
Interpolation function \(\log(\mathcal M)\) for the gereric power\sphinxhyphen{}law hypothesis
\begin{quote}\begin{description}
\item[{Parameters}] \leavevmode\begin{itemize}
\item {} 
\sphinxAtStartPar
\sphinxstyleliteralstrong{\sphinxupquote{Leta}} \textendash{} \(\log(\eta)\)

\item {} 
\sphinxAtStartPar
\sphinxstyleliteralstrong{\sphinxupquote{Gamma}} \textendash{} \(\Gamma\)

\item {} 
\sphinxAtStartPar
\sphinxstyleliteralstrong{\sphinxupquote{delta}} \textendash{} \(\delta\)

\end{itemize}

\item[{Returns}] \leavevmode
\sphinxAtStartPar
\(\log(\mathcal M_\text{gen})\) as a function of \(\log(\eta)\), \(\Gamma\) and \(\delta\)

\end{description}\end{quote}

\end{fulllineitems}



\paragraph{Module contents}
\label{\detokenize{diffsph.spectra:module-diffsph.spectra}}\label{\detokenize{diffsph.spectra:module-contents}}\index{module@\spxentry{module}!diffsph.spectra@\spxentry{diffsph.spectra}}\index{diffsph.spectra@\spxentry{diffsph.spectra}!module@\spxentry{module}}

\subsubsection{diffsph.utils package}
\label{\detokenize{diffsph.utils:diffsph-utils-package}}\label{\detokenize{diffsph.utils::doc}}

\paragraph{Submodules}
\label{\detokenize{diffsph.utils:submodules}}

\paragraph{diffsph.utils.consts module}
\label{\detokenize{diffsph.utils:module-diffsph.utils.consts}}\label{\detokenize{diffsph.utils:diffsph-utils-consts-module}}\index{module@\spxentry{module}!diffsph.utils.consts@\spxentry{diffsph.utils.consts}}\index{diffsph.utils.consts@\spxentry{diffsph.utils.consts}!module@\spxentry{module}}

\paragraph{diffsph.utils.dictionaries module}
\label{\detokenize{diffsph.utils:module-diffsph.utils.dictionaries}}\label{\detokenize{diffsph.utils:diffsph-utils-dictionaries-module}}\index{module@\spxentry{module}!diffsph.utils.dictionaries@\spxentry{diffsph.utils.dictionaries}}\index{diffsph.utils.dictionaries@\spxentry{diffsph.utils.dictionaries}!module@\spxentry{module}}

\paragraph{diffsph.utils.tools module}
\label{\detokenize{diffsph.utils:module-diffsph.utils.tools}}\label{\detokenize{diffsph.utils:diffsph-utils-tools-module}}\index{module@\spxentry{module}!diffsph.utils.tools@\spxentry{diffsph.utils.tools}}\index{diffsph.utils.tools@\spxentry{diffsph.utils.tools}!module@\spxentry{module}}\index{TB() (in module diffsph.utils.tools)@\spxentry{TB()}\spxextra{in module diffsph.utils.tools}}

\begin{fulllineitems}
\phantomsection\label{\detokenize{diffsph.utils:diffsph.utils.tools.TB}}\pysiglinewithargsret{\sphinxcode{\sphinxupquote{diffsph.utils.tools.}}\sphinxbfcode{\sphinxupquote{TB}}}{\emph{\DUrole{n}{brightness}}, \emph{\DUrole{n}{theta}}, \emph{\DUrole{n}{nu}}, \emph{\DUrole{o}{*}\DUrole{n}{args}}, \emph{\DUrole{o}{**}\DUrole{n}{kwargs}}}{}
\sphinxAtStartPar
Brightness temperature conversion
\begin{equation*}
\begin{split}T_B = \frac{c^2}{2\,k\,\nu^2}I_\nu\end{split}
\end{equation*}\begin{quote}\begin{description}
\item[{Parameters}] \leavevmode\begin{itemize}
\item {} 
\sphinxAtStartPar
\sphinxstyleliteralstrong{\sphinxupquote{brightness}} \textendash{} generic brightness function in Jy/sr

\item {} 
\sphinxAtStartPar
\sphinxstyleliteralstrong{\sphinxupquote{theta}} \textendash{} angular radius (as the first argument of the generic brighness function)

\item {} 
\sphinxAtStartPar
\sphinxstyleliteralstrong{\sphinxupquote{nu}} \textendash{} frequency (as the second argument of the generic brighness function)

\end{itemize}

\item[{Returns}] \leavevmode
\sphinxAtStartPar
brightness temperature in mK

\end{description}\end{quote}

\end{fulllineitems}

\index{approxhalo\_fd() (in module diffsph.utils.tools)@\spxentry{approxhalo\_fd()}\spxextra{in module diffsph.utils.tools}}

\begin{fulllineitems}
\phantomsection\label{\detokenize{diffsph.utils:diffsph.utils.tools.approxhalo_fd}}\pysiglinewithargsret{\sphinxcode{\sphinxupquote{diffsph.utils.tools.}}\sphinxbfcode{\sphinxupquote{approxhalo\_fd}}}{\emph{\DUrole{n}{n}}, \emph{\DUrole{n}{theta}}, \emph{\DUrole{n}{dist}}, \emph{\DUrole{n}{rh}}}{}
\sphinxAtStartPar
Partial (\(\theta\)\sphinxhyphen{}dependent) flux\sphinxhyphen{}density halo/bulge factor (approximate formula):
\begin{equation*}
\begin{split}\mathcal H_n(\theta) = \mathcal H_n(r_h,R) - 2\,\int_{R\sin(\theta)}^{r_h}dr\, r\, \kappa_1(r,R,\theta) \frac{\sin\left(\frac{n\pi r}{r_h}\right)}r\end{split}
\end{equation*}
\sphinxAtStartPar
where \(R\), \(rh\) and \(n\) are, respectively the distance, halo radius and Fourier index

\end{fulllineitems}

\index{approxhalo\_fd\_tot() (in module diffsph.utils.tools)@\spxentry{approxhalo\_fd\_tot()}\spxextra{in module diffsph.utils.tools}}

\begin{fulllineitems}
\phantomsection\label{\detokenize{diffsph.utils:diffsph.utils.tools.approxhalo_fd_tot}}\pysiglinewithargsret{\sphinxcode{\sphinxupquote{diffsph.utils.tools.}}\sphinxbfcode{\sphinxupquote{approxhalo\_fd\_tot}}}{\emph{\DUrole{n}{n}}, \emph{\DUrole{n}{dist}}, \emph{\DUrole{n}{rh}}}{}
\sphinxAtStartPar
Total flux\sphinxhyphen{}density halo/bulge factor (approximate formula):
\begin{equation*}
\begin{split}\mathcal H_n(r_h,R) \simeq 4\pi\int_0^{r_h}dr\, r^2 \frac{\sin\left(\frac{n\pi r}{r_h}\right)}r \ ,\end{split}
\end{equation*}
\sphinxAtStartPar
where \(R\), \(rh\) and \(n\) are, respectively the distance, halo radius and Fourier index

\end{fulllineitems}

\index{delta\_float() (in module diffsph.utils.tools)@\spxentry{delta\_float()}\spxextra{in module diffsph.utils.tools}}

\begin{fulllineitems}
\phantomsection\label{\detokenize{diffsph.utils:diffsph.utils.tools.delta_float}}\pysiglinewithargsret{\sphinxcode{\sphinxupquote{diffsph.utils.tools.}}\sphinxbfcode{\sphinxupquote{delta\_float}}}{\emph{\DUrole{n}{inp}}}{}
\sphinxAtStartPar
Float number for variable \sphinxcode{\sphinxupquote{\textquotesingle{}delta\textquotesingle{}}}
\begin{quote}\begin{description}
\item[{Parameters}] \leavevmode
\sphinxAtStartPar
\sphinxstyleliteralstrong{\sphinxupquote{inp}} \textendash{} variable \sphinxcode{\sphinxupquote{\textquotesingle{}delta\textquotesingle{}}} as \sphinxstyleemphasis{str} (\sphinxcode{\sphinxupquote{\textquotesingle{}kol\textquotesingle{}}}, \sphinxcode{\sphinxupquote{\textquotesingle{}kra\textquotesingle{}}}, etc.) or \sphinxstyleemphasis{float}

\item[{Returns}] \leavevmode
\sphinxAtStartPar
float number associated with \sphinxcode{\sphinxupquote{\textquotesingle{}inp\textquotesingle{}}}

\item[{Return type}] \leavevmode
\sphinxAtStartPar
float

\end{description}\end{quote}

\end{fulllineitems}

\index{df() (in module diffsph.utils.tools)@\spxentry{df()}\spxextra{in module diffsph.utils.tools}}

\begin{fulllineitems}
\phantomsection\label{\detokenize{diffsph.utils:diffsph.utils.tools.df}}\pysiglinewithargsret{\sphinxcode{\sphinxupquote{diffsph.utils.tools.}}\sphinxbfcode{\sphinxupquote{df}}}{\emph{\DUrole{n}{func}}, \emph{\DUrole{o}{**}\DUrole{n}{kwargs}}}{}
\end{fulllineitems}

\index{evaluate() (in module diffsph.utils.tools)@\spxentry{evaluate()}\spxextra{in module diffsph.utils.tools}}

\begin{fulllineitems}
\phantomsection\label{\detokenize{diffsph.utils:diffsph.utils.tools.evaluate}}\pysiglinewithargsret{\sphinxcode{\sphinxupquote{diffsph.utils.tools.}}\sphinxbfcode{\sphinxupquote{evaluate}}}{\emph{\DUrole{n}{f}}, \emph{\DUrole{n}{x}}, \emph{\DUrole{o}{**}\DUrole{n}{kwargs}}}{}
\sphinxAtStartPar
Function converts string into a python function’s name and evaluates it
\begin{quote}\begin{description}
\item[{Parameters}] \leavevmode\begin{itemize}
\item {} 
\sphinxAtStartPar
\sphinxstyleliteralstrong{\sphinxupquote{f}} \textendash{} function to be evaluated

\item {} 
\sphinxAtStartPar
\sphinxstyleliteralstrong{\sphinxupquote{x}} \textendash{} first argument of \(f\)

\end{itemize}

\item[{Returns}] \leavevmode
\sphinxAtStartPar
\(f(x)\)

\end{description}\end{quote}

\end{fulllineitems}

\index{f() (in module diffsph.utils.tools)@\spxentry{f()}\spxextra{in module diffsph.utils.tools}}

\begin{fulllineitems}
\phantomsection\label{\detokenize{diffsph.utils:diffsph.utils.tools.f}}\pysiglinewithargsret{\sphinxcode{\sphinxupquote{diffsph.utils.tools.}}\sphinxbfcode{\sphinxupquote{f}}}{\emph{\DUrole{n}{n}}, \emph{\DUrole{n}{x}}}{}
\sphinxAtStartPar
Basis function in Fourier\sphinxhyphen{}expanded brightness formula
\begin{equation*}
\begin{split}f_n(x)=2\int_x^1\frac{\sin(n\pi y) dy}{\sqrt{y^2-x^2}}\end{split}
\end{equation*}\begin{quote}\begin{description}
\item[{Returns}] \leavevmode
\sphinxAtStartPar
\(f_n\) as a function of \(x\)

\end{description}\end{quote}

\end{fulllineitems}

\index{fwhm() (in module diffsph.utils.tools)@\spxentry{fwhm()}\spxextra{in module diffsph.utils.tools}}

\begin{fulllineitems}
\phantomsection\label{\detokenize{diffsph.utils:diffsph.utils.tools.fwhm}}\pysiglinewithargsret{\sphinxcode{\sphinxupquote{diffsph.utils.tools.}}\sphinxbfcode{\sphinxupquote{fwhm}}}{\emph{\DUrole{n}{brightness}}, \emph{\DUrole{n}{thmax}}, \emph{\DUrole{o}{*}\DUrole{n}{args}}, \emph{\DUrole{o}{**}\DUrole{n}{kwargs}}}{}
\sphinxAtStartPar
Full width at half maximum
\begin{quote}\begin{description}
\item[{Parameters}] \leavevmode\begin{itemize}
\item {} 
\sphinxAtStartPar
\sphinxstyleliteralstrong{\sphinxupquote{brightness}} \textendash{} generic brightness function

\item {} 
\sphinxAtStartPar
\sphinxstyleliteralstrong{\sphinxupquote{thmax}} \textendash{} signal’s angular radius

\end{itemize}

\item[{Returns}] \leavevmode
\sphinxAtStartPar
Full width at half maximum in arcmin

\end{description}\end{quote}

\end{fulllineitems}

\index{g() (in module diffsph.utils.tools)@\spxentry{g()}\spxextra{in module diffsph.utils.tools}}

\begin{fulllineitems}
\phantomsection\label{\detokenize{diffsph.utils:diffsph.utils.tools.g}}\pysiglinewithargsret{\sphinxcode{\sphinxupquote{diffsph.utils.tools.}}\sphinxbfcode{\sphinxupquote{g}}}{\emph{\DUrole{n}{n}}, \emph{\DUrole{n}{x}}}{}
\sphinxAtStartPar
Basis function in Fourier\sphinxhyphen{}expanded flux density formula
\begin{equation*}
\begin{split}g_n(x)=2\int_x^1\sqrt{y^2-x^2}\sin(n\pi y) dy\end{split}
\end{equation*}\begin{quote}\begin{description}
\item[{Returns}] \leavevmode
\sphinxAtStartPar
\(g_n\) as a function of \(x\)

\end{description}\end{quote}

\end{fulllineitems}

\index{halo\_fd() (in module diffsph.utils.tools)@\spxentry{halo\_fd()}\spxextra{in module diffsph.utils.tools}}

\begin{fulllineitems}
\phantomsection\label{\detokenize{diffsph.utils:diffsph.utils.tools.halo_fd}}\pysiglinewithargsret{\sphinxcode{\sphinxupquote{diffsph.utils.tools.}}\sphinxbfcode{\sphinxupquote{halo\_fd}}}{\emph{\DUrole{n}{n}}, \emph{\DUrole{n}{theta}}, \emph{\DUrole{n}{dist}}, \emph{\DUrole{n}{rh}}}{}
\sphinxAtStartPar
Partial (\(\theta\)\sphinxhyphen{}dependent) flux\sphinxhyphen{}density halo/bulge factor:
\begin{equation*}
\begin{split}\mathcal H_n(\theta) = \mathcal H_n(r_h,R) - 2\,\int_{R\sin(\theta)}^{r_h}dr\, r\, \kappa_1(r,R,\theta) \frac{\sin\left(\frac{n\pi r}{r_h}\right)}r \ ,\end{split}
\end{equation*}
\sphinxAtStartPar
where \(R\), \(rh\) and \(n\) are, respectively the distance, halo radius and Fourier index

\end{fulllineitems}

\index{halo\_fd\_tot() (in module diffsph.utils.tools)@\spxentry{halo\_fd\_tot()}\spxextra{in module diffsph.utils.tools}}

\begin{fulllineitems}
\phantomsection\label{\detokenize{diffsph.utils:diffsph.utils.tools.halo_fd_tot}}\pysiglinewithargsret{\sphinxcode{\sphinxupquote{diffsph.utils.tools.}}\sphinxbfcode{\sphinxupquote{halo\_fd\_tot}}}{\emph{\DUrole{n}{n}}, \emph{\DUrole{n}{dist}}, \emph{\DUrole{n}{rh}}}{}
\sphinxAtStartPar
Total flux\sphinxhyphen{}density halo/bulge factor:
\begin{equation*}
\begin{split}\mathcal H_n(r_h,R) = 2\,\int_0^{r_h}dr\, r\, \kappa_0(r,R) \frac{\sin\left(\frac{n\pi r}{r_h}\right)}r\ , \end{split}
\end{equation*}
\sphinxAtStartPar
where \(R\), \(rh\) and \(n\) are, respectively the distance, halo radius and Fourier index
\begin{quote}\begin{description}
\item[{Returns}] \leavevmode
\sphinxAtStartPar
Halo flux\sphinxhyphen{}density factor

\end{description}\end{quote}

\end{fulllineitems}

\index{hfd() (in module diffsph.utils.tools)@\spxentry{hfd()}\spxextra{in module diffsph.utils.tools}}

\begin{fulllineitems}
\phantomsection\label{\detokenize{diffsph.utils:diffsph.utils.tools.hfd}}\pysiglinewithargsret{\sphinxcode{\sphinxupquote{diffsph.utils.tools.}}\sphinxbfcode{\sphinxupquote{hfd}}}{\emph{\DUrole{n}{fluxdens}}, \emph{\DUrole{n}{thmax}}, \emph{\DUrole{o}{*}\DUrole{n}{args}}, \emph{\DUrole{o}{**}\DUrole{n}{kwargs}}}{}
\sphinxAtStartPar
Half\sphinxhyphen{}flux diameter
\begin{quote}\begin{description}
\item[{Parameters}] \leavevmode\begin{itemize}
\item {} 
\sphinxAtStartPar
\sphinxstyleliteralstrong{\sphinxupquote{brightness}} \textendash{} generic brightness function

\item {} 
\sphinxAtStartPar
\sphinxstyleliteralstrong{\sphinxupquote{thmax}} \textendash{} signal’s angular radius

\end{itemize}

\item[{Returns}] \leavevmode
\sphinxAtStartPar
Half\sphinxhyphen{}flux diameter in arcmin

\end{description}\end{quote}

\end{fulllineitems}

\index{hypothesis\_index() (in module diffsph.utils.tools)@\spxentry{hypothesis\_index()}\spxextra{in module diffsph.utils.tools}}

\begin{fulllineitems}
\phantomsection\label{\detokenize{diffsph.utils:diffsph.utils.tools.hypothesis_index}}\pysiglinewithargsret{\sphinxcode{\sphinxupquote{diffsph.utils.tools.}}\sphinxbfcode{\sphinxupquote{hypothesis\_index}}}{\emph{\DUrole{n}{hyp}}}{}
\sphinxAtStartPar
Index of the hypothesis (1 for decaying DM or generic scenario, 2 for WIMP self\sphinxhyphen{}annihilation).
\begin{quote}\begin{description}
\item[{Parameters}] \leavevmode
\sphinxAtStartPar
\sphinxstyleliteralstrong{\sphinxupquote{hyp}} (\sphinxstyleliteralemphasis{\sphinxupquote{str}}) \textendash{} hypothesis: \sphinxcode{\sphinxupquote{\textquotesingle{}wimp\textquotesingle{}}}, \sphinxcode{\sphinxupquote{\textquotesingle{}decay\textquotesingle{}}} or \sphinxcode{\sphinxupquote{\textquotesingle{}generic\textquotesingle{}}})

\item[{Returns}] \leavevmode
\sphinxAtStartPar
hypothesis index

\item[{Return type}] \leavevmode
\sphinxAtStartPar
int

\end{description}\end{quote}

\end{fulllineitems}

\index{ker\_0() (in module diffsph.utils.tools)@\spxentry{ker\_0()}\spxextra{in module diffsph.utils.tools}}

\begin{fulllineitems}
\phantomsection\label{\detokenize{diffsph.utils:diffsph.utils.tools.ker_0}}\pysiglinewithargsret{\sphinxcode{\sphinxupquote{diffsph.utils.tools.}}\sphinxbfcode{\sphinxupquote{ker\_0}}}{\emph{\DUrole{n}{r}}, \emph{\DUrole{n}{dist}}}{}~\begin{equation*}
\begin{split}\kappa_0(r,R) = \frac1{R}\log\sqrt{\frac{R+r}{R-r}}\end{split}
\end{equation*}
\end{fulllineitems}

\index{ker\_1() (in module diffsph.utils.tools)@\spxentry{ker\_1()}\spxextra{in module diffsph.utils.tools}}

\begin{fulllineitems}
\phantomsection\label{\detokenize{diffsph.utils:diffsph.utils.tools.ker_1}}\pysiglinewithargsret{\sphinxcode{\sphinxupquote{diffsph.utils.tools.}}\sphinxbfcode{\sphinxupquote{ker\_1}}}{\emph{\DUrole{n}{r}}, \emph{\DUrole{n}{theta}}, \emph{\DUrole{n}{dist}}}{}~\begin{equation*}
\begin{split}\kappa_1(\theta,r,R) = \frac1{R}\log\frac{R\cos\theta+\sqrt{r^2-R^2\sin^2\theta}}{\sqrt{R^2-r^2}}\end{split}
\end{equation*}
\end{fulllineitems}

\index{load\_data() (in module diffsph.utils.tools)@\spxentry{load\_data()}\spxextra{in module diffsph.utils.tools}}

\begin{fulllineitems}
\phantomsection\label{\detokenize{diffsph.utils:diffsph.utils.tools.load_data}}\pysiglinewithargsret{\sphinxcode{\sphinxupquote{diffsph.utils.tools.}}\sphinxbfcode{\sphinxupquote{load\_data}}}{\emph{\DUrole{n}{folder}}}{}
\sphinxAtStartPar
Function loads data from folder

\end{fulllineitems}

\index{sort\_kwargs() (in module diffsph.utils.tools)@\spxentry{sort\_kwargs()}\spxextra{in module diffsph.utils.tools}}

\begin{fulllineitems}
\phantomsection\label{\detokenize{diffsph.utils:diffsph.utils.tools.sort_kwargs}}\pysiglinewithargsret{\sphinxcode{\sphinxupquote{diffsph.utils.tools.}}\sphinxbfcode{\sphinxupquote{sort\_kwargs}}}{\emph{\DUrole{o}{**}\DUrole{n}{kwargs}}}{}
\sphinxAtStartPar
Function sorts keyword arguments alphabetically

\end{fulllineitems}

\index{var\_to\_str() (in module diffsph.utils.tools)@\spxentry{var\_to\_str()}\spxextra{in module diffsph.utils.tools}}

\begin{fulllineitems}
\phantomsection\label{\detokenize{diffsph.utils:diffsph.utils.tools.var_to_str}}\pysiglinewithargsret{\sphinxcode{\sphinxupquote{diffsph.utils.tools.}}\sphinxbfcode{\sphinxupquote{var\_to\_str}}}{\emph{\DUrole{n}{inp}}}{}
\sphinxAtStartPar
Dictionary for variables \sphinxcode{\sphinxupquote{\textquotesingle{}delta\textquotesingle{}}}, \sphinxcode{\sphinxupquote{\textquotesingle{}hyp\textquotesingle{}}}, \sphinxcode{\sphinxupquote{\textquotesingle{}galaxy\textquotesingle{}}}, \sphinxcode{\sphinxupquote{\textquotesingle{}ref\textquotesingle{}}} and \sphinxcode{\sphinxupquote{\textquotesingle{}rad\_temp\textquotesingle{}}}
\begin{quote}\begin{description}
\item[{Parameters}] \leavevmode
\sphinxAtStartPar
\sphinxstyleliteralstrong{\sphinxupquote{inp}} \textendash{} input string or number

\item[{Returns}] \leavevmode
\sphinxAtStartPar
default variable name

\item[{Return type}] \leavevmode
\sphinxAtStartPar
str

\end{description}\end{quote}

\end{fulllineitems}



\paragraph{Module contents}
\label{\detokenize{diffsph.utils:module-diffsph.utils}}\label{\detokenize{diffsph.utils:module-contents}}\index{module@\spxentry{module}!diffsph.utils@\spxentry{diffsph.utils}}\index{diffsph.utils@\spxentry{diffsph.utils}!module@\spxentry{module}}

\subsection{Submodules}
\label{\detokenize{diffsph:submodules}}

\subsection{diffsph.limits module}
\label{\detokenize{diffsph:module-diffsph.limits}}\label{\detokenize{diffsph:diffsph-limits-module}}\index{module@\spxentry{module}!diffsph.limits@\spxentry{diffsph.limits}}\index{diffsph.limits@\spxentry{diffsph.limits}!module@\spxentry{module}}\index{decay\_rate\_gausslim() (in module diffsph.limits)@\spxentry{decay\_rate\_gausslim()}\spxextra{in module diffsph.limits}}

\begin{fulllineitems}
\phantomsection\label{\detokenize{diffsph:diffsph.limits.decay_rate_gausslim}}\pysiglinewithargsret{\sphinxcode{\sphinxupquote{diffsph.limits.}}\sphinxbfcode{\sphinxupquote{decay\_rate\_gausslim}}}{\emph{\DUrole{n}{nu}}, \emph{\DUrole{n}{a\_fit}}, \emph{\DUrole{n}{sigma\_fit}}, \emph{\DUrole{n}{beam\_size}}, \emph{\DUrole{n}{galaxy}}, \emph{\DUrole{n}{rad\_temp}}, \emph{\DUrole{n}{D0}\DUrole{o}{=}\DUrole{default_value}{3e+28}}, \emph{\DUrole{n}{delta}\DUrole{o}{=}\DUrole{default_value}{\textquotesingle{}kol\textquotesingle{}}}, \emph{\DUrole{n}{B}\DUrole{o}{=}\DUrole{default_value}{2}}, \emph{\DUrole{n}{mchi}\DUrole{o}{=}\DUrole{default_value}{50}}, \emph{\DUrole{n}{channel}\DUrole{o}{=}\DUrole{default_value}{\textquotesingle{}mumu\textquotesingle{}}}, \emph{\DUrole{n}{manual}\DUrole{o}{=}\DUrole{default_value}{False}}, \emph{\DUrole{o}{**}\DUrole{n}{kwargs}}}{}
\sphinxAtStartPar
Maximum dark matter decay rate allowed by the exclusion of a Gaussian\sphinxhyphen{}shaped signal
\begin{equation*}
\begin{split}a_\text{fit}\exp\left(-\frac{\theta^2}{2\sigma_\text{fit}^2}\right)\end{split}
\end{equation*}\begin{quote}\begin{description}
\item[{Parameters}] \leavevmode\begin{itemize}
\item {} 
\sphinxAtStartPar
\sphinxstyleliteralstrong{\sphinxupquote{nu}} \textendash{} frequency in GHz

\item {} 
\sphinxAtStartPar
\sphinxstyleliteralstrong{\sphinxupquote{a\_fit}} \textendash{} fitted gaussian amplitude in \(\mu\) Jy / beam

\item {} 
\sphinxAtStartPar
\sphinxstyleliteralstrong{\sphinxupquote{sigma\_fit}} \textendash{} width parameter of the Gaussian template in arcmin

\item {} 
\sphinxAtStartPar
\sphinxstyleliteralstrong{\sphinxupquote{galaxy}} (\sphinxstyleliteralemphasis{\sphinxupquote{str}}) \textendash{} name of the galaxy

\item {} 
\sphinxAtStartPar
\sphinxstyleliteralstrong{\sphinxupquote{rad\_temp}} (\sphinxstyleliteralemphasis{\sphinxupquote{str}}) \textendash{} dark matter halo model (\sphinxcode{\sphinxupquote{\textquotesingle{}NFW\textquotesingle{}}}, \sphinxcode{\sphinxupquote{\textquotesingle{}Einasto\textquotesingle{}}}, etc.)

\item {} 
\sphinxAtStartPar
\sphinxstyleliteralstrong{\sphinxupquote{D0}} \textendash{} magnitude of the diffusion coefficient for a 1 GeV CRE in cm \({}^2\)/s (default value = \(3\times 10^{28}\) cm \({}^2\) /s)

\item {} 
\sphinxAtStartPar
\sphinxstyleliteralstrong{\sphinxupquote{delta}} (\sphinxstyleliteralemphasis{\sphinxupquote{float}}\sphinxstyleliteralemphasis{\sphinxupquote{, }}\sphinxstyleliteralemphasis{\sphinxupquote{str}}) \textendash{} power\sphinxhyphen{}law exponent of the diffusion coefficient as a function of the CRE’s energy (default value = 1/3 or \sphinxcode{\sphinxupquote{\textquotesingle{}kol\textquotesingle{}}})

\item {} 
\sphinxAtStartPar
\sphinxstyleliteralstrong{\sphinxupquote{B}} \textendash{} magnitude of the magnetic field’s smooth component in \(\mu\) G (default value \(= 2 \mu\) G)

\item {} 
\sphinxAtStartPar
\sphinxstyleliteralstrong{\sphinxupquote{mchi}} \textendash{} mass of the DM particle in GeV/c \({}^2\)

\item {} 
\sphinxAtStartPar
\sphinxstyleliteralstrong{\sphinxupquote{channel}} (\sphinxstyleliteralemphasis{\sphinxupquote{str}}) \textendash{} decay channel: \(b\bar b\) (\sphinxcode{\sphinxupquote{\textquotesingle{}bb\textquotesingle{}}}), \(\mu^+ \mu^-\) (\sphinxcode{\sphinxupquote{\textquotesingle{}mumu\textquotesingle{}}}), \(W^+ W^-\) (\sphinxcode{\sphinxupquote{\textquotesingle{}WW\textquotesingle{}}}), etc.

\item {} 
\sphinxAtStartPar
\sphinxstyleliteralstrong{\sphinxupquote{manual}} (\sphinxstyleliteralemphasis{\sphinxupquote{bool}}) \textendash{} manual input of parameter values in rad\_temp (default value = \sphinxcode{\sphinxupquote{\textquotesingle{}False\textquotesingle{}}})

\end{itemize}

\item[{Beam\_size}] \leavevmode
\sphinxAtStartPar
beam size in arcseconds

\end{description}\end{quote}

\sphinxAtStartPar
Keyword arguments
\begin{itemize}
\item {} 
\sphinxAtStartPar
\sphinxcode{\sphinxupquote{manual = \textquotesingle{}False\textquotesingle{}}}

\end{itemize}
\begin{quote}\begin{description}
\item[{Parameters}] \leavevmode
\sphinxAtStartPar
\sphinxstyleliteralstrong{\sphinxupquote{ref}} \textendash{} reference used (\sphinxcode{\sphinxupquote{\textquotesingle{}Martinez\textquotesingle{}}} or \sphinxcode{\sphinxupquote{\textquotesingle{}1309.2641\textquotesingle{}}}, \sphinxcode{\sphinxupquote{\textquotesingle{}Geringer\sphinxhyphen{}Sameth\textquotesingle{}}} or \sphinxcode{\sphinxupquote{\textquotesingle{}1408.0002\textquotesingle{}}}, etc.)

\end{description}\end{quote}
\begin{itemize}
\item {} 
\sphinxAtStartPar
\sphinxcode{\sphinxupquote{manual = \textquotesingle{}True\textquotesingle{}}}

\end{itemize}
\begin{quote}\begin{description}
\item[{Parameters}] \leavevmode\begin{itemize}
\item {} 
\sphinxAtStartPar
\sphinxstyleliteralstrong{\sphinxupquote{rs}} \textendash{} scale radius in kpc

\item {} 
\sphinxAtStartPar
\sphinxstyleliteralstrong{\sphinxupquote{rhos}} \textendash{} characteristic density in GeV/cm \({}^3\)

\item {} 
\sphinxAtStartPar
\sphinxstyleliteralstrong{\sphinxupquote{alpha}} \textendash{} exponent \(\alpha\) in the {\hyperref[\detokenize{diffsph.profiles:diffsph.profiles.templates.hdz}]{\sphinxcrossref{\sphinxcode{\sphinxupquote{diffsph.profiles.templates.hdz()}}}}} profile

\item {} 
\sphinxAtStartPar
\sphinxstyleliteralstrong{\sphinxupquote{beta}} \textendash{} exponent \(\beta\) in the {\hyperref[\detokenize{diffsph.profiles:diffsph.profiles.templates.hdz}]{\sphinxcrossref{\sphinxcode{\sphinxupquote{diffsph.profiles.templates.hdz()}}}}} profile

\item {} 
\sphinxAtStartPar
\sphinxstyleliteralstrong{\sphinxupquote{gamma}} \textendash{} exponent \(\gamma\) in the {\hyperref[\detokenize{diffsph.profiles:diffsph.profiles.templates.hdz}]{\sphinxcrossref{\sphinxcode{\sphinxupquote{diffsph.profiles.templates.hdz()}}}}} profile

\item {} 
\sphinxAtStartPar
\sphinxstyleliteralstrong{\sphinxupquote{alphaE}} \textendash{} parameter \(\alpha_E\) in the {\hyperref[\detokenize{diffsph.profiles:diffsph.profiles.templates.enst}]{\sphinxcrossref{\sphinxcode{\sphinxupquote{diffsph.profiles.templates.enst()}}}}} profile

\item {} 
\sphinxAtStartPar
\sphinxstyleliteralstrong{\sphinxupquote{sigmav}} \textendash{} velocity dispersion in km/s for the isothermal sphere {\hyperref[\detokenize{diffsph.profiles:diffsph.profiles.templates.sis}]{\sphinxcrossref{\sphinxcode{\sphinxupquote{diffsph.profiles.templates.sis()}}}}}

\end{itemize}

\item[{Returns}] \leavevmode
\sphinxAtStartPar
upper limit on the DM decay rate in 1/s

\item[{Return type}] \leavevmode
\sphinxAtStartPar
float

\end{description}\end{quote}

\end{fulllineitems}

\index{decay\_rate\_limest() (in module diffsph.limits)@\spxentry{decay\_rate\_limest()}\spxextra{in module diffsph.limits}}

\begin{fulllineitems}
\phantomsection\label{\detokenize{diffsph:diffsph.limits.decay_rate_limest}}\pysiglinewithargsret{\sphinxcode{\sphinxupquote{diffsph.limits.}}\sphinxbfcode{\sphinxupquote{decay\_rate\_limest}}}{\emph{\DUrole{n}{nu}}, \emph{\DUrole{n}{rms\_noise}}, \emph{\DUrole{n}{beam\_size}}, \emph{\DUrole{n}{galaxy}}, \emph{\DUrole{n}{rad\_temp}}, \emph{\DUrole{n}{ratio}\DUrole{o}{=}\DUrole{default_value}{1}}, \emph{\DUrole{n}{D0}\DUrole{o}{=}\DUrole{default_value}{3e+28}}, \emph{\DUrole{n}{delta}\DUrole{o}{=}\DUrole{default_value}{\textquotesingle{}kol\textquotesingle{}}}, \emph{\DUrole{n}{B}\DUrole{o}{=}\DUrole{default_value}{2}}, \emph{\DUrole{n}{mchi}\DUrole{o}{=}\DUrole{default_value}{50}}, \emph{\DUrole{n}{channel}\DUrole{o}{=}\DUrole{default_value}{\textquotesingle{}mumu\textquotesingle{}}}, \emph{\DUrole{n}{manual}\DUrole{o}{=}\DUrole{default_value}{False}}, \emph{\DUrole{n}{high\_res}\DUrole{o}{=}\DUrole{default_value}{False}}, \emph{\DUrole{n}{accuracy}\DUrole{o}{=}\DUrole{default_value}{1}}, \emph{\DUrole{o}{**}\DUrole{n}{kwargs}}}{}
\sphinxAtStartPar
(Estimated) maximum dark matter decay rate given the rms noise level of an observation
\begin{quote}\begin{description}
\item[{Parameters}] \leavevmode\begin{itemize}
\item {} 
\sphinxAtStartPar
\sphinxstyleliteralstrong{\sphinxupquote{nu}} \textendash{} frequency in GHz

\item {} 
\sphinxAtStartPar
\sphinxstyleliteralstrong{\sphinxupquote{rms\_noise}} \textendash{} RMS noise level of the observation in \(\mu\) Jy / beam

\item {} 
\sphinxAtStartPar
\sphinxstyleliteralstrong{\sphinxupquote{galaxy}} (\sphinxstyleliteralemphasis{\sphinxupquote{str}}) \textendash{} name of the galaxy

\item {} 
\sphinxAtStartPar
\sphinxstyleliteralstrong{\sphinxupquote{rad\_temp}} (\sphinxstyleliteralemphasis{\sphinxupquote{str}}) \textendash{} dark matter halo model (\sphinxcode{\sphinxupquote{\textquotesingle{}NFW\textquotesingle{}}}, \sphinxcode{\sphinxupquote{\textquotesingle{}Einasto\textquotesingle{}}}, etc.)

\item {} 
\sphinxAtStartPar
\sphinxstyleliteralstrong{\sphinxupquote{ratio}} \textendash{} ratio between the diffusion halo and half\sphinxhyphen{}light radii

\item {} 
\sphinxAtStartPar
\sphinxstyleliteralstrong{\sphinxupquote{D0}} \textendash{} magnitude of the diffusion coefficient for a 1 GeV CRE in cm \({}^2\)/s (default value = \(3\times 10^{28}\) cm \({}^2\) /s)

\item {} 
\sphinxAtStartPar
\sphinxstyleliteralstrong{\sphinxupquote{delta}} (\sphinxstyleliteralemphasis{\sphinxupquote{float}}\sphinxstyleliteralemphasis{\sphinxupquote{, }}\sphinxstyleliteralemphasis{\sphinxupquote{str}}) \textendash{} power\sphinxhyphen{}law exponent of the diffusion coefficient as a function of the CRE’s energy (default value = 1/3 or \sphinxcode{\sphinxupquote{\textquotesingle{}kol\textquotesingle{}}})

\item {} 
\sphinxAtStartPar
\sphinxstyleliteralstrong{\sphinxupquote{B}} \textendash{} magnitude of the magnetic field’s smooth component in \(\mu\) G (default value \(= 2 \mu\) G)

\item {} 
\sphinxAtStartPar
\sphinxstyleliteralstrong{\sphinxupquote{mchi}} \textendash{} mass of the DM particle in GeV/c \({}^2\)

\item {} 
\sphinxAtStartPar
\sphinxstyleliteralstrong{\sphinxupquote{channel}} (\sphinxstyleliteralemphasis{\sphinxupquote{str}}) \textendash{} decay channel: \(b\bar b\) (\sphinxcode{\sphinxupquote{\textquotesingle{}bb\textquotesingle{}}}), \(\mu^+ \mu^-\) (\sphinxcode{\sphinxupquote{\textquotesingle{}mumu\textquotesingle{}}}), \(W^+ W^-\) (\sphinxcode{\sphinxupquote{\textquotesingle{}WW\textquotesingle{}}}), etc.

\item {} 
\sphinxAtStartPar
\sphinxstyleliteralstrong{\sphinxupquote{manual}} (\sphinxstyleliteralemphasis{\sphinxupquote{bool}}) \textendash{} manual input of parameter values in rad\_temp (default value = \sphinxcode{\sphinxupquote{\textquotesingle{}False\textquotesingle{}}})

\item {} 
\sphinxAtStartPar
\sphinxstyleliteralstrong{\sphinxupquote{high\_res}} (\sphinxstyleliteralemphasis{\sphinxupquote{bool}}) \textendash{} spatial resolution. If \sphinxcode{\sphinxupquote{\textquotesingle{}True\textquotesingle{}}}, \sphinxcode{\sphinxupquote{synch\_emissivity()}} computes as many terms as needed in order to converge at \(r=0\). (default value = \sphinxcode{\sphinxupquote{\textquotesingle{}False\textquotesingle{}}})

\item {} 
\sphinxAtStartPar
\sphinxstyleliteralstrong{\sphinxupquote{accuracy}} \textendash{} theoretical accuracy in \% (default value = 1\%)

\end{itemize}

\item[{Beam\_size}] \leavevmode
\sphinxAtStartPar
beam size in arcseconds

\end{description}\end{quote}

\sphinxAtStartPar
Keyword arguments
\begin{itemize}
\item {} 
\sphinxAtStartPar
\sphinxcode{\sphinxupquote{manual = \textquotesingle{}False\textquotesingle{}}}

\end{itemize}
\begin{quote}\begin{description}
\item[{Parameters}] \leavevmode
\sphinxAtStartPar
\sphinxstyleliteralstrong{\sphinxupquote{ref}} \textendash{} reference used (\sphinxcode{\sphinxupquote{\textquotesingle{}Martinez\textquotesingle{}}} or \sphinxcode{\sphinxupquote{\textquotesingle{}1309.2641\textquotesingle{}}}, \sphinxcode{\sphinxupquote{\textquotesingle{}Geringer\sphinxhyphen{}Sameth\textquotesingle{}}} or \sphinxcode{\sphinxupquote{\textquotesingle{}1408.0002\textquotesingle{}}}, etc.)

\end{description}\end{quote}
\begin{itemize}
\item {} 
\sphinxAtStartPar
\sphinxcode{\sphinxupquote{manual = \textquotesingle{}True\textquotesingle{}}}

\end{itemize}
\begin{quote}\begin{description}
\item[{Parameters}] \leavevmode\begin{itemize}
\item {} 
\sphinxAtStartPar
\sphinxstyleliteralstrong{\sphinxupquote{rs}} \textendash{} scale radius in kpc

\item {} 
\sphinxAtStartPar
\sphinxstyleliteralstrong{\sphinxupquote{rhos}} \textendash{} characteristic density in GeV/cm \({}^3\)

\item {} 
\sphinxAtStartPar
\sphinxstyleliteralstrong{\sphinxupquote{alpha}} \textendash{} exponent \(\alpha\) in the {\hyperref[\detokenize{diffsph.profiles:diffsph.profiles.templates.hdz}]{\sphinxcrossref{\sphinxcode{\sphinxupquote{diffsph.profiles.templates.hdz()}}}}} profile

\item {} 
\sphinxAtStartPar
\sphinxstyleliteralstrong{\sphinxupquote{beta}} \textendash{} exponent \(\beta\) in the {\hyperref[\detokenize{diffsph.profiles:diffsph.profiles.templates.hdz}]{\sphinxcrossref{\sphinxcode{\sphinxupquote{diffsph.profiles.templates.hdz()}}}}} profile

\item {} 
\sphinxAtStartPar
\sphinxstyleliteralstrong{\sphinxupquote{gamma}} \textendash{} exponent \(\gamma\) in the {\hyperref[\detokenize{diffsph.profiles:diffsph.profiles.templates.hdz}]{\sphinxcrossref{\sphinxcode{\sphinxupquote{diffsph.profiles.templates.hdz()}}}}} profile

\item {} 
\sphinxAtStartPar
\sphinxstyleliteralstrong{\sphinxupquote{alphaE}} \textendash{} parameter \(\alpha_E\) in the {\hyperref[\detokenize{diffsph.profiles:diffsph.profiles.templates.enst}]{\sphinxcrossref{\sphinxcode{\sphinxupquote{diffsph.profiles.templates.enst()}}}}} profile

\item {} 
\sphinxAtStartPar
\sphinxstyleliteralstrong{\sphinxupquote{sigmav}} \textendash{} velocity dispersion in km/s for the isothermal sphere {\hyperref[\detokenize{diffsph.profiles:diffsph.profiles.templates.sis}]{\sphinxcrossref{\sphinxcode{\sphinxupquote{diffsph.profiles.templates.sis()}}}}}

\end{itemize}

\item[{Returns}] \leavevmode
\sphinxAtStartPar
Estimated upper limit on the DM decay rate in 1/s

\item[{Return type}] \leavevmode
\sphinxAtStartPar
float

\end{description}\end{quote}

\end{fulllineitems}

\index{generic\_rate\_gausslim() (in module diffsph.limits)@\spxentry{generic\_rate\_gausslim()}\spxextra{in module diffsph.limits}}

\begin{fulllineitems}
\phantomsection\label{\detokenize{diffsph:diffsph.limits.generic_rate_gausslim}}\pysiglinewithargsret{\sphinxcode{\sphinxupquote{diffsph.limits.}}\sphinxbfcode{\sphinxupquote{generic\_rate\_gausslim}}}{\emph{\DUrole{n}{nu}}, \emph{\DUrole{n}{a\_fit}}, \emph{\DUrole{n}{sigma\_fit}}, \emph{\DUrole{n}{beam\_size}}, \emph{\DUrole{n}{galaxy}}, \emph{\DUrole{n}{rad\_temp}}, \emph{\DUrole{n}{D0}\DUrole{o}{=}\DUrole{default_value}{3e+28}}, \emph{\DUrole{n}{delta}\DUrole{o}{=}\DUrole{default_value}{\textquotesingle{}kol\textquotesingle{}}}, \emph{\DUrole{n}{B}\DUrole{o}{=}\DUrole{default_value}{2}}, \emph{\DUrole{n}{Gamma}\DUrole{o}{=}\DUrole{default_value}{2}}, \emph{\DUrole{o}{**}\DUrole{n}{kwargs}}}{}
\sphinxAtStartPar
Maximum CRE production rate (generic power\sphinxhyphen{}law hypothesis) allowed by the exclusion of a Gaussian\sphinxhyphen{}shaped signal
\begin{equation*}
\begin{split}a_\text{fit}\exp\left(-\frac{\theta^2}{2\sigma_\text{fit}^2}\right)\end{split}
\end{equation*}\begin{quote}\begin{description}
\item[{Parameters}] \leavevmode\begin{itemize}
\item {} 
\sphinxAtStartPar
\sphinxstyleliteralstrong{\sphinxupquote{nu}} \textendash{} frequency in GHz

\item {} 
\sphinxAtStartPar
\sphinxstyleliteralstrong{\sphinxupquote{a\_fit}} \textendash{} fitted gaussian amplitude in \(\mu\) Jy / beam

\item {} 
\sphinxAtStartPar
\sphinxstyleliteralstrong{\sphinxupquote{sigma\_fit}} \textendash{} width parameter of the Gaussian template in arcmin

\item {} 
\sphinxAtStartPar
\sphinxstyleliteralstrong{\sphinxupquote{galaxy}} (\sphinxstyleliteralemphasis{\sphinxupquote{str}}) \textendash{} name of the galaxy

\item {} 
\sphinxAtStartPar
\sphinxstyleliteralstrong{\sphinxupquote{rad\_temp}} (\sphinxstyleliteralemphasis{\sphinxupquote{str}}) \textendash{} dark matter halo model (\sphinxcode{\sphinxupquote{\textquotesingle{}NFW\textquotesingle{}}}, \sphinxcode{\sphinxupquote{\textquotesingle{}Einasto\textquotesingle{}}}, etc.)

\item {} 
\sphinxAtStartPar
\sphinxstyleliteralstrong{\sphinxupquote{D0}} \textendash{} magnitude of the diffusion coefficient for a 1 GeV CRE in cm \({}^2\)/s (default value = \(3\times 10^{28}\) cm \({}^2\) /s)

\item {} 
\sphinxAtStartPar
\sphinxstyleliteralstrong{\sphinxupquote{delta}} (\sphinxstyleliteralemphasis{\sphinxupquote{float}}\sphinxstyleliteralemphasis{\sphinxupquote{, }}\sphinxstyleliteralemphasis{\sphinxupquote{str}}) \textendash{} power\sphinxhyphen{}law exponent of the diffusion coefficient as a function of the CRE’s energy (default value = 1/3 or \sphinxcode{\sphinxupquote{\textquotesingle{}kol\textquotesingle{}}})

\item {} 
\sphinxAtStartPar
\sphinxstyleliteralstrong{\sphinxupquote{B}} \textendash{} magnitude of the magnetic field’s smooth component in \(\mu\) G (default value \(= 2 \mu\) G)

\item {} 
\sphinxAtStartPar
\sphinxstyleliteralstrong{\sphinxupquote{Gamma}} \textendash{} power\sphinxhyphen{}law exponent of the generic CRE source (\(1.1 < \Gamma < 3\), default value = 2)

\item {} 
\sphinxAtStartPar
\sphinxstyleliteralstrong{\sphinxupquote{manual}} (\sphinxstyleliteralemphasis{\sphinxupquote{bool}}) \textendash{} manual input of parameter values in rad\_temp (default value = \sphinxcode{\sphinxupquote{\textquotesingle{}False\textquotesingle{}}})

\end{itemize}

\item[{Beam\_size}] \leavevmode
\sphinxAtStartPar
beam size in arcseconds

\end{description}\end{quote}

\sphinxAtStartPar
Keyword arguments
\begin{itemize}
\item {} 
\sphinxAtStartPar
\sphinxcode{\sphinxupquote{manual = \textquotesingle{}False\textquotesingle{}}}

\end{itemize}
\begin{quote}\begin{description}
\item[{Parameters}] \leavevmode
\sphinxAtStartPar
\sphinxstyleliteralstrong{\sphinxupquote{ref}} \textendash{} reference used (\sphinxcode{\sphinxupquote{\textquotesingle{}Martinez\textquotesingle{}}} or \sphinxcode{\sphinxupquote{\textquotesingle{}1309.2641\textquotesingle{}}}, \sphinxcode{\sphinxupquote{\textquotesingle{}Geringer\sphinxhyphen{}Sameth\textquotesingle{}}} or \sphinxcode{\sphinxupquote{\textquotesingle{}1408.0002\textquotesingle{}}}, etc.)

\end{description}\end{quote}
\begin{itemize}
\item {} 
\sphinxAtStartPar
\sphinxcode{\sphinxupquote{manual = \textquotesingle{}True\textquotesingle{}}}

\end{itemize}
\begin{quote}\begin{description}
\item[{Parameters}] \leavevmode\begin{itemize}
\item {} 
\sphinxAtStartPar
\sphinxstyleliteralstrong{\sphinxupquote{rs}} \textendash{} scale radius in kpc

\item {} 
\sphinxAtStartPar
\sphinxstyleliteralstrong{\sphinxupquote{sigmav}} \textendash{} velocity dispersion in km/s for the isothermal sphere {\hyperref[\detokenize{diffsph.profiles:diffsph.profiles.templates.sis}]{\sphinxcrossref{\sphinxcode{\sphinxupquote{diffsph.profiles.templates.sis()}}}}}

\end{itemize}

\item[{Returns}] \leavevmode
\sphinxAtStartPar
upper limit on the generic CRE production rate in 1/s

\item[{Return type}] \leavevmode
\sphinxAtStartPar
float

\end{description}\end{quote}

\end{fulllineitems}

\index{generic\_rate\_limest() (in module diffsph.limits)@\spxentry{generic\_rate\_limest()}\spxextra{in module diffsph.limits}}

\begin{fulllineitems}
\phantomsection\label{\detokenize{diffsph:diffsph.limits.generic_rate_limest}}\pysiglinewithargsret{\sphinxcode{\sphinxupquote{diffsph.limits.}}\sphinxbfcode{\sphinxupquote{generic\_rate\_limest}}}{\emph{\DUrole{n}{nu}}, \emph{\DUrole{n}{rms\_noise}}, \emph{\DUrole{n}{beam\_size}}, \emph{\DUrole{n}{galaxy}}, \emph{\DUrole{n}{rad\_temp}}, \emph{\DUrole{n}{ratio}\DUrole{o}{=}\DUrole{default_value}{1}}, \emph{\DUrole{n}{D0}\DUrole{o}{=}\DUrole{default_value}{3e+28}}, \emph{\DUrole{n}{delta}\DUrole{o}{=}\DUrole{default_value}{\textquotesingle{}kol\textquotesingle{}}}, \emph{\DUrole{n}{B}\DUrole{o}{=}\DUrole{default_value}{2}}, \emph{\DUrole{n}{Gamma}\DUrole{o}{=}\DUrole{default_value}{2}}, \emph{\DUrole{n}{high\_res}\DUrole{o}{=}\DUrole{default_value}{False}}, \emph{\DUrole{n}{accuracy}\DUrole{o}{=}\DUrole{default_value}{1}}, \emph{\DUrole{o}{**}\DUrole{n}{kwargs}}}{}
\sphinxAtStartPar
(Estimated) maximum CRE production rate (generic power\sphinxhyphen{}law hypothesis) given the rms noise level of an observation
\begin{quote}\begin{description}
\item[{Parameters}] \leavevmode\begin{itemize}
\item {} 
\sphinxAtStartPar
\sphinxstyleliteralstrong{\sphinxupquote{nu}} \textendash{} frequency in GHz

\item {} 
\sphinxAtStartPar
\sphinxstyleliteralstrong{\sphinxupquote{rms\_noise}} \textendash{} RMS noise level of the observation in \(\mu\) Jy / beam

\item {} 
\sphinxAtStartPar
\sphinxstyleliteralstrong{\sphinxupquote{galaxy}} (\sphinxstyleliteralemphasis{\sphinxupquote{str}}) \textendash{} name of the galaxy

\item {} 
\sphinxAtStartPar
\sphinxstyleliteralstrong{\sphinxupquote{rad\_temp}} (\sphinxstyleliteralemphasis{\sphinxupquote{str}}) \textendash{} dark matter halo model (\sphinxcode{\sphinxupquote{\textquotesingle{}NFW\textquotesingle{}}}, \sphinxcode{\sphinxupquote{\textquotesingle{}Einasto\textquotesingle{}}}, etc.)

\item {} 
\sphinxAtStartPar
\sphinxstyleliteralstrong{\sphinxupquote{ratio}} \textendash{} ratio between the diffusion halo and half\sphinxhyphen{}light radii

\item {} 
\sphinxAtStartPar
\sphinxstyleliteralstrong{\sphinxupquote{D0}} \textendash{} magnitude of the diffusion coefficient for a 1 GeV CRE in cm \({}^2\)/s (default value = \(3\times 10^{28}\) cm \({}^2\) /s)

\item {} 
\sphinxAtStartPar
\sphinxstyleliteralstrong{\sphinxupquote{delta}} (\sphinxstyleliteralemphasis{\sphinxupquote{float}}\sphinxstyleliteralemphasis{\sphinxupquote{, }}\sphinxstyleliteralemphasis{\sphinxupquote{str}}) \textendash{} power\sphinxhyphen{}law exponent of the diffusion coefficient as a function of the CRE’s energy (default value = 1/3 or \sphinxcode{\sphinxupquote{\textquotesingle{}kol\textquotesingle{}}})

\item {} 
\sphinxAtStartPar
\sphinxstyleliteralstrong{\sphinxupquote{B}} \textendash{} magnitude of the magnetic field’s smooth component in \(\mu\) G (default value \(= 2 \mu\) G)

\item {} 
\sphinxAtStartPar
\sphinxstyleliteralstrong{\sphinxupquote{Gamma}} \textendash{} power\sphinxhyphen{}law exponent of the generic CRE source (\(1.1 < \Gamma < 3\), default value = 2)

\item {} 
\sphinxAtStartPar
\sphinxstyleliteralstrong{\sphinxupquote{manual}} (\sphinxstyleliteralemphasis{\sphinxupquote{bool}}) \textendash{} manual input of parameter values in rad\_temp (default value = \sphinxcode{\sphinxupquote{\textquotesingle{}False\textquotesingle{}}})

\item {} 
\sphinxAtStartPar
\sphinxstyleliteralstrong{\sphinxupquote{high\_res}} (\sphinxstyleliteralemphasis{\sphinxupquote{bool}}) \textendash{} spatial resolution. If \sphinxcode{\sphinxupquote{\textquotesingle{}True\textquotesingle{}}}, \sphinxcode{\sphinxupquote{synch\_emissivity()}} computes as many terms as needed in order to converge at \(r=0\). (default value = \sphinxcode{\sphinxupquote{\textquotesingle{}False\textquotesingle{}}})

\item {} 
\sphinxAtStartPar
\sphinxstyleliteralstrong{\sphinxupquote{accuracy}} \textendash{} theoretical accuracy in \% (default value = 1\%)

\end{itemize}

\item[{Beam\_size}] \leavevmode
\sphinxAtStartPar
beam size in arcseconds

\end{description}\end{quote}

\sphinxAtStartPar
Keyword arguments
\begin{itemize}
\item {} 
\sphinxAtStartPar
\sphinxcode{\sphinxupquote{manual = \textquotesingle{}False\textquotesingle{}}}

\end{itemize}
\begin{quote}\begin{description}
\item[{Parameters}] \leavevmode
\sphinxAtStartPar
\sphinxstyleliteralstrong{\sphinxupquote{ref}} \textendash{} reference used (\sphinxcode{\sphinxupquote{\textquotesingle{}Martinez\textquotesingle{}}} or \sphinxcode{\sphinxupquote{\textquotesingle{}1309.2641\textquotesingle{}}}, \sphinxcode{\sphinxupquote{\textquotesingle{}Geringer\sphinxhyphen{}Sameth\textquotesingle{}}} or \sphinxcode{\sphinxupquote{\textquotesingle{}1408.0002\textquotesingle{}}}, etc.)

\end{description}\end{quote}
\begin{itemize}
\item {} 
\sphinxAtStartPar
\sphinxcode{\sphinxupquote{manual = \textquotesingle{}True\textquotesingle{}}}

\end{itemize}
\begin{quote}\begin{description}
\item[{Parameters}] \leavevmode\begin{itemize}
\item {} 
\sphinxAtStartPar
\sphinxstyleliteralstrong{\sphinxupquote{rs}} \textendash{} scale radius in kpc

\item {} 
\sphinxAtStartPar
\sphinxstyleliteralstrong{\sphinxupquote{sigmav}} \textendash{} velocity dispersion in km/s for the isothermal sphere {\hyperref[\detokenize{diffsph.profiles:diffsph.profiles.templates.sis}]{\sphinxcrossref{\sphinxcode{\sphinxupquote{diffsph.profiles.templates.sis()}}}}}

\end{itemize}

\item[{Returns}] \leavevmode
\sphinxAtStartPar
Estimated upper limit on the generic CRE production rate in 1/s

\item[{Return type}] \leavevmode
\sphinxAtStartPar
float

\end{description}\end{quote}

\end{fulllineitems}

\index{sigmav\_gausslim() (in module diffsph.limits)@\spxentry{sigmav\_gausslim()}\spxextra{in module diffsph.limits}}

\begin{fulllineitems}
\phantomsection\label{\detokenize{diffsph:diffsph.limits.sigmav_gausslim}}\pysiglinewithargsret{\sphinxcode{\sphinxupquote{diffsph.limits.}}\sphinxbfcode{\sphinxupquote{sigmav\_gausslim}}}{\emph{\DUrole{n}{nu}}, \emph{\DUrole{n}{a\_fit}}, \emph{\DUrole{n}{sigma\_fit}}, \emph{\DUrole{n}{beam\_size}}, \emph{\DUrole{n}{galaxy}}, \emph{\DUrole{n}{rad\_temp}}, \emph{\DUrole{n}{D0}\DUrole{o}{=}\DUrole{default_value}{3e+28}}, \emph{\DUrole{n}{delta}\DUrole{o}{=}\DUrole{default_value}{\textquotesingle{}kol\textquotesingle{}}}, \emph{\DUrole{n}{B}\DUrole{o}{=}\DUrole{default_value}{2}}, \emph{\DUrole{n}{mchi}\DUrole{o}{=}\DUrole{default_value}{50}}, \emph{\DUrole{n}{channel}\DUrole{o}{=}\DUrole{default_value}{\textquotesingle{}mumu\textquotesingle{}}}, \emph{\DUrole{n}{self\_conjugate}\DUrole{o}{=}\DUrole{default_value}{True}}, \emph{\DUrole{n}{manual}\DUrole{o}{=}\DUrole{default_value}{False}}, \emph{\DUrole{o}{**}\DUrole{n}{kwargs}}}{}
\sphinxAtStartPar
Maximum WIMP self\sphinxhyphen{}annihilation cross\sphinxhyphen{}section allowed by the exclusion of a Gaussian\sphinxhyphen{}shaped signal
\begin{equation*}
\begin{split}a_\text{fit}\exp\left(-\frac{\theta^2}{2\sigma_\text{fit}^2}\right)\end{split}
\end{equation*}\begin{quote}\begin{description}
\item[{Parameters}] \leavevmode\begin{itemize}
\item {} 
\sphinxAtStartPar
\sphinxstyleliteralstrong{\sphinxupquote{nu}} \textendash{} frequency in GHz

\item {} 
\sphinxAtStartPar
\sphinxstyleliteralstrong{\sphinxupquote{a\_fit}} \textendash{} fitted gaussian amplitude in \(\mu\) Jy / beam

\item {} 
\sphinxAtStartPar
\sphinxstyleliteralstrong{\sphinxupquote{sigma\_fit}} \textendash{} width parameter of the Gaussian template in arcmin

\item {} 
\sphinxAtStartPar
\sphinxstyleliteralstrong{\sphinxupquote{galaxy}} (\sphinxstyleliteralemphasis{\sphinxupquote{str}}) \textendash{} name of the galaxy

\item {} 
\sphinxAtStartPar
\sphinxstyleliteralstrong{\sphinxupquote{rad\_temp}} (\sphinxstyleliteralemphasis{\sphinxupquote{str}}) \textendash{} dark matter halo model (\sphinxcode{\sphinxupquote{\textquotesingle{}NFW\textquotesingle{}}}, \sphinxcode{\sphinxupquote{\textquotesingle{}Einasto\textquotesingle{}}}, etc.)

\item {} 
\sphinxAtStartPar
\sphinxstyleliteralstrong{\sphinxupquote{D0}} \textendash{} magnitude of the diffusion coefficient for a 1 GeV CRE in cm \({}^2\)/s (default value = \(3\times 10^{28}\) cm \({}^2\) /s)

\item {} 
\sphinxAtStartPar
\sphinxstyleliteralstrong{\sphinxupquote{delta}} (\sphinxstyleliteralemphasis{\sphinxupquote{float}}\sphinxstyleliteralemphasis{\sphinxupquote{, }}\sphinxstyleliteralemphasis{\sphinxupquote{str}}) \textendash{} power\sphinxhyphen{}law exponent of the diffusion coefficient as a function of the CRE’s energy (default value = 1/3 or \sphinxcode{\sphinxupquote{\textquotesingle{}kol\textquotesingle{}}})

\item {} 
\sphinxAtStartPar
\sphinxstyleliteralstrong{\sphinxupquote{B}} \textendash{} magnitude of the magnetic field’s smooth component in \(\mu\) G (default value \(= 2 \mu\) G)

\item {} 
\sphinxAtStartPar
\sphinxstyleliteralstrong{\sphinxupquote{mchi}} \textendash{} mass of the DM particle in GeV/c \({}^2\)

\item {} 
\sphinxAtStartPar
\sphinxstyleliteralstrong{\sphinxupquote{channel}} (\sphinxstyleliteralemphasis{\sphinxupquote{str}}) \textendash{} annihilation channel: \(b\bar b\) (\sphinxcode{\sphinxupquote{\textquotesingle{}bb\textquotesingle{}}}), \(\mu^+ \mu^-\) (\sphinxcode{\sphinxupquote{\textquotesingle{}mumu\textquotesingle{}}}), \(W^+ W^-\) (\sphinxcode{\sphinxupquote{\textquotesingle{}WW\textquotesingle{}}}), etc.

\item {} 
\sphinxAtStartPar
\sphinxstyleliteralstrong{\sphinxupquote{self\_conjugate}} \textendash{} if set \sphinxcode{\sphinxupquote{\textquotesingle{}True\textquotesingle{}}} (default value) the DM particle is its own antiparticle

\item {} 
\sphinxAtStartPar
\sphinxstyleliteralstrong{\sphinxupquote{manual}} (\sphinxstyleliteralemphasis{\sphinxupquote{bool}}) \textendash{} manual input of parameter values in rad\_temp (default value = \sphinxcode{\sphinxupquote{\textquotesingle{}False\textquotesingle{}}})

\end{itemize}

\item[{Beam\_size}] \leavevmode
\sphinxAtStartPar
beam size in arcseconds

\end{description}\end{quote}

\sphinxAtStartPar
Keyword arguments
\begin{itemize}
\item {} 
\sphinxAtStartPar
\sphinxcode{\sphinxupquote{manual = \textquotesingle{}False\textquotesingle{}}}

\end{itemize}
\begin{quote}\begin{description}
\item[{Parameters}] \leavevmode
\sphinxAtStartPar
\sphinxstyleliteralstrong{\sphinxupquote{ref}} \textendash{} reference used (\sphinxcode{\sphinxupquote{\textquotesingle{}Martinez\textquotesingle{}}} or \sphinxcode{\sphinxupquote{\textquotesingle{}1309.2641\textquotesingle{}}}, \sphinxcode{\sphinxupquote{\textquotesingle{}Geringer\sphinxhyphen{}Sameth\textquotesingle{}}} or \sphinxcode{\sphinxupquote{\textquotesingle{}1408.0002\textquotesingle{}}}, etc.)

\end{description}\end{quote}
\begin{itemize}
\item {} 
\sphinxAtStartPar
\sphinxcode{\sphinxupquote{manual = \textquotesingle{}True\textquotesingle{}}}

\end{itemize}
\begin{quote}\begin{description}
\item[{Parameters}] \leavevmode\begin{itemize}
\item {} 
\sphinxAtStartPar
\sphinxstyleliteralstrong{\sphinxupquote{rs}} \textendash{} scale radius in kpc

\item {} 
\sphinxAtStartPar
\sphinxstyleliteralstrong{\sphinxupquote{rhos}} \textendash{} characteristic density in GeV/cm \({}^3\)

\item {} 
\sphinxAtStartPar
\sphinxstyleliteralstrong{\sphinxupquote{alpha}} \textendash{} exponent \(\alpha\) in the {\hyperref[\detokenize{diffsph.profiles:diffsph.profiles.templates.hdz}]{\sphinxcrossref{\sphinxcode{\sphinxupquote{diffsph.profiles.templates.hdz()}}}}} profile

\item {} 
\sphinxAtStartPar
\sphinxstyleliteralstrong{\sphinxupquote{beta}} \textendash{} exponent \(\beta\) in the {\hyperref[\detokenize{diffsph.profiles:diffsph.profiles.templates.hdz}]{\sphinxcrossref{\sphinxcode{\sphinxupquote{diffsph.profiles.templates.hdz()}}}}} profile

\item {} 
\sphinxAtStartPar
\sphinxstyleliteralstrong{\sphinxupquote{gamma}} \textendash{} exponent \(\gamma\) in the {\hyperref[\detokenize{diffsph.profiles:diffsph.profiles.templates.hdz}]{\sphinxcrossref{\sphinxcode{\sphinxupquote{diffsph.profiles.templates.hdz()}}}}} profile

\item {} 
\sphinxAtStartPar
\sphinxstyleliteralstrong{\sphinxupquote{alphaE}} \textendash{} parameter \(\alpha_E\) in the {\hyperref[\detokenize{diffsph.profiles:diffsph.profiles.templates.enst}]{\sphinxcrossref{\sphinxcode{\sphinxupquote{diffsph.profiles.templates.enst()}}}}} profile

\end{itemize}

\item[{Returns}] \leavevmode
\sphinxAtStartPar
upper limit for the WIMP self\sphinxhyphen{}annihilation cross\sphinxhyphen{}section in cm \({}^3\) /s

\item[{Return type}] \leavevmode
\sphinxAtStartPar
float

\end{description}\end{quote}

\end{fulllineitems}

\index{sigmav\_limest() (in module diffsph.limits)@\spxentry{sigmav\_limest()}\spxextra{in module diffsph.limits}}

\begin{fulllineitems}
\phantomsection\label{\detokenize{diffsph:diffsph.limits.sigmav_limest}}\pysiglinewithargsret{\sphinxcode{\sphinxupquote{diffsph.limits.}}\sphinxbfcode{\sphinxupquote{sigmav\_limest}}}{\emph{\DUrole{n}{nu}}, \emph{\DUrole{n}{rms\_noise}}, \emph{\DUrole{n}{beam\_size}}, \emph{\DUrole{n}{galaxy}}, \emph{\DUrole{n}{rad\_temp}}, \emph{\DUrole{n}{ratio}\DUrole{o}{=}\DUrole{default_value}{1}}, \emph{\DUrole{n}{D0}\DUrole{o}{=}\DUrole{default_value}{3e+28}}, \emph{\DUrole{n}{delta}\DUrole{o}{=}\DUrole{default_value}{\textquotesingle{}kol\textquotesingle{}}}, \emph{\DUrole{n}{B}\DUrole{o}{=}\DUrole{default_value}{2}}, \emph{\DUrole{n}{mchi}\DUrole{o}{=}\DUrole{default_value}{50}}, \emph{\DUrole{n}{channel}\DUrole{o}{=}\DUrole{default_value}{\textquotesingle{}mumu\textquotesingle{}}}, \emph{\DUrole{n}{self\_conjugate}\DUrole{o}{=}\DUrole{default_value}{True}}, \emph{\DUrole{n}{manual}\DUrole{o}{=}\DUrole{default_value}{False}}, \emph{\DUrole{n}{high\_res}\DUrole{o}{=}\DUrole{default_value}{False}}, \emph{\DUrole{n}{accuracy}\DUrole{o}{=}\DUrole{default_value}{1}}, \emph{\DUrole{o}{**}\DUrole{n}{kwargs}}}{}
\sphinxAtStartPar
(Estimated) maximum WIMP self\sphinxhyphen{}annihilation cross\sphinxhyphen{}section given the rms noise level of an observation
\begin{quote}\begin{description}
\item[{Parameters}] \leavevmode\begin{itemize}
\item {} 
\sphinxAtStartPar
\sphinxstyleliteralstrong{\sphinxupquote{nu}} \textendash{} frequency in GHz

\item {} 
\sphinxAtStartPar
\sphinxstyleliteralstrong{\sphinxupquote{rms\_noise}} \textendash{} RMS noise level of the observation in \(\mu\) Jy / beam

\item {} 
\sphinxAtStartPar
\sphinxstyleliteralstrong{\sphinxupquote{galaxy}} (\sphinxstyleliteralemphasis{\sphinxupquote{str}}) \textendash{} name of the galaxy

\item {} 
\sphinxAtStartPar
\sphinxstyleliteralstrong{\sphinxupquote{rad\_temp}} (\sphinxstyleliteralemphasis{\sphinxupquote{str}}) \textendash{} dark matter halo model (\sphinxcode{\sphinxupquote{\textquotesingle{}NFW\textquotesingle{}}}, \sphinxcode{\sphinxupquote{\textquotesingle{}Einasto\textquotesingle{}}}, etc.)

\item {} 
\sphinxAtStartPar
\sphinxstyleliteralstrong{\sphinxupquote{ratio}} \textendash{} ratio between the diffusion halo and half\sphinxhyphen{}light radii

\item {} 
\sphinxAtStartPar
\sphinxstyleliteralstrong{\sphinxupquote{D0}} \textendash{} magnitude of the diffusion coefficient for a 1 GeV CRE in cm \({}^2\)/s (default value = \(3\times 10^{28}\) cm \({}^2\) /s)

\item {} 
\sphinxAtStartPar
\sphinxstyleliteralstrong{\sphinxupquote{delta}} (\sphinxstyleliteralemphasis{\sphinxupquote{float}}\sphinxstyleliteralemphasis{\sphinxupquote{, }}\sphinxstyleliteralemphasis{\sphinxupquote{str}}) \textendash{} power\sphinxhyphen{}law exponent of the diffusion coefficient as a function of the CRE’s energy (default value = 1/3 or \sphinxcode{\sphinxupquote{\textquotesingle{}kol\textquotesingle{}}})

\item {} 
\sphinxAtStartPar
\sphinxstyleliteralstrong{\sphinxupquote{B}} \textendash{} magnitude of the magnetic field’s smooth component in \(\mu\) G (default value \(= 2 \mu\) G)

\item {} 
\sphinxAtStartPar
\sphinxstyleliteralstrong{\sphinxupquote{mchi}} \textendash{} mass of the DM particle in GeV/c \({}^2\)

\item {} 
\sphinxAtStartPar
\sphinxstyleliteralstrong{\sphinxupquote{channel}} (\sphinxstyleliteralemphasis{\sphinxupquote{str}}) \textendash{} annihilation channel: \(b\bar b\) (\sphinxcode{\sphinxupquote{\textquotesingle{}bb\textquotesingle{}}}), \(\mu^+ \mu^-\) (\sphinxcode{\sphinxupquote{\textquotesingle{}mumu\textquotesingle{}}}), \(W^+ W^-\) (\sphinxcode{\sphinxupquote{\textquotesingle{}WW\textquotesingle{}}}), etc.

\item {} 
\sphinxAtStartPar
\sphinxstyleliteralstrong{\sphinxupquote{self\_conjugate}} \textendash{} if set \sphinxcode{\sphinxupquote{\textquotesingle{}True\textquotesingle{}}} (default value) the DM particle is its own antiparticle

\item {} 
\sphinxAtStartPar
\sphinxstyleliteralstrong{\sphinxupquote{manual}} (\sphinxstyleliteralemphasis{\sphinxupquote{bool}}) \textendash{} manual input of parameter values in rad\_temp (default value = \sphinxcode{\sphinxupquote{\textquotesingle{}False\textquotesingle{}}})

\item {} 
\sphinxAtStartPar
\sphinxstyleliteralstrong{\sphinxupquote{high\_res}} (\sphinxstyleliteralemphasis{\sphinxupquote{bool}}) \textendash{} spatial resolution. If \sphinxcode{\sphinxupquote{\textquotesingle{}True\textquotesingle{}}}, \sphinxcode{\sphinxupquote{synch\_emissivity()}} computes as many terms as needed in order to converge at \(r=0\). (default value = \sphinxcode{\sphinxupquote{\textquotesingle{}False\textquotesingle{}}})

\item {} 
\sphinxAtStartPar
\sphinxstyleliteralstrong{\sphinxupquote{accuracy}} \textendash{} theoretical accuracy in \% (default value = 1\%)

\end{itemize}

\item[{Beam\_size}] \leavevmode
\sphinxAtStartPar
beam size in arcseconds

\end{description}\end{quote}

\sphinxAtStartPar
Keyword arguments
\begin{itemize}
\item {} 
\sphinxAtStartPar
\sphinxcode{\sphinxupquote{manual = \textquotesingle{}False\textquotesingle{}}}

\end{itemize}
\begin{quote}\begin{description}
\item[{Parameters}] \leavevmode
\sphinxAtStartPar
\sphinxstyleliteralstrong{\sphinxupquote{ref}} \textendash{} reference used (\sphinxcode{\sphinxupquote{\textquotesingle{}Martinez\textquotesingle{}}} or \sphinxcode{\sphinxupquote{\textquotesingle{}1309.2641\textquotesingle{}}}, \sphinxcode{\sphinxupquote{\textquotesingle{}Geringer\sphinxhyphen{}Sameth\textquotesingle{}}} or \sphinxcode{\sphinxupquote{\textquotesingle{}1408.0002\textquotesingle{}}}, etc.)

\end{description}\end{quote}
\begin{itemize}
\item {} 
\sphinxAtStartPar
\sphinxcode{\sphinxupquote{manual = \textquotesingle{}True\textquotesingle{}}}

\end{itemize}
\begin{quote}\begin{description}
\item[{Parameters}] \leavevmode\begin{itemize}
\item {} 
\sphinxAtStartPar
\sphinxstyleliteralstrong{\sphinxupquote{rs}} \textendash{} scale radius in kpc

\item {} 
\sphinxAtStartPar
\sphinxstyleliteralstrong{\sphinxupquote{rhos}} \textendash{} characteristic density in GeV/cm \({}^3\)

\item {} 
\sphinxAtStartPar
\sphinxstyleliteralstrong{\sphinxupquote{alpha}} \textendash{} exponent \(\alpha\) in the {\hyperref[\detokenize{diffsph.profiles:diffsph.profiles.templates.hdz}]{\sphinxcrossref{\sphinxcode{\sphinxupquote{diffsph.profiles.templates.hdz()}}}}} profile

\item {} 
\sphinxAtStartPar
\sphinxstyleliteralstrong{\sphinxupquote{beta}} \textendash{} exponent \(\beta\) in the {\hyperref[\detokenize{diffsph.profiles:diffsph.profiles.templates.hdz}]{\sphinxcrossref{\sphinxcode{\sphinxupquote{diffsph.profiles.templates.hdz()}}}}} profile

\item {} 
\sphinxAtStartPar
\sphinxstyleliteralstrong{\sphinxupquote{gamma}} \textendash{} exponent \(\gamma\) in the {\hyperref[\detokenize{diffsph.profiles:diffsph.profiles.templates.hdz}]{\sphinxcrossref{\sphinxcode{\sphinxupquote{diffsph.profiles.templates.hdz()}}}}} profile

\item {} 
\sphinxAtStartPar
\sphinxstyleliteralstrong{\sphinxupquote{alphaE}} \textendash{} parameter \(\alpha_E\) in the {\hyperref[\detokenize{diffsph.profiles:diffsph.profiles.templates.enst}]{\sphinxcrossref{\sphinxcode{\sphinxupquote{diffsph.profiles.templates.enst()}}}}} profile

\end{itemize}

\item[{Returns}] \leavevmode
\sphinxAtStartPar
Estimated upper limit on WIMP self\sphinxhyphen{}annihilation cross\sphinxhyphen{}section in cm \({}^3\) /s

\item[{Return type}] \leavevmode
\sphinxAtStartPar
float

\end{description}\end{quote}

\end{fulllineitems}



\subsection{diffsph.pyflux module}
\label{\detokenize{diffsph:module-diffsph.pyflux}}\label{\detokenize{diffsph:diffsph-pyflux-module}}\index{module@\spxentry{module}!diffsph.pyflux@\spxentry{diffsph.pyflux}}\index{diffsph.pyflux@\spxentry{diffsph.pyflux}!module@\spxentry{module}}\index{coeff() (in module diffsph.pyflux)@\spxentry{coeff()}\spxextra{in module diffsph.pyflux}}

\begin{fulllineitems}
\phantomsection\label{\detokenize{diffsph:diffsph.pyflux.coeff}}\pysiglinewithargsret{\sphinxcode{\sphinxupquote{diffsph.pyflux.}}\sphinxbfcode{\sphinxupquote{coeff}}}{\emph{\DUrole{n}{n}}, \emph{\DUrole{n}{nu}}, \emph{\DUrole{n}{galaxy}}, \emph{\DUrole{n}{rad\_temp}}, \emph{\DUrole{n}{hyp}}, \emph{\DUrole{n}{ratio}}, \emph{\DUrole{n}{D0}}, \emph{\DUrole{n}{delta}}, \emph{\DUrole{n}{B}}, \emph{\DUrole{n}{manual}\DUrole{o}{=}\DUrole{default_value}{False}}, \emph{\DUrole{o}{**}\DUrole{n}{kwargs}}}{}
\sphinxAtStartPar
n\sphinxhyphen{}th coefficient participating in the Fourier\sphinxhyphen{}expanded Green’s function solution of the CRE transport equation
\begin{equation*}
\begin{split}s_n = h_n\times X_n\end{split}
\end{equation*}\begin{quote}\begin{description}
\item[{Parameters}] \leavevmode\begin{itemize}
\item {} 
\sphinxAtStartPar
\sphinxstyleliteralstrong{\sphinxupquote{n}} \textendash{} order of the halo/bulge factor

\item {} 
\sphinxAtStartPar
\sphinxstyleliteralstrong{\sphinxupquote{theta}} \textendash{} angular radius in arcmin

\item {} 
\sphinxAtStartPar
\sphinxstyleliteralstrong{\sphinxupquote{nu}} \textendash{} frequency in GHz

\item {} 
\sphinxAtStartPar
\sphinxstyleliteralstrong{\sphinxupquote{galaxy}} (\sphinxstyleliteralemphasis{\sphinxupquote{str}}) \textendash{} name of the galaxy

\item {} 
\sphinxAtStartPar
\sphinxstyleliteralstrong{\sphinxupquote{rad\_temp}} (\sphinxstyleliteralemphasis{\sphinxupquote{str}}) \textendash{} radial template (\sphinxcode{\sphinxupquote{\textquotesingle{}NFW\textquotesingle{}}}, \sphinxcode{\sphinxupquote{\textquotesingle{}Einasto\textquotesingle{}}}, etc.)

\item {} 
\sphinxAtStartPar
\sphinxstyleliteralstrong{\sphinxupquote{hyp}} (\sphinxstyleliteralemphasis{\sphinxupquote{str}}) \textendash{} hypothesis: \sphinxcode{\sphinxupquote{\textquotesingle{}wimp\textquotesingle{}}} (\sphinxstylestrong{default}), \sphinxcode{\sphinxupquote{\textquotesingle{}decay\textquotesingle{}}} or \sphinxcode{\sphinxupquote{\textquotesingle{}generic\textquotesingle{}}}

\item {} 
\sphinxAtStartPar
\sphinxstyleliteralstrong{\sphinxupquote{ratio}} \textendash{} ratio between the diffusion halo/bulge and half\sphinxhyphen{}light radii (default value = 1)

\item {} 
\sphinxAtStartPar
\sphinxstyleliteralstrong{\sphinxupquote{D0}} \textendash{} magnitude of the diffusion coefficient for a 1 GeV CRE in cm \({}^2\)/s (default value = \(3\times 10^{28}\) cm \({}^2\) /s)

\item {} 
\sphinxAtStartPar
\sphinxstyleliteralstrong{\sphinxupquote{delta}} (\sphinxstyleliteralemphasis{\sphinxupquote{float}}\sphinxstyleliteralemphasis{\sphinxupquote{, }}\sphinxstyleliteralemphasis{\sphinxupquote{str}}) \textendash{} power\sphinxhyphen{}law exponent of the diffusion coefficient as a function of the CRE’s energy (default value = 1/3 or \sphinxcode{\sphinxupquote{\textquotesingle{}kol\textquotesingle{}}})

\item {} 
\sphinxAtStartPar
\sphinxstyleliteralstrong{\sphinxupquote{B}} \textendash{} magnitude of the magnetic field’s smooth component in \(\mu\) G (default value \(= 2 \mu\) G)

\item {} 
\sphinxAtStartPar
\sphinxstyleliteralstrong{\sphinxupquote{manual}} (\sphinxstyleliteralemphasis{\sphinxupquote{bool}}) \textendash{} manual input of parameter values in rad\_temp (default value = \sphinxcode{\sphinxupquote{\textquotesingle{}False\textquotesingle{}}})

\end{itemize}

\end{description}\end{quote}

\sphinxAtStartPar
Keyword arguments
\begin{itemize}
\item {} 
\sphinxAtStartPar
\sphinxcode{\sphinxupquote{hyp = \textquotesingle{}wimp\textquotesingle{}}}    (default)

\end{itemize}
\begin{quote}\begin{description}
\item[{Parameters}] \leavevmode\begin{itemize}
\item {} 
\sphinxAtStartPar
\sphinxstyleliteralstrong{\sphinxupquote{sv}} \textendash{} annihilation rate (annihilation cross section times relative velocity) \(\sigma v\) in cm \({}^3\)/s (default value = \(3 \times 10^{-26}\) cm \({}^3\) /s)

\item {} 
\sphinxAtStartPar
\sphinxstyleliteralstrong{\sphinxupquote{self\_conjugate}} \textendash{} if set \sphinxcode{\sphinxupquote{\textquotesingle{}True\textquotesingle{}}} (default value) the DM particle is its own antiparticle

\item {} 
\sphinxAtStartPar
\sphinxstyleliteralstrong{\sphinxupquote{mchi}} \textendash{} mass of the DM particle in GeV/c \({}^2\)

\item {} 
\sphinxAtStartPar
\sphinxstyleliteralstrong{\sphinxupquote{channel}} (\sphinxstyleliteralemphasis{\sphinxupquote{str}}) \textendash{} annihilation channel: \(b\bar b\) (\sphinxcode{\sphinxupquote{\textquotesingle{}bb\textquotesingle{}}}), \(\mu^+ \mu^-\) (\sphinxcode{\sphinxupquote{\textquotesingle{}mumu\textquotesingle{}}}), \(W^+ W^-\) (\sphinxcode{\sphinxupquote{\textquotesingle{}WW\textquotesingle{}}}), etc.

\end{itemize}

\end{description}\end{quote}
\begin{itemize}
\item {} 
\sphinxAtStartPar
\sphinxcode{\sphinxupquote{hyp = \textquotesingle{}decay\textquotesingle{}}}

\end{itemize}
\begin{quote}\begin{description}
\item[{Parameters}] \leavevmode\begin{itemize}
\item {} 
\sphinxAtStartPar
\sphinxstyleliteralstrong{\sphinxupquote{width}} \textendash{} decay width of the DM particle in 1/s

\item {} 
\sphinxAtStartPar
\sphinxstyleliteralstrong{\sphinxupquote{mchi}} \textendash{} mass of the DM particle in GeV/c \({}^2\)

\item {} 
\sphinxAtStartPar
\sphinxstyleliteralstrong{\sphinxupquote{channel}} (\sphinxstyleliteralemphasis{\sphinxupquote{str}}) \textendash{} decay channel: \(b\bar b\) (\sphinxcode{\sphinxupquote{\textquotesingle{}bb\textquotesingle{}}}), \(\mu^+ \mu^-\) (\sphinxcode{\sphinxupquote{\textquotesingle{}mumu\textquotesingle{}}}), \(W^+ W^-\) (\sphinxcode{\sphinxupquote{\textquotesingle{}WW\textquotesingle{}}}), etc.

\end{itemize}

\end{description}\end{quote}
\begin{itemize}
\item {} 
\sphinxAtStartPar
\sphinxcode{\sphinxupquote{hyp = \textquotesingle{}generic\textquotesingle{}}}

\end{itemize}
\begin{quote}\begin{description}
\item[{Parameters}] \leavevmode\begin{itemize}
\item {} 
\sphinxAtStartPar
\sphinxstyleliteralstrong{\sphinxupquote{Gamma}} \textendash{} power\sphinxhyphen{}law exponent of the generic CRE source (\(1.1 < \Gamma < 3\))

\item {} 
\sphinxAtStartPar
\sphinxstyleliteralstrong{\sphinxupquote{rate}} \textendash{} CRE production rate in 1/s

\end{itemize}

\end{description}\end{quote}
\begin{itemize}
\item {} 
\sphinxAtStartPar
\sphinxcode{\sphinxupquote{manual = \textquotesingle{}False\textquotesingle{}}}

\end{itemize}
\begin{quote}\begin{description}
\item[{Parameters}] \leavevmode
\sphinxAtStartPar
\sphinxstyleliteralstrong{\sphinxupquote{ref}} \textendash{} reference used (\sphinxcode{\sphinxupquote{\textquotesingle{}Martinez\textquotesingle{}}} or \sphinxcode{\sphinxupquote{\textquotesingle{}1309.2641\textquotesingle{}}}, \sphinxcode{\sphinxupquote{\textquotesingle{}Geringer\sphinxhyphen{}Sameth\textquotesingle{}}} or \sphinxcode{\sphinxupquote{\textquotesingle{}1408.0002\textquotesingle{}}}, etc.)

\end{description}\end{quote}
\begin{itemize}
\item {} 
\sphinxAtStartPar
\sphinxcode{\sphinxupquote{manual = \textquotesingle{}True\textquotesingle{}}}

\end{itemize}
\begin{quote}\begin{description}
\item[{Parameters}] \leavevmode\begin{itemize}
\item {} 
\sphinxAtStartPar
\sphinxstyleliteralstrong{\sphinxupquote{rs}} \textendash{} scale radius in kpc

\item {} 
\sphinxAtStartPar
\sphinxstyleliteralstrong{\sphinxupquote{rhos}} \textendash{} characteristic density in GeV/cm \({}^3\)

\item {} 
\sphinxAtStartPar
\sphinxstyleliteralstrong{\sphinxupquote{alpha}} \textendash{} exponent \(\alpha\) in the {\hyperref[\detokenize{diffsph.profiles:diffsph.profiles.templates.hdz}]{\sphinxcrossref{\sphinxcode{\sphinxupquote{diffsph.profiles.templates.hdz()}}}}} profile

\item {} 
\sphinxAtStartPar
\sphinxstyleliteralstrong{\sphinxupquote{beta}} \textendash{} exponent \(\beta\) in the {\hyperref[\detokenize{diffsph.profiles:diffsph.profiles.templates.hdz}]{\sphinxcrossref{\sphinxcode{\sphinxupquote{diffsph.profiles.templates.hdz()}}}}} profile

\item {} 
\sphinxAtStartPar
\sphinxstyleliteralstrong{\sphinxupquote{gamma}} \textendash{} exponent \(\gamma\) in the {\hyperref[\detokenize{diffsph.profiles:diffsph.profiles.templates.hdz}]{\sphinxcrossref{\sphinxcode{\sphinxupquote{diffsph.profiles.templates.hdz()}}}}} profile

\item {} 
\sphinxAtStartPar
\sphinxstyleliteralstrong{\sphinxupquote{alphaE}} \textendash{} parameter \(\alpha_E\) in the {\hyperref[\detokenize{diffsph.profiles:diffsph.profiles.templates.enst}]{\sphinxcrossref{\sphinxcode{\sphinxupquote{diffsph.profiles.templates.enst()}}}}} profile

\item {} 
\sphinxAtStartPar
\sphinxstyleliteralstrong{\sphinxupquote{rc}} \textendash{} core radius parameter \(r_c\) in the {\hyperref[\detokenize{diffsph.profiles:diffsph.profiles.templates.cnfw}]{\sphinxcrossref{\sphinxcode{\sphinxupquote{diffsph.profiles.templates.cnfw()}}}}} profile

\item {} 
\sphinxAtStartPar
\sphinxstyleliteralstrong{\sphinxupquote{sigmav}} \textendash{} velocity dispersion parameter \(\sigma_v\) in the {\hyperref[\detokenize{diffsph.profiles:diffsph.profiles.templates.sis}]{\sphinxcrossref{\sphinxcode{\sphinxupquote{diffsph.profiles.templates.sis()}}}}} profile

\end{itemize}

\item[{Returns}] \leavevmode
\sphinxAtStartPar
\sphinxtitleref{n}\sphinxhyphen{}th coefficient in the \sphinxtitleref{which\_N} function

\end{description}\end{quote}

\end{fulllineitems}

\index{synch\_TB() (in module diffsph.pyflux)@\spxentry{synch\_TB()}\spxextra{in module diffsph.pyflux}}

\begin{fulllineitems}
\phantomsection\label{\detokenize{diffsph:diffsph.pyflux.synch_TB}}\pysiglinewithargsret{\sphinxcode{\sphinxupquote{diffsph.pyflux.}}\sphinxbfcode{\sphinxupquote{synch\_TB}}}{\emph{\DUrole{n}{theta}}, \emph{\DUrole{n}{nu}}, \emph{\DUrole{n}{galaxy}}, \emph{\DUrole{n}{rad\_temp}}, \emph{\DUrole{n}{hyp}\DUrole{o}{=}\DUrole{default_value}{\textquotesingle{}wimp\textquotesingle{}}}, \emph{\DUrole{n}{ratio}\DUrole{o}{=}\DUrole{default_value}{1}}, \emph{\DUrole{n}{D0}\DUrole{o}{=}\DUrole{default_value}{3e+28}}, \emph{\DUrole{n}{delta}\DUrole{o}{=}\DUrole{default_value}{\textquotesingle{}kol\textquotesingle{}}}, \emph{\DUrole{n}{B}\DUrole{o}{=}\DUrole{default_value}{2}}, \emph{\DUrole{n}{manual}\DUrole{o}{=}\DUrole{default_value}{False}}, \emph{\DUrole{n}{high\_res}\DUrole{o}{=}\DUrole{default_value}{False}}, \emph{\DUrole{n}{accuracy}\DUrole{o}{=}\DUrole{default_value}{1}}, \emph{\DUrole{o}{**}\DUrole{n}{kwargs}}}{}
\sphinxAtStartPar
Model\sphinxhyphen{}specific brightness temperature from synchrotron radiation
\begin{quote}\begin{description}
\item[{Parameters}] \leavevmode\begin{itemize}
\item {} 
\sphinxAtStartPar
\sphinxstyleliteralstrong{\sphinxupquote{theta}} \textendash{} angular radius in arcmin

\item {} 
\sphinxAtStartPar
\sphinxstyleliteralstrong{\sphinxupquote{nu}} \textendash{} frequency in GHz

\item {} 
\sphinxAtStartPar
\sphinxstyleliteralstrong{\sphinxupquote{galaxy}} (\sphinxstyleliteralemphasis{\sphinxupquote{str}}) \textendash{} name of the galaxy

\item {} 
\sphinxAtStartPar
\sphinxstyleliteralstrong{\sphinxupquote{rad\_temp}} (\sphinxstyleliteralemphasis{\sphinxupquote{str}}) \textendash{} radial template (\sphinxcode{\sphinxupquote{\textquotesingle{}NFW\textquotesingle{}}}, \sphinxcode{\sphinxupquote{\textquotesingle{}Einasto\textquotesingle{}}}, etc.)

\item {} 
\sphinxAtStartPar
\sphinxstyleliteralstrong{\sphinxupquote{hyp}} (\sphinxstyleliteralemphasis{\sphinxupquote{str}}) \textendash{} hypothesis: \sphinxcode{\sphinxupquote{\textquotesingle{}wimp\textquotesingle{}}} (\sphinxstylestrong{default}), \sphinxcode{\sphinxupquote{\textquotesingle{}decay\textquotesingle{}}} or \sphinxcode{\sphinxupquote{\textquotesingle{}generic\textquotesingle{}}}

\item {} 
\sphinxAtStartPar
\sphinxstyleliteralstrong{\sphinxupquote{ratio}} \textendash{} ratio between the diffusion halo/bulge and half\sphinxhyphen{}light radii (default value = 1)

\item {} 
\sphinxAtStartPar
\sphinxstyleliteralstrong{\sphinxupquote{D0}} \textendash{} magnitude of the diffusion coefficient for a 1 GeV CRE in cm \({}^2\)/s (default value = \(3\times 10^{28}\) cm \({}^2\) /s)

\item {} 
\sphinxAtStartPar
\sphinxstyleliteralstrong{\sphinxupquote{delta}} (\sphinxstyleliteralemphasis{\sphinxupquote{float}}\sphinxstyleliteralemphasis{\sphinxupquote{, }}\sphinxstyleliteralemphasis{\sphinxupquote{str}}) \textendash{} power\sphinxhyphen{}law exponent of the diffusion coefficient as a function of the CRE’s energy (default value = 1/3 or \sphinxcode{\sphinxupquote{\textquotesingle{}kol\textquotesingle{}}})

\item {} 
\sphinxAtStartPar
\sphinxstyleliteralstrong{\sphinxupquote{B}} \textendash{} magnitude of the magnetic field’s smooth component in \(\mu\) G (default value \(= 2 \mu\) G)

\item {} 
\sphinxAtStartPar
\sphinxstyleliteralstrong{\sphinxupquote{manual}} (\sphinxstyleliteralemphasis{\sphinxupquote{bool}}) \textendash{} manual input of parameter values in rad\_temp (default value = \sphinxcode{\sphinxupquote{\textquotesingle{}False\textquotesingle{}}})

\item {} 
\sphinxAtStartPar
\sphinxstyleliteralstrong{\sphinxupquote{high\_res}} (\sphinxstyleliteralemphasis{\sphinxupquote{bool}}) \textendash{} spatial resolution. If \sphinxcode{\sphinxupquote{\textquotesingle{}True\textquotesingle{}}}, {\hyperref[\detokenize{diffsph:diffsph.pyflux.synch_emissivity}]{\sphinxcrossref{\sphinxcode{\sphinxupquote{synch\_emissivity()}}}}} computes as many terms as needed in order to converge at \(r=0\). (default value = \sphinxcode{\sphinxupquote{\textquotesingle{}False\textquotesingle{}}})

\item {} 
\sphinxAtStartPar
\sphinxstyleliteralstrong{\sphinxupquote{accuracy}} \textendash{} theoretical accuracy in \% (default value = 1\%)

\end{itemize}

\end{description}\end{quote}

\sphinxAtStartPar
Keyword arguments
\begin{itemize}
\item {} 
\sphinxAtStartPar
\sphinxcode{\sphinxupquote{hyp = \textquotesingle{}wimp\textquotesingle{}}}    (default)

\end{itemize}
\begin{quote}\begin{description}
\item[{Parameters}] \leavevmode\begin{itemize}
\item {} 
\sphinxAtStartPar
\sphinxstyleliteralstrong{\sphinxupquote{sv}} \textendash{} annihilation rate (annihilation cross section times relative velocity) \(\sigma v\) in cm \({}^3\)/s (default value = \(3 \times 10^{-26}\) cm \({}^3\) /s)

\item {} 
\sphinxAtStartPar
\sphinxstyleliteralstrong{\sphinxupquote{self\_conjugate}} \textendash{} if set \sphinxcode{\sphinxupquote{\textquotesingle{}True\textquotesingle{}}} (default value) the DM particle is its own antiparticle

\item {} 
\sphinxAtStartPar
\sphinxstyleliteralstrong{\sphinxupquote{mchi}} \textendash{} mass of the DM particle in GeV/c \({}^2\)

\item {} 
\sphinxAtStartPar
\sphinxstyleliteralstrong{\sphinxupquote{channel}} (\sphinxstyleliteralemphasis{\sphinxupquote{str}}) \textendash{} annihilation channel: \(b\bar b\) (\sphinxcode{\sphinxupquote{\textquotesingle{}bb\textquotesingle{}}}), \(\mu^+ \mu^-\) (\sphinxcode{\sphinxupquote{\textquotesingle{}mumu\textquotesingle{}}}), \(W^+ W^-\) (\sphinxcode{\sphinxupquote{\textquotesingle{}WW\textquotesingle{}}}), etc.

\end{itemize}

\end{description}\end{quote}
\begin{itemize}
\item {} 
\sphinxAtStartPar
\sphinxcode{\sphinxupquote{hyp = \textquotesingle{}decay\textquotesingle{}}}

\end{itemize}
\begin{quote}\begin{description}
\item[{Parameters}] \leavevmode\begin{itemize}
\item {} 
\sphinxAtStartPar
\sphinxstyleliteralstrong{\sphinxupquote{width}} \textendash{} decay width of the DM particle in 1/s

\item {} 
\sphinxAtStartPar
\sphinxstyleliteralstrong{\sphinxupquote{mchi}} \textendash{} mass of the DM particle in GeV/c \({}^2\)

\item {} 
\sphinxAtStartPar
\sphinxstyleliteralstrong{\sphinxupquote{channel}} (\sphinxstyleliteralemphasis{\sphinxupquote{str}}) \textendash{} decay channel: \(b\bar b\) (\sphinxcode{\sphinxupquote{\textquotesingle{}bb\textquotesingle{}}}), \(\mu^+ \mu^-\) (\sphinxcode{\sphinxupquote{\textquotesingle{}mumu\textquotesingle{}}}), \(W^+ W^-\) (\sphinxcode{\sphinxupquote{\textquotesingle{}WW\textquotesingle{}}}), etc.

\end{itemize}

\end{description}\end{quote}
\begin{itemize}
\item {} 
\sphinxAtStartPar
\sphinxcode{\sphinxupquote{hyp = \textquotesingle{}generic\textquotesingle{}}}

\end{itemize}
\begin{quote}\begin{description}
\item[{Parameters}] \leavevmode\begin{itemize}
\item {} 
\sphinxAtStartPar
\sphinxstyleliteralstrong{\sphinxupquote{Gamma}} \textendash{} power\sphinxhyphen{}law exponent of the generic CRE source (\(1.1 < \Gamma < 3\))

\item {} 
\sphinxAtStartPar
\sphinxstyleliteralstrong{\sphinxupquote{rate}} \textendash{} CRE production rate in 1/s

\end{itemize}

\end{description}\end{quote}
\begin{itemize}
\item {} 
\sphinxAtStartPar
\sphinxcode{\sphinxupquote{manual = \textquotesingle{}False\textquotesingle{}}}

\end{itemize}
\begin{quote}\begin{description}
\item[{Parameters}] \leavevmode
\sphinxAtStartPar
\sphinxstyleliteralstrong{\sphinxupquote{ref}} \textendash{} reference used (\sphinxcode{\sphinxupquote{\textquotesingle{}Martinez\textquotesingle{}}} or \sphinxcode{\sphinxupquote{\textquotesingle{}1309.2641\textquotesingle{}}}, \sphinxcode{\sphinxupquote{\textquotesingle{}Geringer\sphinxhyphen{}Sameth\textquotesingle{}}} or \sphinxcode{\sphinxupquote{\textquotesingle{}1408.0002\textquotesingle{}}}, etc.)

\end{description}\end{quote}
\begin{itemize}
\item {} 
\sphinxAtStartPar
\sphinxcode{\sphinxupquote{manual = \textquotesingle{}True\textquotesingle{}}}

\end{itemize}
\begin{quote}\begin{description}
\item[{Parameters}] \leavevmode\begin{itemize}
\item {} 
\sphinxAtStartPar
\sphinxstyleliteralstrong{\sphinxupquote{rs}} \textendash{} scale radius in kpc

\item {} 
\sphinxAtStartPar
\sphinxstyleliteralstrong{\sphinxupquote{rhos}} \textendash{} characteristic density in GeV/cm \({}^3\)

\item {} 
\sphinxAtStartPar
\sphinxstyleliteralstrong{\sphinxupquote{alpha}} \textendash{} exponent \(\alpha\) in the {\hyperref[\detokenize{diffsph.profiles:diffsph.profiles.templates.hdz}]{\sphinxcrossref{\sphinxcode{\sphinxupquote{diffsph.profiles.templates.hdz()}}}}} profile

\item {} 
\sphinxAtStartPar
\sphinxstyleliteralstrong{\sphinxupquote{beta}} \textendash{} exponent \(\beta\) in the {\hyperref[\detokenize{diffsph.profiles:diffsph.profiles.templates.hdz}]{\sphinxcrossref{\sphinxcode{\sphinxupquote{diffsph.profiles.templates.hdz()}}}}} profile

\item {} 
\sphinxAtStartPar
\sphinxstyleliteralstrong{\sphinxupquote{gamma}} \textendash{} exponent \(\gamma\) in the {\hyperref[\detokenize{diffsph.profiles:diffsph.profiles.templates.hdz}]{\sphinxcrossref{\sphinxcode{\sphinxupquote{diffsph.profiles.templates.hdz()}}}}} profile

\item {} 
\sphinxAtStartPar
\sphinxstyleliteralstrong{\sphinxupquote{alphaE}} \textendash{} parameter \(\alpha_E\) in the {\hyperref[\detokenize{diffsph.profiles:diffsph.profiles.templates.enst}]{\sphinxcrossref{\sphinxcode{\sphinxupquote{diffsph.profiles.templates.enst()}}}}} profile

\item {} 
\sphinxAtStartPar
\sphinxstyleliteralstrong{\sphinxupquote{rc}} \textendash{} core radius parameter \(r_c\) in the {\hyperref[\detokenize{diffsph.profiles:diffsph.profiles.templates.cnfw}]{\sphinxcrossref{\sphinxcode{\sphinxupquote{diffsph.profiles.templates.cnfw()}}}}} profile

\item {} 
\sphinxAtStartPar
\sphinxstyleliteralstrong{\sphinxupquote{sigmav}} \textendash{} velocity dispersion parameter \(\sigma_v\) in the {\hyperref[\detokenize{diffsph.profiles:diffsph.profiles.templates.sis}]{\sphinxcrossref{\sphinxcode{\sphinxupquote{diffsph.profiles.templates.sis()}}}}} profile

\end{itemize}

\item[{Returns}] \leavevmode
\sphinxAtStartPar
Brightness temperature in mK

\end{description}\end{quote}

\end{fulllineitems}

\index{synch\_TB\_approx() (in module diffsph.pyflux)@\spxentry{synch\_TB\_approx()}\spxextra{in module diffsph.pyflux}}

\begin{fulllineitems}
\phantomsection\label{\detokenize{diffsph:diffsph.pyflux.synch_TB_approx}}\pysiglinewithargsret{\sphinxcode{\sphinxupquote{diffsph.pyflux.}}\sphinxbfcode{\sphinxupquote{synch\_TB\_approx}}}{\emph{\DUrole{n}{theta}}, \emph{\DUrole{n}{nu}}, \emph{\DUrole{n}{galaxy}}, \emph{\DUrole{n}{rad\_temp}}, \emph{\DUrole{n}{hyp}\DUrole{o}{=}\DUrole{default_value}{\textquotesingle{}wimp\textquotesingle{}}}, \emph{\DUrole{n}{ratio}\DUrole{o}{=}\DUrole{default_value}{1}}, \emph{\DUrole{n}{D0}\DUrole{o}{=}\DUrole{default_value}{3e+28}}, \emph{\DUrole{n}{delta}\DUrole{o}{=}\DUrole{default_value}{\textquotesingle{}kol\textquotesingle{}}}, \emph{\DUrole{n}{B}\DUrole{o}{=}\DUrole{default_value}{2}}, \emph{\DUrole{n}{regime}\DUrole{o}{=}\DUrole{default_value}{\textquotesingle{}B\textquotesingle{}}}, \emph{\DUrole{n}{manual}\DUrole{o}{=}\DUrole{default_value}{False}}, \emph{\DUrole{o}{**}\DUrole{n}{kwargs}}}{}
\sphinxAtStartPar
Model\sphinxhyphen{}specific brightness temperature in the Regime “A”, “B” or “C” approximations
\begin{quote}\begin{description}
\item[{Parameters}] \leavevmode\begin{itemize}
\item {} 
\sphinxAtStartPar
\sphinxstyleliteralstrong{\sphinxupquote{theta}} \textendash{} angular radius in arcmin

\item {} 
\sphinxAtStartPar
\sphinxstyleliteralstrong{\sphinxupquote{nu}} \textendash{} frequency in GHz

\item {} 
\sphinxAtStartPar
\sphinxstyleliteralstrong{\sphinxupquote{galaxy}} (\sphinxstyleliteralemphasis{\sphinxupquote{str}}) \textendash{} name of the galaxy

\item {} 
\sphinxAtStartPar
\sphinxstyleliteralstrong{\sphinxupquote{rad\_temp}} (\sphinxstyleliteralemphasis{\sphinxupquote{str}}) \textendash{} radial template (\sphinxcode{\sphinxupquote{\textquotesingle{}NFW\textquotesingle{}}}, \sphinxcode{\sphinxupquote{\textquotesingle{}Einasto\textquotesingle{}}}, etc.)

\item {} 
\sphinxAtStartPar
\sphinxstyleliteralstrong{\sphinxupquote{hyp}} (\sphinxstyleliteralemphasis{\sphinxupquote{str}}) \textendash{} hypothesis: \sphinxcode{\sphinxupquote{\textquotesingle{}wimp\textquotesingle{}}} (\sphinxstylestrong{default}), \sphinxcode{\sphinxupquote{\textquotesingle{}decay\textquotesingle{}}} or \sphinxcode{\sphinxupquote{\textquotesingle{}generic\textquotesingle{}}}

\item {} 
\sphinxAtStartPar
\sphinxstyleliteralstrong{\sphinxupquote{ratio}} \textendash{} ratio between the diffusion halo/bulge and half\sphinxhyphen{}light radii (default value = 1)

\item {} 
\sphinxAtStartPar
\sphinxstyleliteralstrong{\sphinxupquote{D0}} \textendash{} magnitude of the diffusion coefficient for a 1 GeV CRE in cm \({}^2\)/s (default value = \(3\times 10^{28}\) cm \({}^2\) /s)

\item {} 
\sphinxAtStartPar
\sphinxstyleliteralstrong{\sphinxupquote{delta}} (\sphinxstyleliteralemphasis{\sphinxupquote{float}}\sphinxstyleliteralemphasis{\sphinxupquote{, }}\sphinxstyleliteralemphasis{\sphinxupquote{str}}) \textendash{} power\sphinxhyphen{}law exponent of the diffusion coefficient as a function of the CRE’s energy (default value = 1/3 or \sphinxcode{\sphinxupquote{\textquotesingle{}kol\textquotesingle{}}})

\item {} 
\sphinxAtStartPar
\sphinxstyleliteralstrong{\sphinxupquote{B}} \textendash{} magnitude of the magnetic field’s smooth component in \(\mu\) G (default value \(= 2 \mu\) G)

\item {} 
\sphinxAtStartPar
\sphinxstyleliteralstrong{\sphinxupquote{regime}} \textendash{} regime of the approximation. Must be either upper or lower case a, b, c or I/II/III.

\item {} 
\sphinxAtStartPar
\sphinxstyleliteralstrong{\sphinxupquote{manual}} (\sphinxstyleliteralemphasis{\sphinxupquote{bool}}) \textendash{} manual input of parameter values in rad\_temp (default value = \sphinxcode{\sphinxupquote{\textquotesingle{}False\textquotesingle{}}})

\end{itemize}

\end{description}\end{quote}

\sphinxAtStartPar
Keyword arguments
\begin{itemize}
\item {} 
\sphinxAtStartPar
\sphinxcode{\sphinxupquote{hyp = \textquotesingle{}wimp\textquotesingle{}}}    (default)

\end{itemize}
\begin{quote}\begin{description}
\item[{Parameters}] \leavevmode\begin{itemize}
\item {} 
\sphinxAtStartPar
\sphinxstyleliteralstrong{\sphinxupquote{sv}} \textendash{} annihilation rate (annihilation cross section times relative velocity) \(\sigma v\) in cm \({}^3\)/s (default value = \(3 \times 10^{-26}\) cm \({}^3\) /s)

\item {} 
\sphinxAtStartPar
\sphinxstyleliteralstrong{\sphinxupquote{self\_conjugate}} \textendash{} if set \sphinxcode{\sphinxupquote{\textquotesingle{}True\textquotesingle{}}} (default value) the DM particle is its own antiparticle

\item {} 
\sphinxAtStartPar
\sphinxstyleliteralstrong{\sphinxupquote{mchi}} \textendash{} mass of the DM particle in GeV/c \({}^2\)

\item {} 
\sphinxAtStartPar
\sphinxstyleliteralstrong{\sphinxupquote{channel}} (\sphinxstyleliteralemphasis{\sphinxupquote{str}}) \textendash{} annihilation channel: \(b\bar b\) (\sphinxcode{\sphinxupquote{\textquotesingle{}bb\textquotesingle{}}}), \(\mu^+ \mu^-\) (\sphinxcode{\sphinxupquote{\textquotesingle{}mumu\textquotesingle{}}}), \(W^+ W^-\) (\sphinxcode{\sphinxupquote{\textquotesingle{}WW\textquotesingle{}}}), etc.

\end{itemize}

\end{description}\end{quote}
\begin{itemize}
\item {} 
\sphinxAtStartPar
\sphinxcode{\sphinxupquote{hyp = \textquotesingle{}decay\textquotesingle{}}}

\end{itemize}
\begin{quote}\begin{description}
\item[{Parameters}] \leavevmode\begin{itemize}
\item {} 
\sphinxAtStartPar
\sphinxstyleliteralstrong{\sphinxupquote{width}} \textendash{} decay width of the DM particle in 1/s

\item {} 
\sphinxAtStartPar
\sphinxstyleliteralstrong{\sphinxupquote{mchi}} \textendash{} mass of the DM particle in GeV/c \({}^2\)

\item {} 
\sphinxAtStartPar
\sphinxstyleliteralstrong{\sphinxupquote{channel}} (\sphinxstyleliteralemphasis{\sphinxupquote{str}}) \textendash{} decay channel: \(b\bar b\) (\sphinxcode{\sphinxupquote{\textquotesingle{}bb\textquotesingle{}}}), \(\mu^+ \mu^-\) (\sphinxcode{\sphinxupquote{\textquotesingle{}mumu\textquotesingle{}}}), \(W^+ W^-\) (\sphinxcode{\sphinxupquote{\textquotesingle{}WW\textquotesingle{}}}), etc.

\end{itemize}

\end{description}\end{quote}
\begin{itemize}
\item {} 
\sphinxAtStartPar
\sphinxcode{\sphinxupquote{hyp = \textquotesingle{}generic\textquotesingle{}}}

\end{itemize}
\begin{quote}\begin{description}
\item[{Parameters}] \leavevmode\begin{itemize}
\item {} 
\sphinxAtStartPar
\sphinxstyleliteralstrong{\sphinxupquote{Gamma}} \textendash{} power\sphinxhyphen{}law exponent of the generic CRE source (\(1.1 < \Gamma < 3\))

\item {} 
\sphinxAtStartPar
\sphinxstyleliteralstrong{\sphinxupquote{rate}} \textendash{} CRE production rate in 1/s

\end{itemize}

\end{description}\end{quote}
\begin{itemize}
\item {} 
\sphinxAtStartPar
\sphinxcode{\sphinxupquote{manual = \textquotesingle{}False\textquotesingle{}}}

\end{itemize}
\begin{quote}\begin{description}
\item[{Parameters}] \leavevmode
\sphinxAtStartPar
\sphinxstyleliteralstrong{\sphinxupquote{ref}} \textendash{} reference used (\sphinxcode{\sphinxupquote{\textquotesingle{}Martinez\textquotesingle{}}} or \sphinxcode{\sphinxupquote{\textquotesingle{}1309.2641\textquotesingle{}}}, \sphinxcode{\sphinxupquote{\textquotesingle{}Geringer\sphinxhyphen{}Sameth\textquotesingle{}}} or \sphinxcode{\sphinxupquote{\textquotesingle{}1408.0002\textquotesingle{}}}, etc.)

\end{description}\end{quote}
\begin{itemize}
\item {} 
\sphinxAtStartPar
\sphinxcode{\sphinxupquote{manual = \textquotesingle{}True\textquotesingle{}}}

\end{itemize}
\begin{quote}\begin{description}
\item[{Parameters}] \leavevmode\begin{itemize}
\item {} 
\sphinxAtStartPar
\sphinxstyleliteralstrong{\sphinxupquote{rs}} \textendash{} scale radius in kpc

\item {} 
\sphinxAtStartPar
\sphinxstyleliteralstrong{\sphinxupquote{rhos}} \textendash{} characteristic density in GeV/cm \({}^3\)

\item {} 
\sphinxAtStartPar
\sphinxstyleliteralstrong{\sphinxupquote{alpha}} \textendash{} exponent \(\alpha\) in the {\hyperref[\detokenize{diffsph.profiles:diffsph.profiles.templates.hdz}]{\sphinxcrossref{\sphinxcode{\sphinxupquote{diffsph.profiles.templates.hdz()}}}}} profile

\item {} 
\sphinxAtStartPar
\sphinxstyleliteralstrong{\sphinxupquote{beta}} \textendash{} exponent \(\beta\) in the {\hyperref[\detokenize{diffsph.profiles:diffsph.profiles.templates.hdz}]{\sphinxcrossref{\sphinxcode{\sphinxupquote{diffsph.profiles.templates.hdz()}}}}} profile

\item {} 
\sphinxAtStartPar
\sphinxstyleliteralstrong{\sphinxupquote{gamma}} \textendash{} exponent \(\gamma\) in the {\hyperref[\detokenize{diffsph.profiles:diffsph.profiles.templates.hdz}]{\sphinxcrossref{\sphinxcode{\sphinxupquote{diffsph.profiles.templates.hdz()}}}}} profile

\item {} 
\sphinxAtStartPar
\sphinxstyleliteralstrong{\sphinxupquote{alphaE}} \textendash{} parameter \(\alpha_E\) in the {\hyperref[\detokenize{diffsph.profiles:diffsph.profiles.templates.enst}]{\sphinxcrossref{\sphinxcode{\sphinxupquote{diffsph.profiles.templates.enst()}}}}} profile

\item {} 
\sphinxAtStartPar
\sphinxstyleliteralstrong{\sphinxupquote{rc}} \textendash{} core radius parameter \(r_c\) in the {\hyperref[\detokenize{diffsph.profiles:diffsph.profiles.templates.cnfw}]{\sphinxcrossref{\sphinxcode{\sphinxupquote{diffsph.profiles.templates.cnfw()}}}}} profile

\item {} 
\sphinxAtStartPar
\sphinxstyleliteralstrong{\sphinxupquote{sigmav}} \textendash{} velocity dispersion parameter \(\sigma_v\) in the {\hyperref[\detokenize{diffsph.profiles:diffsph.profiles.templates.sis}]{\sphinxcrossref{\sphinxcode{\sphinxupquote{diffsph.profiles.templates.sis()}}}}} profile

\end{itemize}

\item[{Returns}] \leavevmode
\sphinxAtStartPar
Brightness temperature in mK

\end{description}\end{quote}

\end{fulllineitems}

\index{synch\_brightness() (in module diffsph.pyflux)@\spxentry{synch\_brightness()}\spxextra{in module diffsph.pyflux}}

\begin{fulllineitems}
\phantomsection\label{\detokenize{diffsph:diffsph.pyflux.synch_brightness}}\pysiglinewithargsret{\sphinxcode{\sphinxupquote{diffsph.pyflux.}}\sphinxbfcode{\sphinxupquote{synch\_brightness}}}{\emph{\DUrole{n}{theta}}, \emph{\DUrole{n}{nu}}, \emph{\DUrole{n}{galaxy}}, \emph{\DUrole{n}{rad\_temp}}, \emph{\DUrole{n}{hyp}\DUrole{o}{=}\DUrole{default_value}{\textquotesingle{}wimp\textquotesingle{}}}, \emph{\DUrole{n}{ratio}\DUrole{o}{=}\DUrole{default_value}{1}}, \emph{\DUrole{n}{D0}\DUrole{o}{=}\DUrole{default_value}{3e+28}}, \emph{\DUrole{n}{delta}\DUrole{o}{=}\DUrole{default_value}{\textquotesingle{}kol\textquotesingle{}}}, \emph{\DUrole{n}{B}\DUrole{o}{=}\DUrole{default_value}{2}}, \emph{\DUrole{n}{manual}\DUrole{o}{=}\DUrole{default_value}{False}}, \emph{\DUrole{n}{high\_res}\DUrole{o}{=}\DUrole{default_value}{False}}, \emph{\DUrole{n}{accuracy}\DUrole{o}{=}\DUrole{default_value}{1}}, \emph{\DUrole{o}{**}\DUrole{n}{kwargs}}}{}
\sphinxAtStartPar
Model\sphinxhyphen{}specific brightness from synchrotron radiation
\begin{quote}\begin{description}
\item[{Parameters}] \leavevmode\begin{itemize}
\item {} 
\sphinxAtStartPar
\sphinxstyleliteralstrong{\sphinxupquote{theta}} \textendash{} angular radius in arcmin

\item {} 
\sphinxAtStartPar
\sphinxstyleliteralstrong{\sphinxupquote{nu}} \textendash{} frequency in GHz

\item {} 
\sphinxAtStartPar
\sphinxstyleliteralstrong{\sphinxupquote{galaxy}} (\sphinxstyleliteralemphasis{\sphinxupquote{str}}) \textendash{} name of the galaxy

\item {} 
\sphinxAtStartPar
\sphinxstyleliteralstrong{\sphinxupquote{rad\_temp}} (\sphinxstyleliteralemphasis{\sphinxupquote{str}}) \textendash{} radial template (\sphinxcode{\sphinxupquote{\textquotesingle{}NFW\textquotesingle{}}}, \sphinxcode{\sphinxupquote{\textquotesingle{}Einasto\textquotesingle{}}}, etc.)

\item {} 
\sphinxAtStartPar
\sphinxstyleliteralstrong{\sphinxupquote{hyp}} (\sphinxstyleliteralemphasis{\sphinxupquote{str}}) \textendash{} hypothesis: \sphinxcode{\sphinxupquote{\textquotesingle{}wimp\textquotesingle{}}} (\sphinxstylestrong{default}), \sphinxcode{\sphinxupquote{\textquotesingle{}decay\textquotesingle{}}} or \sphinxcode{\sphinxupquote{\textquotesingle{}generic\textquotesingle{}}}

\item {} 
\sphinxAtStartPar
\sphinxstyleliteralstrong{\sphinxupquote{ratio}} \textendash{} ratio between the diffusion halo/bulge and half\sphinxhyphen{}light radii (default value = 1)

\item {} 
\sphinxAtStartPar
\sphinxstyleliteralstrong{\sphinxupquote{D0}} \textendash{} magnitude of the diffusion coefficient for a 1 GeV CRE in cm \({}^2\)/s (default value = \(3\times 10^{28}\) cm \({}^2\) /s)

\item {} 
\sphinxAtStartPar
\sphinxstyleliteralstrong{\sphinxupquote{delta}} (\sphinxstyleliteralemphasis{\sphinxupquote{float}}\sphinxstyleliteralemphasis{\sphinxupquote{, }}\sphinxstyleliteralemphasis{\sphinxupquote{str}}) \textendash{} power\sphinxhyphen{}law exponent of the diffusion coefficient as a function of the CRE’s energy (default value = 1/3 or \sphinxcode{\sphinxupquote{\textquotesingle{}kol\textquotesingle{}}})

\item {} 
\sphinxAtStartPar
\sphinxstyleliteralstrong{\sphinxupquote{B}} \textendash{} magnitude of the magnetic field’s smooth component in \(\mu\) G (default value \(= 2 \mu\) G)

\item {} 
\sphinxAtStartPar
\sphinxstyleliteralstrong{\sphinxupquote{manual}} (\sphinxstyleliteralemphasis{\sphinxupquote{bool}}) \textendash{} manual input of parameter values in rad\_temp (default value = \sphinxcode{\sphinxupquote{\textquotesingle{}False\textquotesingle{}}})

\item {} 
\sphinxAtStartPar
\sphinxstyleliteralstrong{\sphinxupquote{high\_res}} (\sphinxstyleliteralemphasis{\sphinxupquote{bool}}) \textendash{} spatial resolution. If \sphinxcode{\sphinxupquote{\textquotesingle{}True\textquotesingle{}}}, {\hyperref[\detokenize{diffsph:diffsph.pyflux.synch_emissivity}]{\sphinxcrossref{\sphinxcode{\sphinxupquote{synch\_emissivity()}}}}} computes as many terms as needed in order to converge at \(r=0\). (default value = \sphinxcode{\sphinxupquote{\textquotesingle{}False\textquotesingle{}}})

\item {} 
\sphinxAtStartPar
\sphinxstyleliteralstrong{\sphinxupquote{accuracy}} \textendash{} theoretical accuracy in \% (default value = 1\%)

\end{itemize}

\end{description}\end{quote}

\sphinxAtStartPar
Keyword arguments
\begin{itemize}
\item {} 
\sphinxAtStartPar
\sphinxcode{\sphinxupquote{hyp = \textquotesingle{}wimp\textquotesingle{}}}    (default)

\end{itemize}
\begin{quote}\begin{description}
\item[{Parameters}] \leavevmode\begin{itemize}
\item {} 
\sphinxAtStartPar
\sphinxstyleliteralstrong{\sphinxupquote{sv}} \textendash{} annihilation rate (annihilation cross section times relative velocity) \(\sigma v\) in cm \({}^3\)/s (default value = \(3 \times 10^{-26}\) cm \({}^3\) /s)

\item {} 
\sphinxAtStartPar
\sphinxstyleliteralstrong{\sphinxupquote{self\_conjugate}} \textendash{} if set \sphinxcode{\sphinxupquote{\textquotesingle{}True\textquotesingle{}}} (default value) the DM particle is its own antiparticle

\item {} 
\sphinxAtStartPar
\sphinxstyleliteralstrong{\sphinxupquote{mchi}} \textendash{} mass of the DM particle in GeV/c \({}^2\)

\item {} 
\sphinxAtStartPar
\sphinxstyleliteralstrong{\sphinxupquote{channel}} (\sphinxstyleliteralemphasis{\sphinxupquote{str}}) \textendash{} annihilation channel: \(b\bar b\) (\sphinxcode{\sphinxupquote{\textquotesingle{}bb\textquotesingle{}}}), \(\mu^+ \mu^-\) (\sphinxcode{\sphinxupquote{\textquotesingle{}mumu\textquotesingle{}}}), \(W^+ W^-\) (\sphinxcode{\sphinxupquote{\textquotesingle{}WW\textquotesingle{}}}), etc.

\end{itemize}

\end{description}\end{quote}
\begin{itemize}
\item {} 
\sphinxAtStartPar
\sphinxcode{\sphinxupquote{hyp = \textquotesingle{}decay\textquotesingle{}}}

\end{itemize}
\begin{quote}\begin{description}
\item[{Parameters}] \leavevmode\begin{itemize}
\item {} 
\sphinxAtStartPar
\sphinxstyleliteralstrong{\sphinxupquote{width}} \textendash{} decay width of the DM particle in 1/s

\item {} 
\sphinxAtStartPar
\sphinxstyleliteralstrong{\sphinxupquote{mchi}} \textendash{} mass of the DM particle in GeV/c \({}^2\)

\item {} 
\sphinxAtStartPar
\sphinxstyleliteralstrong{\sphinxupquote{channel}} (\sphinxstyleliteralemphasis{\sphinxupquote{str}}) \textendash{} decay channel: \(b\bar b\) (\sphinxcode{\sphinxupquote{\textquotesingle{}bb\textquotesingle{}}}), \(\mu^+ \mu^-\) (\sphinxcode{\sphinxupquote{\textquotesingle{}mumu\textquotesingle{}}}), \(W^+ W^-\) (\sphinxcode{\sphinxupquote{\textquotesingle{}WW\textquotesingle{}}}), etc.

\end{itemize}

\end{description}\end{quote}
\begin{itemize}
\item {} 
\sphinxAtStartPar
\sphinxcode{\sphinxupquote{hyp = \textquotesingle{}generic\textquotesingle{}}}

\end{itemize}
\begin{quote}\begin{description}
\item[{Parameters}] \leavevmode\begin{itemize}
\item {} 
\sphinxAtStartPar
\sphinxstyleliteralstrong{\sphinxupquote{Gamma}} \textendash{} power\sphinxhyphen{}law exponent of the generic CRE source (\(1.1 < \Gamma < 3\))

\item {} 
\sphinxAtStartPar
\sphinxstyleliteralstrong{\sphinxupquote{rate}} \textendash{} CRE production rate in 1/s

\end{itemize}

\end{description}\end{quote}
\begin{itemize}
\item {} 
\sphinxAtStartPar
\sphinxcode{\sphinxupquote{manual = \textquotesingle{}False\textquotesingle{}}}

\end{itemize}
\begin{quote}\begin{description}
\item[{Parameters}] \leavevmode
\sphinxAtStartPar
\sphinxstyleliteralstrong{\sphinxupquote{ref}} \textendash{} reference used (\sphinxcode{\sphinxupquote{\textquotesingle{}Martinez\textquotesingle{}}} or \sphinxcode{\sphinxupquote{\textquotesingle{}1309.2641\textquotesingle{}}}, \sphinxcode{\sphinxupquote{\textquotesingle{}Geringer\sphinxhyphen{}Sameth\textquotesingle{}}} or \sphinxcode{\sphinxupquote{\textquotesingle{}1408.0002\textquotesingle{}}}, etc.)

\end{description}\end{quote}
\begin{itemize}
\item {} 
\sphinxAtStartPar
\sphinxcode{\sphinxupquote{manual = \textquotesingle{}True\textquotesingle{}}}

\end{itemize}
\begin{quote}\begin{description}
\item[{Parameters}] \leavevmode\begin{itemize}
\item {} 
\sphinxAtStartPar
\sphinxstyleliteralstrong{\sphinxupquote{rs}} \textendash{} scale radius in kpc

\item {} 
\sphinxAtStartPar
\sphinxstyleliteralstrong{\sphinxupquote{rhos}} \textendash{} characteristic density in GeV/cm \({}^3\)

\item {} 
\sphinxAtStartPar
\sphinxstyleliteralstrong{\sphinxupquote{alpha}} \textendash{} exponent \(\alpha\) in the {\hyperref[\detokenize{diffsph.profiles:diffsph.profiles.templates.hdz}]{\sphinxcrossref{\sphinxcode{\sphinxupquote{diffsph.profiles.templates.hdz()}}}}} profile

\item {} 
\sphinxAtStartPar
\sphinxstyleliteralstrong{\sphinxupquote{beta}} \textendash{} exponent \(\beta\) in the {\hyperref[\detokenize{diffsph.profiles:diffsph.profiles.templates.hdz}]{\sphinxcrossref{\sphinxcode{\sphinxupquote{diffsph.profiles.templates.hdz()}}}}} profile

\item {} 
\sphinxAtStartPar
\sphinxstyleliteralstrong{\sphinxupquote{gamma}} \textendash{} exponent \(\gamma\) in the {\hyperref[\detokenize{diffsph.profiles:diffsph.profiles.templates.hdz}]{\sphinxcrossref{\sphinxcode{\sphinxupquote{diffsph.profiles.templates.hdz()}}}}} profile

\item {} 
\sphinxAtStartPar
\sphinxstyleliteralstrong{\sphinxupquote{alphaE}} \textendash{} parameter \(\alpha_E\) in the {\hyperref[\detokenize{diffsph.profiles:diffsph.profiles.templates.enst}]{\sphinxcrossref{\sphinxcode{\sphinxupquote{diffsph.profiles.templates.enst()}}}}} profile

\item {} 
\sphinxAtStartPar
\sphinxstyleliteralstrong{\sphinxupquote{rc}} \textendash{} core radius parameter \(r_c\) in the {\hyperref[\detokenize{diffsph.profiles:diffsph.profiles.templates.cnfw}]{\sphinxcrossref{\sphinxcode{\sphinxupquote{diffsph.profiles.templates.cnfw()}}}}} profile

\item {} 
\sphinxAtStartPar
\sphinxstyleliteralstrong{\sphinxupquote{sigmav}} \textendash{} velocity dispersion parameter \(\sigma_v\) in the {\hyperref[\detokenize{diffsph.profiles:diffsph.profiles.templates.sis}]{\sphinxcrossref{\sphinxcode{\sphinxupquote{diffsph.profiles.templates.sis()}}}}} profile

\end{itemize}

\item[{Returns}] \leavevmode
\sphinxAtStartPar
Brightness in Jy/sr

\item[{Return type}] \leavevmode
\sphinxAtStartPar
float

\end{description}\end{quote}

\end{fulllineitems}

\index{synch\_brightness\_approx() (in module diffsph.pyflux)@\spxentry{synch\_brightness\_approx()}\spxextra{in module diffsph.pyflux}}

\begin{fulllineitems}
\phantomsection\label{\detokenize{diffsph:diffsph.pyflux.synch_brightness_approx}}\pysiglinewithargsret{\sphinxcode{\sphinxupquote{diffsph.pyflux.}}\sphinxbfcode{\sphinxupquote{synch\_brightness\_approx}}}{\emph{\DUrole{n}{theta}}, \emph{\DUrole{n}{nu}}, \emph{\DUrole{n}{galaxy}}, \emph{\DUrole{n}{rad\_temp}}, \emph{\DUrole{n}{hyp}\DUrole{o}{=}\DUrole{default_value}{\textquotesingle{}wimp\textquotesingle{}}}, \emph{\DUrole{n}{ratio}\DUrole{o}{=}\DUrole{default_value}{1}}, \emph{\DUrole{n}{D0}\DUrole{o}{=}\DUrole{default_value}{3e+28}}, \emph{\DUrole{n}{delta}\DUrole{o}{=}\DUrole{default_value}{\textquotesingle{}kol\textquotesingle{}}}, \emph{\DUrole{n}{B}\DUrole{o}{=}\DUrole{default_value}{2}}, \emph{\DUrole{n}{regime}\DUrole{o}{=}\DUrole{default_value}{\textquotesingle{}B\textquotesingle{}}}, \emph{\DUrole{n}{manual}\DUrole{o}{=}\DUrole{default_value}{False}}, \emph{\DUrole{o}{**}\DUrole{n}{kwargs}}}{}
\sphinxAtStartPar
Model\sphinxhyphen{}specific brightness from synchrotron radiation in the Regime “A”, “B” or “C” approximations
\begin{quote}\begin{description}
\item[{Parameters}] \leavevmode\begin{itemize}
\item {} 
\sphinxAtStartPar
\sphinxstyleliteralstrong{\sphinxupquote{theta}} \textendash{} angular radius in arcmin

\item {} 
\sphinxAtStartPar
\sphinxstyleliteralstrong{\sphinxupquote{nu}} \textendash{} frequency in GHz

\item {} 
\sphinxAtStartPar
\sphinxstyleliteralstrong{\sphinxupquote{galaxy}} (\sphinxstyleliteralemphasis{\sphinxupquote{str}}) \textendash{} name of the galaxy

\item {} 
\sphinxAtStartPar
\sphinxstyleliteralstrong{\sphinxupquote{rad\_temp}} (\sphinxstyleliteralemphasis{\sphinxupquote{str}}) \textendash{} radial template (\sphinxcode{\sphinxupquote{\textquotesingle{}NFW\textquotesingle{}}}, \sphinxcode{\sphinxupquote{\textquotesingle{}Einasto\textquotesingle{}}}, etc.)

\item {} 
\sphinxAtStartPar
\sphinxstyleliteralstrong{\sphinxupquote{hyp}} (\sphinxstyleliteralemphasis{\sphinxupquote{str}}) \textendash{} hypothesis: \sphinxcode{\sphinxupquote{\textquotesingle{}wimp\textquotesingle{}}} (\sphinxstylestrong{default}), \sphinxcode{\sphinxupquote{\textquotesingle{}decay\textquotesingle{}}} or \sphinxcode{\sphinxupquote{\textquotesingle{}generic\textquotesingle{}}}

\item {} 
\sphinxAtStartPar
\sphinxstyleliteralstrong{\sphinxupquote{ratio}} \textendash{} ratio between the diffusion halo/bulge and half\sphinxhyphen{}light radii (default value = 1)

\item {} 
\sphinxAtStartPar
\sphinxstyleliteralstrong{\sphinxupquote{D0}} \textendash{} magnitude of the diffusion coefficient for a 1 GeV CRE in cm \({}^2\)/s (default value = \(3\times 10^{28}\) cm \({}^2\) /s)

\item {} 
\sphinxAtStartPar
\sphinxstyleliteralstrong{\sphinxupquote{delta}} (\sphinxstyleliteralemphasis{\sphinxupquote{float}}\sphinxstyleliteralemphasis{\sphinxupquote{, }}\sphinxstyleliteralemphasis{\sphinxupquote{str}}) \textendash{} power\sphinxhyphen{}law exponent of the diffusion coefficient as a function of the CRE’s energy (default value = 1/3 or \sphinxcode{\sphinxupquote{\textquotesingle{}kol\textquotesingle{}}})

\item {} 
\sphinxAtStartPar
\sphinxstyleliteralstrong{\sphinxupquote{B}} \textendash{} magnitude of the magnetic field’s smooth component in \(\mu\) G (default value \(= 2 \mu\) G)

\item {} 
\sphinxAtStartPar
\sphinxstyleliteralstrong{\sphinxupquote{regime}} \textendash{} regime of the approximation. Must be either upper or lower case a, b, c or I/II/III.

\item {} 
\sphinxAtStartPar
\sphinxstyleliteralstrong{\sphinxupquote{manual}} (\sphinxstyleliteralemphasis{\sphinxupquote{bool}}) \textendash{} manual input of parameter values in rad\_temp (default value = \sphinxcode{\sphinxupquote{\textquotesingle{}False\textquotesingle{}}})

\end{itemize}

\end{description}\end{quote}

\sphinxAtStartPar
Keyword arguments
\begin{itemize}
\item {} 
\sphinxAtStartPar
\sphinxcode{\sphinxupquote{hyp = \textquotesingle{}wimp\textquotesingle{}}}    (default)

\end{itemize}
\begin{quote}\begin{description}
\item[{Parameters}] \leavevmode\begin{itemize}
\item {} 
\sphinxAtStartPar
\sphinxstyleliteralstrong{\sphinxupquote{sv}} \textendash{} annihilation rate (annihilation cross section times relative velocity) \(\sigma v\) in cm \({}^3\)/s (default value = \(3 \times 10^{-26}\) cm \({}^3\) /s)

\item {} 
\sphinxAtStartPar
\sphinxstyleliteralstrong{\sphinxupquote{self\_conjugate}} \textendash{} if set \sphinxcode{\sphinxupquote{\textquotesingle{}True\textquotesingle{}}} (default value) the DM particle is its own antiparticle

\item {} 
\sphinxAtStartPar
\sphinxstyleliteralstrong{\sphinxupquote{mchi}} \textendash{} mass of the DM particle in GeV/c \({}^2\)

\item {} 
\sphinxAtStartPar
\sphinxstyleliteralstrong{\sphinxupquote{channel}} (\sphinxstyleliteralemphasis{\sphinxupquote{str}}) \textendash{} annihilation channel: \(b\bar b\) (\sphinxcode{\sphinxupquote{\textquotesingle{}bb\textquotesingle{}}}), \(\mu^+ \mu^-\) (\sphinxcode{\sphinxupquote{\textquotesingle{}mumu\textquotesingle{}}}), \(W^+ W^-\) (\sphinxcode{\sphinxupquote{\textquotesingle{}WW\textquotesingle{}}}), etc.

\end{itemize}

\end{description}\end{quote}
\begin{itemize}
\item {} 
\sphinxAtStartPar
\sphinxcode{\sphinxupquote{hyp = \textquotesingle{}decay\textquotesingle{}}}

\end{itemize}
\begin{quote}\begin{description}
\item[{Parameters}] \leavevmode\begin{itemize}
\item {} 
\sphinxAtStartPar
\sphinxstyleliteralstrong{\sphinxupquote{width}} \textendash{} decay width of the DM particle in 1/s

\item {} 
\sphinxAtStartPar
\sphinxstyleliteralstrong{\sphinxupquote{mchi}} \textendash{} mass of the DM particle in GeV/c \({}^2\)

\item {} 
\sphinxAtStartPar
\sphinxstyleliteralstrong{\sphinxupquote{channel}} (\sphinxstyleliteralemphasis{\sphinxupquote{str}}) \textendash{} decay channel: \(b\bar b\) (\sphinxcode{\sphinxupquote{\textquotesingle{}bb\textquotesingle{}}}), \(\mu^+ \mu^-\) (\sphinxcode{\sphinxupquote{\textquotesingle{}mumu\textquotesingle{}}}), \(W^+ W^-\) (\sphinxcode{\sphinxupquote{\textquotesingle{}WW\textquotesingle{}}}), etc.

\end{itemize}

\end{description}\end{quote}
\begin{itemize}
\item {} 
\sphinxAtStartPar
\sphinxcode{\sphinxupquote{hyp = \textquotesingle{}generic\textquotesingle{}}}

\end{itemize}
\begin{quote}\begin{description}
\item[{Parameters}] \leavevmode\begin{itemize}
\item {} 
\sphinxAtStartPar
\sphinxstyleliteralstrong{\sphinxupquote{Gamma}} \textendash{} power\sphinxhyphen{}law exponent of the generic CRE source (\(1.1 < \Gamma < 3\))

\item {} 
\sphinxAtStartPar
\sphinxstyleliteralstrong{\sphinxupquote{rate}} \textendash{} CRE production rate in 1/s

\end{itemize}

\end{description}\end{quote}
\begin{itemize}
\item {} 
\sphinxAtStartPar
\sphinxcode{\sphinxupquote{manual = \textquotesingle{}False\textquotesingle{}}}

\end{itemize}
\begin{quote}\begin{description}
\item[{Parameters}] \leavevmode
\sphinxAtStartPar
\sphinxstyleliteralstrong{\sphinxupquote{ref}} \textendash{} reference used (\sphinxcode{\sphinxupquote{\textquotesingle{}Martinez\textquotesingle{}}} or \sphinxcode{\sphinxupquote{\textquotesingle{}1309.2641\textquotesingle{}}}, \sphinxcode{\sphinxupquote{\textquotesingle{}Geringer\sphinxhyphen{}Sameth\textquotesingle{}}} or \sphinxcode{\sphinxupquote{\textquotesingle{}1408.0002\textquotesingle{}}}, etc.)

\end{description}\end{quote}
\begin{itemize}
\item {} 
\sphinxAtStartPar
\sphinxcode{\sphinxupquote{manual = \textquotesingle{}True\textquotesingle{}}}

\end{itemize}
\begin{quote}\begin{description}
\item[{Parameters}] \leavevmode\begin{itemize}
\item {} 
\sphinxAtStartPar
\sphinxstyleliteralstrong{\sphinxupquote{rs}} \textendash{} scale radius in kpc

\item {} 
\sphinxAtStartPar
\sphinxstyleliteralstrong{\sphinxupquote{rhos}} \textendash{} characteristic density in GeV/cm \({}^3\)

\item {} 
\sphinxAtStartPar
\sphinxstyleliteralstrong{\sphinxupquote{alpha}} \textendash{} exponent \(\alpha\) in the {\hyperref[\detokenize{diffsph.profiles:diffsph.profiles.templates.hdz}]{\sphinxcrossref{\sphinxcode{\sphinxupquote{diffsph.profiles.templates.hdz()}}}}} profile

\item {} 
\sphinxAtStartPar
\sphinxstyleliteralstrong{\sphinxupquote{beta}} \textendash{} exponent \(\beta\) in the {\hyperref[\detokenize{diffsph.profiles:diffsph.profiles.templates.hdz}]{\sphinxcrossref{\sphinxcode{\sphinxupquote{diffsph.profiles.templates.hdz()}}}}} profile

\item {} 
\sphinxAtStartPar
\sphinxstyleliteralstrong{\sphinxupquote{gamma}} \textendash{} exponent \(\gamma\) in the {\hyperref[\detokenize{diffsph.profiles:diffsph.profiles.templates.hdz}]{\sphinxcrossref{\sphinxcode{\sphinxupquote{diffsph.profiles.templates.hdz()}}}}} profile

\item {} 
\sphinxAtStartPar
\sphinxstyleliteralstrong{\sphinxupquote{alphaE}} \textendash{} parameter \(\alpha_E\) in the {\hyperref[\detokenize{diffsph.profiles:diffsph.profiles.templates.enst}]{\sphinxcrossref{\sphinxcode{\sphinxupquote{diffsph.profiles.templates.enst()}}}}} profile

\item {} 
\sphinxAtStartPar
\sphinxstyleliteralstrong{\sphinxupquote{rc}} \textendash{} core radius parameter \(r_c\) in the {\hyperref[\detokenize{diffsph.profiles:diffsph.profiles.templates.cnfw}]{\sphinxcrossref{\sphinxcode{\sphinxupquote{diffsph.profiles.templates.cnfw()}}}}} profile

\item {} 
\sphinxAtStartPar
\sphinxstyleliteralstrong{\sphinxupquote{sigmav}} \textendash{} velocity dispersion parameter \(\sigma_v\) in the {\hyperref[\detokenize{diffsph.profiles:diffsph.profiles.templates.sis}]{\sphinxcrossref{\sphinxcode{\sphinxupquote{diffsph.profiles.templates.sis()}}}}} profile

\end{itemize}

\item[{Returns}] \leavevmode
\sphinxAtStartPar
Brightness in Jy/sr

\end{description}\end{quote}

\end{fulllineitems}

\index{synch\_emissivity() (in module diffsph.pyflux)@\spxentry{synch\_emissivity()}\spxextra{in module diffsph.pyflux}}

\begin{fulllineitems}
\phantomsection\label{\detokenize{diffsph:diffsph.pyflux.synch_emissivity}}\pysiglinewithargsret{\sphinxcode{\sphinxupquote{diffsph.pyflux.}}\sphinxbfcode{\sphinxupquote{synch\_emissivity}}}{\emph{\DUrole{n}{r}}, \emph{\DUrole{n}{nu}}, \emph{\DUrole{n}{galaxy}}, \emph{\DUrole{n}{rad\_temp}}, \emph{\DUrole{n}{hyp}\DUrole{o}{=}\DUrole{default_value}{\textquotesingle{}wimp\textquotesingle{}}}, \emph{\DUrole{n}{ratio}\DUrole{o}{=}\DUrole{default_value}{1}}, \emph{\DUrole{n}{D0}\DUrole{o}{=}\DUrole{default_value}{3e+28}}, \emph{\DUrole{n}{delta}\DUrole{o}{=}\DUrole{default_value}{\textquotesingle{}kol\textquotesingle{}}}, \emph{\DUrole{n}{B}\DUrole{o}{=}\DUrole{default_value}{2}}, \emph{\DUrole{n}{manual}\DUrole{o}{=}\DUrole{default_value}{False}}, \emph{\DUrole{n}{high\_res}\DUrole{o}{=}\DUrole{default_value}{False}}, \emph{\DUrole{n}{accuracy}\DUrole{o}{=}\DUrole{default_value}{1}}, \emph{\DUrole{o}{**}\DUrole{n}{kwargs}}}{}
\sphinxAtStartPar
Model\sphinxhyphen{}specific emissivity from synchrotron radiation
\begin{quote}\begin{description}
\item[{Parameters}] \leavevmode\begin{itemize}
\item {} 
\sphinxAtStartPar
\sphinxstyleliteralstrong{\sphinxupquote{r}} \textendash{} galactocentric distance in kpc

\item {} 
\sphinxAtStartPar
\sphinxstyleliteralstrong{\sphinxupquote{nu}} \textendash{} frequency in GHz

\item {} 
\sphinxAtStartPar
\sphinxstyleliteralstrong{\sphinxupquote{galaxy}} (\sphinxstyleliteralemphasis{\sphinxupquote{str}}) \textendash{} name of the galaxy

\item {} 
\sphinxAtStartPar
\sphinxstyleliteralstrong{\sphinxupquote{rad\_temp}} (\sphinxstyleliteralemphasis{\sphinxupquote{str}}) \textendash{} radial template (\sphinxcode{\sphinxupquote{\textquotesingle{}NFW\textquotesingle{}}}, \sphinxcode{\sphinxupquote{\textquotesingle{}Einasto\textquotesingle{}}}, etc.)

\item {} 
\sphinxAtStartPar
\sphinxstyleliteralstrong{\sphinxupquote{hyp}} (\sphinxstyleliteralemphasis{\sphinxupquote{str}}) \textendash{} hypothesis: \sphinxcode{\sphinxupquote{\textquotesingle{}wimp\textquotesingle{}}} (\sphinxstylestrong{default}), \sphinxcode{\sphinxupquote{\textquotesingle{}decay\textquotesingle{}}} or \sphinxcode{\sphinxupquote{\textquotesingle{}generic\textquotesingle{}}}

\item {} 
\sphinxAtStartPar
\sphinxstyleliteralstrong{\sphinxupquote{ratio}} \textendash{} ratio between the diffusion halo/bulge and half\sphinxhyphen{}light radii (default value = 1)

\item {} 
\sphinxAtStartPar
\sphinxstyleliteralstrong{\sphinxupquote{D0}} \textendash{} magnitude of the diffusion coefficient for a 1 GeV CRE in cm \({}^2\)/s (default value = \(3\times 10^{28}\) cm \({}^2\) /s)

\item {} 
\sphinxAtStartPar
\sphinxstyleliteralstrong{\sphinxupquote{delta}} (\sphinxstyleliteralemphasis{\sphinxupquote{float}}\sphinxstyleliteralemphasis{\sphinxupquote{, }}\sphinxstyleliteralemphasis{\sphinxupquote{str}}) \textendash{} power\sphinxhyphen{}law exponent of the diffusion coefficient as a function of the CRE’s energy (default value = 1/3 or \sphinxcode{\sphinxupquote{\textquotesingle{}kol\textquotesingle{}}})

\item {} 
\sphinxAtStartPar
\sphinxstyleliteralstrong{\sphinxupquote{B}} \textendash{} magnitude of the magnetic field’s smooth component in \(\mu\) G (default value \(= 2 \mu\) G)

\item {} 
\sphinxAtStartPar
\sphinxstyleliteralstrong{\sphinxupquote{manual}} (\sphinxstyleliteralemphasis{\sphinxupquote{bool}}) \textendash{} manual input of parameter values in rad\_temp (default value = \sphinxcode{\sphinxupquote{\textquotesingle{}False\textquotesingle{}}})

\item {} 
\sphinxAtStartPar
\sphinxstyleliteralstrong{\sphinxupquote{high\_res}} (\sphinxstyleliteralemphasis{\sphinxupquote{bool}}) \textendash{} spatial resolution. If \sphinxcode{\sphinxupquote{\textquotesingle{}True\textquotesingle{}}}, {\hyperref[\detokenize{diffsph:diffsph.pyflux.synch_emissivity}]{\sphinxcrossref{\sphinxcode{\sphinxupquote{synch\_emissivity()}}}}} computes as many terms as needed in order to converge at \(r=0\) (default value = \sphinxcode{\sphinxupquote{\textquotesingle{}False\textquotesingle{}}})

\item {} 
\sphinxAtStartPar
\sphinxstyleliteralstrong{\sphinxupquote{accuracy}} \textendash{} theoretical accuracy in \% (default value = 1\%)

\end{itemize}

\end{description}\end{quote}

\sphinxAtStartPar
Keyword arguments
\begin{itemize}
\item {} 
\sphinxAtStartPar
\sphinxcode{\sphinxupquote{hyp = \textquotesingle{}wimp\textquotesingle{}}}    (default)

\end{itemize}
\begin{quote}\begin{description}
\item[{Parameters}] \leavevmode\begin{itemize}
\item {} 
\sphinxAtStartPar
\sphinxstyleliteralstrong{\sphinxupquote{sv}} \textendash{} annihilation rate (annihilation cross section times relative velocity) \(\sigma v\) in cm \({}^3\)/s (default value = \(3 \times 10^{-26}\) cm \({}^3\) /s)

\item {} 
\sphinxAtStartPar
\sphinxstyleliteralstrong{\sphinxupquote{self\_conjugate}} \textendash{} if set \sphinxcode{\sphinxupquote{\textquotesingle{}True\textquotesingle{}}} (default value) the DM particle is its own antiparticle

\item {} 
\sphinxAtStartPar
\sphinxstyleliteralstrong{\sphinxupquote{mchi}} \textendash{} mass of the DM particle in GeV/c \({}^2\)

\item {} 
\sphinxAtStartPar
\sphinxstyleliteralstrong{\sphinxupquote{channel}} (\sphinxstyleliteralemphasis{\sphinxupquote{str}}) \textendash{} annihilation channel: \(b\bar b\) (\sphinxcode{\sphinxupquote{\textquotesingle{}bb\textquotesingle{}}}), \(\mu^+ \mu^-\) (\sphinxcode{\sphinxupquote{\textquotesingle{}mumu\textquotesingle{}}}), \(W^+ W^-\) (\sphinxcode{\sphinxupquote{\textquotesingle{}WW\textquotesingle{}}}), etc.

\end{itemize}

\end{description}\end{quote}
\begin{itemize}
\item {} 
\sphinxAtStartPar
\sphinxcode{\sphinxupquote{hyp = \textquotesingle{}decay\textquotesingle{}}}

\end{itemize}
\begin{quote}\begin{description}
\item[{Parameters}] \leavevmode\begin{itemize}
\item {} 
\sphinxAtStartPar
\sphinxstyleliteralstrong{\sphinxupquote{width}} \textendash{} decay width of the DM particle in 1/s

\item {} 
\sphinxAtStartPar
\sphinxstyleliteralstrong{\sphinxupquote{mchi}} \textendash{} mass of the DM particle in GeV/c \({}^2\)

\item {} 
\sphinxAtStartPar
\sphinxstyleliteralstrong{\sphinxupquote{channel}} (\sphinxstyleliteralemphasis{\sphinxupquote{str}}) \textendash{} decay channel: \(b\bar b\) (\sphinxcode{\sphinxupquote{\textquotesingle{}bb\textquotesingle{}}}), \(\mu^+ \mu^-\) (\sphinxcode{\sphinxupquote{\textquotesingle{}mumu\textquotesingle{}}}), \(W^+ W^-\) (\sphinxcode{\sphinxupquote{\textquotesingle{}WW\textquotesingle{}}}), etc.

\end{itemize}

\end{description}\end{quote}
\begin{itemize}
\item {} 
\sphinxAtStartPar
\sphinxcode{\sphinxupquote{hyp = \textquotesingle{}generic\textquotesingle{}}}

\end{itemize}
\begin{quote}\begin{description}
\item[{Parameters}] \leavevmode\begin{itemize}
\item {} 
\sphinxAtStartPar
\sphinxstyleliteralstrong{\sphinxupquote{Gamma}} \textendash{} power\sphinxhyphen{}law exponent of the generic CRE source (\(1.1 < \Gamma < 3\))

\item {} 
\sphinxAtStartPar
\sphinxstyleliteralstrong{\sphinxupquote{rate}} \textendash{} CRE production rate in 1/s

\end{itemize}

\end{description}\end{quote}
\begin{itemize}
\item {} 
\sphinxAtStartPar
\sphinxcode{\sphinxupquote{manual = \textquotesingle{}False\textquotesingle{}}}

\end{itemize}
\begin{quote}\begin{description}
\item[{Parameters}] \leavevmode
\sphinxAtStartPar
\sphinxstyleliteralstrong{\sphinxupquote{ref}} \textendash{} reference used (\sphinxcode{\sphinxupquote{\textquotesingle{}Martinez\textquotesingle{}}} or \sphinxcode{\sphinxupquote{\textquotesingle{}1309.2641\textquotesingle{}}}, \sphinxcode{\sphinxupquote{\textquotesingle{}Geringer\sphinxhyphen{}Sameth\textquotesingle{}}} or \sphinxcode{\sphinxupquote{\textquotesingle{}1408.0002\textquotesingle{}}}, etc.)

\end{description}\end{quote}
\begin{itemize}
\item {} 
\sphinxAtStartPar
\sphinxcode{\sphinxupquote{manual = \textquotesingle{}True\textquotesingle{}}}

\end{itemize}
\begin{quote}\begin{description}
\item[{Parameters}] \leavevmode\begin{itemize}
\item {} 
\sphinxAtStartPar
\sphinxstyleliteralstrong{\sphinxupquote{rs}} \textendash{} scale radius in kpc

\item {} 
\sphinxAtStartPar
\sphinxstyleliteralstrong{\sphinxupquote{rhos}} \textendash{} characteristic density in GeV/cm \({}^3\)

\item {} 
\sphinxAtStartPar
\sphinxstyleliteralstrong{\sphinxupquote{alpha}} \textendash{} exponent \(\alpha\) in the {\hyperref[\detokenize{diffsph.profiles:diffsph.profiles.templates.hdz}]{\sphinxcrossref{\sphinxcode{\sphinxupquote{diffsph.profiles.templates.hdz()}}}}} profile

\item {} 
\sphinxAtStartPar
\sphinxstyleliteralstrong{\sphinxupquote{beta}} \textendash{} exponent \(\beta\) in the {\hyperref[\detokenize{diffsph.profiles:diffsph.profiles.templates.hdz}]{\sphinxcrossref{\sphinxcode{\sphinxupquote{diffsph.profiles.templates.hdz()}}}}} profile

\item {} 
\sphinxAtStartPar
\sphinxstyleliteralstrong{\sphinxupquote{gamma}} \textendash{} exponent \(\gamma\) in the {\hyperref[\detokenize{diffsph.profiles:diffsph.profiles.templates.hdz}]{\sphinxcrossref{\sphinxcode{\sphinxupquote{diffsph.profiles.templates.hdz()}}}}} profile

\item {} 
\sphinxAtStartPar
\sphinxstyleliteralstrong{\sphinxupquote{alphaE}} \textendash{} parameter \(\alpha_E\) in the {\hyperref[\detokenize{diffsph.profiles:diffsph.profiles.templates.enst}]{\sphinxcrossref{\sphinxcode{\sphinxupquote{diffsph.profiles.templates.enst()}}}}} profile

\item {} 
\sphinxAtStartPar
\sphinxstyleliteralstrong{\sphinxupquote{rc}} \textendash{} core radius parameter \(r_c\) in the {\hyperref[\detokenize{diffsph.profiles:diffsph.profiles.templates.cnfw}]{\sphinxcrossref{\sphinxcode{\sphinxupquote{diffsph.profiles.templates.cnfw()}}}}} profile

\item {} 
\sphinxAtStartPar
\sphinxstyleliteralstrong{\sphinxupquote{sigmav}} \textendash{} velocity dispersion parameter \(\sigma_v\) in the {\hyperref[\detokenize{diffsph.profiles:diffsph.profiles.templates.sis}]{\sphinxcrossref{\sphinxcode{\sphinxupquote{diffsph.profiles.templates.sis()}}}}} profile

\end{itemize}

\item[{Returns}] \leavevmode
\sphinxAtStartPar
Emissivity in erg/cm \({}^3\) /Hz/s/sr

\item[{Return type}] \leavevmode
\sphinxAtStartPar
float

\end{description}\end{quote}

\end{fulllineitems}

\index{synch\_emissivity\_approx() (in module diffsph.pyflux)@\spxentry{synch\_emissivity\_approx()}\spxextra{in module diffsph.pyflux}}

\begin{fulllineitems}
\phantomsection\label{\detokenize{diffsph:diffsph.pyflux.synch_emissivity_approx}}\pysiglinewithargsret{\sphinxcode{\sphinxupquote{diffsph.pyflux.}}\sphinxbfcode{\sphinxupquote{synch\_emissivity\_approx}}}{\emph{\DUrole{n}{r}}, \emph{\DUrole{n}{nu}}, \emph{\DUrole{n}{galaxy}}, \emph{\DUrole{n}{rad\_temp}}, \emph{\DUrole{n}{hyp}\DUrole{o}{=}\DUrole{default_value}{\textquotesingle{}wimp\textquotesingle{}}}, \emph{\DUrole{n}{ratio}\DUrole{o}{=}\DUrole{default_value}{1}}, \emph{\DUrole{n}{D0}\DUrole{o}{=}\DUrole{default_value}{3e+28}}, \emph{\DUrole{n}{delta}\DUrole{o}{=}\DUrole{default_value}{\textquotesingle{}kol\textquotesingle{}}}, \emph{\DUrole{n}{B}\DUrole{o}{=}\DUrole{default_value}{2}}, \emph{\DUrole{n}{regime}\DUrole{o}{=}\DUrole{default_value}{\textquotesingle{}B\textquotesingle{}}}, \emph{\DUrole{n}{manual}\DUrole{o}{=}\DUrole{default_value}{False}}, \emph{\DUrole{o}{**}\DUrole{n}{kwargs}}}{}
\sphinxAtStartPar
Model\sphinxhyphen{}specific emissivity from synchrotron radiation in the Regime “A”, “B” or “C” approximations
\begin{quote}\begin{description}
\item[{Parameters}] \leavevmode\begin{itemize}
\item {} 
\sphinxAtStartPar
\sphinxstyleliteralstrong{\sphinxupquote{r}} \textendash{} galactocentric distance in kpc

\item {} 
\sphinxAtStartPar
\sphinxstyleliteralstrong{\sphinxupquote{nu}} \textendash{} frequency in GHz

\item {} 
\sphinxAtStartPar
\sphinxstyleliteralstrong{\sphinxupquote{galaxy}} (\sphinxstyleliteralemphasis{\sphinxupquote{str}}) \textendash{} name of the galaxy

\item {} 
\sphinxAtStartPar
\sphinxstyleliteralstrong{\sphinxupquote{rad\_temp}} (\sphinxstyleliteralemphasis{\sphinxupquote{str}}) \textendash{} radial template (\sphinxcode{\sphinxupquote{\textquotesingle{}NFW\textquotesingle{}}}, \sphinxcode{\sphinxupquote{\textquotesingle{}Einasto\textquotesingle{}}}, etc.)

\item {} 
\sphinxAtStartPar
\sphinxstyleliteralstrong{\sphinxupquote{hyp}} (\sphinxstyleliteralemphasis{\sphinxupquote{str}}) \textendash{} hypothesis: \sphinxcode{\sphinxupquote{\textquotesingle{}wimp\textquotesingle{}}} (\sphinxstylestrong{default}), \sphinxcode{\sphinxupquote{\textquotesingle{}decay\textquotesingle{}}} or \sphinxcode{\sphinxupquote{\textquotesingle{}generic\textquotesingle{}}}

\item {} 
\sphinxAtStartPar
\sphinxstyleliteralstrong{\sphinxupquote{ratio}} \textendash{} ratio between the diffusion halo/bulge and half\sphinxhyphen{}light radii (default value = 1)

\item {} 
\sphinxAtStartPar
\sphinxstyleliteralstrong{\sphinxupquote{D0}} \textendash{} magnitude of the diffusion coefficient for a 1 GeV CRE in cm \({}^2\)/s (default value = \(3\times 10^{28}\) cm \({}^2\) /s)

\item {} 
\sphinxAtStartPar
\sphinxstyleliteralstrong{\sphinxupquote{delta}} (\sphinxstyleliteralemphasis{\sphinxupquote{float}}\sphinxstyleliteralemphasis{\sphinxupquote{, }}\sphinxstyleliteralemphasis{\sphinxupquote{str}}) \textendash{} power\sphinxhyphen{}law exponent of the diffusion coefficient as a function of the CRE’s energy (default value = 1/3 or \sphinxcode{\sphinxupquote{\textquotesingle{}kol\textquotesingle{}}})

\item {} 
\sphinxAtStartPar
\sphinxstyleliteralstrong{\sphinxupquote{B}} \textendash{} magnitude of the magnetic field’s smooth component in \(\mu\) G (default value \(= 2 \mu\) G)

\item {} 
\sphinxAtStartPar
\sphinxstyleliteralstrong{\sphinxupquote{regime}} \textendash{} regime of the approximation. Must be either upper or lower case a, b, c or I/II/III.

\item {} 
\sphinxAtStartPar
\sphinxstyleliteralstrong{\sphinxupquote{manual}} (\sphinxstyleliteralemphasis{\sphinxupquote{bool}}) \textendash{} manual input of parameter values in rad\_temp (default value = \sphinxcode{\sphinxupquote{\textquotesingle{}False\textquotesingle{}}})

\end{itemize}

\end{description}\end{quote}

\sphinxAtStartPar
Keyword arguments
\begin{itemize}
\item {} 
\sphinxAtStartPar
\sphinxcode{\sphinxupquote{hyp = \textquotesingle{}wimp\textquotesingle{}}}    (default)

\end{itemize}
\begin{quote}\begin{description}
\item[{Parameters}] \leavevmode\begin{itemize}
\item {} 
\sphinxAtStartPar
\sphinxstyleliteralstrong{\sphinxupquote{sv}} \textendash{} annihilation rate (annihilation cross section times relative velocity) \(\sigma v\) in cm \({}^3\)/s (default value = \(3 \times 10^{-26}\) cm \({}^3\) /s)

\item {} 
\sphinxAtStartPar
\sphinxstyleliteralstrong{\sphinxupquote{self\_conjugate}} \textendash{} if set \sphinxcode{\sphinxupquote{\textquotesingle{}True\textquotesingle{}}} (default value) the DM particle is its own antiparticle

\item {} 
\sphinxAtStartPar
\sphinxstyleliteralstrong{\sphinxupquote{mchi}} \textendash{} mass of the DM particle in GeV/c \({}^2\)

\item {} 
\sphinxAtStartPar
\sphinxstyleliteralstrong{\sphinxupquote{channel}} (\sphinxstyleliteralemphasis{\sphinxupquote{str}}) \textendash{} annihilation channel: \(b\bar b\) (\sphinxcode{\sphinxupquote{\textquotesingle{}bb\textquotesingle{}}}), \(\mu^+ \mu^-\) (\sphinxcode{\sphinxupquote{\textquotesingle{}mumu\textquotesingle{}}}), \(W^+ W^-\) (\sphinxcode{\sphinxupquote{\textquotesingle{}WW\textquotesingle{}}}), etc.

\end{itemize}

\end{description}\end{quote}
\begin{itemize}
\item {} 
\sphinxAtStartPar
\sphinxcode{\sphinxupquote{hyp = \textquotesingle{}decay\textquotesingle{}}}

\end{itemize}
\begin{quote}\begin{description}
\item[{Parameters}] \leavevmode\begin{itemize}
\item {} 
\sphinxAtStartPar
\sphinxstyleliteralstrong{\sphinxupquote{width}} \textendash{} decay width of the DM particle in 1/s

\item {} 
\sphinxAtStartPar
\sphinxstyleliteralstrong{\sphinxupquote{mchi}} \textendash{} mass of the DM particle in GeV/c \({}^2\)

\item {} 
\sphinxAtStartPar
\sphinxstyleliteralstrong{\sphinxupquote{channel}} (\sphinxstyleliteralemphasis{\sphinxupquote{str}}) \textendash{} decay channel: \(b\bar b\) (\sphinxcode{\sphinxupquote{\textquotesingle{}bb\textquotesingle{}}}), \(\mu^+ \mu^-\) (\sphinxcode{\sphinxupquote{\textquotesingle{}mumu\textquotesingle{}}}), \(W^+ W^-\) (\sphinxcode{\sphinxupquote{\textquotesingle{}WW\textquotesingle{}}}), etc.

\end{itemize}

\end{description}\end{quote}
\begin{itemize}
\item {} 
\sphinxAtStartPar
\sphinxcode{\sphinxupquote{hyp = \textquotesingle{}generic\textquotesingle{}}}

\end{itemize}
\begin{quote}\begin{description}
\item[{Parameters}] \leavevmode\begin{itemize}
\item {} 
\sphinxAtStartPar
\sphinxstyleliteralstrong{\sphinxupquote{Gamma}} \textendash{} power\sphinxhyphen{}law exponent of the generic CRE source (\(1.1 < \Gamma < 3\))

\item {} 
\sphinxAtStartPar
\sphinxstyleliteralstrong{\sphinxupquote{rate}} \textendash{} CRE production rate in 1/s

\end{itemize}

\end{description}\end{quote}
\begin{itemize}
\item {} 
\sphinxAtStartPar
\sphinxcode{\sphinxupquote{manual = \textquotesingle{}False\textquotesingle{}}}

\end{itemize}
\begin{quote}\begin{description}
\item[{Parameters}] \leavevmode
\sphinxAtStartPar
\sphinxstyleliteralstrong{\sphinxupquote{ref}} \textendash{} reference used (\sphinxcode{\sphinxupquote{\textquotesingle{}Martinez\textquotesingle{}}} or \sphinxcode{\sphinxupquote{\textquotesingle{}1309.2641\textquotesingle{}}}, \sphinxcode{\sphinxupquote{\textquotesingle{}Geringer\sphinxhyphen{}Sameth\textquotesingle{}}} or \sphinxcode{\sphinxupquote{\textquotesingle{}1408.0002\textquotesingle{}}}, etc.)

\end{description}\end{quote}
\begin{itemize}
\item {} 
\sphinxAtStartPar
\sphinxcode{\sphinxupquote{manual = \textquotesingle{}True\textquotesingle{}}}

\end{itemize}
\begin{quote}\begin{description}
\item[{Parameters}] \leavevmode\begin{itemize}
\item {} 
\sphinxAtStartPar
\sphinxstyleliteralstrong{\sphinxupquote{rs}} \textendash{} scale radius in kpc

\item {} 
\sphinxAtStartPar
\sphinxstyleliteralstrong{\sphinxupquote{rhos}} \textendash{} characteristic density in GeV/cm \({}^3\)

\item {} 
\sphinxAtStartPar
\sphinxstyleliteralstrong{\sphinxupquote{alpha}} \textendash{} exponent \(\alpha\) in the {\hyperref[\detokenize{diffsph.profiles:diffsph.profiles.templates.hdz}]{\sphinxcrossref{\sphinxcode{\sphinxupquote{diffsph.profiles.templates.hdz()}}}}} profile

\item {} 
\sphinxAtStartPar
\sphinxstyleliteralstrong{\sphinxupquote{beta}} \textendash{} exponent \(\beta\) in the {\hyperref[\detokenize{diffsph.profiles:diffsph.profiles.templates.hdz}]{\sphinxcrossref{\sphinxcode{\sphinxupquote{diffsph.profiles.templates.hdz()}}}}} profile

\item {} 
\sphinxAtStartPar
\sphinxstyleliteralstrong{\sphinxupquote{gamma}} \textendash{} exponent \(\gamma\) in the {\hyperref[\detokenize{diffsph.profiles:diffsph.profiles.templates.hdz}]{\sphinxcrossref{\sphinxcode{\sphinxupquote{diffsph.profiles.templates.hdz()}}}}} profile

\item {} 
\sphinxAtStartPar
\sphinxstyleliteralstrong{\sphinxupquote{alphaE}} \textendash{} parameter \(\alpha_E\) in the {\hyperref[\detokenize{diffsph.profiles:diffsph.profiles.templates.enst}]{\sphinxcrossref{\sphinxcode{\sphinxupquote{diffsph.profiles.templates.enst()}}}}} profile

\item {} 
\sphinxAtStartPar
\sphinxstyleliteralstrong{\sphinxupquote{rc}} \textendash{} core radius parameter \(r_c\) in the {\hyperref[\detokenize{diffsph.profiles:diffsph.profiles.templates.cnfw}]{\sphinxcrossref{\sphinxcode{\sphinxupquote{diffsph.profiles.templates.cnfw()}}}}} profile

\item {} 
\sphinxAtStartPar
\sphinxstyleliteralstrong{\sphinxupquote{sigmav}} \textendash{} velocity dispersion parameter \(\sigma_v\) in the {\hyperref[\detokenize{diffsph.profiles:diffsph.profiles.templates.sis}]{\sphinxcrossref{\sphinxcode{\sphinxupquote{diffsph.profiles.templates.sis()}}}}} profile

\end{itemize}

\item[{Returns}] \leavevmode
\sphinxAtStartPar
Emissivity in erg/cm \({}^3\) /Hz/s/sr

\end{description}\end{quote}

\end{fulllineitems}

\index{synch\_flux\_density() (in module diffsph.pyflux)@\spxentry{synch\_flux\_density()}\spxextra{in module diffsph.pyflux}}

\begin{fulllineitems}
\phantomsection\label{\detokenize{diffsph:diffsph.pyflux.synch_flux_density}}\pysiglinewithargsret{\sphinxcode{\sphinxupquote{diffsph.pyflux.}}\sphinxbfcode{\sphinxupquote{synch\_flux\_density}}}{\emph{\DUrole{n}{theta}}, \emph{\DUrole{n}{nu}}, \emph{\DUrole{n}{galaxy}}, \emph{\DUrole{n}{rad\_temp}}, \emph{\DUrole{n}{hyp}\DUrole{o}{=}\DUrole{default_value}{\textquotesingle{}wimp\textquotesingle{}}}, \emph{\DUrole{n}{ratio}\DUrole{o}{=}\DUrole{default_value}{1}}, \emph{\DUrole{n}{D0}\DUrole{o}{=}\DUrole{default_value}{3e+28}}, \emph{\DUrole{n}{delta}\DUrole{o}{=}\DUrole{default_value}{\textquotesingle{}kol\textquotesingle{}}}, \emph{\DUrole{n}{B}\DUrole{o}{=}\DUrole{default_value}{2}}, \emph{\DUrole{n}{manual}\DUrole{o}{=}\DUrole{default_value}{False}}, \emph{\DUrole{n}{high\_res}\DUrole{o}{=}\DUrole{default_value}{False}}, \emph{\DUrole{n}{accuracy}\DUrole{o}{=}\DUrole{default_value}{1}}, \emph{\DUrole{o}{**}\DUrole{n}{kwargs}}}{}
\sphinxAtStartPar
Model\sphinxhyphen{}specific flux density from synchrotron radiation
\begin{quote}\begin{description}
\item[{Parameters}] \leavevmode\begin{itemize}
\item {} 
\sphinxAtStartPar
\sphinxstyleliteralstrong{\sphinxupquote{theta}} \textendash{} angular radius in arcmin

\item {} 
\sphinxAtStartPar
\sphinxstyleliteralstrong{\sphinxupquote{nu}} \textendash{} frequency in GHz

\item {} 
\sphinxAtStartPar
\sphinxstyleliteralstrong{\sphinxupquote{galaxy}} (\sphinxstyleliteralemphasis{\sphinxupquote{str}}) \textendash{} name of the galaxy

\item {} 
\sphinxAtStartPar
\sphinxstyleliteralstrong{\sphinxupquote{rad\_temp}} (\sphinxstyleliteralemphasis{\sphinxupquote{str}}) \textendash{} radial template (\sphinxcode{\sphinxupquote{\textquotesingle{}NFW\textquotesingle{}}}, \sphinxcode{\sphinxupquote{\textquotesingle{}Einasto\textquotesingle{}}}, etc.)

\item {} 
\sphinxAtStartPar
\sphinxstyleliteralstrong{\sphinxupquote{hyp}} (\sphinxstyleliteralemphasis{\sphinxupquote{str}}) \textendash{} hypothesis: \sphinxcode{\sphinxupquote{\textquotesingle{}wimp\textquotesingle{}}} (\sphinxstylestrong{default}), \sphinxcode{\sphinxupquote{\textquotesingle{}decay\textquotesingle{}}} or \sphinxcode{\sphinxupquote{\textquotesingle{}generic\textquotesingle{}}}

\item {} 
\sphinxAtStartPar
\sphinxstyleliteralstrong{\sphinxupquote{ratio}} \textendash{} ratio between the diffusion halo/bulge and half\sphinxhyphen{}light radii (default value = 1)

\item {} 
\sphinxAtStartPar
\sphinxstyleliteralstrong{\sphinxupquote{D0}} \textendash{} magnitude of the diffusion coefficient for a 1 GeV CRE in cm \({}^2\)/s (default value = \(3\times 10^{28}\) cm \({}^2\) /s)

\item {} 
\sphinxAtStartPar
\sphinxstyleliteralstrong{\sphinxupquote{delta}} (\sphinxstyleliteralemphasis{\sphinxupquote{float}}\sphinxstyleliteralemphasis{\sphinxupquote{, }}\sphinxstyleliteralemphasis{\sphinxupquote{str}}) \textendash{} power\sphinxhyphen{}law exponent of the diffusion coefficient as a function of the CRE’s energy (default value = 1/3 or \sphinxcode{\sphinxupquote{\textquotesingle{}kol\textquotesingle{}}})

\item {} 
\sphinxAtStartPar
\sphinxstyleliteralstrong{\sphinxupquote{B}} \textendash{} magnitude of the magnetic field’s smooth component in \(\mu\) G (default value \(= 2 \mu\) G)

\item {} 
\sphinxAtStartPar
\sphinxstyleliteralstrong{\sphinxupquote{manual}} (\sphinxstyleliteralemphasis{\sphinxupquote{bool}}) \textendash{} manual input of parameter values in rad\_temp (default value = \sphinxcode{\sphinxupquote{\textquotesingle{}False\textquotesingle{}}})

\item {} 
\sphinxAtStartPar
\sphinxstyleliteralstrong{\sphinxupquote{high\_res}} (\sphinxstyleliteralemphasis{\sphinxupquote{bool}}) \textendash{} spatial resolution. If \sphinxcode{\sphinxupquote{\textquotesingle{}True\textquotesingle{}}}, {\hyperref[\detokenize{diffsph:diffsph.pyflux.synch_emissivity}]{\sphinxcrossref{\sphinxcode{\sphinxupquote{synch\_emissivity()}}}}} computes as many terms as needed in order to converge at \(r=0\). (default value = \sphinxcode{\sphinxupquote{\textquotesingle{}False\textquotesingle{}}})

\item {} 
\sphinxAtStartPar
\sphinxstyleliteralstrong{\sphinxupquote{accuracy}} \textendash{} theoretical accuracy in \% (default value = 1\%)

\end{itemize}

\end{description}\end{quote}

\sphinxAtStartPar
Keyword arguments
\begin{itemize}
\item {} 
\sphinxAtStartPar
\sphinxcode{\sphinxupquote{hyp = \textquotesingle{}wimp\textquotesingle{}}}    (default)

\end{itemize}
\begin{quote}\begin{description}
\item[{Parameters}] \leavevmode\begin{itemize}
\item {} 
\sphinxAtStartPar
\sphinxstyleliteralstrong{\sphinxupquote{sv}} \textendash{} annihilation rate (annihilation cross section times relative velocity) \(\sigma v\) in cm \({}^3\)/s (default value = \(3 \times 10^{-26}\) cm \({}^3\) /s)

\item {} 
\sphinxAtStartPar
\sphinxstyleliteralstrong{\sphinxupquote{self\_conjugate}} \textendash{} if set \sphinxcode{\sphinxupquote{\textquotesingle{}True\textquotesingle{}}} (default value) the DM particle is its own antiparticle

\item {} 
\sphinxAtStartPar
\sphinxstyleliteralstrong{\sphinxupquote{mchi}} \textendash{} mass of the DM particle in GeV/c \({}^2\)

\item {} 
\sphinxAtStartPar
\sphinxstyleliteralstrong{\sphinxupquote{channel}} (\sphinxstyleliteralemphasis{\sphinxupquote{str}}) \textendash{} annihilation channel: \(b\bar b\) (\sphinxcode{\sphinxupquote{\textquotesingle{}bb\textquotesingle{}}}), \(\mu^+ \mu^-\) (\sphinxcode{\sphinxupquote{\textquotesingle{}mumu\textquotesingle{}}}), \(W^+ W^-\) (\sphinxcode{\sphinxupquote{\textquotesingle{}WW\textquotesingle{}}}), etc.

\end{itemize}

\end{description}\end{quote}
\begin{itemize}
\item {} 
\sphinxAtStartPar
\sphinxcode{\sphinxupquote{hyp = \textquotesingle{}decay\textquotesingle{}}}

\end{itemize}
\begin{quote}\begin{description}
\item[{Parameters}] \leavevmode\begin{itemize}
\item {} 
\sphinxAtStartPar
\sphinxstyleliteralstrong{\sphinxupquote{width}} \textendash{} decay width of the DM particle in 1/s

\item {} 
\sphinxAtStartPar
\sphinxstyleliteralstrong{\sphinxupquote{mchi}} \textendash{} mass of the DM particle in GeV/c \({}^2\)

\item {} 
\sphinxAtStartPar
\sphinxstyleliteralstrong{\sphinxupquote{channel}} (\sphinxstyleliteralemphasis{\sphinxupquote{str}}) \textendash{} decay channel: \(b\bar b\) (\sphinxcode{\sphinxupquote{\textquotesingle{}bb\textquotesingle{}}}), \(\mu^+ \mu^-\) (\sphinxcode{\sphinxupquote{\textquotesingle{}mumu\textquotesingle{}}}), \(W^+ W^-\) (\sphinxcode{\sphinxupquote{\textquotesingle{}WW\textquotesingle{}}}), etc.

\end{itemize}

\end{description}\end{quote}
\begin{itemize}
\item {} 
\sphinxAtStartPar
\sphinxcode{\sphinxupquote{hyp = \textquotesingle{}generic\textquotesingle{}}}

\end{itemize}
\begin{quote}\begin{description}
\item[{Parameters}] \leavevmode\begin{itemize}
\item {} 
\sphinxAtStartPar
\sphinxstyleliteralstrong{\sphinxupquote{Gamma}} \textendash{} power\sphinxhyphen{}law exponent of the generic CRE source (\(1.1 < \Gamma < 3\))

\item {} 
\sphinxAtStartPar
\sphinxstyleliteralstrong{\sphinxupquote{rate}} \textendash{} CRE production rate in 1/s

\end{itemize}

\end{description}\end{quote}
\begin{itemize}
\item {} 
\sphinxAtStartPar
\sphinxcode{\sphinxupquote{manual = \textquotesingle{}False\textquotesingle{}}}

\end{itemize}
\begin{quote}\begin{description}
\item[{Parameters}] \leavevmode
\sphinxAtStartPar
\sphinxstyleliteralstrong{\sphinxupquote{ref}} \textendash{} reference used (\sphinxcode{\sphinxupquote{\textquotesingle{}Martinez\textquotesingle{}}} or \sphinxcode{\sphinxupquote{\textquotesingle{}1309.2641\textquotesingle{}}}, \sphinxcode{\sphinxupquote{\textquotesingle{}Geringer\sphinxhyphen{}Sameth\textquotesingle{}}} or \sphinxcode{\sphinxupquote{\textquotesingle{}1408.0002\textquotesingle{}}}, etc.)

\end{description}\end{quote}
\begin{itemize}
\item {} 
\sphinxAtStartPar
\sphinxcode{\sphinxupquote{manual = \textquotesingle{}True\textquotesingle{}}}

\end{itemize}
\begin{quote}\begin{description}
\item[{Parameters}] \leavevmode\begin{itemize}
\item {} 
\sphinxAtStartPar
\sphinxstyleliteralstrong{\sphinxupquote{rs}} \textendash{} scale radius in kpc

\item {} 
\sphinxAtStartPar
\sphinxstyleliteralstrong{\sphinxupquote{rhos}} \textendash{} characteristic density in GeV/cm \({}^3\)

\item {} 
\sphinxAtStartPar
\sphinxstyleliteralstrong{\sphinxupquote{alpha}} \textendash{} exponent \(\alpha\) in the {\hyperref[\detokenize{diffsph.profiles:diffsph.profiles.templates.hdz}]{\sphinxcrossref{\sphinxcode{\sphinxupquote{diffsph.profiles.templates.hdz()}}}}} profile

\item {} 
\sphinxAtStartPar
\sphinxstyleliteralstrong{\sphinxupquote{beta}} \textendash{} exponent \(\beta\) in the {\hyperref[\detokenize{diffsph.profiles:diffsph.profiles.templates.hdz}]{\sphinxcrossref{\sphinxcode{\sphinxupquote{diffsph.profiles.templates.hdz()}}}}} profile

\item {} 
\sphinxAtStartPar
\sphinxstyleliteralstrong{\sphinxupquote{gamma}} \textendash{} exponent \(\gamma\) in the {\hyperref[\detokenize{diffsph.profiles:diffsph.profiles.templates.hdz}]{\sphinxcrossref{\sphinxcode{\sphinxupquote{diffsph.profiles.templates.hdz()}}}}} profile

\item {} 
\sphinxAtStartPar
\sphinxstyleliteralstrong{\sphinxupquote{alphaE}} \textendash{} parameter \(\alpha_E\) in the {\hyperref[\detokenize{diffsph.profiles:diffsph.profiles.templates.enst}]{\sphinxcrossref{\sphinxcode{\sphinxupquote{diffsph.profiles.templates.enst()}}}}} profile

\item {} 
\sphinxAtStartPar
\sphinxstyleliteralstrong{\sphinxupquote{rc}} \textendash{} core radius parameter \(r_c\) in the {\hyperref[\detokenize{diffsph.profiles:diffsph.profiles.templates.cnfw}]{\sphinxcrossref{\sphinxcode{\sphinxupquote{diffsph.profiles.templates.cnfw()}}}}} profile

\item {} 
\sphinxAtStartPar
\sphinxstyleliteralstrong{\sphinxupquote{sigmav}} \textendash{} velocity dispersion parameter \(\sigma_v\) in the {\hyperref[\detokenize{diffsph.profiles:diffsph.profiles.templates.sis}]{\sphinxcrossref{\sphinxcode{\sphinxupquote{diffsph.profiles.templates.sis()}}}}} profile

\end{itemize}

\item[{Returns}] \leavevmode
\sphinxAtStartPar
Flux density in µJy

\end{description}\end{quote}

\end{fulllineitems}

\index{synch\_flux\_density\_approx() (in module diffsph.pyflux)@\spxentry{synch\_flux\_density\_approx()}\spxextra{in module diffsph.pyflux}}

\begin{fulllineitems}
\phantomsection\label{\detokenize{diffsph:diffsph.pyflux.synch_flux_density_approx}}\pysiglinewithargsret{\sphinxcode{\sphinxupquote{diffsph.pyflux.}}\sphinxbfcode{\sphinxupquote{synch\_flux\_density\_approx}}}{\emph{\DUrole{n}{theta}}, \emph{\DUrole{n}{nu}}, \emph{\DUrole{n}{galaxy}}, \emph{\DUrole{n}{rad\_temp}}, \emph{\DUrole{n}{hyp}\DUrole{o}{=}\DUrole{default_value}{\textquotesingle{}wimp\textquotesingle{}}}, \emph{\DUrole{n}{ratio}\DUrole{o}{=}\DUrole{default_value}{1}}, \emph{\DUrole{n}{D0}\DUrole{o}{=}\DUrole{default_value}{3e+28}}, \emph{\DUrole{n}{delta}\DUrole{o}{=}\DUrole{default_value}{\textquotesingle{}kol\textquotesingle{}}}, \emph{\DUrole{n}{B}\DUrole{o}{=}\DUrole{default_value}{2}}, \emph{\DUrole{n}{regime}\DUrole{o}{=}\DUrole{default_value}{\textquotesingle{}B\textquotesingle{}}}, \emph{\DUrole{n}{manual}\DUrole{o}{=}\DUrole{default_value}{False}}, \emph{\DUrole{o}{**}\DUrole{n}{kwargs}}}{}
\sphinxAtStartPar
Model\sphinxhyphen{}specific flux density from synchrotron radiation in the Regime “A”, “B” or “C” approximations
\begin{quote}\begin{description}
\item[{Parameters}] \leavevmode\begin{itemize}
\item {} 
\sphinxAtStartPar
\sphinxstyleliteralstrong{\sphinxupquote{theta}} \textendash{} angular radius in arcmin

\item {} 
\sphinxAtStartPar
\sphinxstyleliteralstrong{\sphinxupquote{nu}} \textendash{} frequency in GHz

\item {} 
\sphinxAtStartPar
\sphinxstyleliteralstrong{\sphinxupquote{galaxy}} (\sphinxstyleliteralemphasis{\sphinxupquote{str}}) \textendash{} name of the galaxy

\item {} 
\sphinxAtStartPar
\sphinxstyleliteralstrong{\sphinxupquote{rad\_temp}} (\sphinxstyleliteralemphasis{\sphinxupquote{str}}) \textendash{} radial template (\sphinxcode{\sphinxupquote{\textquotesingle{}NFW\textquotesingle{}}}, \sphinxcode{\sphinxupquote{\textquotesingle{}Einasto\textquotesingle{}}}, etc.)

\item {} 
\sphinxAtStartPar
\sphinxstyleliteralstrong{\sphinxupquote{hyp}} (\sphinxstyleliteralemphasis{\sphinxupquote{str}}) \textendash{} hypothesis: \sphinxcode{\sphinxupquote{\textquotesingle{}wimp\textquotesingle{}}} (\sphinxstylestrong{default}), \sphinxcode{\sphinxupquote{\textquotesingle{}decay\textquotesingle{}}} or \sphinxcode{\sphinxupquote{\textquotesingle{}generic\textquotesingle{}}}

\item {} 
\sphinxAtStartPar
\sphinxstyleliteralstrong{\sphinxupquote{ratio}} \textendash{} ratio between the diffusion halo/bulge and half\sphinxhyphen{}light radii (default value = 1)

\item {} 
\sphinxAtStartPar
\sphinxstyleliteralstrong{\sphinxupquote{D0}} \textendash{} magnitude of the diffusion coefficient for a 1 GeV CRE in cm \({}^2\)/s (default value = \(3\times 10^{28}\) cm \({}^2\) /s)

\item {} 
\sphinxAtStartPar
\sphinxstyleliteralstrong{\sphinxupquote{delta}} (\sphinxstyleliteralemphasis{\sphinxupquote{float}}\sphinxstyleliteralemphasis{\sphinxupquote{, }}\sphinxstyleliteralemphasis{\sphinxupquote{str}}) \textendash{} power\sphinxhyphen{}law exponent of the diffusion coefficient as a function of the CRE’s energy (default value = 1/3 or \sphinxcode{\sphinxupquote{\textquotesingle{}kol\textquotesingle{}}})

\item {} 
\sphinxAtStartPar
\sphinxstyleliteralstrong{\sphinxupquote{B}} \textendash{} magnitude of the magnetic field’s smooth component in \(\mu\) G (default value \(= 2 \mu\) G)

\item {} 
\sphinxAtStartPar
\sphinxstyleliteralstrong{\sphinxupquote{regime}} \textendash{} regime of the approximation. Must be either upper or lower case a, b, c or I/II/III.

\item {} 
\sphinxAtStartPar
\sphinxstyleliteralstrong{\sphinxupquote{manual}} (\sphinxstyleliteralemphasis{\sphinxupquote{bool}}) \textendash{} manual input of parameter values in rad\_temp (default value = \sphinxcode{\sphinxupquote{\textquotesingle{}False\textquotesingle{}}})

\end{itemize}

\end{description}\end{quote}

\sphinxAtStartPar
Keyword arguments
\begin{itemize}
\item {} 
\sphinxAtStartPar
\sphinxcode{\sphinxupquote{hyp = \textquotesingle{}wimp\textquotesingle{}}}    (default)

\end{itemize}
\begin{quote}\begin{description}
\item[{Parameters}] \leavevmode\begin{itemize}
\item {} 
\sphinxAtStartPar
\sphinxstyleliteralstrong{\sphinxupquote{sv}} \textendash{} annihilation rate (annihilation cross section times relative velocity) \(\sigma v\) in cm \({}^3\)/s (default value = \(3 \times 10^{-26}\) cm \({}^3\) /s)

\item {} 
\sphinxAtStartPar
\sphinxstyleliteralstrong{\sphinxupquote{self\_conjugate}} \textendash{} if set \sphinxcode{\sphinxupquote{\textquotesingle{}True\textquotesingle{}}} (default value) the DM particle is its own antiparticle

\item {} 
\sphinxAtStartPar
\sphinxstyleliteralstrong{\sphinxupquote{mchi}} \textendash{} mass of the DM particle in GeV/c \({}^2\)

\item {} 
\sphinxAtStartPar
\sphinxstyleliteralstrong{\sphinxupquote{channel}} (\sphinxstyleliteralemphasis{\sphinxupquote{str}}) \textendash{} annihilation channel: \(b\bar b\) (\sphinxcode{\sphinxupquote{\textquotesingle{}bb\textquotesingle{}}}), \(\mu^+ \mu^-\) (\sphinxcode{\sphinxupquote{\textquotesingle{}mumu\textquotesingle{}}}), \(W^+ W^-\) (\sphinxcode{\sphinxupquote{\textquotesingle{}WW\textquotesingle{}}}), etc.

\end{itemize}

\end{description}\end{quote}
\begin{itemize}
\item {} 
\sphinxAtStartPar
\sphinxcode{\sphinxupquote{hyp = \textquotesingle{}decay\textquotesingle{}}}

\end{itemize}
\begin{quote}\begin{description}
\item[{Parameters}] \leavevmode\begin{itemize}
\item {} 
\sphinxAtStartPar
\sphinxstyleliteralstrong{\sphinxupquote{width}} \textendash{} decay width of the DM particle in 1/s

\item {} 
\sphinxAtStartPar
\sphinxstyleliteralstrong{\sphinxupquote{mchi}} \textendash{} mass of the DM particle in GeV/c \({}^2\)

\item {} 
\sphinxAtStartPar
\sphinxstyleliteralstrong{\sphinxupquote{channel}} (\sphinxstyleliteralemphasis{\sphinxupquote{str}}) \textendash{} decay channel: \(b\bar b\) (\sphinxcode{\sphinxupquote{\textquotesingle{}bb\textquotesingle{}}}), \(\mu^+ \mu^-\) (\sphinxcode{\sphinxupquote{\textquotesingle{}mumu\textquotesingle{}}}), \(W^+ W^-\) (\sphinxcode{\sphinxupquote{\textquotesingle{}WW\textquotesingle{}}}), etc.

\end{itemize}

\end{description}\end{quote}
\begin{itemize}
\item {} 
\sphinxAtStartPar
\sphinxcode{\sphinxupquote{hyp = \textquotesingle{}generic\textquotesingle{}}}

\end{itemize}
\begin{quote}\begin{description}
\item[{Parameters}] \leavevmode\begin{itemize}
\item {} 
\sphinxAtStartPar
\sphinxstyleliteralstrong{\sphinxupquote{Gamma}} \textendash{} power\sphinxhyphen{}law exponent of the generic CRE source (\(1.1 < \Gamma < 3\))

\item {} 
\sphinxAtStartPar
\sphinxstyleliteralstrong{\sphinxupquote{rate}} \textendash{} CRE production rate in 1/s

\end{itemize}

\end{description}\end{quote}
\begin{itemize}
\item {} 
\sphinxAtStartPar
\sphinxcode{\sphinxupquote{manual = \textquotesingle{}False\textquotesingle{}}}

\end{itemize}
\begin{quote}\begin{description}
\item[{Parameters}] \leavevmode
\sphinxAtStartPar
\sphinxstyleliteralstrong{\sphinxupquote{ref}} \textendash{} reference used (\sphinxcode{\sphinxupquote{\textquotesingle{}Martinez\textquotesingle{}}} or \sphinxcode{\sphinxupquote{\textquotesingle{}1309.2641\textquotesingle{}}}, \sphinxcode{\sphinxupquote{\textquotesingle{}Geringer\sphinxhyphen{}Sameth\textquotesingle{}}} or \sphinxcode{\sphinxupquote{\textquotesingle{}1408.0002\textquotesingle{}}}, etc.)

\end{description}\end{quote}
\begin{itemize}
\item {} 
\sphinxAtStartPar
\sphinxcode{\sphinxupquote{manual = \textquotesingle{}True\textquotesingle{}}}

\end{itemize}
\begin{quote}\begin{description}
\item[{Parameters}] \leavevmode\begin{itemize}
\item {} 
\sphinxAtStartPar
\sphinxstyleliteralstrong{\sphinxupquote{rs}} \textendash{} scale radius in kpc

\item {} 
\sphinxAtStartPar
\sphinxstyleliteralstrong{\sphinxupquote{rhos}} \textendash{} characteristic density in GeV/cm \({}^3\)

\item {} 
\sphinxAtStartPar
\sphinxstyleliteralstrong{\sphinxupquote{alpha}} \textendash{} exponent \(\alpha\) in the {\hyperref[\detokenize{diffsph.profiles:diffsph.profiles.templates.hdz}]{\sphinxcrossref{\sphinxcode{\sphinxupquote{diffsph.profiles.templates.hdz()}}}}} profile

\item {} 
\sphinxAtStartPar
\sphinxstyleliteralstrong{\sphinxupquote{beta}} \textendash{} exponent \(\beta\) in the {\hyperref[\detokenize{diffsph.profiles:diffsph.profiles.templates.hdz}]{\sphinxcrossref{\sphinxcode{\sphinxupquote{diffsph.profiles.templates.hdz()}}}}} profile

\item {} 
\sphinxAtStartPar
\sphinxstyleliteralstrong{\sphinxupquote{gamma}} \textendash{} exponent \(\gamma\) in the {\hyperref[\detokenize{diffsph.profiles:diffsph.profiles.templates.hdz}]{\sphinxcrossref{\sphinxcode{\sphinxupquote{diffsph.profiles.templates.hdz()}}}}} profile

\item {} 
\sphinxAtStartPar
\sphinxstyleliteralstrong{\sphinxupquote{alphaE}} \textendash{} parameter \(\alpha_E\) in the {\hyperref[\detokenize{diffsph.profiles:diffsph.profiles.templates.enst}]{\sphinxcrossref{\sphinxcode{\sphinxupquote{diffsph.profiles.templates.enst()}}}}} profile

\item {} 
\sphinxAtStartPar
\sphinxstyleliteralstrong{\sphinxupquote{rc}} \textendash{} core radius parameter \(r_c\) in the {\hyperref[\detokenize{diffsph.profiles:diffsph.profiles.templates.cnfw}]{\sphinxcrossref{\sphinxcode{\sphinxupquote{diffsph.profiles.templates.cnfw()}}}}} profile

\item {} 
\sphinxAtStartPar
\sphinxstyleliteralstrong{\sphinxupquote{sigmav}} \textendash{} velocity dispersion parameter \(\sigma_v\) in the {\hyperref[\detokenize{diffsph.profiles:diffsph.profiles.templates.sis}]{\sphinxcrossref{\sphinxcode{\sphinxupquote{diffsph.profiles.templates.sis()}}}}} profile

\end{itemize}

\item[{Returns}] \leavevmode
\sphinxAtStartPar
Flux density in µJy

\end{description}\end{quote}

\end{fulllineitems}

\index{which\_N() (in module diffsph.pyflux)@\spxentry{which\_N()}\spxextra{in module diffsph.pyflux}}

\begin{fulllineitems}
\phantomsection\label{\detokenize{diffsph:diffsph.pyflux.which_N}}\pysiglinewithargsret{\sphinxcode{\sphinxupquote{diffsph.pyflux.}}\sphinxbfcode{\sphinxupquote{which\_N}}}{\emph{\DUrole{n}{nu}}, \emph{\DUrole{n}{galaxy}}, \emph{\DUrole{n}{rad\_temp}}, \emph{\DUrole{n}{hyp}}, \emph{\DUrole{n}{ratio}}, \emph{\DUrole{n}{D0}}, \emph{\DUrole{n}{delta}}, \emph{\DUrole{n}{B}}, \emph{\DUrole{n}{manual}\DUrole{o}{=}\DUrole{default_value}{False}}, \emph{\DUrole{n}{high\_res}\DUrole{o}{=}\DUrole{default_value}{False}}, \emph{\DUrole{n}{accuracy}\DUrole{o}{=}\DUrole{default_value}{1}}, \emph{\DUrole{o}{**}\DUrole{n}{kwargs}}}{}
\sphinxAtStartPar
Determines at which order should the Fourier\sphinxhyphen{}expanded Green’s function solution be truncated and stores the associated \(s_n = h_n\times X_n\) coefficients as an array in the \sphinxcode{\sphinxupquote{/cache}} folder
\begin{quote}\begin{description}
\item[{Parameters}] \leavevmode\begin{itemize}
\item {} 
\sphinxAtStartPar
\sphinxstyleliteralstrong{\sphinxupquote{nu}} \textendash{} frequency in GHz

\item {} 
\sphinxAtStartPar
\sphinxstyleliteralstrong{\sphinxupquote{galaxy}} (\sphinxstyleliteralemphasis{\sphinxupquote{str}}) \textendash{} name of the galaxy

\item {} 
\sphinxAtStartPar
\sphinxstyleliteralstrong{\sphinxupquote{rad\_temp}} (\sphinxstyleliteralemphasis{\sphinxupquote{str}}) \textendash{} radial template (\sphinxcode{\sphinxupquote{\textquotesingle{}NFW\textquotesingle{}}}, \sphinxcode{\sphinxupquote{\textquotesingle{}Einasto\textquotesingle{}}}, etc.)

\item {} 
\sphinxAtStartPar
\sphinxstyleliteralstrong{\sphinxupquote{hyp}} (\sphinxstyleliteralemphasis{\sphinxupquote{str}}) \textendash{} hypothesis: \sphinxcode{\sphinxupquote{\textquotesingle{}wimp\textquotesingle{}}} (\sphinxstylestrong{default}), \sphinxcode{\sphinxupquote{\textquotesingle{}decay\textquotesingle{}}} or \sphinxcode{\sphinxupquote{\textquotesingle{}generic\textquotesingle{}}}

\item {} 
\sphinxAtStartPar
\sphinxstyleliteralstrong{\sphinxupquote{ratio}} \textendash{} ratio between the diffusion halo/bulge and half\sphinxhyphen{}light radii (default value = 1)

\item {} 
\sphinxAtStartPar
\sphinxstyleliteralstrong{\sphinxupquote{D0}} \textendash{} magnitude of the diffusion coefficient for a 1 GeV CRE in cm \({}^2\)/s (default value = \(3\times 10^{28}\) cm \({}^2\) /s)

\item {} 
\sphinxAtStartPar
\sphinxstyleliteralstrong{\sphinxupquote{delta}} (\sphinxstyleliteralemphasis{\sphinxupquote{float}}\sphinxstyleliteralemphasis{\sphinxupquote{, }}\sphinxstyleliteralemphasis{\sphinxupquote{str}}) \textendash{} power\sphinxhyphen{}law exponent of the diffusion coefficient as a function of the CRE’s energy (default value = 1/3 or \sphinxcode{\sphinxupquote{\textquotesingle{}kol\textquotesingle{}}})

\item {} 
\sphinxAtStartPar
\sphinxstyleliteralstrong{\sphinxupquote{B}} \textendash{} magnitude of the magnetic field’s smooth component in \(\mu\) G (default value \(= 2 \mu\) G)

\item {} 
\sphinxAtStartPar
\sphinxstyleliteralstrong{\sphinxupquote{manual}} (\sphinxstyleliteralemphasis{\sphinxupquote{bool}}) \textendash{} manual input of parameter values in rad\_temp (default value = \sphinxcode{\sphinxupquote{\textquotesingle{}False\textquotesingle{}}})

\item {} 
\sphinxAtStartPar
\sphinxstyleliteralstrong{\sphinxupquote{high\_res}} (\sphinxstyleliteralemphasis{\sphinxupquote{bool}}) \textendash{} spatial resolution. If \sphinxcode{\sphinxupquote{\textquotesingle{}True\textquotesingle{}}}, {\hyperref[\detokenize{diffsph:diffsph.pyflux.synch_emissivity}]{\sphinxcrossref{\sphinxcode{\sphinxupquote{synch\_emissivity()}}}}} computes as many terms as needed in order to converge at \(r=0\). (default value = \sphinxcode{\sphinxupquote{\textquotesingle{}False\textquotesingle{}}})

\item {} 
\sphinxAtStartPar
\sphinxstyleliteralstrong{\sphinxupquote{accuracy}} \textendash{} theoretical accuracy in \% (default value = 1\%)

\end{itemize}

\end{description}\end{quote}

\sphinxAtStartPar
Keyword arguments
\begin{itemize}
\item {} 
\sphinxAtStartPar
\sphinxcode{\sphinxupquote{hyp = \textquotesingle{}wimp\textquotesingle{}}}    (default)

\end{itemize}
\begin{quote}\begin{description}
\item[{Parameters}] \leavevmode\begin{itemize}
\item {} 
\sphinxAtStartPar
\sphinxstyleliteralstrong{\sphinxupquote{sv}} \textendash{} annihilation rate (annihilation cross section times relative velocity) \(\sigma v\) in cm \({}^3\)/s (default value = \(3 \times 10^{-26}\) cm \({}^3\) /s)

\item {} 
\sphinxAtStartPar
\sphinxstyleliteralstrong{\sphinxupquote{self\_conjugate}} \textendash{} if set \sphinxcode{\sphinxupquote{\textquotesingle{}True\textquotesingle{}}} (default value) the DM particle is its own antiparticle

\item {} 
\sphinxAtStartPar
\sphinxstyleliteralstrong{\sphinxupquote{mchi}} \textendash{} mass of the DM particle in GeV/c \({}^2\)

\item {} 
\sphinxAtStartPar
\sphinxstyleliteralstrong{\sphinxupquote{channel}} (\sphinxstyleliteralemphasis{\sphinxupquote{str}}) \textendash{} annihilation channel: \(b\bar b\) (\sphinxcode{\sphinxupquote{\textquotesingle{}bb\textquotesingle{}}}), \(\mu^+ \mu^-\) (\sphinxcode{\sphinxupquote{\textquotesingle{}mumu\textquotesingle{}}}), \(W^+ W^-\) (\sphinxcode{\sphinxupquote{\textquotesingle{}WW\textquotesingle{}}}), etc.

\end{itemize}

\end{description}\end{quote}
\begin{itemize}
\item {} 
\sphinxAtStartPar
\sphinxcode{\sphinxupquote{hyp = \textquotesingle{}decay\textquotesingle{}}}

\end{itemize}
\begin{quote}\begin{description}
\item[{Parameters}] \leavevmode\begin{itemize}
\item {} 
\sphinxAtStartPar
\sphinxstyleliteralstrong{\sphinxupquote{width}} \textendash{} decay width of the DM particle in 1/s

\item {} 
\sphinxAtStartPar
\sphinxstyleliteralstrong{\sphinxupquote{mchi}} \textendash{} mass of the DM particle in GeV/c \({}^2\)

\item {} 
\sphinxAtStartPar
\sphinxstyleliteralstrong{\sphinxupquote{channel}} (\sphinxstyleliteralemphasis{\sphinxupquote{str}}) \textendash{} decay channel: \(b\bar b\) (\sphinxcode{\sphinxupquote{\textquotesingle{}bb\textquotesingle{}}}), \(\mu^+ \mu^-\) (\sphinxcode{\sphinxupquote{\textquotesingle{}mumu\textquotesingle{}}}), \(W^+ W^-\) (\sphinxcode{\sphinxupquote{\textquotesingle{}WW\textquotesingle{}}}), etc.

\end{itemize}

\end{description}\end{quote}
\begin{itemize}
\item {} 
\sphinxAtStartPar
\sphinxcode{\sphinxupquote{hyp = \textquotesingle{}generic\textquotesingle{}}}

\end{itemize}
\begin{quote}\begin{description}
\item[{Parameters}] \leavevmode\begin{itemize}
\item {} 
\sphinxAtStartPar
\sphinxstyleliteralstrong{\sphinxupquote{Gamma}} \textendash{} power\sphinxhyphen{}law exponent of the generic CRE source (\(1.1 < \Gamma < 3\))

\item {} 
\sphinxAtStartPar
\sphinxstyleliteralstrong{\sphinxupquote{rate}} \textendash{} CRE production rate in 1/s

\end{itemize}

\end{description}\end{quote}
\begin{itemize}
\item {} 
\sphinxAtStartPar
\sphinxcode{\sphinxupquote{manual = \textquotesingle{}False\textquotesingle{}}}

\end{itemize}
\begin{quote}\begin{description}
\item[{Parameters}] \leavevmode
\sphinxAtStartPar
\sphinxstyleliteralstrong{\sphinxupquote{ref}} \textendash{} reference used (\sphinxcode{\sphinxupquote{\textquotesingle{}Martinez\textquotesingle{}}} or \sphinxcode{\sphinxupquote{\textquotesingle{}1309.2641\textquotesingle{}}}, \sphinxcode{\sphinxupquote{\textquotesingle{}Geringer\sphinxhyphen{}Sameth\textquotesingle{}}} or \sphinxcode{\sphinxupquote{\textquotesingle{}1408.0002\textquotesingle{}}}, etc.)

\end{description}\end{quote}
\begin{itemize}
\item {} 
\sphinxAtStartPar
\sphinxcode{\sphinxupquote{manual = \textquotesingle{}True\textquotesingle{}}}

\end{itemize}
\begin{quote}\begin{description}
\item[{Parameters}] \leavevmode\begin{itemize}
\item {} 
\sphinxAtStartPar
\sphinxstyleliteralstrong{\sphinxupquote{rs}} \textendash{} scale radius in kpc

\item {} 
\sphinxAtStartPar
\sphinxstyleliteralstrong{\sphinxupquote{rhos}} \textendash{} characteristic density in GeV/cm \({}^3\)

\item {} 
\sphinxAtStartPar
\sphinxstyleliteralstrong{\sphinxupquote{alpha}} \textendash{} exponent \(\alpha\) in the {\hyperref[\detokenize{diffsph.profiles:diffsph.profiles.templates.hdz}]{\sphinxcrossref{\sphinxcode{\sphinxupquote{diffsph.profiles.templates.hdz()}}}}} profile

\item {} 
\sphinxAtStartPar
\sphinxstyleliteralstrong{\sphinxupquote{beta}} \textendash{} exponent \(\beta\) in the {\hyperref[\detokenize{diffsph.profiles:diffsph.profiles.templates.hdz}]{\sphinxcrossref{\sphinxcode{\sphinxupquote{diffsph.profiles.templates.hdz()}}}}} profile

\item {} 
\sphinxAtStartPar
\sphinxstyleliteralstrong{\sphinxupquote{gamma}} \textendash{} exponent \(\gamma\) in the {\hyperref[\detokenize{diffsph.profiles:diffsph.profiles.templates.hdz}]{\sphinxcrossref{\sphinxcode{\sphinxupquote{diffsph.profiles.templates.hdz()}}}}} profile

\item {} 
\sphinxAtStartPar
\sphinxstyleliteralstrong{\sphinxupquote{alphaE}} \textendash{} parameter \(\alpha_E\) in the {\hyperref[\detokenize{diffsph.profiles:diffsph.profiles.templates.enst}]{\sphinxcrossref{\sphinxcode{\sphinxupquote{diffsph.profiles.templates.enst()}}}}} profile

\item {} 
\sphinxAtStartPar
\sphinxstyleliteralstrong{\sphinxupquote{rc}} \textendash{} core radius parameter \(r_c\) in the {\hyperref[\detokenize{diffsph.profiles:diffsph.profiles.templates.cnfw}]{\sphinxcrossref{\sphinxcode{\sphinxupquote{diffsph.profiles.templates.cnfw()}}}}} profile

\item {} 
\sphinxAtStartPar
\sphinxstyleliteralstrong{\sphinxupquote{sigmav}} \textendash{} velocity dispersion parameter \(\sigma_v\) in the {\hyperref[\detokenize{diffsph.profiles:diffsph.profiles.templates.sis}]{\sphinxcrossref{\sphinxcode{\sphinxupquote{diffsph.profiles.templates.sis()}}}}} profile

\end{itemize}

\item[{Returns}] \leavevmode
\sphinxAtStartPar
series truncation order \sphinxstyleemphasis{N}

\end{description}\end{quote}

\end{fulllineitems}



\subsection{Module contents}
\label{\detokenize{diffsph:module-diffsph}}\label{\detokenize{diffsph:module-contents}}\index{module@\spxentry{module}!diffsph@\spxentry{diffsph}}\index{diffsph@\spxentry{diffsph}!module@\spxentry{module}}
\begin{DUlineblock}{0em}
\item[] 
\end{DUlineblock}


\chapter{Architecture}
\label{\detokenize{index:architecture}}
\sphinxAtStartPar
The code’s structure can be summarized as follows:

\noindent\sphinxincludegraphics[width=1000\sphinxpxdimen]{{flowchart}.png}

\sphinxAtStartPar
Besides being able to compute fluxes with the \sphinxcode{\sphinxupquote{pyflux}} ({\hyperref[\detokenize{diffsph:module-diffsph.pyflux}]{\sphinxcrossref{\sphinxcode{\sphinxupquote{diffsph.pyflux}}}}}) module, the \sphinxcode{\sphinxupquote{limits}} ({\hyperref[\detokenize{diffsph:module-diffsph.limits}]{\sphinxcrossref{\sphinxcode{\sphinxupquote{diffsph.limits}}}}}) module enables users to produce 2\(\sigma\) limits on e.\textasciitilde{}g. annihilation cross sections or decay rates of dark matter particles.

\begin{DUlineblock}{0em}
\item[] 
\end{DUlineblock}


\chapter{Examples}
\label{\detokenize{index:examples}}

\section{\sphinxstyleliteralintitle{\sphinxupquote{pyflux}} module}
\label{\detokenize{index:pyflux-module}}
\sphinxAtStartPar
In order to get familiar with the code, the user can use the following set of commands in order to generate the figure below:

\begin{sphinxVerbatim}[commandchars=\\\{\}]
\PYG{k+kn}{from} \PYG{n+nn}{diffsph} \PYG{k+kn}{import} \PYG{n}{pyflux} \PYG{k}{as} \PYG{n}{pf}
\PYG{k+kn}{import} \PYG{n+nn}{matplotlib}\PYG{n+nn}{.}\PYG{n+nn}{pyplot} \PYG{k}{as} \PYG{n+nn}{plt}
\PYG{o}{\PYGZpc{}}\PYG{n}{matplotlib} \PYG{n}{inline}

\PYG{c+c1}{\PYGZsh{} Angle grid in arcmin}

\PYG{n}{theta\PYGZus{}grid} \PYG{o}{=} \PYG{p}{[}\PYG{l+m+mi}{15} \PYG{o}{*} \PYG{n}{i} \PYG{o}{/} \PYG{l+m+mi}{1000} \PYG{k}{for} \PYG{n}{i} \PYG{o+ow}{in} \PYG{n+nb}{range}\PYG{p}{(}\PYG{l+m+mi}{0}\PYG{p}{,}\PYG{l+m+mi}{1000}\PYG{p}{)}\PYG{p}{]}

\PYG{c+c1}{\PYGZsh{} List of satellite galaxies}

\PYG{n}{dsph\PYGZus{}list} \PYG{o}{=} \PYG{p}{[}\PYG{l+s+s1}{\PYGZsq{}}\PYG{l+s+s1}{Ursa Major II}\PYG{l+s+s1}{\PYGZsq{}}\PYG{p}{,} \PYG{l+s+s1}{\PYGZsq{}}\PYG{l+s+s1}{Fornax}\PYG{l+s+s1}{\PYGZsq{}}\PYG{p}{,} \PYG{l+s+s1}{\PYGZsq{}}\PYG{l+s+s1}{Ursa Minor}\PYG{l+s+s1}{\PYGZsq{}}\PYG{p}{,} \PYG{l+s+s1}{\PYGZsq{}}\PYG{l+s+s1}{Sextans}\PYG{l+s+s1}{\PYGZsq{}}\PYG{p}{]}

\PYG{c+c1}{\PYGZsh{} diffsph\PYGZsq{}s computations at nu = 150 MHz and for the given model}

\PYG{n}{Inu} \PYG{o}{=} \PYG{p}{[}\PYG{p}{[}\PYG{n}{pf}\PYG{o}{.}\PYG{n}{synch\PYGZus{}brightness}\PYG{p}{(}\PYG{n}{th}\PYG{p}{,} \PYG{n}{nu} \PYG{o}{=} \PYG{l+m+mf}{.150}\PYG{p}{,} \PYG{n}{galaxy} \PYG{o}{=} \PYG{n}{gal}\PYG{p}{,} \PYG{n}{rad\PYGZus{}temp} \PYG{o}{=} \PYG{l+s+s1}{\PYGZsq{}}\PYG{l+s+s1}{HDZ}\PYG{l+s+s1}{\PYGZsq{}}\PYG{p}{,}
                                 \PYG{n}{hyp} \PYG{o}{=} \PYG{l+s+s1}{\PYGZsq{}}\PYG{l+s+s1}{wimp}\PYG{l+s+s1}{\PYGZsq{}}\PYG{p}{,} \PYG{n}{ref} \PYG{o}{=} \PYG{l+s+s1}{\PYGZsq{}}\PYG{l+s+s1}{1408.0002}\PYG{l+s+s1}{\PYGZsq{}}\PYG{p}{,} \PYG{n}{sv} \PYG{o}{=} \PYG{l+m+mf}{3e\PYGZhy{}26}\PYG{p}{,}
                                 \PYG{n}{mchi} \PYG{o}{=} \PYG{l+m+mi}{10}\PYG{p}{,} \PYG{n}{channel} \PYG{o}{=} \PYG{l+s+s1}{\PYGZsq{}}\PYG{l+s+s1}{mumu}\PYG{l+s+s1}{\PYGZsq{}}\PYG{p}{,} \PYG{n}{high\PYGZus{}res} \PYG{o}{=} \PYG{k+kc}{True}\PYG{p}{,}
                                 \PYG{n}{accuracy} \PYG{o}{=} \PYG{l+m+mf}{.1}\PYG{p}{)}
             \PYG{k}{for} \PYG{n}{th} \PYG{o+ow}{in} \PYG{n}{theta\PYGZus{}grid}\PYG{p}{]}
            \PYG{k}{for} \PYG{n}{gal} \PYG{o+ow}{in} \PYG{n}{dsph\PYGZus{}list}\PYG{p}{]}

\PYG{c+c1}{\PYGZsh{} Plots}

\PYG{n}{plt}\PYG{o}{.}\PYG{n}{plot}\PYG{p}{(}\PYG{n}{theta\PYGZus{}grid}\PYG{p}{,} \PYG{n}{Inu}\PYG{p}{[}\PYG{l+m+mi}{0}\PYG{p}{]}\PYG{p}{,} \PYG{l+s+s2}{\PYGZdq{}}\PYG{l+s+s2}{k}\PYG{l+s+s2}{\PYGZdq{}}\PYG{p}{,} \PYG{n}{label} \PYG{o}{=} \PYG{n}{dsph\PYGZus{}list}\PYG{p}{[}\PYG{l+m+mi}{0}\PYG{p}{]}\PYG{p}{)}
\PYG{n}{plt}\PYG{o}{.}\PYG{n}{plot}\PYG{p}{(}\PYG{n}{theta\PYGZus{}grid}\PYG{p}{,} \PYG{n}{Inu}\PYG{p}{[}\PYG{l+m+mi}{1}\PYG{p}{]}\PYG{p}{,} \PYG{l+s+s2}{\PYGZdq{}}\PYG{l+s+s2}{\PYGZhy{}\PYGZhy{}k}\PYG{l+s+s2}{\PYGZdq{}}\PYG{p}{,} \PYG{n}{label} \PYG{o}{=} \PYG{n}{dsph\PYGZus{}list}\PYG{p}{[}\PYG{l+m+mi}{1}\PYG{p}{]}\PYG{p}{)}
\PYG{n}{plt}\PYG{o}{.}\PYG{n}{plot}\PYG{p}{(}\PYG{n}{theta\PYGZus{}grid}\PYG{p}{,} \PYG{n}{Inu}\PYG{p}{[}\PYG{l+m+mi}{2}\PYG{p}{]}\PYG{p}{,} \PYG{l+s+s2}{\PYGZdq{}}\PYG{l+s+s2}{:k}\PYG{l+s+s2}{\PYGZdq{}}\PYG{p}{,} \PYG{n}{label} \PYG{o}{=} \PYG{n}{dsph\PYGZus{}list}\PYG{p}{[}\PYG{l+m+mi}{2}\PYG{p}{]}\PYG{p}{)}
\PYG{n}{plt}\PYG{o}{.}\PYG{n}{plot}\PYG{p}{(}\PYG{n}{theta\PYGZus{}grid}\PYG{p}{,} \PYG{n}{Inu}\PYG{p}{[}\PYG{l+m+mi}{3}\PYG{p}{]}\PYG{p}{,} \PYG{l+s+s2}{\PYGZdq{}}\PYG{l+s+s2}{\PYGZhy{}.k}\PYG{l+s+s2}{\PYGZdq{}}\PYG{p}{,} \PYG{n}{label} \PYG{o}{=} \PYG{n}{dsph\PYGZus{}list}\PYG{p}{[}\PYG{l+m+mi}{3}\PYG{p}{]}\PYG{p}{)}
\PYG{n}{plt}\PYG{o}{.}\PYG{n}{legend}\PYG{p}{(}\PYG{p}{)}
\PYG{n}{plt}\PYG{o}{.}\PYG{n}{xlabel}\PYG{p}{(}\PYG{l+s+s1}{\PYGZsq{}}\PYG{l+s+s1}{\PYGZdl{}}\PYG{l+s+se}{\PYGZbs{}\PYGZbs{}}\PYG{l+s+s1}{theta\PYGZdl{} (arcmin)}\PYG{l+s+s1}{\PYGZsq{}}\PYG{p}{,} \PYG{n}{size} \PYG{o}{=} \PYG{l+s+s1}{\PYGZsq{}}\PYG{l+s+s1}{large}\PYG{l+s+s1}{\PYGZsq{}}\PYG{p}{)}\PYG{p}{;}
\PYG{n}{plt}\PYG{o}{.}\PYG{n}{ylabel}\PYG{p}{(}\PYG{l+s+s1}{\PYGZsq{}}\PYG{l+s+s1}{\PYGZdl{}I\PYGZus{}}\PYG{l+s+se}{\PYGZbs{}\PYGZbs{}}\PYG{l+s+s1}{nu\PYGZdl{} (Jy/sr)}\PYG{l+s+s1}{\PYGZsq{}}\PYG{p}{,} \PYG{n}{size} \PYG{o}{=} \PYG{l+s+s1}{\PYGZsq{}}\PYG{l+s+s1}{large}\PYG{l+s+s1}{\PYGZsq{}}\PYG{p}{)}\PYG{p}{;}
\PYG{n}{plt}\PYG{o}{.}\PYG{n}{title}\PYG{p}{(}\PYG{l+s+s1}{\PYGZsq{}}\PYG{l+s+s1}{Brightness profiles with diffsph}\PYG{l+s+s1}{\PYGZsq{}}\PYG{p}{)}\PYG{p}{;}
\PYG{n}{plt}\PYG{o}{.}\PYG{n}{text}\PYG{p}{(}\PYG{l+m+mf}{7.5}\PYG{p}{,} \PYG{l+m+mi}{630}\PYG{p}{,} \PYG{l+s+s1}{\PYGZsq{}}\PYG{l+s+s1}{\PYGZdl{}}\PYG{l+s+s1}{\PYGZbs{}}\PYG{l+s+s1}{chi}\PYG{l+s+s1}{\PYGZbs{}}\PYG{l+s+s1}{chi}\PYG{l+s+s1}{\PYGZbs{}}\PYG{l+s+s1}{, }\PYG{l+s+se}{\PYGZbs{}\PYGZbs{}}\PYG{l+s+s1}{to}\PYG{l+s+s1}{\PYGZbs{}}\PYG{l+s+s1}{, }\PYG{l+s+s1}{\PYGZbs{}}\PYG{l+s+s1}{mu\PYGZca{}+}\PYG{l+s+s1}{\PYGZbs{}}\PYG{l+s+s1}{mu\PYGZca{}\PYGZhy{}\PYGZdl{}}\PYG{l+s+s1}{\PYGZsq{}}\PYG{p}{,}
         \PYG{n}{horizontalalignment} \PYG{o}{=} \PYG{l+s+s1}{\PYGZsq{}}\PYG{l+s+s1}{center}\PYG{l+s+s1}{\PYGZsq{}}\PYG{p}{,} \PYG{n}{size} \PYG{o}{=} \PYG{l+s+s1}{\PYGZsq{}}\PYG{l+s+s1}{large}\PYG{l+s+s1}{\PYGZsq{}}\PYG{p}{)}\PYG{p}{;}
\end{sphinxVerbatim}

\noindent\sphinxincludegraphics[width=1000\sphinxpxdimen]{{example_pyflux}.png}

\sphinxAtStartPar
In this example, the ordering of the elements in the list \sphinxcode{\sphinxupquote{dshp\_list}} is not arbitrary. It was deliberately chosen in such a way that the total flux density of the first element is the largest while the rest are sorted in decreasing order. The following command line allows one to assess this quantitatively:

\begin{sphinxVerbatim}[commandchars=\\\{\}]
\PYG{c+c1}{\PYGZsh{} Total flux density for each galaxy in mJy}
\PYG{n}{Snu} \PYG{o}{=} \PYG{p}{[}\PYG{p}{[}\PYG{n}{gal}\PYG{p}{,} \PYG{l+m+mf}{1e\PYGZhy{}3} \PYG{o}{*} \PYG{n}{pf}\PYG{o}{.}\PYG{n}{synch\PYGZus{}flux\PYGZus{}density}\PYG{p}{(}\PYG{l+m+mi}{30}\PYG{p}{,} \PYG{n}{nu} \PYG{o}{=} \PYG{l+m+mf}{.150}\PYG{p}{,} \PYG{n}{galaxy} \PYG{o}{=} \PYG{n}{gal}\PYG{p}{,}
                                          \PYG{n}{rad\PYGZus{}temp} \PYG{o}{=} \PYG{l+s+s1}{\PYGZsq{}}\PYG{l+s+s1}{HDZ}\PYG{l+s+s1}{\PYGZsq{}}\PYG{p}{,} \PYG{n}{hyp} \PYG{o}{=} \PYG{l+s+s1}{\PYGZsq{}}\PYG{l+s+s1}{wimp}\PYG{l+s+s1}{\PYGZsq{}}\PYG{p}{,}
                                          \PYG{n}{ref} \PYG{o}{=} \PYG{l+s+s1}{\PYGZsq{}}\PYG{l+s+s1}{1408.0002}\PYG{l+s+s1}{\PYGZsq{}}\PYG{p}{,} \PYG{n}{sv} \PYG{o}{=} \PYG{l+m+mf}{3e\PYGZhy{}26}\PYG{p}{,}
                                          \PYG{n}{mchi} \PYG{o}{=} \PYG{l+m+mi}{10}\PYG{p}{,} \PYG{n}{channel} \PYG{o}{=} \PYG{l+s+s1}{\PYGZsq{}}\PYG{l+s+s1}{mumu}\PYG{l+s+s1}{\PYGZsq{}}\PYG{p}{,}
                                          \PYG{n}{high\PYGZus{}res} \PYG{o}{=} \PYG{k+kc}{True}\PYG{p}{,} \PYG{n}{accuracy} \PYG{o}{=} \PYG{l+m+mf}{.1}\PYG{p}{)}\PYG{p}{]}
            \PYG{k}{for} \PYG{n}{gal} \PYG{o+ow}{in} \PYG{n}{dsph\PYGZus{}list}\PYG{p}{]}

\PYG{c+c1}{\PYGZsh{} Print}
\PYG{n+nb}{print}\PYG{p}{(}\PYG{n}{Snu}\PYG{p}{)}

\PYG{o}{\PYGZgt{}\PYGZgt{}}\PYG{o}{\PYGZgt{}} \PYG{p}{[}\PYG{p}{[}\PYG{l+s+s1}{\PYGZsq{}}\PYG{l+s+s1}{Ursa Major II}\PYG{l+s+s1}{\PYGZsq{}}\PYG{p}{,} \PYG{l+m+mf}{5.205501952938485}\PYG{p}{]}\PYG{p}{,} \PYG{p}{[}\PYG{l+s+s1}{\PYGZsq{}}\PYG{l+s+s1}{Fornax}\PYG{l+s+s1}{\PYGZsq{}}\PYG{p}{,} \PYG{l+m+mf}{3.2288461400011492}\PYG{p}{]}\PYG{p}{,}
\PYG{p}{[}\PYG{l+s+s1}{\PYGZsq{}}\PYG{l+s+s1}{Ursa Minor}\PYG{l+s+s1}{\PYGZsq{}}\PYG{p}{,} \PYG{l+m+mf}{2.8903428197065293}\PYG{p}{]}\PYG{p}{,} \PYG{p}{[}\PYG{l+s+s1}{\PYGZsq{}}\PYG{l+s+s1}{Sextans}\PYG{l+s+s1}{\PYGZsq{}}\PYG{p}{,} \PYG{l+m+mf}{1.6467270289999836}\PYG{p}{]}\PYG{p}{]}
\end{sphinxVerbatim}


\section{\sphinxstyleliteralintitle{\sphinxupquote{limits}} module}
\label{\detokenize{index:limits-module}}
\sphinxAtStartPar
The following example shows how to obtain limits on e.\textasciitilde{}g. the decay rate of dark matter particles using the given noise level of a non\sphinxhyphen{}detection image of Draco. It takes (without parallelization) about one hour to compute all:

\begin{sphinxVerbatim}[commandchars=\\\{\}]
\PYG{k+kn}{from} \PYG{n+nn}{diffsph} \PYG{k+kn}{import} \PYG{n}{limits} \PYG{k}{as} \PYG{n}{lims}

\PYG{c+c1}{\PYGZsh{} DM mass grid in GeV}

\PYG{n}{mass\PYGZus{}grid} \PYG{o}{=} \PYG{p}{[}\PYG{l+m+mi}{100} \PYG{o}{*} \PYG{l+m+mi}{10} \PYG{o}{*}\PYG{o}{*} \PYG{p}{(}\PYG{l+m+mi}{3} \PYG{o}{*} \PYG{n}{i} \PYG{o}{/} \PYG{l+m+mi}{1000}\PYG{p}{)} \PYG{k}{for} \PYG{n}{i} \PYG{o+ow}{in} \PYG{n+nb}{range}\PYG{p}{(}\PYG{l+m+mi}{0}\PYG{p}{,}\PYG{l+m+mi}{1000}\PYG{p}{)}\PYG{p}{]}

\PYG{c+c1}{\PYGZsh{} List of decay channels}

\PYG{n}{ch\PYGZus{}list} \PYG{o}{=} \PYG{p}{[}\PYG{l+s+s1}{\PYGZsq{}}\PYG{l+s+s1}{WW}\PYG{l+s+s1}{\PYGZsq{}}\PYG{p}{,} \PYG{l+s+s1}{\PYGZsq{}}\PYG{l+s+s1}{ZZ}\PYG{l+s+s1}{\PYGZsq{}}\PYG{p}{,} \PYG{l+s+s1}{\PYGZsq{}}\PYG{l+s+s1}{hh}\PYG{l+s+s1}{\PYGZsq{}}\PYG{p}{,} \PYG{l+s+s1}{\PYGZsq{}}\PYG{l+s+s1}{nunu}\PYG{l+s+s1}{\PYGZsq{}}\PYG{p}{,} \PYG{l+s+s1}{\PYGZsq{}}\PYG{l+s+s1}{mumu}\PYG{l+s+s1}{\PYGZsq{}}\PYG{p}{,} \PYG{l+s+s1}{\PYGZsq{}}\PYG{l+s+s1}{tautau}\PYG{l+s+s1}{\PYGZsq{}}\PYG{p}{,} \PYG{l+s+s1}{\PYGZsq{}}\PYG{l+s+s1}{qq}\PYG{l+s+s1}{\PYGZsq{}}\PYG{p}{,} \PYG{l+s+s1}{\PYGZsq{}}\PYG{l+s+s1}{cc}\PYG{l+s+s1}{\PYGZsq{}}\PYG{p}{,} \PYG{l+s+s1}{\PYGZsq{}}\PYG{l+s+s1}{bb}\PYG{l+s+s1}{\PYGZsq{}}\PYG{p}{,} \PYG{l+s+s1}{\PYGZsq{}}\PYG{l+s+s1}{tt}\PYG{l+s+s1}{\PYGZsq{}}\PYG{p}{]}

\PYG{c+c1}{\PYGZsh{} diffsph\PYGZsq{}s computations at nu = 150 MHz and for the given image}

\PYG{n}{rates} \PYG{o}{=} \PYG{p}{[}\PYG{p}{[}\PYG{n}{lims}\PYG{o}{.}\PYG{n}{decay\PYGZus{}rate\PYGZus{}limest}\PYG{p}{(}\PYG{n}{nu} \PYG{o}{=} \PYG{l+m+mf}{.15}\PYG{p}{,} \PYG{n}{rms\PYGZus{}noise} \PYG{o}{=} \PYG{l+m+mi}{100}\PYG{p}{,} \PYG{n}{beam\PYGZus{}size} \PYG{o}{=} \PYG{l+m+mi}{20}\PYG{p}{,}
                                 \PYG{n}{galaxy} \PYG{o}{=} \PYG{l+s+s1}{\PYGZsq{}}\PYG{l+s+s1}{Draco}\PYG{l+s+s1}{\PYGZsq{}}\PYG{p}{,} \PYG{n}{rad\PYGZus{}temp} \PYG{o}{=} \PYG{l+s+s1}{\PYGZsq{}}\PYG{l+s+s1}{HDZ}\PYG{l+s+s1}{\PYGZsq{}}\PYG{p}{,} \PYG{n}{mchi} \PYG{o}{=} \PYG{n}{mch}\PYG{p}{,}
                                 \PYG{n}{channel} \PYG{o}{=} \PYG{n}{ch}\PYG{p}{,} \PYG{n}{high\PYGZus{}res} \PYG{o}{=} \PYG{k+kc}{True}\PYG{p}{,} \PYG{n}{accuracy} \PYG{o}{=} \PYG{l+m+mf}{.1}\PYG{p}{,}
                                 \PYG{n}{ref} \PYG{o}{=} \PYG{l+s+s1}{\PYGZsq{}}\PYG{l+s+s1}{1408.0002}\PYG{l+s+s1}{\PYGZsq{}}\PYG{p}{)}
          \PYG{k}{for} \PYG{n}{mch} \PYG{o+ow}{in} \PYG{n}{mass\PYGZus{}grid}\PYG{p}{]}
         \PYG{k}{for} \PYG{n}{ch} \PYG{o+ow}{in} \PYG{n}{ch\PYGZus{}list}\PYG{p}{]}

\PYG{c+c1}{\PYGZsh{} Plots}

\PYG{p}{[}\PYG{n}{plt}\PYG{o}{.}\PYG{n}{loglog}\PYG{p}{(}\PYG{n}{mass\PYGZus{}grid}\PYG{p}{,} \PYG{n}{rates\PYGZus{}Draco}\PYG{p}{[}\PYG{n}{i}\PYG{p}{]}\PYG{p}{,} \PYG{n}{label} \PYG{o}{=} \PYG{n}{ch\PYGZus{}list}\PYG{p}{[}\PYG{n}{i}\PYG{p}{]}\PYG{p}{,} \PYG{n}{ls} \PYG{o}{=} \PYG{l+s+s1}{\PYGZsq{}}\PYG{l+s+s1}{\PYGZhy{}\PYGZhy{}}\PYG{l+s+s1}{\PYGZsq{}}\PYG{p}{)} \PYG{k}{for} \PYG{n}{i} \PYG{o+ow}{in} \PYG{n+nb}{range}\PYG{p}{(}\PYG{l+m+mi}{0}\PYG{p}{,} \PYG{l+m+mi}{3}\PYG{p}{)}\PYG{p}{]}
\PYG{p}{[}\PYG{n}{plt}\PYG{o}{.}\PYG{n}{loglog}\PYG{p}{(}\PYG{n}{mass\PYGZus{}grid}\PYG{p}{,} \PYG{n}{rates\PYGZus{}Draco}\PYG{p}{[}\PYG{n}{i}\PYG{p}{]}\PYG{p}{,} \PYG{n}{label} \PYG{o}{=} \PYG{n}{ch\PYGZus{}list}\PYG{p}{[}\PYG{n}{i}\PYG{p}{]}\PYG{p}{,} \PYG{n}{ls} \PYG{o}{=} \PYG{l+s+s1}{\PYGZsq{}}\PYG{l+s+s1}{:}\PYG{l+s+s1}{\PYGZsq{}}\PYG{p}{)} \PYG{k}{for} \PYG{n}{i} \PYG{o+ow}{in} \PYG{n+nb}{range}\PYG{p}{(}\PYG{l+m+mi}{3}\PYG{p}{,} \PYG{l+m+mi}{6}\PYG{p}{)}\PYG{p}{]}
\PYG{p}{[}\PYG{n}{plt}\PYG{o}{.}\PYG{n}{loglog}\PYG{p}{(}\PYG{n}{mass\PYGZus{}grid}\PYG{p}{,} \PYG{n}{rates\PYGZus{}Draco}\PYG{p}{[}\PYG{n}{i}\PYG{p}{]}\PYG{p}{,} \PYG{n}{label} \PYG{o}{=} \PYG{n}{ch\PYGZus{}list}\PYG{p}{[}\PYG{n}{i}\PYG{p}{]}\PYG{p}{)} \PYG{k}{for} \PYG{n}{i} \PYG{o+ow}{in} \PYG{n+nb}{range}\PYG{p}{(}\PYG{l+m+mi}{6}\PYG{p}{,} \PYG{n+nb}{len}\PYG{p}{(}\PYG{n}{ch\PYGZus{}list}\PYG{p}{)}\PYG{p}{)}\PYG{p}{]}
\PYG{n}{plt}\PYG{o}{.}\PYG{n}{ylim}\PYG{p}{(}\PYG{p}{[}\PYG{l+m+mf}{2e\PYGZhy{}24}\PYG{p}{,}\PYG{l+m+mf}{8e\PYGZhy{}21}\PYG{p}{]}\PYG{p}{)}\PYG{p}{;}
\PYG{n}{plt}\PYG{o}{.}\PYG{n}{xlim}\PYG{p}{(}\PYG{p}{[}\PYG{l+m+mi}{200}\PYG{p}{,}\PYG{l+m+mf}{1e5}\PYG{p}{]}\PYG{p}{)}\PYG{p}{;}
\PYG{n}{plt}\PYG{o}{.}\PYG{n}{legend}\PYG{p}{(}\PYG{n}{loc} \PYG{o}{=} \PYG{l+s+s1}{\PYGZsq{}}\PYG{l+s+s1}{upper right}\PYG{l+s+s1}{\PYGZsq{}}\PYG{p}{,} \PYG{n}{ncols} \PYG{o}{=} \PYG{l+m+mi}{5}\PYG{p}{)}
\PYG{n}{plt}\PYG{o}{.}\PYG{n}{xlabel}\PYG{p}{(}\PYG{l+s+s1}{\PYGZsq{}}\PYG{l+s+s1}{\PYGZdl{}m\PYGZus{}}\PYG{l+s+s1}{\PYGZbs{}}\PYG{l+s+s1}{chi\PYGZdl{} (GeV)}\PYG{l+s+s1}{\PYGZsq{}}\PYG{p}{,} \PYG{n}{size} \PYG{o}{=} \PYG{l+s+s1}{\PYGZsq{}}\PYG{l+s+s1}{large}\PYG{l+s+s1}{\PYGZsq{}}\PYG{p}{)}\PYG{p}{;}
\PYG{n}{plt}\PYG{o}{.}\PYG{n}{ylabel}\PYG{p}{(}\PYG{l+s+s1}{\PYGZsq{}}\PYG{l+s+s1}{\PYGZdl{}}\PYG{l+s+s1}{\PYGZbs{}}\PYG{l+s+s1}{Gamma\PYGZus{}}\PYG{l+s+si}{\PYGZob{}dec\PYGZcb{}}\PYG{l+s+s1}{\PYGZdl{} (s\PYGZdl{}}\PYG{l+s+si}{\PYGZob{}\PYGZcb{}}\PYG{l+s+s1}{\PYGZca{}}\PYG{l+s+s1}{\PYGZob{}}\PYG{l+s+s1}{\PYGZhy{}1\PYGZcb{}\PYGZdl{})}\PYG{l+s+s1}{\PYGZsq{}}\PYG{p}{,} \PYG{n}{size} \PYG{o}{=} \PYG{l+s+s1}{\PYGZsq{}}\PYG{l+s+s1}{large}\PYG{l+s+s1}{\PYGZsq{}}\PYG{p}{)}\PYG{p}{;}
\PYG{n}{plt}\PYG{o}{.}\PYG{n}{title}\PYG{p}{(}\PYG{l+s+s1}{\PYGZsq{}}\PYG{l+s+s1}{diffsph 2\PYGZdl{}}\PYG{l+s+s1}{\PYGZbs{}}\PYG{l+s+s1}{sigma\PYGZdl{} limit estimates on DM decay rates}\PYG{l+s+s1}{\PYGZsq{}}\PYG{p}{)}\PYG{p}{;}
\PYG{n}{plt}\PYG{o}{.}\PYG{n}{text}\PYG{p}{(}\PYG{l+m+mf}{8e4}\PYG{p}{,} \PYG{l+m+mf}{4e\PYGZhy{}24}\PYG{p}{,} \PYG{l+s+s1}{\PYGZsq{}}\PYG{l+s+s1}{Draco}\PYG{l+s+s1}{\PYGZsq{}}\PYG{p}{,} \PYG{n}{horizontalalignment} \PYG{o}{=} \PYG{l+s+s1}{\PYGZsq{}}\PYG{l+s+s1}{right}\PYG{l+s+s1}{\PYGZsq{}}\PYG{p}{,} \PYG{n}{size} \PYG{o}{=} \PYG{l+s+s1}{\PYGZsq{}}\PYG{l+s+s1}{large}\PYG{l+s+s1}{\PYGZsq{}}\PYG{p}{)}\PYG{p}{;}
\end{sphinxVerbatim}

\noindent\sphinxincludegraphics[width=1000\sphinxpxdimen]{{example_limits}.png}


\chapter{Indices and tables}
\label{\detokenize{index:indices-and-tables}}\begin{itemize}
\item {} 
\sphinxAtStartPar
\DUrole{xref,std,std-ref}{genindex}

\item {} 
\sphinxAtStartPar
\DUrole{xref,std,std-ref}{modindex}

\item {} 
\sphinxAtStartPar
\DUrole{xref,std,std-ref}{search}

\end{itemize}
\subsubsection*{References}


\renewcommand{\indexname}{Python Module Index}
\begin{sphinxtheindex}
\let\bigletter\sphinxstyleindexlettergroup
\bigletter{d}
\item\relax\sphinxstyleindexentry{diffsph}\sphinxstyleindexpageref{diffsph:\detokenize{module-diffsph}}
\item\relax\sphinxstyleindexentry{diffsph.limits}\sphinxstyleindexpageref{diffsph:\detokenize{module-diffsph.limits}}
\item\relax\sphinxstyleindexentry{diffsph.profiles}\sphinxstyleindexpageref{diffsph.profiles:\detokenize{module-diffsph.profiles}}
\item\relax\sphinxstyleindexentry{diffsph.profiles.analytics}\sphinxstyleindexpageref{diffsph.profiles:\detokenize{module-diffsph.profiles.analytics}}
\item\relax\sphinxstyleindexentry{diffsph.profiles.hfactors}\sphinxstyleindexpageref{diffsph.profiles:\detokenize{module-diffsph.profiles.hfactors}}
\item\relax\sphinxstyleindexentry{diffsph.profiles.massmodels}\sphinxstyleindexpageref{diffsph.profiles:\detokenize{module-diffsph.profiles.massmodels}}
\item\relax\sphinxstyleindexentry{diffsph.profiles.templates}\sphinxstyleindexpageref{diffsph.profiles:\detokenize{module-diffsph.profiles.templates}}
\item\relax\sphinxstyleindexentry{diffsph.pyflux}\sphinxstyleindexpageref{diffsph:\detokenize{module-diffsph.pyflux}}
\item\relax\sphinxstyleindexentry{diffsph.spectra}\sphinxstyleindexpageref{diffsph.spectra:\detokenize{module-diffsph.spectra}}
\item\relax\sphinxstyleindexentry{diffsph.spectra.analytics}\sphinxstyleindexpageref{diffsph.spectra:\detokenize{module-diffsph.spectra.analytics}}
\item\relax\sphinxstyleindexentry{diffsph.spectra.synchrotron}\sphinxstyleindexpageref{diffsph.spectra:\detokenize{module-diffsph.spectra.synchrotron}}
\item\relax\sphinxstyleindexentry{diffsph.utils}\sphinxstyleindexpageref{diffsph.utils:\detokenize{module-diffsph.utils}}
\item\relax\sphinxstyleindexentry{diffsph.utils.consts}\sphinxstyleindexpageref{diffsph.utils:\detokenize{module-diffsph.utils.consts}}
\item\relax\sphinxstyleindexentry{diffsph.utils.dictionaries}\sphinxstyleindexpageref{diffsph.utils:\detokenize{module-diffsph.utils.dictionaries}}
\item\relax\sphinxstyleindexentry{diffsph.utils.tools}\sphinxstyleindexpageref{diffsph.utils:\detokenize{module-diffsph.utils.tools}}
\end{sphinxtheindex}

\renewcommand{\indexname}{Index}
\printindex
\end{document}